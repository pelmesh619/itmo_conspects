\clearpage

\section{X. Программа экзамена в 2024/2025}


\begin{enumerate}
\subsection{X.1. Числовые ряды.}

    \item Определение числового ряда, понятие суммы ряда.

    \hyperlink{numberseriesdefinition}{Определение числового ряда}: $\{u_1, u_2, \dots, u_n, \dots\} = \{u_n\}$ называется числовым рядом

    $u_n$ называется общим членом ряда
    
    \hyperlink{sumofseriesdefinition}{Понятие суммы ряда}: Частичная сумма ряда $S_n \stackrel{def}{=} \sum_{k = 1}^{n} u_k$

    Если $\exists \lim_{n \to\infty} S_n = S \in \Real$, то ряд $\sum_{n = 1}^\infty u_n$ называют сходящимся,
    а $S$ называют суммой ряда $\sum_{n = 1}^\infty u_n = S$

    \item Сходимость числового ряда. Эталонные ряды: геометрический, гармонический.

    \hyperlink{seriesconvergence}{Сходимость числового ряда}: Если $\exists \lim_{n \to\infty} S_n = S \in \Real$, то ряд $\sum_{n = 1}^\infty u_n$ называют сходящимся

    \hyperlink{geometricseries}{Геометрический ряд}: $\sum_{n = 0}^\infty b q^n$ - сходится при $|q| < 1$, тогда $S = \frac{b}{1 - q}$
    
    \hyperlink{harmonicseries}{Гармонический ряд}: $\sum_{n = 1}^\infty \frac{1}{n}$ - расходится

    \item Условия сходимости рядов: необходимое условие, критерий Коши.

    \hyperlink{necessarycondition}{Необходимое условие}: 

    \Ths Если $\sum_{n = 1}^\infty u_n$ сходится, то верно, что $\lim_{n \to \infty} u_n = 0$

    \hyperlink{cauchycriteria}{Критерий Коши}:

    \Ths $\sum_{n = 1}^\infty u_n$ сходится $\Longleftrightarrow \forall \varepsilon > 0 \ \exists \underset{n_0 = n_0 (\varepsilon)}{n_0 \in \Natural} \ | \ \forall m > n > n_0 \ \ \underset{|S_m - S_n| < \varepsilon}{|u_{n + 1} + \dots + u_m|} < \varepsilon$

    \item Знакоположительные числовые ряды, свойства.

    Знакоположительный ряд - ряд $\sum_{n = 1}^\infty u_n$ такой, что $u_n > 0$

    \hyperlink{seriesproperties}{Свойства рядов}:

    \begin{enumerate}
        \item Отбрасывание или добавление конечного числа членов ряда не влияет на сходимость, но влияет на сумму

        \item $\sum_{n = 1}^\infty u_n = S \in \Real, \quad \alpha \in \Real$

        Тогда $\alpha \sum_{n = 1}^\infty u_n = \sum_{n = 1}^\infty \alpha u_n = \alpha S$

        \item $\sum_{n = 1}^\infty u_n = S \in \Real$, $\sum_{n = 1}^\infty v_n = \sigma \in \Real$

        Тогда $\sum_{n = 1}^\infty (u_n \pm v_n) = S \pm \sigma$ - сходится
    \end{enumerate}

    \item Признаки сходимости знакоположительных числовых рядов: признаки сравнения.

    \hyperlink{comparisonsign}{Признак сравнения в неравенствах}:

    a) $\letsymbol 0 < u_n \leq v_n. \quad$ Тогда $\sum v_n$ сходится $\Longrightarrow \sum u_n$ сходится

    б) $\letsymbol 0 \leq v_n \leq u_n. \quad$ Тогда $\sum v_n$ расходится $\Longrightarrow \sum u_n$ расходится

    \hyperlink{limitcomparisonsign}{Предельный признак сравнения}:

    $\lim_{n \to \infty} \frac{u_n}{v_n} = q \in \Real \setminus \{0\} \ \Longrightarrow \
    \begin{sqcases}
        \sum u_n \text{ сходится, если } \sum v_n \text{ сходится} \\
        \sum u_n \text{ расходится, если } \sum v_n \text{ расходится}
    \end{sqcases}$
    
    \item Признак Даламбера, радикальный признак Коши.

    \hyperlink{dalambersign}{Признак Даламбера}:

    $\sum_{n = 1}^\infty u_n$ - исследуемый, $\exists \mathcal{D} = \lim_{n \to \infty} \frac{u_{n + 1}}{u_n} \in \Real^+$

    \begin{tabular}{ll}
        а) $0 \leq \mathcal{D} < 1$ & $\Longrightarrow \sum u_n$ сходится                                \\

        б) $\mathcal{D} > 1$        & $\Longrightarrow \sum u_n$ расходится                              \\

        в) $\mathcal{D} = 1$        & $\Longrightarrow$ ничего не следует, требуется другое исследование \\
    \end{tabular}

    \hyperlink{cauchyradicalsign}{Радикальный признак Коши}:

    $\sum_{n = 1}^\infty u_n \quad\quad u_n \geq 0$ и $\exists \lim_{n \to \infty} \sqrt[n]{u_n} = K \in \Real$

    а) $0 \leq K < 1 \Longrightarrow \sum u_n$ сходится

    б) $K > 1 \Longrightarrow \sum u_n$ расходится

    в) $K = 1 \Longrightarrow$ требуется другое исследование

    \item Интегральный признак сходимости.

    \hyperlink{cauchyintegralsign}{Интегральный признак Коши}:

    Если существует $f(x) : [1; +\infty] \to \Real^+, f(x)$ монотонно убывает, $f(n) = u_n$, то $\sum_{n = 1}^\infty u_n$ и $\int_{1}^\infty f(x) dx$ одновременно сходятся или расходятся

    \item Знакочередующиеся ряды. Теорема Лейбница. Оценка остатка ряда.

    \hyperlink{alternatingsignseries}{Знакочередующиеся ряды}: $\sum_{n = 0}^\infty (-1)^n u_n$ ($u_n > 0$) - знакочередующийся ряд

    \hyperlink{leibniztheorem}{Признак Лейбница}: Если для знакочередующегося ряда $\sum_{n = 0}^\infty (-1)^n u_n$ верно,
    что \ $\underset{n \to \infty}{u_n \to 0}$ \ и \ $|u_1| > |u_2| > \dots > |u_n|$,
    то ряд $\sum_{n = 0}^\infty (-1)^n u_n$ сходится

    \hyperlink{seriesremainderevaluation}{Оценка остатка ряда}: для знакочередующегося ряда $R_{n + 1} < u_{n + 1}$

    \item Знакопеременные ряды. Абсолютная и условная сходимость.

    \hyperlink{changingsignseries}{Знакопеременные ряды}: $\sum_{n = 1}^\infty u_n$, где $u_n$ - любого знака и не все $u_n$ одного знака

    \hyperlink{absoluteconvergence}{Абсолютная сходимость}: $\sum_{n = 1}^\infty |u_n|$ сходится $\Longrightarrow \sum_{n = 1}^\infty u_n$ сходится абсолютно

    \hyperlink{conditionalconvergence}{Условная сходимость}: $\sum_{n = 1}^\infty u_n$ сходится условно, если $\sum_{n = 1}^\infty |u_n|$ расходится

\subsection{X.2. Функциональные ряды.}

    \item Функциональные ряды. Сходимость. Поточечная и равномерная сходимость ряда.
    Мажорирующий ряд.

    \hyperlink{functionalseries}{Функциональные ряды}: $\sum_{n = 1}^\infty u_n(x)$, где $u_n(x) : E \subset \Real \to \Real$ называется функциональным

    \hyperlink{functionalseriesconvergence}{Сходимость}: Если ряд $\sum_{n = 1}^\infty u_n(x)$ сходится при всех $x$ из некоторого множества $E$, то сумма ряда -
    функция $S(x)$

    Поточечная сходимость: Ряд сходится поточечно, если $\forall x \in D \ \exists \lim_{n \to \infty} S_n(x)$

    \hyperlink{uniformconvergence}{Равномерная сходимость}:  $\sum_{n = 1}^\infty u_n(x)$ равномерно сходится в области $D \overset{def}{\Longleftrightarrow}$

    $\forall \varepsilon > 0 \ \exists \underset{n_0 = n_0(\varepsilon)}{n_0 \in \Natural} \ | \ \forall n > n_0 \ |R_n(x)| < \varepsilon$

    Пример: $\sum_{n = 1}^\infty x^n$ на $[0,1)$ сходится поточечно, но не сходится равномерно

    \hyperlink{majorseries}{Мажорирующий ряд}: ряд $\sum_{n = 1}^\infty \alpha_n$ называется мажорирующим, если по 
    признаку Вейерштрасса с помощью него можно сказать, что $\sum_{n = 1}^\infty u_n(x)$ сходится

    \item Признак Вейерштрасса.

    \hyperlink{weierstrassign}{Признак Вейерштрасса}: $\exists \sum_{n = 1}^\infty \alpha_n$ - числовой ряд такой, что $\alpha_n > 0$, $\sum \alpha_n$ сходится,
    $|u_n(x)| \leq \alpha_n \ \forall n$

    Тогда $\sum_{n = 1}^\infty u_n(x)$ равномерно сходящийся, а $\sum_{n = 1}^\infty \alpha_n$ называют мажорирующим

    \item Непрерывность суммы ряда.

    \hyperlink{functionalsumcontinuity}{Непрерывность суммы ряда}: \Ths Если ряд $\sum_{n = 1}^\infty u_n(x) \ (u_n(x) \in C_{[a, b]})$ мажорируем в $D = [a, b]$, то 
    его сумма $S(x)$ непрерывна на $[a, b]$

    \item Свойства равномерно сходящихся рядов (дифференцирование и интегрирование суммы
    ряда).

    \hyperlink{functionalsumintegral}{Интегрирование}: \Ths Если ряд мажорируется на $[a, b]$ и $u_n(x)$ непрерывна на $[a, b]$, то определен $\int_{x_0}^y S(x)dx$ и 
    $\int_{x_0}^x S(x)dx  = \sum_{n = 1}^\infty \int_{x_0}^x u_n(x) dx$

    \hyperlink{functionalsumderivative}{Дифференцирование}: \Ths $\sum_{n = 1}^\infty u_n(x)$ мажорируем на $[a, b]$ и $u_n(x) \in C^\prime_{[a, b]}$.
    Тогда $S^\prime(x) = \sum_{n = 1}^\infty u^\prime_n(x)$

    \item Степенные ряды. Теорема Абеля. Радиус сходимости.

    \hyperlink{powerseries}{Степенной ряд}: $\sum_{n = 0}^\infty c_n(x - x_0)^n, \ c_n \in \Real, x_0 \in \Real$ - степенной ряд с центром $x_0$ (в точке $x_0$, по степеням $(x - x_0)$)

    \hyperlink{abelstheorem}{Теорема Абеля}: \ThNs{Абеля} 

    1) $\sum_{n = 0}^\infty c_n x^n$ сходится в точке $x_1$. Тогда ряд сходится абсолютно для любого $x$, который $|x| < |x_1|$

    2) $\sum_{n = 0}^\infty c_n x^n$ расходится в точке $x_2$. Тогда ряд расходится для любого $x$, который $|x| > |x_2|$

    \hyperlink{convergenceradius}{Радиус сходимости}: $R \in \Real^+ \ \Big| \ \forall |x| < R $ ряд сходится, а $\forall |x| > R$ ряд расходится, тогда $R$ называют радиусом сходимости

    \item Ряд Тейлора. Стандартные разложения элементарных функций.

    \hyperlink{taylorsseries}{Ряд Тейлора}: $f(x) \in C^\infty_{U_0(x_0)}$ и $R_n(x) = \frac{f^{(n + 1)}(\xi)}{(n + 1)!} (x - x_0)^{n + 1}$, $\xi$ между $x$ и $x_0$
    
    \Ths Если $R_n(x) \underset{n \to \infty}{\longrightarrow} 0$, то $f(x) = \sum_{n = 0}^\infty \frac{f^{(n)}(x_0)}{n!} (x - x_0)^n$ - ряд Тейлора

    \hyperlink{taylorsseriesoffunctions}{Разложения функций}:

    \begin{tabular}{rclcl}
        Функция & & \multicolumn{3}{c}{Ряд Тейлора} \\
        \hline
        $e^x$ & $=$ & $\sum_{n = 0}^\infty \frac{x^n}{n!}$ & $=$ & $1 + x + \frac{x^2}{2} + \frac{x^3}{6} + \dots$ \\
        \hline
        $\sin x$ & $=$ & $\sum_{n = 0}^\infty \frac{(-1)^n}{(2n + 1)!}x^{2n + 1}$ & $=$ & $x - \frac{x^3}{3!} + \frac{x^5}{5!} + \dots$ \\
        \hline
        $\cos x$ & $=$ & $\sum_{n = 0}^\infty \frac{(-1)^n}{(2n)!}x^{2n}$ & $=$ & $1 - \frac{x^2}{2!} + \frac{x^4}{4!} + \dots$ \\
        \hline
        $\mathrm{sh} x$ & $=$ & $\sum_{n = 0}^\infty \frac{x^{(2n + 1)}}{(2n + 1)!}$ & $=$ & $x + \frac{x^3}{3!} + \frac{x^5}{5!} + \dots$\\
        \hline
        $\mathrm{ch} x$ & $=$ & $\sum_{n = 0}^\infty \frac{x^{(2n)}}{(2n)!}$  & $=$ & $1 + \frac{x^2}{2!} + \frac{x^4}{4!} + \dots$\\
        \hline 
        $(1 + x)^m$ & $=$ & $\sum_{k = 0}^\infty C_m^k x^k$ & $=$ & $1 + mx + m(m - 1)x^2 + \dots$\\
        \hline
        $\ln(1 + x)$ & $=$ & $\sum_{n = 0}^\infty (-1)^n \frac{x^{n + 1}}{n + 1}$ & $=$ & $x - \frac{x^2}{2} + \frac{x^3}{3} - \frac{x^4}{4} + \dots$ \\
    \end{tabular}

    \item Ортогональные системы функций и ряды Фурье. Определение тригонометрического ряда
    Фурье для функции на отрезке $[-\pi, \pi]$. Теорема Дирихле.

    Оргогональные системы функций: 
    Система функций $\{e_n\} \ (e_n \in C_{[a, b]})$ называется ортогональной, 
    если $\forall i \neq j \ (e_i, e_i) \neq 0$, $(e_i, e_j) = 0$, где
    $(f(x), g(x)) = \int_{a}^{b} f(x) g(x) dx$

    \hyperlink{fouriersseries}{Ряд Фурье}: Пусть $f(x)$ $2\pi$-периодична на интервале $[-\pi;\pi]$, тогда ее ряд Фурье - $f(x) = \frac{a_0}{2} + \sum_{n = 1}^\infty (a_n \cos nx + b_n \sin nx)$, где 
    
    $\frac{a_0}{2} = \frac{1}{2\pi} \int_{-\pi}^\pi f(x) dx$ 
    
    $a_k = \frac{1}{\pi} \int_{-\pi}^\pi f(x) \cos kx dx$
    
    $b_k = \frac{1}{\pi} \int_{-\pi}^\pi f(x) \sin kx dx$

    \hyperlink{dirichletstheorem}{Теорема Дирихле}:

    \Ths $f(x)$ - $2\pi$-периодична, на $[-\pi, \pi]$ $f(x)$ - кусочно монотонна и ограничена (то есть имеет конечное число конечных разрывов). 
    Тогда в точках непрерывности $f(x)$ представляется рядом Фурье $S(x) = \frac{a_0}{2} + \sum_{n = 1}^\infty (a_n \cos nx + b_n \sin nx)$,
    а в точках разрыва $x_0$ $S(x_0) = \frac{1}{2} (f(x_0 + 0) + f(x_0 - 0))$

    \item Тригонометрический ряд Фурье на произвольном отрезке (сдвиг, растяжение)

    \hyperlink{shifttheorem}{Теорема о сдвиге}: \ThNs{о сдвиге} Ряд Фурье не изменится, если $[-\pi, \pi]$ заменить на $[a; a + 2\pi]$

    \hyperlink{stretchingtheorem}{Теорема о растяжении}: \ThNs{о растяжении} Для $f : [a, b] \to \Real$ ($2l$-периодична, где $l = \frac{b - a}{2}$) растяжение промежутка приводит к разложению
    $\varphi(t) = \frac{a_0}{2} + \sum_{k = 1}^\infty a_k \cos \frac{\pi k}{l} t + b_k \sin \frac{\pi k}{l} t$, где

    $a_0 = \frac{1}{l} \int_{-l}^l f(x) dx$

    $a_k = \frac{1}{l} \int_{-l}^l f(x) \cos \frac{\pi k}{l} x dx$

    $b_k = \frac{1}{l} \int_{-l}^l f(x) \sin \frac{\pi k}{l} x dx$

    

\end{enumerate}