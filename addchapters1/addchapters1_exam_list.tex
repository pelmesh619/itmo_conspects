\section{X. Программа экзамена в 2024/2025}

\begin{center}
    \textbf{1. Числовые ряды.}
\end{center}

\begin{enumerate}
    \item Определение числового ряда, понятие суммы ряда.
    \item Сходимость числового ряда. Эталонные ряды: геометрический, гармонический.
    \item Условия сходимости рядов: необходимое условие, критерий Коши.
    \item Знакоположительные числовые ряды, свойства.
    \item Признаки сходимости знакоположительных числовых рядов: признаки сравнения.
    \item Признак Даламбера, радикальный признак Коши.
    \item Интегральный признак сходимости.
    \item Знакочередующиеся ряды. Теорема Лейбница. Оценка остатка ряда.
    \item Знакопеременные ряды. Абсолютная и условная сходимость.

\begin{center}
    \textbf{2. Функциональные ряды.}
\end{center}

    \item Функциональные ряды. Сходимость. Поточечная и равномерная сходимость ряда.
    Мажорирующий ряд.
    \item Признак Вейерштрасса.
    \item Непрерывность суммы ряда.
    \item Свойства равномерно сходящихся рядов (дифференцирование и интегрирование суммы
    ряда).
    \item Степенные ряды. Теорема Абеля. Радиус сходимости.
    \item Ряд Тейлора. Стандартные разложения элементарных функций.
    \item Ортогональные системы функций и ряды Фурье. Определение тригонометрического ряда
    Фурье для функции на отрезке $[-\pi, \pi]$. Теорема Дирихле.
    \item Тригонометрический ряд Фурье на произвольном отрезке (сдвиг, растяжение)
\end{enumerate}