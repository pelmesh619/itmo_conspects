\documentclass[12pt]{article}
\usepackage{preamble}

\pagestyle{fancy}
\fancyhead[LO,LE]{Дополнительные главы \\ высшей математики}
\fancyhead[RO,RE]{Лекции Далевская О. П.}

\fancyfoot[L]{\scriptsize исходники найдутся тут: \\ \url{https://github.com/pelmesh619/itmo_conspects} \Cat}

\renewcommand{\thesection}{}

\begin{document}

    \tableofcontents
    \clearpage

        % begin addchapters1_2024_09_06.tex

    \section{1. Ряды}

    \subsection{1.1 Числовые ряды}

    \tedxtbf{1. Определения}

    \Mems Числовая последовательность: $\{u_n\} = \{u_1, u_2, \dots, u_n, \dots\}, u_n \in \Real$

    \ExNs{1} Бесконечно убывающая геометрическая прогрессия: $u_n = b q^n, \quad
    \frac{1}{2^n} \stackrel{n = 0,1,\dots}{=} \{1, \frac{1}{2}, \frac{1}{4}, \dots\}$

    \ExNs{2} $u_n = 1, -1, 1, -1, \dots$

    \Def $\{u_n\}$ - последовательность

    $\sum_{n = 1}^{\infty} u_n = u_1 + u_2 + \dots + u_n + \dots$ называется числовым рядом

    \Notas Начальное значение $n$ произвольно (целое)

    \Ex $u_n = \frac{1}{(n - 4)^3}, \quad n = 5, 6, \dots$

    $u_n = \frac{1}{n^3}, \quad n = 2024, 2025, \dots$

    \Notas $u_n$ называется общим членом ряда

    \Nota Существует ли сумма $\sum_{n = 1}^{\infty} u_n$ и в каком смысле?

    \ExN{3} $\sum_{n = 1}^{\infty} n = 1 + 2 + 3 + \dots = \infty$ - существует, но бесконечная

    \ExNs{4} $\sum_{n = 0}^{\infty} (-1)^n = 1 - 1 + 1 - 1 + 1 - 1 + \dots =
    \begin{sqcases}
        0 + 0 + \dots = 0 \\
        1 + 0 + 0 + \dots = 1
    \end{sqcases}$

    \ExNs{5} $\sum_{n = 0}^\infty \frac{1}{2^n} = 1 + \frac{1}{2} + \frac{1}{4} + \dots = 2$

    \Def Частичная сумма ряда $S_n \stackrel{def}{=} \sum_{k = 1}^{n} u_k$

    \Notas Последовательность частичных сумм - $S_1, S_2, S_3, S_4, \dots$

    \Ex $1 + \frac{1}{2} + \frac{1}{4} + \frac{1}{8} + \dots$

    $S_1 = u_1 = 1 \quad S_2 = \frac{3}{2} \quad S_3 = \frac{7}{4} \quad S_4 = \frac{15}{8}$

    $\lim_{n \to \infty} S_n = ?$, но проблема заключается в том, что бы найти формулу для $S_n$

    \Def Если $\exists \lim_{n \to\infty} S_n = S \in \Real$, то ряд $\sum_{n = 1}^\infty u_n$ называют сходящимся,
    а $S$ называют суммой ряда $\sum_{n = 1}^\infty u_n = S$

    \Notas В противном случае ряд расходится, суммы не может быть или она бесконечна

    \Ex Поиск суммы по определению

    $\sum_{n = 1}^\infty \frac{1}{n (n + 1)}$

    $u_n = \frac{1}{n(n + 1)} = \frac{1}{n} - \frac{1}{n + 1}$

    $S_n = \sum_{k = 1}^n u_k = \sum_{k = 1}^n \left(\frac{1}{k} - \frac{1}{k + 1}\right) = 1 - \frac{1}{2} + \frac{1}{2} - \frac{1}{3} + \frac{1}{3} + \dots + \frac{1}{n} - \frac{1}{n + 1} = 1 - \frac{1}{n + 1}$

    $\lim_{n \to \infty} S_n = \lim_{n \to \infty} \left(1 - \frac{1}{n + 1}\right) = 1 = S = \sum_{n = 1}^\infty \frac{1}{n(n + 1)}$

    \Nota При исследовании на сходимость используются эталонные ряды

    \Ex Геометрический ряд (эталонный): \ \ $\sum_{n = 0}^\infty b q^n$

    $S_n = \sum_{k = 0}^n b q^k = b (1 + q + q^2 + q^3 + \dots + q^n) = b \frac{1 - q^n}{1 - q}$

    Исследуем предел $\lim_{n\to\infty} S_n$:

    $|q| < 1 \quad\quad \lim_{n\to\infty} S_n = \frac{b}{1 - q} \lim_{n \to\infty} (1 - q^n) = \frac{b}{1 - q}$

    $|q| > 1 \quad\quad \lim_{n\to\infty} S_n = \infty (q^n \to\infty)$

    $|q| = 1 \quad\quad \lim_{n\to\infty} b \frac{0}{0} ? \quad\quad \sum_{n = 0}^\infty b q^n = \sum_{n = 0}^\infty b = \infty \quad (b \neq 0)$

    $q = -1 \quad\quad \sum_{n = 0}^\infty b (-1)^n$ - расходится (из четвертого примера)

    \Lab Доказать при $q = -1$ по def $(S_n = ?)$

    \clearpage

    \section{2. Свойства числовых рядов}

    \Notas Свойства рядов используются в арифметических операциях с рядами и при исследовании на сходимость

    \begin{MyTheorem}
        \ThNs{1} Отбрасывание или добавление конечного числа членов ряда не влияет на сходимость, но влияет на сумму

        $\sum_{n = 1}^\infty u_n$ и $\sum_{n = k > 1}^\infty u_n$ одновременно сходятся или расходятся
    \end{MyTheorem}

    \begin{tcolorbox}
        $\Box$

        $S^u_n = \sum_{n = 1}^{\infty} u_n = u_1 + u_2 + u_3 + \dots + u_k + u_{k + 1} + \dots + u_n + \dots$

        $S^v_n = \sum_{n = k}^\infty v_n \quad\quad u_n = v_n \quad \forall n \geq k$

        $S_n^u = \underset{\sigma \in \Real}{\underbrace{u_1 + u_2 + \dots + u_{k - 1}}} + \underset{S^v_n}{\underbrace{u_{k} + \dots + u_n}} = \sigma + S^v_n$

        $\lim_{n \to \infty} S_n^u = \lim_{n \to \infty} (\sigma + S^v_n) = \sigma + \lim_{n + \infty} S_n^v$

        \smallvspace

        Оба предела либо существуют (либо конечны, либо нет), либо не существуют

        % old proof (basically right but too complicated):

%        $S_n = \sum_{k = 1}^n u_k \quad\quad \sigma_t = \sum_{r = 1}^t v_r \quad\quad v_r = u_{k - 1 + r}$
%
%        $\lim_{n \to \infty} S_n = \lim_{n \to\infty} (u_1 + u_2 + u_3 + \dots + u_k + u_{k + 1} + \dots + u_n) = u_1 + \dots + u_{k - 1} + \lim_{t \to \infty} \sigma_t$

        $\Box$
    \end{tcolorbox}

    \begin{MyTheorem}
        \ThNs{2} $\sum_{n = 1}^\infty u_n = S \in \Real, \quad \alpha \in \Real$

        Тогда $\alpha \sum_{n = 1}^\infty u_n = \sum_{n = 1}^\infty \alpha u_n = \alpha S$
    \end{MyTheorem}

    \begin{tcolorbox}
        $\Box$ По свойству пределов $\Box$
    \end{tcolorbox}

    \begin{MyTheorem}
        \ThNs{3} $\sum_{n = 1}^\infty u_n = S \in \Real$, $\sum_{n = 1}^\infty v_n = \sigma \in \Real$

        Тогда $\sum_{n = 1}^\infty (u_n \pm v_n) = S \pm \sigma$ - сходится
    \end{MyTheorem}

    \begin{tcolorbox}
        $\Box$ По свойству пределов $\lim_{n \to\infty} (S_n \pm \sigma_n) = \lim_{n \to\infty} S_n \pm \lim_{n \to\infty} \sigma_n = S \pm \sigma$ $\Box$
    \end{tcolorbox}

    \Nota Обратное неверно! Теорема разрешает складывать и вычитать сходящиеся ряды, но из сходимости суммы рядов не следует сходимость каждого из них

    \Exs $\sum_{n = 1}^\infty \frac{1}{n (n + 1)} = 1$, \quad но: $\frac{1}{n (n + 1)} = \frac{1}{n} - \frac{1}{n + 1}$,
    а $\sum_{n = 1}^\infty \frac{1}{n}$ и $\sum_{n = 1}^\infty \frac{1}{n + 1}$ расходятся

    \Nota Докажем расходимость $\sum_{n = 1}^\infty \frac{1}{n}$

    \Exs Гармонический ряд (эталонный)

    $\sum_{n = 1}^\infty u_n = 1 + \frac{1}{2} + \frac{1}{3} + \frac{1}{4} + \frac{1}{5} + \frac{1}{6} + \frac{1}{7} + \frac{1}{8} + \frac{1}{9} + \frac{1}{10} + \frac{1}{11} + \frac{1}{12} + \frac{1}{13} + \frac{1}{14} + \frac{1}{15} + \frac{1}{16} + \dots$

    $\sum_{n = 1}^\infty v_n = 1 + \frac{1}{2} + \frac{1}{4} + \frac{1}{4} + \frac{1}{8} + \frac{1}{8} + \frac{1}{8} + \frac{1}{8} + \frac{1}{16} + \frac{1}{16} + \frac{1}{16} + \frac{1}{16} + \frac{1}{16} + \frac{1}{16} + \frac{1}{16} + \frac{1}{16} + \dots = \\
    \quad = 1 + \frac{1}{2} \cdot 1 + \frac{1}{4} \cdot 2 + \frac{1}{8} \cdot 4 + \frac{1}{16} \cdot 8 + \dots = 1 + \sum_{n = 1}^\infty \frac{1}{2} = \infty$

    А так как нижний ряд почленно меньше верхнего, а нижний расходится, то и верхний расходится

    Так как $u_n \geq v_n$, то $S_n \geq \sigma_n$, тогда $\lim_{n \to\infty} S_n \geq \lim_{n \to\infty} \sigma_n$

    $\sigma_n = 1 + \frac{1}{2} \cdot n \to \infty \Longrightarrow S_n \to \infty$

    \begin{MyTheorem}
        \ThNs{4} Если ряд сходится к числу $S$, то члены ряда можно группировать произвольным образом, не переставляя, и сумма всех рядов будет равна $S$

        Группировка означает выделение различных подпоследовательностей из последовательности частичных сумм
    \end{MyTheorem}

    \begin{tcolorbox}
        $\Box$

        Так как $\lim_{n \to\infty} S_n = S$, то $\lim_{k \to\infty} S_n^{(k)} = S$, где $S_n^{(k)}$ - подпоследовательность $S_n$

        $\Box$
    \end{tcolorbox}

    \Exs Было $\sum (-1)^n = 1 - 1 + 1 - 1 + 1 - 1 + 1 - \dots = \begin{sqcases}
                                                                     0, \\ 1,
    \end{sqcases}$ так как ряд расходится

    \Nota В условиях \Ths важно, что переставлять члены ряда нельзя

    \Exs $\sum_{n = 1}^\infty \frac{(-1)^{n - 1}}{n} = 1 - \frac{1}{2} + \frac{1}{3} - \frac{1}{4} + \frac{1}{5} - \frac{1}{6} + \frac{1}{7} - \frac{1}{8} + \frac{1}{9} - \frac{1}{10} + \frac{1}{11} - \frac{1}{12} + \frac{1}{13} - \frac{1}{14} + \frac{1}{15} + \dots$

    Далее будет доказано, что этот ряд сходится

    Найдем сумму, переставив члены ряда

    $S = \sum_{n = 1}^\infty \frac{(-1)^{n - 1}}{n} = 1 - \frac{1}{2} + \left(\frac{1}{3} - \frac{1}{6}\right) - \frac{1}{4} + \left(\frac{1}{5} - \frac{1}{10}\right) - \frac{1}{8} + \left(\frac{1}{7} - \frac{1}{14}\right) - \frac{1}{12} + \left(\frac{1}{9} - \frac{1}{18}\right) + \dots\\
    S = 1 - \frac{1}{2} \left(1 + \frac{1}{2} - \frac{1}{3} + \frac{1}{4} - \frac{1}{5} + \frac{1}{6}\right) = 1 + \frac{1}{2} \left(-1 - \frac{1}{2} + \frac{1}{3} - \dots\right) =
    1 + \frac{1}{2} \left(-2 + 1 - \frac{1}{2} + \frac{1}{3} + \dots\right) = \frac{1}{2} \left(1 - \frac{1}{2} + \frac{1}{3} - \dots\right) = \frac{1}{2} S$ \ ?!

    \Notas Можно доказать, что в подобных рядах перестановкой членов можно получить любое наперед заданное число

    \Notas Сходящиеся ряды допускают умножение, но непочленное. В действительности используют формулы перемножения рядов (см. литературу)

    $\sum_{n = 1}^\infty u_n = S, \sum_{n = 1}^\infty v_n = \sigma$

    Тогда $\left(\sum_{n = 1}^\infty u_n\right)\left(\sum_{n = 1}^\infty v_n\right) = S\sigma$

    \mediumvspace

    \section{3. Условия сходимости рядов}

    \subsection{3.1. Необходимое}

    \begin{MyTheorem}
        \Ths $\sum_{n = 1}^\infty u_n = S \in \Real \Longrightarrow \lim_{n \to \infty} u_n = 0$
    \end{MyTheorem}

    \begin{tcolorbox}
        $\Box$

        $\lim_{n \to \infty} S_n = S, \quad \lim_{n\to\infty} (S_n - S_{n - 1}) = 0$

        $\Box$
    \end{tcolorbox}

    \Nota Обратное неверно! (см. гармонический ряд)

    \Ex $\sum_{n = 1}^\infty (2n + 3) \sin \frac{1}{n}$

    $\lim_{n \to \infty} (2n + 3) \sin \frac{1}{n} = \lim_{n \to \infty} (2 + \frac{3}{n}) = 2 \neq 0$

    \subsection{3.2. Критерии (Необходимое и Достаточное условия)}

    \Mem Критерий Коши для последовательности: $\{x_n\}$ сходится $\Longleftrightarrow \forall \varepsilon > 0 \ \exists \underset{n_0 = n_0 (\varepsilon)}{n_0 \in \Natural} \ | \ \forall m > n > n_0 \ |x_m - x_k| < \varepsilon$

    \begin{MyTheorem}
        \Ths (без док-ва) $\quad \sum_{n = 1}^\infty u_n$ сходится $\Longleftrightarrow \forall \varepsilon > 0 \ \exists \underset{n_0 = n_0 (\varepsilon)}{n_0 \in \Natural} \ | \ \forall m > n > n_0 \ \underset{|S_m - S_k| < \varepsilon}{|u_n + \dots + u_m|} < \varepsilon$
    \end{MyTheorem}

    \Nota Хвост ряда попадает в $\varepsilon$-трубу

    \Notas Критерий не удобен для непосредственного исследования на сходимость, в отличии от признаков
    % end addchapters1_2024_09_06.tex

    % begin addchapters1_2024_09_20.tex

    \mediumvspace

    \textbf{3.3. Достаточное условие (признаки сходимости)}

    Здесь мы рассмотрим:

    \begin{enumerate}
        \item Признак сравнения (в неравенствах)
        \item Предельный признак сравнения
        \item Признак Даламбера
        \item Признак Коши (радикальный)
        \item Признак Коши (интегральный)
    \end{enumerate}

    Далее $\sum_{n = 1}^\infty u_n$ - исследуемый ряд, $\sum_{n = 1}^\infty v_n$ - вспомогательный (уже исследован на сходимость),
    для простоты $v_n, u_n > 0$ (для отрицательных доказывается аналогично)

    \mediumvspace

    \begin{MyTheorem}
        \ThNs{1} Признак сравнения (в неравенствах)

        a) $\letsymbol 0 < u_n \leq v_n. \quad$ Тогда $\sum v_n$ сходится $\Longrightarrow \sum u_n$ сходится

        б) $\letsymbol 0 \leq v_n \leq u_n. \quad$ Тогда $\sum v_n$ расходится $\Longrightarrow \sum u_n$ расходится
    \end{MyTheorem}

    \begin{MyProof}
        $\Box$

        а) Строим частичные суммы:

        $\sum v_n$ сходится $\Longleftrightarrow \exists \lim_{n \to \infty} \sigma_n = \sigma \in \Real$

        $S_n, \sigma_n$ возрастают и обе ограничены числом $\sigma$

        Следовательно $\exists \lim_{n \to \infty} S_n = S \leq \sigma$

        Аналогично пункт б)

        $\Box$
    \end{MyProof}

    \begin{MyTheorem}
        \ThNs{2} Предельный признак

        $\lim_{n \to \infty} \frac{u_n}{v_n} = q \in \Real \setminus \{0\} \ \Longrightarrow \
        \begin{sqcases}
            \sum u_n \text{ сходится, если } \sum v_n \text{ сходится} \\
            \sum u_n \text{ расходится, если } \sum v_n \text{ расходится}
        \end{sqcases}$
    \end{MyTheorem}

    \begin{MyProof}
        $\Box$

        По определению предела

        $\exists \lim_{n \to \infty} \frac{u_n}{v_n} = q \in \Real \Longleftrightarrow \forall \varepsilon > 0 \ \exists n_0 \in \Natural \ | \ \forall n > n_0 \ |\frac{u_n}{v_n} - q| < \varepsilon$

        $|\frac{u_n}{v_n} - q| < \varepsilon \Longleftrightarrow q - \varepsilon < \frac{u_n}{v_n} < q + \varepsilon$

        $(q - \varepsilon) v_n < u_n < (q + \varepsilon) v_n$

        а) Если $\sum v_n$ сходится, то из правой части неравенства: $0 < u_n < (q + \varepsilon) v_n$

        По признаку сравнения $\sum u_n$ также сходится

        б) Если $\sum v_n$ расходится, то из левой части неравенства: $0 < (q - \varepsilon) v_n < u_n$

        Тогда по пункту б) \ThNs{1} $\sum u_n$ расходится

        $\Box$
    \end{MyProof}

    \ExN{1} $\sum_{n = 1}^\infty \frac{1}{n^2} = \sum_{n = 1}^\infty u_n$

    $\sum_{n = 1}^\infty \frac{1}{n(n + 1)} = \sum_{n = 1}^\infty v_n$ сходится

    $\frac{1}{n(n + 1)} = \frac{1}{n^2 + n} > \frac{1}{n^2 + 2n + 1} = \frac{1}{(n + 1)^2}$

    $\sum_{n = 0}^\infty \frac{1}{(n + 1)^2} = \sum_{n = 1}^\infty \frac{1}{n^2}$ сходится по признаку сравнения

    \ExN{2} $\sum_{n = 1}^\infty \frac{1}{n!} = \sum_{n = 1}^\infty u_n$

    $\sum_{n = 1}^\infty \frac{1}{2^n} = \sum_{n = 1}^\infty v_n$ сходится

    Начиная с некоторого $n_0$ $n! > 2^n$. Тогда $u_n < v_n$ при $n > n_0$, по признаку сравнения $\sum_{n = 1}^\infty \frac{1}{n!}$ сходится


    \ExN{3} $\sum_{n = 1}^\infty$ \asin \frac{1}{n}

    \Nota Члены рядов $u_n$ и $v_n$ - бесконечно малые последовательности. Иначе ряды расходятся по необходимому условию.
    Тогда в \ThNs{2} сравниваются порядки бесконечно малых, и ряды одновременно сходятся или расходятся, если $u_n$ и $v_n$ одного
    порядка малости. По этому принципу подбирается вспомогательный ряд

    $u_n = \arcsin \frac{1}{n} \underset{n \to \infty}{\sim} \frac{1}{n} = v_n \quad\quad \sum_{n = 1}^\infty v_n = \sum_{n = 1}^\infty \frac{1}{n}$ расходится

    \begin{MyTheorem}
        \ThNs{3} Признак Даламбера

        $\sum_{n = 1}^\infty u_n$ - исследуемый, $\exists \mathcal{D} = \lim_{n \to \infty} \frac{u_{n + 1}}{u_n} \in \Real^+$

        \begin{tabular}{ll}
            а) $0 \leq \mathcal{D} < 1$ & $\Longrightarrow \sum u_n$ сходится                                \\

            б) $\mathcal{D} > 1$        & $\Longrightarrow \sum u_n$ расходится                              \\

            в) $\mathcal{D} = 1$        & $\Longrightarrow$ ничего не следует, требуется другое исследование \\
        \end{tabular}
    \end{MyTheorem}

    \begin{MyProof}
        $\Box$

        а) По определению предела $\mathcal{D} = \lim_{n \to \infty} \frac{u_{n + 1}}{u_n}, \ 0 \leq \mathcal{D} < 1 \Longleftrightarrow$

        $\forall \varepsilon > 0 \ \exists n_0 \in \Natural \ | \ \forall n > n_0 \ |\frac{u_{n + 1}}{u_n} - \mathcal{D}| < \varepsilon \quad \Longleftrightarrow \quad \mathcal{D} - \varepsilon < \frac{u_{n + 1}}{u_n} < \mathcal{D} + \varepsilon$

        Так как $0 \leq \mathcal{D} < 1$, можно втиснуть число $r$ между $\mathcal{D}$ и 1: $\mathcal{D} < r < 1$

        Положим $\varepsilon = r - \mathcal{D}$, то есть $\mathcal{D} + \varepsilon = r$

        Смотрим правую часть $\frac{u_{n + 1}}{u_n} < r$ для $\forall n > n_0$, где $n_0 = n_0(\varepsilon), \varepsilon = r - \mathcal{D}$

        $u_{n_0 + 1} < r u_{n_0}$

        $u_{n_0 + 2} < r u_{n_0 + 1} < r^2 u_{n_0}$

        $u_{n_0 + l} < r^l u_{n_0}$

        $\sum_{n = 1}^\infty u_n = \underset{k}{\underbrace{u_1 + u_2 + \dots + u_{n_0 - 1}}} + u_{n_0} + \dots = k + \sum_{m = 1}^\infty v_m $ % u_{n_0} + u_{n_0 + 1} + \dots

        Члены $v_m < r^l u_{n_0}$; $u_{n_0}$ - фикс. число, а $\sum_{l = 1}^\infty r^l$ сходится как геометрический при $|r| < 1$

        Итак ряд $\sum_{l = 1}^\infty r^l u_{n_0}$ сходится и почленно превышает $\sum v_m = \left(\sum u_n\right) - k$

        То есть $\sum u_n$ сходится

        б) \Lab (взять $r$ между $\mathcal{D}$ и $1$, $1 < r < \mathcal{D}$, $\mathcal{D} - r = \varepsilon$)

        Сравнить $\sum u_n$ с $\sum r^l$ (расходящимся)

        $\Box$
    \end{MyProof}

    \ExN{1} $\sum_{n = 1}^\infty \frac{1}{n!} \quad\quad \mathcal{D} = \lim_{n \to \infty} \frac{u_{n + 1}}{u_n} = \lim_{n \to \infty} \frac{n!}{(n + 1)!} = \lim_{n \to \infty} \frac{1}{n + 1} = 0$ - сходится

    \ExN{2} $\sum_{n = 1}^\infty \frac{1}{n} \quad\quad \mathcal{D} = \lim_{n \to \infty} \frac{n}{n + 1} = 1$ - расходится

    $\sum_{n = 1}^\infty \frac{1}{n^2} \quad\quad \mathcal{D} = \lim_{n \to \infty} \frac{n^2}{(n + 1)^2} = 1$ - сходится

    \begin{MyTheorem}
        \ThNs{4} Радикальный признак Коши

        $\sum_{n = 1}^\infty u_n \quad\quad u_n \geq 0$ и $\exists \lim_{n \to \infty} \sqrt[n]{u_n} = K \in \Real$

        а) $0 \leq K < 1 \Longrightarrow \sum u_n$ сходится

        б) $K > 1 \Longrightarrow \sum u_n$ расходится
    \end{MyTheorem}

    \Notas $K = 1$ - ничего не следует

    \begin{MyProof}
        $\Box$

        а) По определению предела $\forall \varepsilon > 0 \ \exists n_0 \in \Natural \ | \ \forall n > n_0 \ |\sqrt[n]{u_n} - K| < \varepsilon$

        $\Longleftrightarrow k - \varepsilon < \sqrt[n]{u_n} < k + \varepsilon$ Положим $\varepsilon = r - K$, где $K < r < 1$

        $\Longrightarrow 0 \leq u_n < r^n$ - геом. ряд с $|r| < 1$, то есть $\sum r^n$ сходится

        б) Аналогично

        $\Box$
    \end{MyProof}

    \ExN{1} $\sum_{n \to \infty} \left(\frac{n}{n + 1}\right)^n \quad\quad K = \lim_{n \to \infty} \sqrt[n]{\left(\frac{n}{n + 1}\right)^n} = \lim_{n \to \infty} \frac{n}{n + 1} = 1$

    $\mathcal{D} = \lim_{n \to \infty} \frac{\left(1 - \frac{1}{n + 2}\right)^{n + 1}}{\left(1 - \frac{1}{n + 1}\right)^{n}}$

    Но $\lim_{n \to \infty} u_n = e^{-1} \neq 0$ - необходимое условие не выполняется

    \ExN{2} $\sum_{n = 1}^\infty \left(\frac{n}{n + 1}\right)^{n^2}, \quad\quad K = \lim_{n \to \infty} \sqrt[n]{\left(\frac{n}{n + 1}\right)^n} = e^{-1} < 1$ - сходится

    \begin{MyTheorem}
        \ThNs{5} Интегральный признак Коши

        Если существует $f(x) : [1; +\infty] \to \Real^+, f(x)$ монотонно убывает, $f(n) = u_n$, то $\sum_{n = 1}^\infty u_n$ и $\int_{1}^\infty f(x) dx$ одновременно сходятся или расходятся
    \end{MyTheorem}

    \begin{MyProof}
        $\Box$

        $\int_1^{+\infty} f(x) dx = \lim_{b \to \infty} \int_1^b f(x) dx$

        $\sum_{n = 2}^b u_n = u_2 \cdot 1 + u_3 \cdot 1 + \dots < \int_1^b f(x) dx < u_1 \cdot 1 + u_2 \cdot 1 + \dots = \sum_{n = 1}^{b - 1} u_n$

        Обозначим $\sum_{n = 1}^{b - 1} u_n = S_{b - 1}, \quad \sum_{n = 2}^{b} u_n = S_{b - 1} - u_1 + u_b$

        $0 < S_{b - 1} - u_1 + u_b < \int_1^b f(x) dx < S_{b - 1}$

        $0 < \sum_{n = 1}^{\infty} u_n - u_1 + u_b < \int_1^\infty f(x) dx < \sum_{n = 1}^\infty u_n$

        Если $\int$ сходится, то смотрим правую часть

        Если $\int$ расходится, то смотрим левую часть неравенства

        $\Box$
    \end{MyProof}

    \clearpage


    \section{4. Знакочередующиеся ряды}

    \Def $\sum_{n = 0}^\infty (-1)^n u_n$ ($u_n > 0$) - знакочередующийся ряд

    \begin{MyTheorem}
        \Ths Признак Лейбница

        Если для знакочередующегося ряда $\sum_{n = 0}^\infty (-1)^n u_n$ верно,
        что \ $\underset{n \to \infty}{u_n \to 0}$ \ и \ $|u_1| > |u_2| > \dots > |u_n|$,
        то ряд $\sum_{n = 0}^\infty (-1)^n u_n$ сходится
    \end{MyTheorem}
    % end addchapters1_2024_09_20.tex



\end{document}

