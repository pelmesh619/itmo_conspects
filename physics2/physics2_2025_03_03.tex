$subject$=Физические основы компьютерных \\ и сетевых технологий
$teacher$=Лекции Герта А. В.
$date$=03.03.2025


\section{Лекция 4. Электромагнитная индукция и магнетики}

\subsection{Электромагнитная индукция}

Закон Фарадея связывает ЭДС индукции с изменением магнитного потока через контур:

\[
    \varepsilon_{\text{инд}} = \oint \vec{E} \cdot d\vec{l} = -\frac{\partial \Phi}{\partial t},
\]

где магнитный поток $\Phi = \int_S \vec{B} \cdot d\vec{S}$. 
Переменное магнитное поле создаёт вихревое электрическое поле: 
силовые линии такого поля замкнуты, а само поле действует на заряды даже в отсутствие проводника. 
Это явление лежит в основе работы трансформаторов и генераторов. 
Например, если быстро вращать магнит рядом с катушкой, возникающая ЭДС заставляет ток течь в цепи.

Для произвольного контура изменение потока связано с локальным изменением $\vec{B}$:

\[
    \frac{\partial \Phi}{\partial t} = \int_S \frac{\partial \vec{B}}{\partial t} \cdot d\vec{S}.
\]

ЭДС можно выразить через интеграл от производной магнитного поля:

\[
    \varepsilon_S = -\int_S \frac{\partial \vec{B}}{\partial t} \cdot d\vec{S}.
\]

Это означает, что чем быстрее меняется магнитное поле или чем больше площадь контура, 
тем сильнее индуцированный ток.

\subsection{Магнитное поле в веществе}

В материальной среде полное поле $\vec{B}$ складывается из внешнего $\vec{B}_0$ и поля $\vec{B}'$, 
создаваемого намагниченностью вещества. Магнитные свойства описываются вектором намагниченности $\vec{J}$ - 
магнитным моментом единицы объема:

\[
    \vec{J} = n\vec{p}_m, \quad [\vec{J}] = \text{А/м},
\]


где $n$ — концентрация атомов, $\vec{p}_m$ — магнитный момент одного атома. Намагниченность связана с молекулярными токами: каждый атом ведёт себя как микроскопический виток с током $I' = \vec{J} \cdot \vec{l}$ (рис. 1). Суммарный молекулярный ток через поверхность $S$ равен циркуляции $\vec{J}$ по контуру:

\[
    I' = \oint \vec{J} \cdot d\vec{l}.
\]

\vspace{0.5cm}

\noindent\textbf{Основные уравнения.} Полный ток (внешний $I$ и молекулярный $I'$) создаёт магнитное поле:

\[
    \oint \vec{B} \cdot d\vec{l} = \mu_0(I + I').
\]

Чтобы исключить молекулярные токи, вводят \textbf{напряжённость магнитного поля} $\vec{H}$:

\[
    \vec{H} = \frac{\vec{B}}{\mu_0} - \vec{J}, \quad \oint \vec{H} \cdot d\vec{l} = I.
\]

В дифференциальной форме: $\nabla \times \vec{H} = \vec{j}$, где $\vec{j}$ — плотность внешних токов. 
Для большинства веществ $\vec{B}$ и $\vec{H}$ связаны линейно:

\[
    \vec{B} = \mu_0\mu\vec{H}, \quad \vec{J} = \chi\vec{H},
\]

где $\mu = 1 + \chi$ — относительная магнитная проницаемость, $\chi$ — магнитная восприимчивость. 

\vspace{0.5cm}

\noindent\textbf{Граничные условия.} По аналогии с электрическим полем при переходе между средами с 
разными коэффициентами магнитной проницаемости:

\begin{itemize}
    \item Нормальная компонента $B$ непрерывна: $B_{1n} = B_{2n}$ (магнитные заряды не существуют).
    \item Касательная компонента $H$ скачкообразно меняется при наличии поверхностных токов: $H_{2\tau} - H_{1\tau} = I_{\text{пов}}$. 
    Если токов нет, $H_{1\tau} = H_{2\tau}$, 
    а $B$-поле меняется пропорционально $\mu$: $B_{2\tau} = \frac{\mu_2}{\mu_1} B_{1\tau}$.
\end{itemize}

\subsection{Типы магнетиков и энергия поля}

\noindent\textbf{Диамагнетики} (медь, вода) слабо выталкиваются из поля ($\chi < 0$, $|\chi| \sim 10^{-5}$). 
Их атомы не имеют собственного момента; наведённые токи ослабляют внешнее поле. 

\textbf{Парамагнетики} (алюминий) слабо втягиваются ($\chi > 0$, $\chi \sim 10^{-3}$): 
тепловое движение мешает ориентации атомных моментов. 

\textbf{Ферромагнетики} (железо) сильно усиливают поле ($\mu \gg 1$) за счёт доменов - 
областей спонтанной намагниченности. При циклическом перемагничивании наблюдается гистерезис: 
зависимость $B(H)$ образует петлю, что приводит к потерям энергии.

\vspace{0.5cm}

\noindent\textbf{Энергия магнитного поля.} При изменении тока в катушке совершается работа против ЭДС индукции:

\[
    dW = I \cdot d\Phi = L I \, dI \quad \Rightarrow \quad W = \frac{LI^2}{2}.
\]

Эта энергия «запасена» в поле: плотность энергии $w = \frac{B^2}{2\mu_0\mu}$. 
В ферромагнетиках часть энергии тратится на переориентацию доменов (гистерезисные потери).
