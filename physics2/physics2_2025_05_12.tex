$subject$=Физические основы компьютерных \\ и сетевых технологий
$teacher$=Лекции Герта А. В.
$date$=12.05.2025

\section{Лекция 14. Уравнение Шрёдингера}

Свет проявляет как волновые, так и корпускулярные свойства. Эта дуалистическая природа описывается следующими соотношениями:

\[
E = \hbar \omega, \quad \abs{\vec{p}} = \hbar \abs{\vec{k}} = \frac{2\pi \hbar}{\lambda} = \frac{h}{\lambda}
\]

Здесь $E$ — энергия, $\omega$ — угловая частота, $\vec{p}$ — импульс, $\vec{k}$ — волновой вектор, $\lambda$ — длина волны, $h = 6.626 \cdot 10^{-34} \text{ Дж}\cdot\text{c}$ — постоянная Планка, $\hbar = \frac{h}{2\pi}$.

Луи де Бройль предположил, что если свет обладает как волновыми, так и корпускулярными свойствами, то подобный дуализм должен быть присущ и обычной материи: электронам, протонам и другим частицам. Он ввёл понятие волны, соответствующей частице. Её длина определяется соотношением:

\[
\lambda = \frac{h}{p}
\]

а частота:

\[
\omega = \frac{E}{\hbar}
\]

где $p$ — импульс частицы, $E$ — её энергия. Для свободной частицы с массой $m$ и без потенциальной энергии:

\[
E = \frac{p^2}{2m}
\]

Таким образом, даже электрону можно сопоставить волну, и он может проявлять интерференцию и дифракцию. Это было экспериментально подтверждено.

\textbf{Опыт Дэвиссона и Джермера (1927):} при прохождении пучка электронов через кристалл никеля была обнаружена дифракционная картина, аналогичная картине от рентгеновского излучения. Это стало доказательством волновых свойств электронов.

\textbf{Электрон как облако вероятностей:} из-за волновой природы частицы нельзя точно указать её положение и импульс одновременно. Электрон описывается не как точечный объект, а как облако вероятностей, где выше вероятность нахождения — там выше $|\psi|^2$.

\textbf{Принцип неопределённости Гейзенберга:}

\[
\Delta x \cdot \Delta p \gtrsim \frac{\hbar}{2}
\]

где $\Delta x$ — неопределённость координаты, $\Delta p$ — неопределённость импульса. Это фундаментальное ограничение, вытекающее из самой природы квантовых объектов.

\textbf{Волновая функция $\psi(\vec{r}, t)$} — центральное понятие квантовой механики. Её квадрат модуля $|\psi|^2$ показывает вероятность обнаружить частицу в данной точке пространства в данный момент времени. Волновая функция может быть комплексной, но физически измеримыми являются только производные от неё величины.

\textbf{Уравнение Шрёдингера} описывает эволюцию волновой функции. Оно заменяет законы Ньютона в квантовом мире:

\[
i\hbar \frac{\partial \psi(\vec{r}, t)}{\partial t} = \hat{H} \psi(\vec{r}, t)
\]

Здесь $\hat{H}$ — гамильтониан — оператор полной энергии. Он состоит из оператора кинетической энергии и потенциальной:

\[
\hat{H} = -\frac{\hbar^2}{2m} \nabla^2 + V(\vec{r})
\]

Первая часть — кинетическая энергия, выраженная через оператор Лапласа, вторая — потенциальная энергия $V(\vec{r})$.

Таким образом, уравнение Шрёдингера показывает, как изменяется волновая функция во времени под действием полной энергии системы.

\textbf{Стационарное уравнение Шрёдингера} (если потенциал не зависит от времени):

\[
\hat{H} \psi(\vec{r}) = E \psi(\vec{r})
\]

Это уравнение на собственные значения: мы ищем такие функции $\psi$, при которых действие оператора $\hat{H}$ приводит к умножению на число $E$ — энергию.

\textbf{Свойства волновой функции:}
\begin{itemize}
    \item $\psi$ должна быть нормируемой: $\int |\psi|^2 dV = 1$
    \item должна быть конечной, непрерывной и однозначной во всех точках
    \item производные $\psi$ также должны быть непрерывными (кроме точек, где потенциал имеет особые особенности)
\end{itemize}

Всё поведение микрочастиц в квантовой механике может быть выведено из уравнения Шрёдингера, что делает его фундаментом современной теоретической физики.
