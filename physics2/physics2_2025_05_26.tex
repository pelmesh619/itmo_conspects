$subject$=Физические основы компьютерных \\ и сетевых технологий
$teacher$=Лекции Герта А. В.
$date$=26.05.2025

\section{Лекция 16. }

Квантово-механическое описание атома водорода начинается с фундаментального уравнения, описывающего поведение микрочастиц — уравнения Шрёдингера. В случае стационарных (не зависящих от времени) состояний оно записывается в виде:

\[
- \frac{\hbar^2}{2\mu} \Delta \psi(\mathbf{r}) + U(r) \psi(\mathbf{r}) = E \psi(\mathbf{r}),
\]

где:

\begin{itemize}
    \item $\psi(\mathbf{r})$ — волновая функция электрона, зависящая от координаты;
    \item $\mu$ — приведённая масса системы (электрон и протон);
    \item $\Delta$ — оператор Лапласа (в трёхмерном пространстве);
    \item $U(r)$ — потенциальная энергия взаимодействия между электроном и ядром (протоном);
    \item $E$ — энергия стационарного состояния;
    \item $\hbar = \frac{h}{2\pi}$ — приведённая постоянная Планка.
\end{itemize}

Потенциальная энергия взаимодействия двух противоположно заряженных частиц определяется кулоновским законом:

\[
U(r) = -\frac{1}{4\pi \varepsilon_0} \frac{e^2}{r},
\]

где:

\begin{itemize}
    \item $e$ — элементарный заряд;
    \item $\varepsilon_0$ — электрическая постоянная;
    \item $r$ — расстояние между электроном и ядром.
\end{itemize}

Так как $U(r)$ зависит только от расстояния, задача обладает сферической симметрией. Это позволяет перейти к сферическим координатам ($r$, $\theta$, $\varphi$), и искать решение уравнения в виде произведения:

$$
\psi(r, \theta, \varphi) = R(r) \cdot Y_l^m(\theta, \varphi),
$$

где:

\begin{itemize}
    \item $R(r)$ — радиальная часть волновой функции;
    \item $Y_l^m(\theta, \varphi)$ — сферические гармоники, описывающие угловую зависимость.
\end{itemize}

Подстановка этого выражения в уравнение Шрёдингера и разделение переменных приводит к радиальному уравнению. Введя функцию $u(r) = r R(r)$, получаем дифференциальное уравнение второго порядка:

$$
\frac{d^2 u}{dr^2} + \left[ \frac{2\mu}{\hbar^2} \left( E + \frac{e^2}{4\pi \varepsilon_0 r} \right) - \frac{l(l+1)}{r^2} \right] u(r) = 0.
$$

Решения этого уравнения возможны только при определённых значениях энергии $E$. Эти разрешённые значения образуют дискретный спектр:

$$
E_n = -\frac{\mu e^4}{2 (4\pi \varepsilon_0)^2 \hbar^2 n^2} = -\frac{13.6~\text{эВ}}{n^2}, \quad n = 1, 2, 3, \dots
$$

Это и есть энергетические уровни атома водорода. Отрицательный знак означает, что частица связана с ядром — полная энергия системы меньше нуля.

Минимальный возможный уровень энергии (при $n = 1$) соответствует основному состоянию. Энергия стремится к нулю при $n \to \infty$ — это соответствует ионизации атома (электрон уходит на бесконечность).

Особый интерес представляет радиус, на котором вероятность нахождения электрона максимальна. Это — радиус Бора:

$$
a_0 = \frac{4 \pi \varepsilon_0 \hbar^2}{\mu e^2} \approx 0.529~\text{\AA}. \text{(ангстрема)}
$$

Вероятность обнаружить электрон в тонкой сферической оболочке радиуса $r$ и толщины $dr$ определяется плотностью вероятности:

$$
P(r)dr = |\psi(r,\theta,\varphi)|^2 dV = |R(r)|^2 \cdot |Y_l^m(\theta, \varphi)|^2 \cdot r^2 \sin \theta \, dr \, d\theta \, d\varphi.
$$

Если усреднить по углам, остаётся:

$$
P(r) dr = |R(r)|^2 4\pi r^2 dr.
$$

Рассмотрим теперь квантовые числа, которыми описывается каждое состояние:

\begin{itemize}
    \item $n$ — главное квантовое число: определяет уровень энергии;
    \item $l$ — орбитальное квантовое число ($0 \le l \le n - 1$): определяет форму орбитали;
    \item $m$ — магнитное квантовое число ($-l \le m \le l$): определяет ориентацию орбитали;
    \item $s$ — спиновое квантовое число (для электрона всегда $\frac{1}{2}$);
    \item $m_s$ — проекция спина ($\pm\frac{1}{2}$).
\end{itemize}

В соответствии с квантовой механикой, механический орбитальный момент электрона определяется выражением:

$$
|\vec{L}| = \hbar \sqrt{l(l+1)}.
$$

С орбитальным движением электрона связан также магнитный момент:

$$
|\vec{\mu}_L| = \mu_B \sqrt{l(l+1)},
$$

где $\mu_B = \frac{e\hbar}{2m_e}$ — магнетон Бора.

В 1922 году был поставлен опыт Штерна и Герлаха. Пучок атомов серебра проходил через неоднородное магнитное поле. По классическим представлениям ожидалось размытие пучка, но на экране появлялись две чёткие полоски. Это доказывало, что проекция магнитного момента (и, следовательно, спина) квантуется. Спин электрона принимает только два значения: $+\frac{1}{2}$ и $-\frac{1}{2}$.

Если атом помещается во внешнее магнитное поле, уровни с разными $m$ расщепляются. Это явление называется эффектом Зеемана:

$$
\Delta E = m \mu_B B.
$$

Правила отбора указывают, какие переходы между уровнями возможны при испускании или поглощении фотона:

$$
\Delta l = \pm 1, \quad \Delta m = 0, \pm 1.
$$

Наконец, в многоэлектронных атомах действует правило Хунда: при заполнении подуровней с одинаковыми $n$ и $l$ электроны занимают орбитали так, чтобы суммарный спин был максимальным. Это минимизирует энергию системы благодаря принципу запрета Паули.
