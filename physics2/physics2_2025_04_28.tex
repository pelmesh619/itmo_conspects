$subject$=Физические основы компьютерных \\ и сетевых технологий
$teacher$=Лекции Герта А. В.
$date$=28.04.2025

\section{Лекция 12. Дифракция и поляризация}

\subsection{Дифракция на системе щелей}

Дифракционная решётка состоит из большого числа одинаковых щелей, разделённых непрозрачными промежутками. Дифракционные картины, создаваемые каждой щелью, совпадают по виду и интерферируют друг с другом. Суммарная картина наблюдается как результат интерференции когерентных волн, выходящих из всех щелей решётки.

Пусть ширина каждой щели равна $b$, ширина непрозрачной прослойки между щелями — $a$. Тогда полный период решётки $d = a + b$. Это расстояние между центрами двух соседних щелей. Также $d = \frac{1}{N_0}$, где $N_0$ — число щелей на единицу длины решётки (пространственная плотность щелей).

Разность хода между волнами, идущими из двух соседних щелей под углом $\theta$ к нормали, равна:
\[
\Delta = d \sin \theta
\]

Условия наблюдения дифракционной картины:
\begin{itemize}
    \item \textbf{Главные максимумы (интерференционные пики)} наблюдаются, когда волны от всех $N$ щелей приходят в фазе:
    \[
    d \sin \theta = m \lambda, \quad m \in \mathbb{Z}
    \]
    
    \item \textbf{Главные минимумы} для одной щели возникают, если края щели создают волны в противофазе:
    \[
    b \sin \theta = m \lambda
    \]
    
    Это определяет подавление интенсивности для каждой отдельной щели. При этом даже если выполняется условие максимума решётки, максимум может исчезнуть, если $\theta$ также соответствует минимуму одиночной щели.

    \item \textbf{Дополнительные минимумы} (между главными максимумами) появляются при:
    \[
    d \sin \theta = \left(2m + 1\right) \frac{\lambda}{2}
    \]
    Их физическая природа связана с частичной компенсацией амплитуд от разных щелей. Между каждыми двумя соседними главными максимумами находится $N - 1$ дополнительных минимума.
\end{itemize}

Интенсивность в направлении главного максимума в $N^2$ раз превышает интенсивность от одной щели:
\[
I_{\max} = N^2 I_1
\]
где $I_1$ — интенсивность от одной щели в направлении главного максимума. Это происходит потому, что амплитуды складываются:
\[
A_{\max} = N A_1
\Rightarrow I \propto A^2
\]

Фазовые сдвиги при интерференции можно учесть, считая колебания от первой щели:
\[
A_1(t) = A_0 \cos \omega t
\]
От второй щели:
\[
A_2(t) = A_0 \cos(\omega t - \Delta \varphi)
\]
От $k$-й щели:
\[
A_k(t) = A_0 \cos\left(\omega t - (k - 1)\Delta \varphi\right)
\]

Эти колебания можно представить как сумму комплексных экспонент:
\[
\sum_{k=1}^N A_0 e^{i(\omega t - (k - 1)\Delta \varphi)} = A_0 e^{i\omega t} \sum_{k=0}^{N-1} e^{-i k \Delta \varphi}
\]

Это конечная геометрическая прогрессия, которая может быть просуммирована. Таким образом можно получить зависимость амплитуды и интенсивности от угла $\theta$.

Центральный (нулевой) максимум наблюдается при $\theta = 0$ и не зависит от длины волны $\lambda$. Поэтому он содержит все длины волн и выглядит белым на экране при освещении белым светом.


\subsection{Поляризация света}

Свет — это электромагнитная волна, в которой переменные электрическое и магнитное поля колеблются перпендикулярно направлению распространения. При этом электрическое поле играет основную роль при взаимодействии с веществом, поэтому его направление называют направлением поляризации.

Если вектор электрического поля сохраняет своё направление при распространении, такая волна называется линейно или плоско-поляризованной. Поляризация — это проявление поперечной природы световых волн.

Поляризатор — это устройство, преобразующее неполяризованный свет в поляризованный. Анализатор — прибор, позволяющий проверить, поляризован ли свет.

Явление, при котором степень поглощения света зависит от направления колебаний электрического поля, называется дихроизмом. 

Поляроид — это полимерная плёнка, содержащая ориентированные кристаллы дихроичного вещества. Он пропускает колебания только в одном направлении и поглощает в перпендикулярном, эффективно действуя как поляризатор.

\textbf{Закон Малюса:}
Если на анализатор падает линейно-поляризованный свет, и угол между направлениями поляризации и пропускания анализатора равен $\varphi$, то:
\[
I = I_0 \cos^2 \varphi
\]
где $I_0$ — интенсивность падающего света, $I$ — интенсивность прошедшего через анализатор.

\textbf{Угол Брюстера:} при падении света на границу двух сред под некоторым углом, отражённый свет становится полностью поляризованным. Этот угол определяется из условия:
\[
\tan \theta_B = \frac{n_2}{n_1}
\]

\textbf{Формулы Френеля:} описывают амплитуды отражённых и преломлённых волн при падении под произвольным углом на границу двух диэлектриков. Они позволяют рассчитать степень поляризации отражённого света.

\textbf{Степень поляризации} света определяется как:
\[
P = \frac{I_{\max} - I_{\min}}{I_{\max} + I_{\min}}
\]
где $I_{\max}$ и $I_{\min}$ — максимальная и минимальная интенсивности при вращении анализатора.

\textbf{Эффект Керра:} в сильных электрических полях в некоторых веществах возникает искусственная двойная лучепреломляемость. Свет, проходящий через такое вещество, становится частично поляризованным.

