$subject$=Физические основы компьютерных \\ и сетевых технологий
$teacher$=Лекции Герта А. В.
$date$=21.04.2025

\section{Лекция 11. Дифракция}

Дифракция в оптике — это совокупность явлений, связанных с отклонением света от прямолинейного пути распространения. Особенно ярко дифракционные эффекты проявляются при прохождении света мимо непрозрачных препятствий, хотя дифракция может возникать и при взаимодействии света с прозрачными объектами. В узком смысле дифракция — это огибание волнами препятствий, что характерно для всех типов волн, включая световые.

Принцип Гюйгенса утверждает: каждая точка среды, до которой дошла волна, становится источником вторичных сферических волн. Огибающая этих волн определяет форму волнового фронта в следующий момент времени.

Френель развил этот принцип, сделав его более количественным и применимым к объяснению дифракции:
\begin{itemize}
  \item Все вторичные источники на волновом фронте когерентны между собой, если исходная волна была когерентной.
  \item Равные по площади участки фронта излучают равные по мощности вторичные волны.
  \item Излучение каждого вторичного источника направлено преимущественно вдоль нормали к фронту.
  \item Действует принцип суперпозиции: волны от разных участков фронта складываются независимо. Если часть фронта экранируется, остальные участки продолжают излучать, как если бы экрана не было.
\end{itemize}

На основе этих положений формулируется принцип Гюйгенса–Френеля: каждый элемент волнового фронта можно рассматривать как центр вторичного возмущения, излучающего сферические волны. Амплитуда в некоторой точке $P$ определяется суперпозицией всех таких волн.

Амплитуда сферической волны, приходящей в точку $P$ от малого элемента поверхности $\Delta S$, зависит от расстояния $r$ до точки $P$, угла $\theta$ между нормалью к $\Delta S$ и направлением $r$, и пропорциональна $\Delta S$:
\[ E_P = \int_S K(\theta) \frac{E_0}{r} \cos(kr + \varphi_0)\, dS \]

Здесь $K(\theta)$ — коэффициент, зависящий от угла, $k = \frac{2\pi}{\lambda}$ — волновое число, $E_0$ — амплитуда первичной волны, $\varphi_0$ — её начальная фаза.

Дифракция Френеля — это дифракция сферических волн, когда источник и экран находятся на конечном расстоянии. Если же свет представлен параллельными пучками, и источник и экран расположены на бесконечности (или фокусируются линзой), говорят о дифракции Фраунгофера.

Волновая поверхность сферической волны симметрична относительно оси $SP$. Френель предложил разбивать волновую поверхность на шаровые зоны так, чтобы разность хода между волнами от границ соседних зон была равна $\lambda/2$. Вклад от каждой последующей зоны идёт с чередующейся фазой, и суммарная амплитуда зависит от числа таких зон.

Рассмотрим дифракцию плоской волны на бесконечно длинную щель шириной $b$. Пусть плоская волна падает на щель под углом $\theta$. Тогда оптическая разность хода между волнами, идущими от противоположных краёв щели, будет:
\[ \Delta = b \sin \theta \]

Если $\Delta = k\frac{\lambda}{2}$, где $k$ — целое число, то волны от краёв щели интерферируют, и картина зависит от того, сколько зон Френеля укладывается в ширину щели:
\begin{itemize}
  \item если число зон чётное — волны взаимно гасятся, и в точке наблюдается минимум;
  \item если число зон нечётное — центральная зона не компенсирована, и наблюдается максимум.
\end{itemize}

В направлении $\theta = 0$ вся щель действует как одна зона Френеля. Это даёт центральный максимум — он самый яркий и широкий.
