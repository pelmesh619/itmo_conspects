$subject$=Физические основы компьютерных \\ и сетевых технологий
$teacher$=Лекции Герта А. В.
$date$=10.03.2025

\section{Лекция 5. Колебания}

Колебаниями называются процессы (изменения состояния тела или системы), 
обладающие той или иной степенью повторяемости во времени. 
Повторение может быть строго периодическим (например, движение маятника) 
или хаотическим (апериодические колебания). Различают:

\begin{itemize}
    \item \textbf{Периодические:} значения физических величин повторяются через равные интервалы времени $T$
    
    \item \textbf{Апериодические:} нет строгой периодичности, например, затухающие колебания груза

    \item \textbf{Свободные (собственные):} возникают после выведения системы из равновесия \textit{без внешнего воздействия}
    
    \item \textbf{Вынужденные:} поддерживаются внешней периодической силой. Амплитуда зависит от соотношения частоты силы и собственной частоты системы (явление резонанса).
    
    \item \textbf{Автоколебания:} энергия поступает в систему по её внутренним законам. Пример: генератор на транзисторе, где обратная связь поддерживает незатухающие колебания.
\end{itemize}

Колебательная система - физическая система, в которой могут существовать свободные колебания. 
Реальные системы всегда имеют затухание из-за диссипативных сил (трение, сопротивление).
Систему, в которой описывающие ее величины совершают колебания около точки равновесия, называют осциллятором

Системы, в которых возможны колебательные процессы, подразделяются на линейные и нелинейные.
В первом случае дифференциальные уравнения, описывающие поведение системы, являются линейными, 
и система подчиняется принципу суперпозиции.

Во втором случае такие дифференциальные уравнения нелинейны и принцип суперпозиции не справедлив. Большинство
физических систем нелинейны, однако, при малых отклонениях от состояний равновесия они демонстрируют линейное поведение

\mediumvspace

Простейшими являются гармонические колебания, которые описываются формулой $x = A \cos (\omega_0 t + \alpha)$ 
(или $x = A \sin (\omega_0 t + \alpha)$). Обычно точка $x = 0$ считается положением равновесия. 

Такие колебания часто встречаются в природе (например, маятник). К тому же, другие периодические процессы 
могут быть представлены как комбинация гармонических (подобно рядам Фурье)

Величина $\alpha$ называется начальной фазой колебаний, а $A$ - амплитуда, наибольшее значение колебания. 
Косинус - $2\pi$-периодичная функция, из этого можно найти период колебаний: $(\omega_0 (t + T) + \alpha) - (\omega_0 t + \alpha) = 2\pi \Longrightarrow T = \frac{2\pi}{\omega_0}$

Период отражает величину времени, через которое система придет в исходное положение. Обратная величина - частота $\nu = \frac{1}{T}$

Можем получить скорость колебаний: $v_x = \dot x = -A \omega_0 \sin (\omega_0 t + \alpha)$

И ускорение: $a_x = \ddot x = -A \omega_0^2 \cos (\omega_0 t + \alpha)$

Из этого $\ddot x + \omega_0^2 x = 0$

По формуле Эйлера функцию гармонических колебаний можно представить как $x(t) = \operatorname{Re} (Ae^{i(\omega_0 t + \alpha)})$

\ExN{Пружинный маятник} Сила упругости по закону Гука равна $F = -kx$. Подставляя в $F = m\ddot{x}$, получаем:

\[
m\ddot{x} + kx = 0 \quad \Longrightarrow \quad \omega_0 = \sqrt{\frac{k}{m}}, \quad T = 2\pi\sqrt{\frac{m}{k}}
\]

Полная энергия в отсутствие трения сохраняется:$W = \frac{mv^2}{2} + \frac{kx^2}{2}$.
И в пике достигает $W_{\text{пот}} = W_{\text{кин}} = \frac{mv^2}{2} = \frac{mA^2\omega_0^2}{2} = \frac{kA^2}{2}$
    
\ExN{Математический маятник} - подвешенный грузик на нерастяжимой нити, совершающий движение по окружности.
Момент силы равет $M = m g l \sin \varphi, I = m l^2$, угловое ускорение - $\varepsilon = \frac{d^2 \varphi}{d t^2}$

По основному уравнению вращательного движения $I \varepsilon = M \Longrightarrow \ddot \varphi l = g \sin \varphi$

При малых колебаниях $\sin \varphi \sim \varphi$ и получаем $\varphi = \varphi_{\max} \cos (\omega_0 t + \alpha)$

Отсюда период $T = 2\pi\sqrt{\frac{l}{g}}, \omega_0 = \sqrt{\frac{g}{l}}$

\ExN{Электрический $LC$-контур:} 
Пусть конденсатор ёмкостью $C$ заряжен до напряжения $U_0$. 
При соединении конденсатора с катушкой индуктивности в цепи потечёт ток $I$, 
что вызовет в катушке ЭДС самоиндукции, направленную на уменьшение тока в цепи. 
Ток, вызванный этой ЭДС (при отсутствии потерь в индуктивности), 
в начальный момент будет равен току разряда конденсатора, то есть результирующий ток будет равен нулю. 
Магнитная энергия катушки в этот (начальный) момент равна нулю.

Затем результирующий ток в цепи будет возрастать, а энергия из конденсатора будет переходить 
в катушку до полного разряда конденсатора. В этот момент электрическая энергия конденсатора равна нулю. 
Магнитная же энергия, сосредоточенная в катушке, напротив, максимальна

После этого начнётся перезарядка конденсатора, то есть зарядка конденсатора напряжением другой полярности. 
Перезарядка будет проходить до тех пор, пока магнитная энергия катушки не перейдёт в электрическую энергию 
конденсатора. Конденсатор в этом случае снова будет заряжен до напряжения $-U_0$

\[
L\frac{d^2q}{dt^2} + \frac{q}{C} = 0 \quad \Longrightarrow \quad \ddot{q} + \frac{1}{LC}q = 0
\]

Отсюда циклическая частота: $\omega_0 = \frac{1}{\sqrt{LC}}$, период - $T = 2\pi\sqrt{LC}$

Энергия переходит между конденсатором $\left(W_E = \frac{q^2}{2C}\right)$ и катушкой $\left(W_M = \frac{LI^2}{2}\right)$.


