\clearpage

\section{X. Программа экзамена в 2024/2025}

\begin{enumerate}

    \item Гармонический осциллятор. Уравнение гармонических колебаний. Энергия гармонических колебаний. Примеры механических колебаний: пружинный и математический маятник. Физический маятник. Частота, круговая частота и период колебаний. Фазовый портрет маятника. Адиабатические инварианты.

    \item Затухающие механические колебания. Анализ затухающих колебаний. Сухое и вязкое трение. Коэффициент затухания, логарифмический декремент затухания, добротность.

    \item Вынужденные колебания. Колебания материальной точки под действием вынуждающей синусоидальной силы. Резонанс. Резонансные кривые. Параметрический резонанс.

    \item Сложение колебаний. Биения. Моды колебаний связанных математических маятников.

    \item Волны: волновое уравнение, свойства решений волнового уравнения, основные понятия - волновой вектор, длина волны, фазовая скорость. Продольные и поперечные волны. Плоские и сферические волны.

    \item Вывод уравнения электромагнитных (эл/м) волн из уравнений Максвелла.

    \item Свойства эл/м волн (поперечность, связь электрического и магнитного полей, синфазность колебаний электрического и магнитного полей).

    \item Плотность потока энергии в эл/м волне (вектор Пойтинга).

    \item Условия интерференции. Интерференция монохроматических световых волн. Двухлучевая интерференция: схема опыта Юнга, интерференционные схемы.

    \item Интерференция в тонких пленках. Кольца Ньютона.

    \item Принцип Гюйгенса-Френеля (определение и математическая формулировка).

    \item Дифракция Френеля на круглом отверстии: зоны Френеля, спираль Френеля, пятно Пуассона.

    \item Режимы дифракции (Френеля, Фраунгофера).

    \item Дифракция Фраунгофера на щели, свойства дифракционной картины.

    \item Дифракционная решетка в монохроматическом свете, свойства дифракционной картины.

    \item Дифракционная решетка как спектральный прибор: угловая дисперсия при дифракции на решетке, разрешающая способность дифракционной решетки.

    \item Поляризованный и естественный свет, линейная и круговая поляризации, закон Малюса, степень поляризации.

    \item Законы отражения и преломления э/м волн.

    \item Поляризация при отражении и преломлении: формулы Френеля.

    \item Применение формул Френеля: угол Брюстера, скачок фазы при отражении, коэффициент отражения.

    \item Основы кристаллооптики: тензор диэлектрической проницаемости, уравнение распространения плоской волны в анизотропной среде.

    \item Двулучепреломление: основные понятия, одноосные и двуосные кристаллы, поляризаторы (призма Николя, призма Фуко, поляроиды), фазовые пластинки (λ/4 и λ/2).
\end{enumerate}
