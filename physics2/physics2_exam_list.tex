\clearpage

\section{X. Программа экзамена в 2024/2025}

\begin{enumerate}

    \item Магнитное поле в вакууме. Вектор магнитной индукции. Магнитное поле движущегося 
    заряда. Закон Био-Савара-Лапласа. Закон Ампера. Сила Лоренца. Движение зарядов в 
    электрических и магнитных полях. 

    \item Теорема о циркуляции и теорема Гаусса для вектора магнитной индукции в интегральной и 
    дифференциальной формах. Работа при движении проводника с током в магнитном поле. 

    \item Магнитное поле в веществе. Магнитное поле и магнитный дипольный момент кругового тока. 
    Вектор намагниченности и его связь с плотностью молекулярных токов. Напряженность 
    магнитного поля. Магнитная восприимчивость и магнитная проницаемость. Диамагнетики, 
    парамагнетики и ферромагнетики. Граничные условия на поверхности раздела двух магнетиков. 

    \item Электромагнитная индукция. Закон электромагнитной индукции Фарадея. Правило Ленца. 
    Объяснение явления для проводника, движущегося в магнитном поле. Заряд, протекающий в 
    проводнике при возникновении ЭДС индукции. 

    \item Электромагнитная индукция. Явления самоиндукции и взаимной индукции. Индуктивность и 
    взаимная индуктивность контуров. Индуктивности длинного прямого соленоида. Токи при 
    замыкании и размыкании цепи, связанные с ее индуктивностью. Энергия магнитного поля: 
    энергия магнитного поля проводника с током, объемная плотность энергии.

    \item Уравнения Максвелла. Максвелловская трактовка явления электромагнитной индукции. Ток 
    смещения. Теорема о циркуляции вектора напряженности магнитного поля при наличии 
    переменного электрического поля в интегральной и дифференциальной формах. Система 
    уравнений Максвелла в интегральной и дифференциальной формах. Физический смысл 
    входящих в нее уравнений. Материальные уравнения.

    \item Механические и электромагнитные гармонические колебания. Идеальный гармонический 
    осциллятор. Уравнение идеального осциллятора и его решение. Амплитуда, частота и фаза 
    колебания. Энергия колебаний. Примеры колебательных движений различной физической 
    природы. Свободные затухающие колебания осциллятора с потерями. Вынужденные колебания. 
    Сложение колебаний (биения, фигуры Лиссажу). 

    \item Продольные и поперечные волны. Волновое уравнение. Волновые поверхности. Уравнение 
    плоской монохроматической волны, ее характеристики. Фазовая скорость волны. Сферические 
    волны. Энергия упругой волны, вектор Умова для упругой волны. Стоячие волны. Акустический 
    эффект Доплера. 

    \item Электромагнитные волны. Существование электромагнитных волн как следствие уравнений 
    Максвелла. Свойства электромагнитных волн. Энергия электромагнитных волн. Вектор 
    Пойнтинга для электромагнитной волны. Шкала электромагнитных волн. Оптический диапазон 
    длин волн. 

    \item Волновая оптика. Интерференция света. Условия максимумов и минимумов. Опыт Юнга. 
    Способы наблюдения интерференции света (зеркала Френеля, бипризма Френеля, зеркало 
    Ллойда, тонкие пленки, кольца Ньютона). Дифракция света. Принцип Гюйгенса-Френеля. Зоны 
    Френеля. Дифракция Френеля на круглом отверстии и диске. Дифракция Фраунгофера на 
    длинной прямоугольной щели. Дифракционная решетка. 
    
    \item Поляризация света. Закон Малюса. Степень поляризации. Способы получения 
    поляризованного света. Закон Брюстера. Дисперсия света. Разложение света в спектр. 
    Нормальная и аномальная дисперсия.

    \item Тепловое излучение и его характеристики. Законы излучения абсолютно черного тела. 
    Гипотеза Планка о квантах света. Формула Планка для излучательной способности абсолютно 
    черного тела. Свойства фотонов. Внешний фотоэффект, законы Столетова. Уравнение 
    Эйнштейна для внешнего фотоэффекта. Эффект Комптона. Тормозное рентгеновское излучение. 
    Давление света. Корпускулярно-волновой дуализм природы света.

    \item Экспериментальные данные о структуре атомов. Линейчатые спектры атомов. Модель атома 
    Резерфорда. Постулаты Бора. Теория Бора для водородоподобных ионов. Квантование энергии 
    электрона в атоме. Опыт Франка и Герца. 

    \item Элементы квантовой механики. Гипотеза де Бройля. Волны де Бройля. Экспериментальное 
    подтверждение гипотезы де Бройля. Принцип неопределенностей Гейзенберга. Волновая 
    функция, ее свойства и физический смысл. Временное и стационарное уравнения Шредингера. 
    Квантовая частица в одномерной потенциальной яме. Одномерный потенциальный барьер. 
    Квантовый гармонический осциллятор. 

    \item Квантово-механическое описание атомов. Стационарное уравнение Шредингера для атома 
    водорода. Волновые функции и квантовые числа. Правила отбора для квантовых переходов. 
    Опыт Штерна и Герлаха. Спин электрона.

\end{enumerate}
