$subject$=Физические основы компьютерных \\ и сетевых технологий
$teacher$=Лекции Герта А. В.
$date$=14.04.2025

\section{Лекция 10. Интерференция в разных опытах}

\subsection{Оптическая разность хода и разность фаз}

Чтобы наблюдать интерференцию света, необходимо использовать когерентные источники, т.е. источники с постоянной разностью фаз. Получить два совершенно одинаковых независимых источника практически невозможно, поэтому используют один источник и делят его волновой фронт -- например, с помощью двух щелей. В этом случае два полученных пучка будут когерентными.

Важно понимать, что сама по себе оптическая разность хода между лучами не гарантирует интерференции. Для наблюдения интерференционной картины необходимо, чтобы лучи пересекались в пространстве и накладывались друг на друга. Если лучи в дальнейшем не пересекаются, интерференционные эффекты в этих областях не наблюдаются, даже несмотря на существование фиксированной разности хода.

\textbf{Оптическая разность хода} $\Delta$ -- это разность между произведениями геометрических длин путей, пройденных волнами, и показателей преломления сред, через которые проходят волны. Иначе говоря, она показывает, насколько одна волна \enquote{отстаёт} от другой по фазе из-за различий в длине пути и/или свойствах среды.

Если волна проходит путь $S$ в среде с показателем преломления $n$, её фаза:
\[
\varphi = \omega t - k S + \varphi_0 = \omega t - \frac{2\pi n}{\lambda_0} S + \varphi_0
\]

Для двух лучей разность фаз будет:
\[
\Delta = \varphi_2 - \varphi_1 = \frac{2\pi}{\lambda_0}(n_1 S_1 - n_2 S_2)
\]

Алгоритм вычисления оптической разности хода:
\begin{enumerate}
  \item Определить пути $S_1$ и $S_2$, пройденные волнами.
  \item Определить показатели преломления $n_1$ и $n_2$ сред, через которые проходят волны.
  \item Вычислить: \[ \Delta = n_1 S_1 - n_2 S_2. \]
  \item Если волны идут по воздуху ($n_1 = n_2 = 1$), формула упрощается до разности геометрических длин.
\end{enumerate}


\subsection{Интерференционные полосы (опыт Юнга)}

Томас Юнг в своём знаменитом эксперименте использовал одну щель как источник света, проходящего через две узкие щели. На экране за ними возникала интерференционная картина -- чередующиеся светлые и тёмные полосы.

Пусть расстояние между щелями $d$, расстояние до экрана $L$, а $x$ -- поперечная координата на экране:
\[
\Delta = d \sin \theta \approx \frac{d x}{L}, \quad \text{где } x \text{ --- координата полосы}
\]

Для максимумов интерференции:
\[
\Delta = m\lambda \Rightarrow x_m = \frac{m\lambda L}{d}
\]

\subsection{Призма}

При прохождении света через тонкую призму с углом преломления $\alpha$ и показателем преломления $n$:
\[
\alpha_1 = n\beta_1, \quad \alpha_2 = n\beta_2, \quad \beta_1 + \beta_2 = \alpha
\]

Итоговая разность хода:
\[
\Delta = (\alpha_1 + \alpha_2) - (\beta_1 + \beta_2) = n\alpha - \alpha = \alpha(n - 1)
\]

\subsection{Билинза Френеля}

Это приспособление, создающее два когерентных пучка с помощью симметрично расположенных линз. Оптическая разность хода:
\[
\Delta = 2\alpha(n - 1)a
\]

Если $a + b = L$, где $b$ -- расстояние от билинзы до экрана, то координаты интерференционных максимумов:
\[
x_m = m \frac{l}{d} \lambda = m \frac{a + b}{2 \alpha (n - 1) a} \lambda
\]
Для центрального максимума ($m = 0$), центр находится в точке:
\[
x_0 = \frac{b}{2 \alpha (n - 1)}
\]


\subsection{Интерференция от полупрозрачного зеркала}

При отражении света от границы между двумя средами, если свет отражается от более оптически плотной среды (с большим показателем преломления), то он приобретает сдвиг фазы на $\frac{\lambda}{2}$. Это означает, что волна, отразившаяся от более плотной среды, будет иметь фазовый сдвиг относительно той, которая прошла через границу.

\[
\Delta = a - b + \frac{\lambda}{2}
\]

\subsection{Интерференция в тонкой пластине}

Свет частично отражается от верхней и нижней поверхностей тонкой плёнки с показателем преломления $n$. Разность хода:
\[
\sin \alpha = n \sin \gamma
\]

\[
\Delta = 2nh\cos \gamma = \frac{2nh}{\sqrt{1 - \frac{\sin^2 \alpha}{n^2}}}
\]

\subsection{Клиновидная пластина}

\textbf{Клиновидная пластина} -- тонкий слой материала с переменной толщиной, вызывает интерференцию с переменной разностью хода, что приводит к чередованию полос. Разность хода в этом случае будет:
\[
\Delta(x) = 2n x \tan \theta
\]

Положение тёмных полос определяется условием разрушительной интерференции:
\[
\x_m = \frac{(2m + 1)\lambda}{4n \tan \theta}
\]

\subsection{Кольца Ньютона}

\textbf{Кольца Ньютона} -- интерференционная картина, возникающая при наложении выпуклой линзы на плоскую пластину. Возникают концентрические кольца из-за различной толщины воздушного зазора $h(r)$:
\[
h(r) = \frac{r^2}{2R} \Rightarrow \Delta = 2n h = \frac{n r^2}{R}
\]

Для тёмных колец:
\[
\frac{n r_m^2}{R} = (2m + 1)\frac{\lambda}{2} \Rightarrow r_m = \sqrt{\frac{(2m + 1)\lambda R}{2n}}
\]

\subsection{Дифракция света}

Дифракция -- явление огибания светом препятствий, не объяснимое законами геометрической оптики. Свидетельствует о волновой природе света и проявляется, например, в виде характерных полос за узкими щелями и объектами.

Типичные примеры:
\begin{itemize}
  \item дифракция на щели;
  \item дифракция на проволоке;
  \item дифракция Фраунгофера и Френеля.
\end{itemize}

\textbf{Условие минимума при дифракции на щели:} $a \sin \theta = m \lambda, \quad m \in \mathbb{Z}$, где $a$ -- ширина щели.

\textbf{Условие максимума при дифракции на щели:} $d \sin \theta = m \lambda, \quad m \in \mathbb{Z}$, где $d$ -- период решетки (ширина щели + ширина препятствия).

Или $a \sin \theta = \frac{1}{2} (2m + 1) \lambda, \quad m \in \mathbb{Z}$
