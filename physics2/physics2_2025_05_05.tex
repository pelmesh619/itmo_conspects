$subject$=Физические основы компьютерных \\ и сетевых технологий
$teacher$=Лекции Герта А. В.
$date$=05.05.2025

\section{Лекция 13. Модель атома}

Модель Томпсона представляла собой «пудинг с изюмом»: атом — это положительно заряженная сфера, внутри которой равномерно распределены отрицательные электроны. Такая модель могла объяснить нейтральность атома, но не допускала существования чёткой структуры внутри атома и не давала объяснения наблюдаемым спектрам.

Опыт Резерфорда (1911) стал поворотным моментом. Альфа-частицы (ядра гелия) направлялись на тонкую золотую фольгу. Большинство частиц проходило без отклонения, но некоторые отклонялись на большие углы, а отдельные — почти обратно. Это можно объяснить только в предположении, что в центре атома сосредоточен положительный заряд и почти вся масса — то есть существует ядро. Модель Томпсона не может это объяснить: в ней нет плотного центра, который мог бы отклонить тяжёлую альфа-частицу.

Так появилась модель Резерфорда: атом состоит из тяжёлого положительно заряженного ядра, вокруг которого по орбитам движутся электроны, как планеты вокруг Солнца. Но классическая электродинамика предсказывает, что ускоренно движущийся электрон должен излучать энергию, терять её и, за долю секунды, упасть на ядро. Это противоречит стабильности атомов.

Чтобы объяснить устойчивость атома и его спектры, Нильс Бор в 1913 году предложил свою модель. Он ввёл постулаты:

\begin{enumerate}
    \item Электрон может двигаться по стационарной орбите без излучения.
    \item Излучение или поглощение происходит при переходе между орбитами.
    \item Угловой момент электрона на орбите квантован:
    \[
    L = m v r = n \hbar, \quad n \in \mathbb{N}
    \]
\end{enumerate}

Для наименьшей орбиты (основное состояние, $n = 1$) получим радиус Бора:
\[
r_n = \frac{n^2 \hbar^2}{m k e^2} = n^2 r_1, \quad r_1 \approx 5.29 \cdot 10^{-11} \text{ м}
\]

Здесь $k = \frac{1}{4\pi\varepsilon_0}$ — коэффициент из закона Кулона.

Полная энергия электрона в $n$-й орбите:
\[
E_n = - \frac{m e^4}{2 \hbar^2 n^2 (4 \pi \varepsilon_0)^2} = -\frac{13.6\ \text{эВ}}{n^2}
\]

При переходе с уровня $n_2$ на $n_1$ (где $n_2 > n_1$), излучается фотон энергии:
\[
h \nu = E_{n_2} - E_{n_1}
\]

Эта формула объясняет спектральные линии водорода.

Постоянная Ридберга — это универсальная константа в формулах для спектров:
\[
\frac{1}{\lambda} = R \left( \frac{1}{n_1^2} - \frac{1}{n_2^2} \right), \quad R \approx 1.097 \cdot 10^7\ \text{м}^{-1}
\]

Серии в спектре водорода:

\begin{itemize}
    \item серия Лаймана: $n_1 = 1$, $n_2 = 2, 3, \ldots$ (ультрафиолет)  
    \item серия Бальмера: $n_1 = 2$, $n_2 = 3, 4, \ldots$ (видимый диапазон)  
    \item серия Пашена: $n_1 = 3$, $n_2 = 4, 5, \ldots$ (инфракрасный диапазон)
\end{itemize}

Состояние атома, в котором электрон находится на орбите с минимально возможной энергией, называется \textbf{основным}. Все остальные орбиты соответствуют \textbf{возбуждённым} состояниям. Атом может находиться в возбуждённом состоянии конечное время, после чего спонтанно переходит в основное, испуская фотон.

Опыт Франка и Герца (1914) экспериментально подтвердил существование дискретных энергетических уровней в атоме. Электроны ускорялись и сталкивались с атомами. При достижении определённой энергии (около 4.9 эВ для ртути), электроны теряли энергию, возбуждая атомы. Это означало, что атомы могут поглощать энергию только порциями — квантуемыми значениями, — соответствующими разности уровней.

Термин \textbf{электронное облако} — это современное представление о распределении вероятности нахождения электрона в атоме. Оно заменяет точечное положение орбиты и говорит о том, где с большей вероятностью можно найти электрон. Это уже относится к квантово-механической модели атома, основанной на уравнении Шрёдингера, а не на модели Бора.

Сила Ван-дер-Ваальса — это слабое взаимодействие между нейтральными атомами и молекулами, обусловленное флуктуациями электронных оболочек. Это не имеет прямого отношения к атомным моделям, но важно для понимания межмолекулярных взаимодействий.
