\documentclass[12pt]{article}
\usepackage{preamble}

\pagestyle{fancy}
\fancyhead[LO,LE]{Специальные разделы \\ высшей математики}
\fancyhead[CO,CE]{19.04.2024}
\fancyhead[RO,RE]{Лекции Далевской О. П.}


\begin{document}
    \section[p4]{4. Дифференциальные уравнения}

    \section[p4\_1]{4.1 Общие понятия}

    \subsection{1* Постановка задачи}

    \textit{Pr. 1.} Скорость распада радия в текущий момент времени $t$ пропорциональна его наличному количеству $Q$. Требуется найти закон распада радия:

    \[Q = Q(t),\]

    если в начальный момент времени $\displaystyle t_0 = 0$ количество равнялось $\displaystyle Q_0$

    Коэффициент пропорциональности $k$ найден эмпирически.

    \underline{Решение.} Скорость распада.

    $\displaystyle \frac{dQ(t)}{dt} = kQ \quad$ - ищем $Q(t)$

    $dQ(t) = kQdt$

    $\displaystyle \underset{\text{содержит только }Q}{\undergroup{\frac{dQ(t)}{Q}}} = \underset{\text{содержит только }t}{\undergroup{kdt}}$ - \enquote{разделение переменных}

    Внесем все в дифференциал:

    $d \ln Q = kdt = dkt$

    $d(\ln Q - kt) = 0$

    Нашли семейство первообразных:

    $\ln Q - kt = \tilde{C}$

    $\ln Q = \tilde{C} + kt$

    $\displaystyle Q = e^{\tilde{C} + kt} \stackrel{e^\tilde{C} = C}{=\joinrel=\joinrel=\joinrel=} Ce^{kt}$

    По смыслу $k < 0$, так как $Q$ уменьшается. Обозначим $n = -k, n > 0$

    \vspace{5mm}

    Тогда \fbox{$\displaystyle Q(t) = Ce^{-nt}$}

    \vspace{5mm}

    Получили вид закона распада. Выбор константы $C$ определен Н.У. (начальными условиями):

    $\displaystyle t_0 = 0 \quad Q(t_0) = Q_0 = C$

    Тогда, закон - \fbox{$\displaystyle Q^*(t) = Q_0 e^{-nt}$}

    \vspace{3mm}
\textit{Nota}. Оба закона: общий $\displaystyle Q(t) = Ce^{-nt}$ и частный $\displaystyle Q^*(t) = Q_0 e^{-nt}$ -
    являются решением дифференциального уравнения:

    $\displaystyle Q^\prime(t) = kQ$  (явный вид)

    $d \ln Q(t) - kdt = 0$ (в дифференциалах)

    \vspace{5mm}

    \textit{Pr. 2} \quad Тело массой $m$ брошено вверх с начальной скоростью $\displaystyle v_0$. Нужно найти закон движения $y = y(t)$.
    Сопротивлением воздуха пренебречь.

    По II закону Ньютона:

    $m\overrightarrow{a} = m\overrightarrow{g}$

    $\overrightarrow{a} = \overrightarrow{g}$

    $\overrightarrow{a} = \overrightarrow{g}$

    $a = $\fbox{$\displaystyle \frac{d^2 y}{dt^2} = -g$} - ДУ

    \underline{Решение.} \quad $\displaystyle y^{\prime\prime}(t) = -g$

    $\displaystyle (y^{\prime}(t))^\prime = -g$

    $\displaystyle y^{\prime}(t) = -\int g dt = -gt + C_1$

    $\displaystyle y(t) = \int (-gt + C_1) dt = $\fbox{$\displaystyle -\frac{gt^2}{2} + C_1 t + C_2 = y(t)$} - общий закон

    $\displaystyle C_{1,2}$ ищем из Н.У.

    В задаче нет условия для $\displaystyle y(t_0)$. Возьмем $\displaystyle y_0 = y(t_0) = 0$

    Кроме того $\displaystyle y^\prime(t_0) = v(t_0) = v_0$

    Таким образом, $\displaystyle \begin{cases}y(t_0) = 0 \\ y^\prime(t_0) = v_0\end{cases}$

    Найдем $\displaystyle C_1$: $\displaystyle y^\prime(t_0) = y(0) = -gt_0 + C_1 = v_0 \quad C_1 = v_0$

    Найдем $\displaystyle C_2$: $\displaystyle y(t_0) = y(0) = -\frac{gt^2}{2} + C_1 t + C_2 = C_2 = 0$

    Частный закон: \fbox{$\displaystyle y^*(t) = v_0 t - \frac{gt^2}{2}$}

    \vspace{5mm}

    \subsection{2* Основные определения}

    \vspace{3mm}
\textit{Def} 1. Уравнение $\displaystyle F(x, y(x), y^\prime(x), \dots, y^{(n)}(x)) = 0$ - называется обыкновенным ДУ $n$-ого порядка $(*)$

    \vspace{3mm}
\textit{Ex}. $\displaystyle Q^\prime + nQ = 0 \quad$ и $\displaystyle \quad y^{\prime\prime} + g = 0$

    \vspace{3mm}
\textit{Def} 2. Решением ДУ $(*)$ называется функция $y(x)$, которая при подстановке обращает $(*)$ в тождество

    \vspace{3mm}
\textit{Def} $\displaystyle 2^\prime$. Если $y(x)$ имеет неявное задание $\Phi(x, y(x)) = 0$, то $\Phi(x, y)$ называется интегралом уравнения $(*)$

    \vspace{3mm}
\textit{Nota}. Разделяют общее решение ДУ - семейство функций, при этом каждое из них - решение; и
    частное решение - отдельная функция

    \vspace{3mm}
\textit{Def} 3. Кривая с уравнением $y = y(x)$ или $\Phi(x, y(x)) = 0$ называют интегральной кривой

    \vspace{3mm}
\textit{Def} 4. $\displaystyle \begin{cases}y(x_0) = y_0 \\ \vdots \\ y^{(n - 1)}(x_0) = y_0^{(n - 1)}\end{cases}$ - система начальных условий $(**)$

    Тогда $\begin{cases}(*) \\ (**)\end{cases}$ - задача Коши (ЗК)

    \vspace{3mm}
\textit{Nota}. Задача Коши может не иметь решений или иметь множество решений

    \vspace{3mm}
\textit{Th}. $\displaystyle y^\prime = f(x, y)$ - ДУ

    $\displaystyle M_0(x_0, y_0) \in D$ - точка, принадлежащая ОДЗ

    Если $f(x, y)$ и $\displaystyle \frac{\partial f}{\partial y}$ непрерывны в $\displaystyle M_0$, то ЗК

    $\displaystyle \begin{cases}y^\prime = f(x, y) \\ y(x_0) = y_0\end{cases}$

    имеет единственное решение $\varphi(x, y) = 0$, удовлетворяющее Н.У. (без док-ва)

    \vspace{3mm}
\textit{Nota}. Преобразуем ДУ: $\displaystyle \underset{F(x, y(x), y^\prime(x))}{\undergroup{y^\prime - f(x, y)}} = 0$

    См. определения обыкн. и особых точек

    \vspace{3mm}
\textit{Def} 5. Точки, в которых нарушаются условия теоремы называются особыми, а решения, у которых каждая точка особая,
    называются особыми

    \vspace{3mm}
\textit{Def} 6. Общим решением ДУ $(*)$ называется $\displaystyle y = f(x, C_1, C_2, \dots, C_n)$

    \vspace{3mm}
\textit{Nota}. $\displaystyle \Phi(x, y(x), C_1, \dots, C_n) = 0$ - общий интеграл

    \vspace{3mm}
\textit{Def} 7. Решением $(*)$ с определенными значениями $\displaystyle C_1^*, \dots, C_n^*$ называется частным

    \vspace{3mm}
\textit{Nota}. Форма записи:

    Разрешенное относительно производной $\displaystyle y^\prime = f(x, y)$

    Сведем к виду: $\displaystyle \frac{dy}{dx} = \frac{P(x, y)}{-Q(x, y)} \Longrightarrow -Q(x, y)dy = P(x, y)dx \Longrightarrow $

    \fbox{$P(x, y)dx + Q(x, y)dy = 0$} - форма в дифференциалах



\end{document}

