\documentclass[12pt]{article}
\usepackage{preamble}

\pagestyle{fancy}
\fancyhead[LO,LE]{$\mathcal{D}$искретная математика}
\fancyhead[RO,RE]{Лекции Чухарева К. И.}

\renewcommand{\thesection}{}

\begin{document}

    \tableofcontents
    \clearpage

    
    \section{7. Комбинаторика}

    \textbf{Базовые понятия:}

    \begin{itemize}
        \item \textbf{Алфавит} (Alphabet) $\Sigma$ (или $X$, \Exs $X = \Set{a, b, c}$) - множество символов в нашей системе

        \vspace{5mm}

        \item \textbf{Диапазон} (Range) $[n] = \Set{1, \dots, n}$ - конечное множество последовательных натуральных чисел

        \vspace{5mm}

        \item \textbf{Расстановка} (Ordered arrangement) - последовательность каких-либо элементов (тоже самое, что кортеж),
        \Exs $x = (a, b, c, d, b, b, c) \quad |x| = n$

        Расстановку можно представить как функцию $f : \underset{\text{domain}}{\undergroup{[n]}} \to \underset{\text{codomain}}{\undergroup{\Sigma}}$, которая по порядковому номеру выдает символ

        $ran f = \Set{c \in \Sigma \ | \ \exists i \in [n]\ :\ f(i) = c}$

        \vspace{5mm}

        \item \textbf{Перестановка} (Permutation) - $\pi : [n] \to \Sigma$, где $n = |\Sigma|$

        Расстановка $\pi$ - биекция между $[n]$ и $\Sigma$

        \Ex $\pi = \mathtt{2713546}$

        \vspace{3mm}

        \begin{tabular}{l|ccccccc}
            i      & 1          & 2          & 3          & 4          & 5          & 6          & 7          \\
            \hline
            \pi(i) & \mathtt{2} & \mathtt{7} & \mathtt{1} & \mathtt{3} & \mathtt{5} & \mathtt{4} & \mathtt{6}
        \end{tabular}

        \vspace{5mm}

        \underline{Одна из задач комбинаторики} - посчитать количество различных расстановок или перестановок при заданных $n$ и $\Sigma$

        \vspace{5mm}

        \item \textbf{$k$-перестановка} (k-permutation) - расстановка из $k$ различных элементов из $\Sigma$

        \Ex $\underset{\text{5-perm из } \Sigma = [7]}{\undergroup{|31475|}} = 5$

        $k$-перестановка - инъекция $\pi : [k] \to \Sigma$ ($k \leq n = |\Sigma|$)

        \vspace{5mm}

        \item $P(n, k)$ - множество всех $k$-перестановок алфавита $\Sigma = [n]$ (если исходный алфавит не состоит из чисел, то мы можем сделать биекцию между ним и $[n]$)

        $P(n, k) = \Set{f \ | \ f : [k] \to [n]}$

        Чаще интересует не само множество, а его размер, поэтому под обозначением $P(n, k)$ подразумевается $|P(n, k)|$

        \vspace{5mm}

        \item $\displaystyle S_n = P_n = P(n, n)$ - множество всех перестановок. Также чаще всего нас будет интересовать не множество, а его размер

        $\displaystyle |S_n| = n!$ - всего существует $n!$ перестановок

        $\displaystyle |P(n, k)| = n \cdot (n - 1) \cdot (n - 2) \cdot \dots \cdot (n - k + 1) = \frac{n!}{(n - k)!}$

        \vspace{5mm}

        \item \textbf{Циклические $k$-перестановки} (Circular $k$-permutations)

        $\displaystyle \pi_1, \pi_2 \in P(n, k)$ - циклически эквивалентны тогда и только тогда:

        $\displaystyle \exists s \ | \ \forall i \ \pi_1((i + s) \% k) = \pi_2(i)$


        \Ex $\displaystyle \pi_1 = \mathtt{76123}, \pi_2 = \mathtt{12376}$

        \begin{tikzpicture}
        \node[circle, draw=black!60, thick, minimum size=0.5cm] (s00) {\mathtt{7}};
        \node[below=1cm of s00] (pi1) {$\displaystyle \pi_1$};
        \node[below left=0.66cm and 1cm of s00] (s01) {\mathtt{6}};
        \node[below right=1cm and 0.13cm of s01] (s02) {\mathtt{1}};
        \node[below right=0.66cm and 1cm of s00] (s04) {\mathtt{3}};
        \node[below left=1cm and 0.13cm of s04] (s03) {\mathtt{2}};
        \path[->]
        (s00) edge [bend right] node {} (s01)
        (s01) edge [bend right] node {} (s02)
        (s02) edge [bend right] node {} (s03)
        (s03) edge [bend right] node {} (s04)
        (s04) edge [bend right] node {} (s00)
        ;

        \node[circle, draw=black!60, thick, minimum size=0.5cm, right=3cm of s00] (s0) {\mathtt{1}};
        \node[below=1cm of s0] (pi) {$\displaystyle \pi_2$};
        \node[below left=0.66cm and 1cm of s0] (s1) {\mathtt{2}};
        \node[below right=1cm and 0.13cm of s1] (s2) {\mathtt{3}};
        \node[below right=0.66cm and 1cm of s0] (s4) {\mathtt{6}};
        \node[below left=1cm and 0.13cm of s4] (s3) {\mathtt{7}};
        \path[->]
        (s0) edge [bend right] node {} (s1)
        (s1) edge [bend right] node {} (s2)
        (s2) edge [bend right] node {} (s3)
        (s3) edge [bend right] node {} (s4)
        (s4) edge [bend right] node {} (s0)
        ;
        \end{tikzpicture}

        $\displaystyle P_C(n, k)$ - множество всех циклических $k$-перестановок в $\Sigma$

        $\displaystyle |P_C(n, k)| \cdot k = |P(n, k)|$

        $\displaystyle |P_C(n, k)| = \frac{|P(n, k)|}{k} = \frac{n!}{k(n - k)!}$

        \vspace{5mm}

        \item \textbf{Неупорядоченная расстановка $k$ элементов} (Unordered arrangement of $k$ elements) - мультимножество $\displaystyle \Sigma^*$ размера $k$

        \Ex $\displaystyle \Sigma^* = \Set{\triangle, \triangle, \Box, \triangle, \circ, \Box}^* = \Set{3 \cdot \triangle, 2 \cdot \Box, 1 \cdot \circ} = (\Sigma, r)$

        Неупорядоченную расстановку можно представить как функцию:

        $r : \Sigma \to \Natural, \quad r(x)$ - кол-во повторений объекта $x$

        \vspace{5mm}

        \item \textbf{$k$-сочетание} ($k$-combination) - неупорядоченная перестановка из $k$ различных элементов из $\Sigma$ (еще называют $k$-подмножеством, $k$-subset)

        Соответственно $C(n, k)$ - множество всех таких $k$-сочетаний

        $\displaystyle |C(n, k)| = C^k_n = \begin{pmatrix}n \\ k\end{pmatrix}$

        $C(n, k) = \begin{pmatrix}\Sigma \\ k\end{pmatrix}$

        $\begin{pmatrix}n \\ k\end{pmatrix} \cdot k! = |P(n, k)|$

        $\displaystyle |C(n, k)| = \begin{pmatrix}n \\ k\end{pmatrix} = \frac{n!}{k!(n - k)!}$

    \end{itemize}

    \Th Биномиальная теорема (Binomial theorem):

    \[(x + y)^n = \sum_{k=0}^n \begin{pmatrix}n \\ k\end{pmatrix} x^k y^{n - k}\]

    $\begin{pmatrix}n \\ k\end{pmatrix}$ - биномиальный коэффициент

    \Th Мультиномиальная теорема (Multinomial theorem)

    \[(x_1 + \dots + x_r)^n = \sum_{\substack{k_i \in 1..n, \\ k_1 + \dots + k_r = n}} \begin{pmatrix}n \\ k_1, \dots, k_r\end{pmatrix} x^{k_1}_1 \cdot \dots \cdot x^{k_r}_r\]

    $\displaystyle \begin{pmatrix}n \\ k_1, \dots, k_r\end{pmatrix} = \frac{n!}{k_1! \dots k_r!}$ - мультиномиальный коэффициент



    \Ex мультиномиальной теоремы:

    $\displaystyle (x + y + z)^4 = 1 (x^4 + y^4 + z^4) + 4 (xy^3 + xz^3 + x^3y + yz^3 + y^3z + yz^3) +
    6(x^2y^2 + y^2z^2 + x^2z^2) + 12 (xyz^2 + xy^2z + x^2yz)$

    Доказательство:

    $\Box$

    $\displaystyle (x_1 + \dots + x_r)^n = \sum_{\substack{i_j \in [r] \\ j \in [n]}} x_{i_1}^1 \cdot \dots \cdot x_{i_n}^1 =
    \sum_{\substack{i_j \in [r] \\ j \in [n]}} x_1^{k_1} \cdot \dots \cdot x_r^{k_r}$, где $\displaystyle k_t$ - количество $x$ с индексом $t$ в одночлене ($\displaystyle k_t = |\Set{j \in [n] | i_j = t}|$)

    Получается мультиномиальный коэффицциент $\displaystyle \begin{pmatrix}
                                                  n \\ k_1, \dots, k_r
    \end{pmatrix}$
    будет равен количество способов поставить $\displaystyle k_1$ единиц в индексы в $\displaystyle x_{i_1}^1 \cdot \dots \cdot x_{i_n}^1$, $\displaystyle k_2$ двоек в индексы и так далее

    У нас есть $\displaystyle \begin{pmatrix}
                    n \\ k_1
    \end{pmatrix}$ способов поставить единицу в индексы в одночлен,
    $\displaystyle \begin{pmatrix}
         n - k_1 \\ k_2
    \end{pmatrix}$ способов поставить двойку и т. д., получаем:

    $\displaystyle \begin{pmatrix}
         n \\ k_1, \dots, k_r
    \end{pmatrix} = \begin{pmatrix}
                        n \\ k_1
    \end{pmatrix} \begin{pmatrix}
                      n - k_1 \\ k_2
    \end{pmatrix} \dots \begin{pmatrix}
                            n - k_1 - \dots - k_{r - 1} \\ k_r
    \end{pmatrix} = [n - k_1 - \dots - k_r = 0] = \\
    \frac{n!}{k_1! (n - k_1)!} \frac{(n - k_1)!}{k_2! (n - k_1 - k_2)!} \dots \frac{(n - k_1 - \dots - k_{r - 1})!}{k_r! 0!} = \frac{n!}{k_1! \dots k_r!}$

    $\Box$

    \begin{itemize}

        \item \textbf{Перестановка мультимножества $\displaystyle \Sigma^*$} (Permutations of a multiset $\displaystyle \Sigma^*$)

        $\displaystyle \Sigma^* = \Set{\triangle^1, \triangle^2, \Box, \star} = (\Sigma, r) \quad r : \Sigma \to \Natural_0 \quad n = |\Sigma^*| = 4 \quad s = |\Sigma| = 3$

        \Nota \begin{cases}
                  \triangle^1, \triangle^2, \Box, \star \\
                  \triangle^2, \triangle^1, \Box, \star
        \end{cases} считаются равными перестановками

        $\displaystyle |P^*(\Sigma^*, n)| = \frac{n!}{r_1! \dots r_s!} = \begin{pmatrix}
                                                               n \\ r_1, \dots, r_s
        \end{pmatrix}$ - количество перестановок мультимножества, где $\displaystyle r_i$ - количество $i$-ого элемента в мультимножестве

        \item \textbf{$k$-комбинация бесконечного мультимножества} ($k$-combinations of infinite multiset) -
        такое субмультимножество размера $k$, содержащее элементы из исходного мультимножества.
        При этом соблюдается, что количество какого-либо элемента $\displaystyle r_i$ в исходном мультимножестве не больше размера комбинации $k$

        $\displaystyle \Sigma^* = \Set{\infty \cdot \triangle, \infty \cdot \Box, \infty \cdot \star, \infty \cdot \Cat}^* \quad n = |\Sigma^*| = \infty$

        $\Sigma = \Set{\triangle, \Box, \star, \Cat} \quad s = |\Sigma| = 4$

        \Ex $5$-комбинация: $\Set{\triangle, \star, \Box, \star, \Box}$

        Разделяем на группы по $\Sigma$ палочками:

        $\triangle \Big| \Box \Box \Big| \star \star \Big| $

        Заменяем элементы на точечки - нам уже не так важен тип элемента, потому что мы знаем из разделения:

        $\bullet \Big| \bullet \bullet \Big| \bullet \bullet \Big| $

        (другой \Exs $\bullet \bullet \bullet \bullet \Big| \Big| \Big| \bullet = \Set{4 \cdot \triangle, 1 \cdot \Cat}$)

        Получается всего $k$ точечек и $s - 1$ палочек, всего $k + s - 1$ объектов. Получаем мультимножество $\Set{k \cdot \bullet, (s - 1) \cdot \Big|}$ (\textit{Star and Bars method})

        Получаем количество перестановок этого мультимножества:
        $\displaystyle \frac{(k + s - 1)!}{k!(s - 1)!} = \begin{pmatrix}
                                               k + s - 1 \\ k, s - 1
        \end{pmatrix} =
        \begin{pmatrix}
            k + s - 1 \\ k
        \end{pmatrix} = \begin{pmatrix}
                            k + s - 1 \\ s - 1
        \end{pmatrix}$

        что и является количеством возможных $k$-комбинаций бесконечного мультимножества

        \vspace{5mm}

        \item \textbf{Слабая композиция} (Weak composition) неотрицательного целого числа $n$ в $k$ частей -
        это решение $\displaystyle (b_1, \dots, b_k)$ уравнение $\displaystyle b_1 + \dots + b_k = n$, где $\displaystyle b_i \geq 0$

        $|\Set{\text{слабая композиция } n \text{ в } k \text{ частей}}| = \begin{pmatrix}
                                                                               n + k - 1 \\ n, k - 1
        \end{pmatrix}$

        Для решения воспользуемся аналогичным из доказательства мультиномиальной теоремы приемом:

        $n = 1 + 1 + 1 + \dots + 1$

        Поставим палочки:

        $n = 1 + 1 \Big| 1 \Big| \dots + 1$

        Получаем задачу поиска количеств $k$-комбинаций в мультимножестве: $\Set{n \cdot 1, (k - 1) \cdot \Big|}$; получаем $\begin{pmatrix}
                                                                                                                                 n + k - 1 \\ n, k - 1
        \end{pmatrix}$

        \vspace{5mm}
        \item \textbf{Композиция} (Composition) - решение для $\displaystyle b_1 + \dots + b_k = n$, где $\displaystyle b_i > 0$

        $|\Set{\text{композиция } n \text{ в } k \text{ частей}}| = \begin{pmatrix}
                                                                        n - k + k - 1 \\ n - k, k - 1
        \end{pmatrix}$

        Мы знаем, что одну единичку получит каждая $\displaystyle b_i$, поэтому мы решаем это как слабую композицию для $n - k$ в $k$ частей

        \vspace{5mm}
        \item \textbf{Число композиций $n$ в некоторой число частей} (Number of all compositions into some number of positive parts)

        $\displaystyle \sum_{k=1}^n \begin{pmatrix}
                          n - 1 \\ k - 1
        \end{pmatrix} = 2^{n-1}$

        Пусть $t = k - 1$, тогда $\displaystyle \sum_{t = 0}^{n-1} \begin{pmatrix}
                                                         n - 1 \\ t
        \end{pmatrix} = 2^{n - 1}$

        \vspace{5mm}
        \item \textbf{Разбиения множества} (Set partitions) - множество размера $k$ непересекающихся непустых подмножеств

        \begin{tabular}{cp}
            \Exs $\Set{1, 2, 3, 4}, n = 4, k = 2 \rightarrow [\text{разбиение в 2 части}] \rightarrow & \Set{\Set{1}, \Set{2, 3, 4}}, \\
            & \Set{\Set{1, 2}, \Set{3, 4}}, \\
            & \Set{\Set{1, 2, 3}, \Set{4}}, \\
            & \Set{\Set{1, 3}, \Set{2, 4}}, \\
            & \Set{\Set{1, 4}, \Set{2, 3}}, \\
            & \Set{\Set{2}, \Set{1, 3, 4}}, \\
            & \Set{\Set{3}, \Set{1, 2, 4}}$
        \end{tabular}

        $\displaystyle |\Set{\text{разбиение } n \text{ элементов в } k \text{ частей}}| = \begin{Bmatrix}
                                                                                 n \\ k
        \end{Bmatrix} = S^{II}_k (n) = S(n, k)$ - число Стирлинга второго рода

        Для примера выше число Стирлинга $S(4, 2) = \begin{Bmatrix}
                                                        4 \\ 2
        \end{Bmatrix} = 7$

        Согласно Википедии \href{https://ru.wikipedia.org/wiki/%D0%A7%D0%B8%D1%81%D0%BB%D0%B0_%D0%A1%D1%82%D0%B8%D1%80%D0%BB%D0%B8%D0%BD%D0%B3%D0%B0_%D0%B2%D1%82%D0%BE%D1%80%D0%BE%D0%B3%D0%BE_%D1%80%D0%BE%D0%B4%D0%B0}{для формулы Стирлинга}
        есть формула: $\displaystyle S(n, k) = \frac{1}{k!} \sum_{j=0}^k (-1)^{k+j} \begin{pmatrix}
                                                                          k \\ j
        \end{pmatrix}j^n$

        \vspace{5mm}
        \item \textbf{Формула Паскаля} (Pascal's formula)

        $\begin{pmatrix}
             n \\ k
        \end{pmatrix} = \begin{pmatrix}
                            n - 1 \\ k - 1
        \end{pmatrix} + \begin{pmatrix}
                            n - 1 \\ k
        \end{pmatrix}$

        \vspace{5mm}
        \item \textbf{Рекуррентное отношение для чисел Стирлинга} (Recurrence relation for Stirling$\displaystyle ^{(2)}$ number):

        $\begin{Bmatrix}
             n \\ k
        \end{Bmatrix} = \begin{Bmatrix}
                            n - 1 \\ k - 1
        \end{Bmatrix} + k \cdot \begin{Bmatrix}
                                    n - 1 \\ k
        \end{Bmatrix}$

        Возьмем какое-либо разбиение для $n - 1$ элементов на $k$ частей, тогда возможны два случая:

        1) В $k$-ое множество нет ни одного элемента, тогда мы обязаны в него положить наш $n$-ый элемент по определению,
        количество перестановок будет равно $\begin{Bmatrix}
                                                 n - 1 \\ k - 1
        \end{Bmatrix} \cdot 1$

        2) В $k$-ом множестве уже есть элементы, тогда все множества будут заполнены и у нас будет выбор из $k$ множеств,
        куда положить $k$-ый элемент, то есть $k \cdot \begin{Bmatrix}
                                                           n - 1 \\ k
        \end{Bmatrix}$

        Эти два случая независимы, поэтому получаем $\begin{Bmatrix}
                                                         n - 1 \\ k - 1
        \end{Bmatrix} + k \cdot \begin{Bmatrix}
                                    n - 1 \\ k
        \end{Bmatrix}$

        \vspace{5mm}
        \item \textbf{Число Белла} (Bell number) - количество всех неупорядоченных разбиений множества размера $n$

        Число Белла вычисляется по формуле: $\displaystyle B_n = \sum_{m=0}^n S(n, m)$

        \vspace{5mm}
        \item \textbf{Целочисленное разбиение} (Integer partition) - решение для $\displaystyle a_1 + \dots + a_k = n$, где $\displaystyle a_1 \geq a_2 \geq \dots \geq a_k \geq 1$

        $p(n, k)$ - число целочисленных разбиений $n$ в $k$ частей

        $\displaystyle p(n) = \sum_{k = 1}^n p(n, k)$ - число всех разбиений для $n$

        \Ex $5 = 5 = 4 + 1 = 3 + 2 = 3 + 1 + 1 = 2 + 2 + 1 = 2 + 1 + 1 + 1 = 1 + 1 + 1 + 1 + 1$

        \vspace{5mm}

    \end{itemize}



    \begin{itemize}
        \item \textbf{Принцип включения/исключения} (Principle of Incusion/Exclusion (PIE))

        $|A \union B \union C| = |A| + |B| + |C| - |A \cap B| - |B \cap C| - |A \cap C| + |A \cap B \cap C|$

        \Ex есть $n = 11$ объектов, нужно распределить их между $k = 3$ группами $A$, $B$ и $C$

        Эту задачу можно решить с помощью \textit{Stars and bars method}, тогда мы получим $
        \begin{pmatrix} n + k - 1 \\ n, k - 1 \end{pmatrix} = \begin{pmatrix} 13 \\ 2 \end{pmatrix} = 78$

        Введем ограничение: пусть мощность каждого множества будет не больше 4.

        Посчитаем количество неподходящих вариантов:

        $\displaystyle |A| = |\Set{b_A \geq 5}| = 1 \cdot
        \begin{pmatrix} 11 - 5 + 3 - 1 \\ 3 - 1 \end{pmatrix} =
        \begin{pmatrix} 8 \\ 2 \end{pmatrix} = 28$

        $\displaystyle |A \cap B| = |\Set{b_A \geq 5 \land b_B \geq 5}| =
        \begin{pmatrix} 3 \\ 2 \end{pmatrix} = 3$

        $\displaystyle |A \cap B \cap C| = |\Set{b_A \geq 5 \land b_B \geq 5 \land b_C \geq 5}| = 0$

        Итого получаем $28 \cdot 3 + 3 \cdot 3 + 0 = 75$ вариантов.

        Далее исключаем эти варианты из количества всех вариантов, а значит подходящих вариантов всего $78 - 75 = 3$

        \vspace{5mm}
        \item \textbf{Принцип включения/исключения} (Inclusion/Exclusion Principle (PIE))

        \begin{itemize}
            \item $X$ - начальное множество элементов
            \item $\displaystyle P_1, \dots, P_m$ - свойства
            \item Пусть $\displaystyle X_i = \Set{x \in X \ | \ P_i\text{ - свойство для } x}$
            \item Пусть $S \in [m]$ - множество свойств
            \item Пусть $\displaystyle N(S) = \bigcap_{i \in S} X_i = \Set{x \in X\ | \ x \text{ имеет все свойства } P_1, \dots, P_m}$
        \end{itemize}

        $N(\emptyset) = X \quad |N(\emptyset)| = |X| = n$

        \vspace{5mm}
        \item \textbf{Теорема ПВ/И} (Theorem PIE)

        $\displaystyle |X \setminus (X_1 \union X_2 \union \dots \union X_m)| = \sum_{S \subseteq [m]} (-1)^{|S|} |N(S)|$ - количество элементов множества $X$, не имеющих никакое из свойств

        Доказательство:

        Пусть $x \in X$

        Если $x$ не имеет свойств $\displaystyle P_1,\dots,P_m$, то $x \in N(\emptyset)$ и $x \notin N(S) \ \forall S \neq \emptyset$

        Поэтому $x$ дает в общую сумму $1$

        Иначе, если $x$ имеет $k \geq 1$ свойств $T \in \begin{pmatrix} [m] \\ k\end{pmatrix}$,

        то $x \in N(S)$ тогда и только тогда, когда $S \subseteq T$.

        Поэтому $x$ дает в сумму $\displaystyle \sum_{S \subseteq T} (-1)^{|S|} = \sum_{i = 0}^k \begin{pmatrix} k \\ i \end{pmatrix} (-1)^i = 0$

        \vspace{5mm}
        \item \textbf{Следствие}

        $\displaystyle |\bigunion_{i \in [m]} X_i| = |X| - \sum_{S \subseteq [m]} (-1)^{|S|} |N(S)| = \sum_{S \subseteq [m], S \neq \emptyset} (-1)^{|S| - 1} |N(S)|$

        \vspace{5mm}
        \item \textbf{Приложения}:

        * Определяете \enquote{плохие} свойства $\displaystyle P_1, \dots, P_m$

        * Посчитываете $N(S)$

        * Применяете ПВ/И

        \vspace{5mm}
        \item \textbf{Количество сюръекций (правототальных функций)}

        * $X = \Set{\text{функция } f : [k] \to [n]}$

        * Плохое свойство $\displaystyle P_i \ : \ X_i = \Set{f : [k] \to [n] \ | \ \nexists j \in [k] : f(j) = i}$

        * $\displaystyle |\Set{\text{сюръекции } f : [k] \to [n]}| = |X \setminus (X_1 \union \dots \union X_m)| \stackrel{\text{PIE}}{=}
        \sum_{S \subseteq [m]} (-1)^{|S|} |N(S)| = \sum_{S \subseteq [m]} (-1)^{|S|} (n - |S|)^k =
        \sum^k_{i = 0} (-1)^{i} \begin{pmatrix} k \\ i \end{pmatrix} (k - i)^n$

        \vspace{5mm}
        \item \textbf{Количество биекций}

        $\displaystyle n! = \sum_{i=0}^n (-1)^i \begin{pmatrix}
                                      n \\ i
        \end{pmatrix} (n - i)^n$

        \item \textbf{Число Стирлинга} (опять)

        Заметим, что сюръекция = разбиение, тогда:

        $\displaystyle \sum^k_{i = 0} (-1)^{i} \begin{pmatrix} k \\ i \end{pmatrix} (k - i)^n = n! S^{II}_n (k)$

        \vspace{5mm}
        \item \textbf{Беспорядки} (Derangements) - перестановка без фиксированных точек

        Если $f(i) = i$, то $i$ - фиксированная точка

        * $X = $ все $n!$ перестановок

        * Плохие свойства $\displaystyle P_1,\dots,P_m : \pi \in X$ имеет свойство $\displaystyle P_i$ \Longleftrightarrow $\pi(i) = i$

        * Посчитаем $N(S): \quad N(S) = (n - |S|)!$

        * Применяем ПВ/И: $\displaystyle X \setminus (X_1 \union \dots \union X_n) = \sum_{S \subseteq [n]} (-1)^{|S|} N(S) =
        \sum_{S \subseteq [n]} (-1)^{|S|} (n - |S|)! = \sum_{i = 0}^n (-1)^{i} \begin{pmatrix}
                                                                                   n \\ i
        \end{pmatrix} (n - i)!$

    \end{itemize}

    \clearpage


    \section{8. Рекуррентности и производящие функции}

    \begin{itemize}
        \item \textbf{Производящие функции} (Generating Functions)

        $\displaystyle \sum_{n = 0}^\infty a_n x^n = a_0 + a_1 x + a_2 x^2 + \dots$

        Функция выше задает последовательность $\displaystyle a_0, a_1, a_2, \dots$

        \Ex $\displaystyle 3 + 8x^2 + x^3 + \frac{1}{7}x^5 + 100x^6 + \dots \implies (3, 0, 8, 1, 0, \frac{1}{7}, 100, \dots)$

        \Ex Последовательность $(1, 1, 1, \dots)$ задает функцию $\displaystyle 1 + x + x^2 + \dots = \sum_{n = 0}^\infty x^n$

        Пусть $\displaystyle S = 1 + x + x^2 + \dots$, тогда $\displaystyle xS = x + x^2 + \dots$, $(1 - x) S = 1 \Longrightarrow $

        \fbox{$\displaystyle S = \frac{1}{1 - x}$ задает последовательность $(1, 1, 1, \dots)$}
    
        \Ex

        $\displaystyle \frac{1}{1 + x} = 1 - x + x^2 - x^3 + \dots = \sum_{n = 0}^\infty (-1)^n x^n$

        $\displaystyle \frac{1}{1 - 3x} = 1 + 3x + 9x^2 + 27x^3 + \dots = \sum_{n = 0}^\infty 3^n x^n$

        $\displaystyle \frac{2}{1 - x} = 2 + 2x + 2x^2 + 2x^3 + \dots = \sum_{n = 0}^\infty 2 x^n$

        $(2, 4, 10, 28, 82, \dots) = (1, 1, 1, 1, 1, \dots) + (1, 3, 9, 27, 81, \dots)$

        $\displaystyle \frac{1}{1 - x} + \frac{1}{1 - 3x} = \frac{2 - 4x}{(1 - x)(1 - 3x)}$

        $\displaystyle \frac{1}{1 - x^2} = 1 + x^2 + x^4 + x^6 + \dots = \sum_{n = 0}^\infty x^{2n} \implies (1, 0, 1, 0, \dots)$

        $\displaystyle \frac{x}{1 - x^2} = x + x^3 + x^5 + \dots = \sum_{n = 0}^\infty x^{2n + 1} \implies (0, 1, 0, 1, \dots)$

        \textbf{Взятие производной}:

        $\displaystyle \frac{d}{dx} (\frac{1}{1 - x}) = \frac{1}{1 - x^2} = \frac{d}{dx} (1 + x + x^2 + \dots) = 1 + 2x + 3x^2 + 4x^3 + \dots \implies (1, 2, 3, 4, \dots)$

        \Ex Найти ПФ для $(1, 3, 5, 7, 9, \dots)$

        $\displaystyle A(x) = 1 + 3x + 5x^2 + \dots$

        $\displaystyle xA = 0 + x + 3x^2 + 5x^3 + \dots$

        $\displaystyle (1 - x)A = 1 + 2x + 2x^2 + 2x^3 + \dots$

        $\displaystyle (1 - x)A = 1 + \frac{2x}{1 - x} \quad A = \frac{1 + \frac{2x}{1 - x}}{1 - x} = \frac{1 + x}{(1 - x)^2}$

        \Ex Найти ПФ для $(1, 4, 9, 16, \dots)$

        $\displaystyle A = 1 + 4x + 9x^2 + 16x^3 + \dots \quad (1 - x)A = $

        \item \textbf{Подсчет, используя производящие функции}

        Найти число решений для $\displaystyle x_1 + x_2 + x_3 = 6$, где $\displaystyle x_i \geq 0, x_1 \leq 4, x_2 \leq 3, x_3 \leq 5$

        $\displaystyle A_1(x) = 1 + x + x^2 + x^3 + x^4$

        $\displaystyle A_2(x) = 1 + x + x^2 + x^3$

        $\displaystyle A_3(x) = 1 + x + x^2 + x^3 + x^4 + x^5$

        $\displaystyle A(x) = A_1 \cdot A_2 \cdot A_3 = 1 + 3x + 6x^2 + 10x^3 + 14x^4 + 17x^5 + \underline{18x^6} + 17x^7 + \dots$

        Ответ - 18

    \end{itemize}



    \begin{itemize}
        \item \textbf{Рекуррентные соотношения} (Recurrence relations)

        \underline{Решить рекуррентное соотношение} - найти закрытую формулу

        \Ex Арифметическая прогрессия

        $\displaystyle a_n = \begin{cases}a_0 = const \quad n = 0 \\ a_{n - 1} + d, \quad n > 0\end{cases}$

        Решение: $\displaystyle a_n = a_0 + nd$ - анзац (Ansatz, догадка)

        Проверка: $\displaystyle a_n = a_0 + nd = a_{n - 1} + d = a_0 + (n - 1)d + d = a_0 + nd \quad$ - {\Large👍👍}

        \item Метод характеристического уравнения

        \substack{\text{Рекуррентное} \\ \text{соотношение}} \ $\displaystyle \stackrel{a_n \to r^n}{\rightsquigarrow}$ \ \substack{\text{Характеристическое} \\ \text{уравнение}} \ $\stackrel{\text{решение}}{\rightsquigarrow}$ Корни $\stackrel{\text{магия}}{\rightsquigarrow}$ Решение $\rightsquigarrow$ Проверка

        \Ex $\displaystyle a_n = a_{n - 1} + 6a_{n - 2}$

        $\displaystyle r^n - r^{n - 1} - 6r^{n - 2} = 0$

        $\displaystyle r^{n-  2} (r^2 - r - 6) = 0$

        $\displaystyle r_{1,2} = -2, 3$

        \fbox{Если $\displaystyle r_1 \neq r_2$, то $\displaystyle a_n = ar_1^n + br_2^n$ - общее решение \\

        Если $\displaystyle r_1 = r_2 = r$, то $\displaystyle a_n = ar^n + bnr^n$}

        $\displaystyle a_n = a(-2)^n + b(3)^n$

        Пусть $\displaystyle \begin{cases}a_0 = 1 = a + b \\ a_1 = 8 = -2a + 3b\end{cases}$

        $\displaystyle -5a = 5 \Longrightarrow \begin{cases}a = -1 \\ b = 2\end{cases} \Longrightarrow a_n = -(-2)^n + 2 \cdot 3^n$

        \item \textbf{Разделяй и властвуй} (Divide-and-Conquer)

        $\displaystyle T(n) = \underset{\text{работа рекурсии}}{\undergroup{2T\left(\frac{n}{2}\right)}} + \underset{\text{работа разделения/слияния}}{\undergroup{\theta(n)}}$

        \item \textbf{Основная теорема о рекуррентных соотношениях} (Master Theorem)
        \hfill\href{https://ru.wikipedia.org/wiki/%D0%9E%D1%81%D0%BD%D0%BE%D0%B2%D0%BD%D0%B0%D1%8F_%D1%82%D0%B5%D0%BE%D1%80%D0%B5%D0%BC%D0%B0_%D0%BE_%D1%80%D0%B5%D0%BA%D1%83%D1%80%D1%80%D0%B5%D0%BD%D1%82%D0%BD%D1%8B%D1%85_%D1%81%D0%BE%D0%BE%D1%82%D0%BD%D0%BE%D1%88%D0%B5%D0%BD%D0%B8%D1%8F%D1%85}{*тык*}


        $\displaystyle T(n) = aT\left(\frac{n}{b}\right) + f(n)$

        Из этого, $\displaystyle c_{crit} = \log_b a$

        \vspace{5mm}

        \underline{I случай}: слияние $<$ рекурсия

        $\displaystyle f(n) \in O(n^c)$, где $\displaystyle c < c_{crit} \Longrightarrow T(n) \in \Theta(n^{c_{crit}})$

        $\displaystyle f(n) \in O(n^c) \Longleftrightarrow f(n) \in o(n^{c_{crit}})$

        \vspace{5mm}

        \underline{II случай}: слияние $\approx$ рекурсия

        $\displaystyle f(n) \in \Theta(n^{c_{crit}} \log^k n) \Longrightarrow T(n) \in \Theta(n^{c_{crit}} \log^{k + 1} n)$

        Здесь $k \geq 0$. В общем случае см. википедию

        \vspace{5mm}

        \underline{III случай}: слияние $>$ рекурсия

        $\displaystyle f(n) \in \Omega(n^c)$, где $\displaystyle c > c_{crit} \Longrightarrow T(n) \in \Theta(f(n))$

        \item \textbf{Метод Акра-Бацци} (Akra-Bazzi method)
        \hfill\href{https://en.wikipedia.org/wiki/Akra%E2%80%93Bazzi_method}{*тык*}


        $\displaystyle T(n) = f(n) + \sum_{i = 1}^k a_i T(b_i n + h_i(n)) \Longrightarrow T(n) \in \Theta\left(n^p \cdot \left(1 + \int_1^n \frac{f(x)}{x^{p + 1}} dx\right)\right)$, где $p$ - решение для $\displaystyle \sum_{i = 1}^k a_i b_i^p = 1$

        \begin{cases}
            k = const \\
            a_i > 0 \\
            0 < b_i < 1 \\
            h_1(n) \in O(\frac{n}{\log^2 n}) \text{ - малые возмущения}
        \end{cases}

        \Ex $\displaystyle T(n) = T\left(\lfloor\frac{n}{2}\rfloor\right) + T\left(\lceil\frac{n}{2}\rceil\right) + n$ - асимптотика сортировки слиянием

        $\displaystyle T(n) = T\left(\frac{n}{2} + O(1)\right) + T\left(\frac{n}{2} - O(1)\right) + \theta(n)$

        Здесь $\displaystyle b_i = \frac{1}{2}, \quad h = \pm O(1) \in O\left(\frac{n}{\log^2 n}\right)$

        \Ex $\displaystyle T(n) = T\left(\frac{3n}{4}\right) + T\left(\frac{n}{4}\right) + n$

        $\displaystyle a_1 = 1, b_1 = \frac{3}{4}, a_2 = 1, b_2 = \frac{1}{4}, f(n) = n$

        $\displaystyle (\frac{3}{4})^p + \left(\frac{1}{4}\right)^p = 1$

        $p = 1$

        $\displaystyle \int_1^n \frac{x}{x^{1 + 1}}dx = \int_1^n \frac{dx}{x} = \ln x \Big|_1^n = \ln n$

        $T(n) \in \Theta(n \cdot (1 + \ln n))$

        $T(n) \in \Theta(n \ln n)$

        \item Решить рекуррентное соотношение $\displaystyle a_n = 3a_{n-1} - 2a_{n-1}$, где $\displaystyle a_0 = 1, a_1 = 3$

        Используем производящие функции:

        $\displaystyle A(x) = \frac{1}{1 - 3x + 2x^2} = \frac{1}{(1 - x)(1 - 2x)} = \frac{-1}{1 - x} + \frac{2}{1 - 2x} \to 2^{n + 1} - 1$

    \end{itemize}




\end{document}

