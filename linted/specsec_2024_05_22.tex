\documentclass[12pt]{article}
\usepackage{preamble}

\pagestyle{fancy}
\fancyhead[LO,LE]{Специальные разделы \\ высшей математики}
\fancyhead[CO,CE]{22.05.2024}
\fancyhead[RO,RE]{Лекции Далевской О. П.}


\begin{document}
    \Th $\displaystyle Ly = f(x), y = \overline{y} + y^*$ - решение $Ly = f(x)$.

    Тогда $\displaystyle \overline{y} + y^*$ - общее решение

    $\Box$

    Правда ли, что найдется единственный набор констант $\displaystyle C_1, \dots, C_n$, которое удовлетворяет НУ $\displaystyle \begin{cases}y(x_0) = y_0 \\ y^\prime(x_0) = y_0^\prime \\ \vdots\end{cases}$

    Так как $\displaystyle \overline{y} + y^*$ - решение, то
    $\displaystyle \begin{cases}
         y_0 = C_1 y_{01} + C_2 y_{02} + \dots + C_n y_{0n} + y_0^* \\
         y_0^\prime = C_1 y_{01}^\prime + \dots + y_0^*^{\prime}
    \end{cases} \Longleftrightarrow
    \begin{cases}
        y_0 - y_0^* = \sum C_i y_{0i} \\
        y_0^\prime - y_0^*^\prime = \sum C_i y_{0i}^\prime \\
    \end{cases} \Longleftrightarrow
    \underset{\det W \neq 0}{\undergroup{
    \begin{pmatrix}
        y_{01} & y_{02} & \dots & y_{0n} \\
        y_{01}^\prime & y_{02}^\prime & \dots & y_{0n}^\prime \\
        \vdots & \vdots & \ddots & \vdots \\
        y_{01}^{(n)} & y_{02}^{(n)} & \dots & y_{0n}^{(n)} \\
    \end{pmatrix}}}
    \begin{pmatrix}
        C_1 \\ C_2 \\ \vdots \\ C_n
    \end{pmatrix} =
    \begin{pmatrix}
        y_0 - y_0^* \\ \vdots \\ \vdots \\ \vdots
    \end{pmatrix}
    $

    Таким образом система имеет единое решение $\displaystyle (C_1, \dots, C_n)$, которое удовлетворяет НУ

    $\Box$

    \Th $\displaystyle Ly = f_1(x) + f_2(x)$

    Пусть $\displaystyle Ly_1^* = f_1(x)$ и $\displaystyle Ly^*_2 = f_2(x)$, тогда $\displaystyle Ly^* = f_1 + f_2$, где $\displaystyle y^* = y_1^* + y_2^*$

    $\Box$

    $\displaystyle Ly^* = L(y^*_1 + y^*_2) = Ly^*_1 + Ly^*_2 = f_1(x) + f_2(x)$

    $\Box$

    \section{4.6. Системы ДУ}

    \Def Набор функций $\displaystyle y_1, \dots, y_n$.

    Система дифференциальных уравнений, связывающие эти функции, то есть

    $\displaystyle \begin{cases}
         F_1(x_1, y_1, \dots y_n, \dots, y_1^{(n)}, \dots y_n^{(n)}) = 0
         \vdots
    \end{cases}$ называется системой ДУ

    \vspace{5mm}

    \underline{Механический смысл}

    $\displaystyle \Real^n$ - фазовое пространство - пространство состояний системы

    $t$ - время, $\displaystyle x_i$ - координаты точки $M$ в $\displaystyle \Real^n$

    $\displaystyle \begin{cases}
         \frac{dx_1}{dt} = \varphi_1(t, \Set{x_i}) \\
         \frac{dx_2}{dt} = \varphi_2(t, \Set{x_i}) \\
         \vdots \\
         \frac{dx_n}{dt} = \varphi_n(t, \Set{x_i}) \\
    \end{cases}$ \begin{tabular}{r} - СДУ описывает состояние исследуемой системы во времени, \\ причем $\displaystyle \frac{dx_i}{dt} = \dot{x_i}$ - скорости \end{tabular}

    \Nota Такая система называется нормальной, то есть все уравнения разрешены относительно производных

    \Nota Всякое ДУ$\displaystyle _n$ можно рассмотреть как СДУ: $\displaystyle y^{(n)} = f(x, y, y^\prime, \dots, y^{(n - 1)}) \Longleftrightarrow y = y_1(x), y^\prime = y_2(x, y_1), \dots$

    Можно сделать и обратное - свести СДУ к ДУ$\displaystyle _n$

    \underline{Метод исключения} Рассмотрим на примере СДУ 2-ого порядка

    $\displaystyle \begin{cases}
         \frac{dy}{dt} = f(x, y, t) \\
         \frac{dx}{dt} = g(x, y, t)
    \end{cases} \Longleftrightarrow \begin{cases}
         \dot{y} = f(x, y, t) \\
         \dot{x} = g(x, y, t)
    \end{cases} \Longleftrightarrow \begin{cases}
         \ddot y = \frac{\partial f}{\partial t} + \frac{\partial f}{\partial x}\frac{dx}{dt} + \frac{\partial f}{\partial y}\frac{dy}{dt} = \frac{\partial f}{\partial t} + \frac{\partial f}{\partial x}g + \frac{\partial f}{\partial y}f \\
         \dot{x} = g(x, y, t)
    \end{cases}$

    Свели СДУ к ДУ$\displaystyle _2$: $\displaystyle \ddot y = \frac{\partial f}{\partial t} + \frac{\partial f}{\partial x}g + \frac{\partial f}{\partial y}f$

    \Nota Чтобы свести к ДУ СДУ $\displaystyle \begin{cases}
         \dot x_1 = \varphi_1(t, x_1, \dots, x_n) \\
         \vdots \\
         \dot x_n = \varphi_n(t, x_1, \dots, x_n) \\
    \end{cases}$ \begin{tabular}{l}\\ нужно исключить $n - 1$ \\ выражение $\displaystyle \dot x_i$, для этого взять $\displaystyle \frac{d^{n - 1} \dot x_1}{dt^{n - 1}}$\end{tabular}

    Таким образом общий порядок СДУ (сумма порядков старших производных) будет равен порядку ДУ

    \Ex
    $\begin{cases}
        \dot y = y + 5x \\
        \dot x = -y - 3x
    \end{cases} \Longleftrightarrow
    \begin{cases}
        \ddot y = \dot y + 5\dot x \\
        \dot x = -y - 3x
    \end{cases} \Longleftrightarrow
    \begin{cases}
        \ddot y = \dot y + 5\dot (-y - 3x) \\
        \dot x = -y - 3x
    \end{cases} \Longleftrightarrow
    \begin{cases}
        \ddot y = \dot y - 5y - 15x \\
        \dot x = -y - 3x
    \end{cases} \Longleftrightarrow
    \begin{cases}
        \ddot y = \dot y - 5y - 3(\dot y - y) \\
        \dot x = -y - 3x
    \end{cases} \Longleftrightarrow \ddot y + 2\dot y + 2y = 0$

    ХрУ 😼: $\displaystyle \lambda_{1,2} = -1 \pm i \rightarrow \overline{y} = e^{-t} (C_1 \cos t + C_2 \sin t)$

    Найдем $x(t)$ из 1-ого ДУ: $\displaystyle \dot{\overline{y}} = -e^{-t} (C_1 \cos t + C_2 \sin t) + e^{-t} (-C_1 \sin t + C_2 \cos t) = e^{-t} ((C_2 - C_1) \cos t - (C_1 + C_2) \sin t)$

    $\displaystyle 5x = \dot{\overline{y}} - \overline{y} = e^{-t} ((C_2 - 2C_1) \cos t - (C_1 + 2C_2) \sin t)$

    $\displaystyle \begin{cases}
         y(t) = e^{-t} (C_1 \cos t + C_2 \sin t) \\
         x(t) = \frac{1}{5} e^{-t} ((C_2 - 2C_1) \cos t - (C_1 + 2C_2) \sin t)
    \end{cases}$

    \Nota Метод исключения сохраняет линейность, поэтому линейная СДУ (с постоян. коэфф.) сводится к ЛДУ (с пост. коэфф.)

    \Nota СДУ из \Exs не содержала $t$ в явном виде. Такие СДУ называются автономными

    \vspace{5mm}

    \underline{Матричный метод}

    $\displaystyle \begin{cases}
         y^{\prime}_1 = a_{11}y_1 + a_{12}y_2 + \dots + a_{1n} y_n \\
         \vdots \\
         y^{\prime}_n = a_{n1}y_n + a_{n2}y_2 + \dots + a_{nn} y_n
    \end{cases} \quad a_{ij} \in \Real$

    Обозначим $\displaystyle (y_1, \dots, y_n) = Y$, $\displaystyle \Set{a_ij} = A_{\text{(матрица СДУ)}}$

    Тогда СДУ запишется $\displaystyle Y^\prime = AY$ (однородная СДУ, так как нет $f(x)$)

    $\displaystyle \letsymbol \lambda_1, \dots, \lambda_n$ - собственные числа $A$ и $\displaystyle h_i$ - собственный вектор для $\displaystyle \lambda_i$

    Будем искать решение $Y$ в виде $\displaystyle Y = \ln e^{\lambda_i x}$

    Подставим в СДУ: $\displaystyle Y^\prime = \lambda_i h_i = e^{\lambda_i x} = A \underset{Y}{\undergroup{h_i e^{\lambda_i x}}} = AY$

    \Ex
    $\begin{cases}
         \dot x = x + y \\
         \dot y = 8x + 3y
    \end{cases} \quad x(0) = 0, y(0) = 2$


    $A =
    \begin{pmatrix}
        1 & 1 \\ 8 & 3
    \end{pmatrix} \quad
    \begin{vmatrix}
        1 - \lambda & 1 \\ 8 & 3 - \lambda
    \end{vmatrix} = 0$

    $\displaystyle \lambda^2 - 4\lambda - 5 = 0, \lambda_1 = -1, \lambda_2 = 5$

    $\displaystyle h_1: \begin{pmatrix}[cc|c] 2 & 1 & 0 \\ 8 & 4 & 0\end{pmatrix} \sim \begin{pmatrix}[cc|c] 2 & 1 & 0 \\ 0 & 0 & 0\end{pmatrix} \to
    h_1 = \begin{pmatrix}1 \\ -2\end{pmatrix}$

    $\displaystyle h_2: \begin{pmatrix}[cc|c] -4 & 1 & 0 \\ 8 & -2 & 0\end{pmatrix} \sim \begin{pmatrix}[cc|c] -4 & 1 & 0 \\ 0 & 0 & 0\end{pmatrix} \to
    h_1 = \begin{pmatrix}1 \\ 4\end{pmatrix}$

    $\displaystyle \begin{pmatrix}x \\ y\end{pmatrix} = C_1 h_1 e^{\lambda_1 t} + C_2 h_2 e^{\lambda_2 t} =
    C_1 \begin{pmatrix}1 \\ -2\end{pmatrix} e^{-t} = C_2 \begin{pmatrix}1 \\ 4\end{pmatrix} e^{5t}$

    Задача Коши: $\displaystyle \begin{pmatrix}0 \\ 2\end{pmatrix} = \begin{pmatrix}
                                                           C_1 + C_2 \\ -2C_1 + 4C_2
    \end{pmatrix} \to C_1 = -\frac{1}{3}, C_2 = \frac{1}{3}$

    Итак $\displaystyle \begin{cases}
              x(t) = -\frac{1}{3} e^{-t} + \frac{1}{3} e^{5t} \\
              y(t) = \frac{2}{3} e^{-t} + \frac{4}{3} e^{5t} \\
    \end{cases}$

    Решения в \Exs линейно независимы (то есть $\displaystyle Y = C_1 Y_1 + C_2 Y_2$, где $\displaystyle Y_1 = h_i e^{\lambda_i t}$), так как $\displaystyle \lambda_1 \neq \lambda_2, \lambda_{1,2} \in \Real$

    Для кратных собственных $\Real$-чисел нельзя построить базис из $\displaystyle h_i$, а чтобы составить общее решение СДУ,
    нужно $n$ линейно независимых решений $\displaystyle Y_i$ (ФСР). В этом случае используют жорданов базис (см. литературу)

    Для $\displaystyle \lambda_{1,2} \in \mathbb{C}$ можно искать решения в том же виде, но потом свести к вещественным функциям (см. литературу{\huge 🧐})

    \section{4.7. Теория устойчивости (элементы)}

    Наводящие соображения:

    Возьмем грузик, подвешенный на стержне. Когда он находится снизу, он находится в устойчивом равновесии, но когда сверху - в неустойчивом

    \Def СДУ$\displaystyle _2$:
    \begin{cases}
        \dot x = f_1(t, x, y) \\
        \dot y = f_2(t, x, y)
    \end{cases} и НУ$\displaystyle _1$:
    \begin{cases}
        x(0) = x_0 \\
        y(0) = y_0
    \end{cases} и НУ$\displaystyle _2$:
    \begin{cases}
        \tilde{x}(0) = \tilde{x}_0 \\
        \tilde{y}(0) = \tilde{y}_0
    \end{cases}

    Решение СДУ $x = x(t), y = y(t)$ называется устойчивым по Ляпунову при $t \to +\infty$

    $\displaystyle \forall \varepsilon > 0 \ \exists \delta > 0 \ | \ \underset{\begin{cases}|\tilde{x}_0 - x_0| < \delta \\ |\tilde{y}_0 - y_0| < \delta\end{cases}}{\forall x, y} \forall t > 0 \begin{cases}|\tilde{x}_0 - x_0| < \varepsilon \\ |\tilde{y}_0 - y_0| < \varepsilon\end{cases}$

    Или $\begin{matrix}\Delta x (t) \to 0 \\ \Delta y (t) \to 0\end{matrix}$ при $t \to +\infty$ и $\displaystyle \begin{cases}\Delta x_0 \to 0 \\ \Delta y_0 \to 0\end{cases}$

    \Nota Малое воздействие приводит к малым отклонениям от исходной траектории

    \Nota Обычно рассматривают отклонение решений от нулевого, то есть $\displaystyle \begin{matrix}x_0 = 0 \\ y_0 = 0\end{matrix}$

    \Ex $\dot y + y = 1$, НУ: $\displaystyle y(0) = 1, \tilde{y}(0) = \tilde{y}_0$ (малое отклонение)

    $\displaystyle \begin{cases}y = Ce^{-t} + 1 \\ y(0) = 1\end{cases} \rightarrow C = 0 \quad
    \begin{cases}y = Ce^{-t} + 1 \\ \tilde{y}(0) = \tilde{y}_0\end{cases} \to C = \tilde{y} - 1$

    $\displaystyle \tilde{y} - y = (\tilde{y}_0 - y) e^{-t} + 1 - 1 = (\tilde{y}_0 - 1)e^{-t} \stackrel{t \to +\infty}{\longrightarrow} 0$ - устойчива{\huge 🥳}

\end{document}

