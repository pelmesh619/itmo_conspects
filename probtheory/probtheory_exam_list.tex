\clearpage

\section{X. Программа экзамена в 2024/2025}

\begin{enumerate}
    \item Пространство элементарных исходов. Случайные события. Операции над событиями.

    \hyperlink{spaceofelementaryoutcomes}{Пространство элементарных исходов}: Пространством элементарных исходов $\Omega$ называется множество, содержащее все возможные исходы
    экспериментов, из которых при испытании происходит ровно один. Элементы этого множества называются
    элементарными исходами и обозначаются $\omega$

    \hyperlink{randomeventdefinition}{Случайное событие}: Случайными событиями называется подмножество $A \subset \Omega$. События $A$ наступают, если произошел один из
    элементарных исходов из множества $A$

    \hyperlink{randomeventoperations}{Операции над событиями}: Суммой $A + B$ называется событие, состоящее в том, что произошло события $A$ или событие $B$ (хотя бы одно из них)

    Произведением $A \cdot B$ называется событие, состоящее в том, что произошло событие $A$ и событие $B$ (оба из них)

    Противоположным $A$ событием называется событие $\overline{A}$, состоящее в том, что событие $A$ не произошло

    Дополнение (разность) $A \setminus B$ называется событие $A \cdot \overline{B}$

    События $A$ и $B$ называются несовместными, если их произведение - пустое множество
    (не могут произойти одновременно при одной эксперименте)

    События $A$ влечет события $B$, если $A \subset B$ (если наступает $A$, то наступит $B$)

    \item Статистическое определение вероятности. Классическое определение вероятности.

    \hyperlink{statisticaldefinitionofprobability}{Статистическое определение вероятности}: Пусть проводится $n$ реальных экспериментов, при которых событие $A$ появилось $n_A$ раз.
    Отношение $\frac{n_A}{n}$ называется частотой события $A$.
    Эксперименты показывают, что при увеличении числа $n$ частота стабилизируется у некоторого числа,
    при котором мы понимаем статистическую вероятность: $P(A) \approx \frac{n_A}{n}$ при $n \to \infty$

    \hyperlink{classicdefinitionofprobability}{Классическое определение вероятности}: Пусть пространство случайных событий $\Omega$ содержит конечное число равновозможных исходов,
    тогда применимо классическое определение вероятности: \fbox{$P(A) = \frac{|A|}{|\Omega|} = \frac{m}{n}$}, где $n$ - число всех возможных исходов, $m$ - число благоприятных исходов

    \item Геометрическое определение вероятности. Задача Бюффона об игле.

    \hyperlink{geometricdefinitionofprobability}{Геометрическое определение вероятности}: Пусть $\Omega \subset \Real^n$ - замкнутая ограниченная область,
    $\mu(\Omega)$ - мера $\Omega$ в $\Real^n$ (например, длина отрезка, площадь области на плоскости, объем тела в пространстве), в этом случае применимо геометрическое определение вероятности: $P(A) = \frac{\mu(A)}{\mu(\Omega)}$

    \hyperlink{buffonsproblem}{Задача Бюффона об игле}: пусть пол вымощен ламинатом, $2l$ - ширина доски, на пол бросается игла длины, равной ширине доски,
    найти вероятность того, что игла пересечет стык доски

    Определим положение иглы координатами центра и углом, между иглой и стыком доски, причем можно считать, что эти величины независимы

    $\letsymbol x \in [0; l]$ - расстояние от центра до ближайшего края, $\varphi \in [0; \pi]$ - угол

    $\Omega = [0; l] \times [0; \pi]$

    Событие $A$ (пересечет стык) наступает, если $x \leq l \sin \varphi$

    $P(A) = \frac{S(A)}{S(\Omega)} = \frac{\int_0^\pi l \sin \varphi d \varphi}{\pi l} = \frac{2l}{\pi l} = \frac{2}{\pi}$

    \item Аксиоматическое определение вероятности. Вероятностное пространство. Свойства вероятности.
    \item Аксиома непрерывности. Ее смысл и вывод.
    \item Свойства операций сложения и умножения. Формула сложения вероятностей.
    \item Независимость событий. Независимые события в совокупности и попарно.                   Пример Бернштейна. 
    \item Условная вероятность. Формула умножения событий.
    \item Полная группа событий. Формула полной вероятности. Формула Байеса.
    \item Последовательность независимых испытаний. Формула Бернулли. Наиболее вероятное число успехов в схеме Бернулли.
    \item Локальная и интегральная формулы Муавра-Лапласа (без док-ва).
    \item Вероятность отклонения относительной частоты от вероятности события.                            Закон больших чисел Бернулли.
    \item Схемы испытаний: Бернулли, до первого успеха. Биномиальное и геометрическое распределения. Свойство отсутствия последействия.
    \item Урновая схема с возвратом и без возврата. Гипергеометрическое распределение. Теорема об его асимптотическом приближении к биномиальному.
    \item Схема Пуассона. Формула Пуассона. Оценка погрешности в формуле Пуассона.
    \item Случайные величины, определение. Измеримость функции, ее смысл. Вероятностное пространство (R, B, P). Распределение случайной величины.
    \item Дискретные случайные величины. Определение, закон распределения, числовые характеристики.
    \item Свойства математического ожидания и дисперсии дискретной случайной величины.
    \item Стандартные дискретные распределения и их числовые характеристики (Бернулли, биномиальное, геометрическое, Пуассона).
    \item Функция распределения и ее свойства (в свойствах 4, 5, 6 достаточно привести одно из доказательств).
    \item Абсолютно непрерывные случайные величины. Плотность и ее свойства.
    \item Числовые характеристики абсолютно непрерывной случайной величины, их свойства.
    \item Равномерное распределение. 
    \item Показательное распределение. Свойство нестарения.
    \item Нормальное распределение. Стандартное нормальное распределение, его числовые характеристики.
    \item Связь между стандартным нормальным и нормальным распределениями. Следствия.
    \item Сингулярные распределения. Теорема Лебега (без док-ва).
    \item Преобразования случайных величин. Стандартизация случайной величины. 
    \item Теорема о монотонном преобразовании. Линейное преобразование случайной величины. (без док-ва).
    \item Квантильное преобразование. Моделирование случайной величины с помощью датчика случайных чисел.
    \item Виды сходимостей случайных величин, связь между ними. Теорема об эквивалентности сходимостей к константе (все без док-ва).
    \item Математическое ожидание преобразованной случайной величины. Свойства моментов.
    \item Неравенство Йенсена, следствие.
    \item Неравенства Маркова, Чебышева, правило трех сигм.
    \item Среднее арифметическое одинаковых независимых случайных величин. Закон больших чисел Чебышева.
    \item Вывод закона больших чисел Бернулли из закона больших чисел Чебышева. Законы больших чисел Хинчина и Колмогорова (только формулировки).
    \item Совместные распределения случайных величин. Функция совместного распределения, ее свойства. Независимость случайных величин.
    \item Дискретная система двух случайных величин. Закон совместного распределения. Маргинальные распределения.
    \item Абсолютно непрерывная система двух случайных величин. Плотность совместного распределения, ее свойства.
    \item Функции от двух случайных величин. Теорема о функции распределения. Формула свертки.
    \item Суммы стандартных распределений, устойчивость по суммированию (биномиальное, Пуассона, стандартное нормальное).
    \item Условные распределения и условные математические ожидания. Случаи дискретной и абсолютно непрерывной систем двух случайных величин.
    \item Пространство случайных величин. Скалярное произведение, неравенство Коши-Буняковского-Шварца. 
    \item Условное математическое ожидание как случайная величина, его свойства. Формула полного математического ожидания.
    \item Условная дисперсия. Закон полной дисперсии. Смысл второго слагаемого в разложении дисперсии.
    \item Числовые характеристики зависимости случайных величин. Ковариация, ее свойства. Коэффициент корреляции, его свойства. Корреляция случайных величин.
    \item Характеристическая функция случайной величины, ее свойства. Теорема о непрерывном соответствии (формулировка).
    \item Характеристические функции стандартных распределений (Бернулли, биномиальное, Пуассона, нормальное). Следствия.
    \item Доказательство закона больших чисел Хинчина.
    \item Центральная предельная теорема. Вывод из нее предельной теоремы Муавра-Лапласа. Неравенство Берри-Ессеена (формулировка). 
\end{enumerate}
