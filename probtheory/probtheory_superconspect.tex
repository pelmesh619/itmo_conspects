\documentclass[12pt]{article}
\usepackage{preamble}

\pagestyle{fancy}
\fancyhead[LO,LE]{Теория вероятности}
\fancyhead[RO,RE]{Лекции Блаженова А. В.}

\fancyfoot[L]{\scriptsize исходники найдутся тут: \\ \url{https://github.com/pelmesh619/itmo_conspects} \Cat}

\renewcommand{\thesection}{}

\begin{document}

    \tableofcontents
    \clearpage

    % begin probtheory_2024_09_03.tex

    \section{Лекция 1}

    В теории вероятности обычно изучают случайные события

    Обычно наука занимается закономерностями, но так как в случайных экспериментах нет закономерностей, теория
    вероятности занимается поисков закономерности в сериях случайных экспериментах

    Итак, в XVI веке начали с экспериментов бросков монеты:

    \begin{tabular}{ccc}
        число бросков & число гербов & частота \\
        4040 & 2048 & 0.5069 \\
        12000 & 6019 & 0.5016 \\
        24000 & 12012 & 0.5005 \\
    \end{tabular}

    Как можно видеть, частота стремится к 0.5 - появляется статистическая закономерность

    \subsection{Статистическое определение вероятности}

    Пусть проводится $n$ реальных экспериментов, при которых событие $A$ появилось $n_A$ раз

    Отношение $\frac{n_A}{n}$ называется частотой события $A$

    Эксперименты показывают, что при увеличении числа $n$ частота стабилизируется у некоторого числа,
    при котором мы понимаем статистическую вероятность: $P(A) \approx \frac{n_A}{n}$ при $n \to \infty$

    \subsection{Пространство элементарных исходов. Случайные события}

    \Def Пространством элементарных исходов $\Omega$ называется множество, содержащее все возможные исходы
    экспериментов, из которых при испытании происходит ровно один. Элементы этого множества называются
    элементарными исходами и обозначаются $\omega$

    \Def Случайными событиями называется подмножество $A \subset \Omega$. События $A$ наступают, если произошел один из
    элементарных исходов из множества $A$

    \ExN{1} Бросок монеты: $\Omega = \{\text{Г}, \text{Р}\}$, $A = \{\text{Г}\}$ - выпал герб

    \ExN{2} Игральная кость: $\Omega = \{1, 2, 3, 4, 5, 6\}$, $A = \{\text{выпало четное число}\} = \{2, 4, 6\}$

    \ExN{3} Монета бросается дважды.

    a) Учитываем порядок: $\Omega = \{\text{ГГ}, \text{РР}, \text{РГ}, \text{ГР}\}$

    a) Не учитываем порядок: $\Omega = \{\text{ГГ}, \text{РР}, \text{ГР}\}$

    \ExN{4} Кубик дважды: $\Omega = \{\Pair{i, j} \ | \ 1 \leq i, j \leq 6\}$

    $A = \{\text{разность} \ \vdots \ 3\} = \{\Pair{1, 4}; \Pair{4, 1}; \Pair{2, 5}; \Pair{5, 2}; \dots\}$

    \ExN{5} Монета бросается до первого герба: $\Omega = \{\text{Г}, \text{РГ}, \text{РРГ}, \dots\}$ - счетно-бесконечное множество

    \ExN{6} Монета бросается на плоскость: $\Omega = \{\Pair{x, y} \ | \ x, y \in \Real, \Pair{x, y} \text{ - центр монеты}\}$ - несчетное число исходов

    \sebsection{Операции над событиями}

    $\Omega$ - достоверные события (наступают всегда)

    $\emptyset$ - невозможное события (никогда не наступает, так как не содержит ни одного элем. исхода)

    Введем операции:

    \DefN{1} Суммой $A + B$ называется событие, состоящее в том, что произошло события $A$ или событие $B$ (хотя бы одно из них)

    \DefN{2} Произведением $A \cdot B$ называется событие, состоящее в том, что произошло событие $A$ и событие $B$ (оба из них)

    \Notas $A_1 + A_2 + \dots + A_n + \dots$ - произошло хотя бы одно из этих событий

    $A_1 \cdot A_2 \cdot \dots \cdot A_n \cdot \dots$ - произошли все эти события

    \DefN{3} Противоположным $A$ событием называется событие $\overline{A}$, состоящее в том, что событие $A$ не произошло

    \Notas $\overline{A} = A$ % не понял почему

    \DefN{4} Дополнение (разность) $A \setminus B$ называется событие $A \cdot \overline{B}$

    \DefN{5} События $A$ и $B$ называются несовместными, если их произведение - пустое множество
    (не могут произойти одновременно при одной эксперименте)

    \DefN{6} События $A$ влечет события $B$, если $A \subset B$ (если наступает $A$, то наступит $B$)

    \subsection{Вероятность}

    Мы хотим присвоить какую-то числовую характеристику к каждому событию,
    отражающее его частоту наступления: $0 \leq P(A) \leq 1$ - вероятность наступления события $A$

    \mediumvspace

    \textbf{Классическое определение вероятности}

    Пусть пространство случайных событий $\Omega$ содержит конечное число равновозможных исходов,
    тогда применимо классическое определение вероятности

    \Def \fbox{$P(A) = \frac{|A|}{|\Omega|} = \frac{m}{n}$}, где $n$ - число всех возможных исходов, $m$ - число благоприятных исходов

    В частности, если $\Omega = n$ и $A_i$ - элем. исх., то $P(A_i) = \frac{1}{n}$

    Свойства:

    1) $0 \leq P(A) \leq 1$

    2) $P(\Omega) = 1 \quad (m = n)$

    3) $P(\emptyset) = 0 \quad (m = 0)$

    4) Если события $A$ и $B$ несовместны, то $P(A + B) = P(A) + P(B)$

    \mediumvspace

    \textbf{Геометрическое определение вероятности (граф де Бюффон)}

    Пусть $\Omega \subset \Real^n$ - замкнутая ограниченная область

    $\mu(\Omega)$ - мера $\Omega$ в $\Real^n$ (например, длина отрезка, площадь области на плоскости, объем тела в пространстве)

    В эту область наугад бросаем точку. \enquote{Наугад} означает, что вероятность попадания в $A$ зависит только от меры $A$ и не зависит от ее расположения

    В этом случае применимо геометрическое определение вероятности

    $P(A) = \frac{\mu(A)}{\mu(\Omega)}$

    \ExN{1} Монета диаметром в 6 см бросается на пол, вымощенной квадратной плиткой со стороной 20 см, какова вероятность,
    что монета окажется целиком внутри одной плитки

    $\mu(\Omega) = 20^2 = 400$

    $\mu(A) = (20 - 3 - 3)^2 = 196$

    $P(A) = \frac{\mu(A)}{\mu(\Omega)} = \frac{196}{400} = 0.49$

    \hypertarget{buffonsproblem}{}

    \ExN{2} Задача Бюффона об игле: пусть пол вымощен ламинатом, $2l$ - ширина доски, на пол бросается игла длины, равной ширине доски,
    найти вероятность того, что игла пересечет стык доски

    \smallvspace

    \begin{minipage}{\linewidth}
        \begin{wrapfigure}{r}{0pt}
            \includegraphics[height=6cm]{probtheory/images/probtheory_2024_09_03_1}
        \end{wrapfigure}

        Определим положение иглы координатами центра и углом, между иглой и стыком доски, причем можно считать, что эти величины независимы

        $\letsymbol x \in [0; l]$ - расстояние от центра до ближайшего края, $\varphi \in [0; \pi]$ - угол

        $\Omega = [0; l] \times [0; \pi]$

        Событие $A$ (пересечет стык) наступает, если $x \leq l \sin \varphi$

        $P(A) = \frac{S(A)}{S(\Omega)}$

        $S(\Omega) = \pi l$

        $S(A) = \int_0^\pi l \sin \varphi d\varphi = -l \cos \varphi \Big|_0^\pi = -l (-1 - 1) = 2l$

        $P(A) = \frac{2l}{\pi l} = \frac{2}{\pi}$
    \end{minipage}
% end probtheory_2024_09_03.tex

% begin probtheory_2024_09_10.tex

    \section{Лекция 2}
    
    \subsection{Построение модели случайных явлений}

    \begin{enumerate}
        \item Задаем пространство элементарных исходов $\Omega$

        \item \Defs Система $\mathcal{F}$ подмножеств $\Omega$ называется $\sigma$-алгеброй событий, если:

        1) $\Omega \in \mathcal{F}$;

        2) $A \in \mathcal{F} \Longrightarrow \overline{A} \in \mathcal{F}$;

        3) $A_1, A_2, \dots, A_n, \dots \in \mathcal{F} \Longrightarrow \bigunion_{i = 1}^\infty A_i \in \mathcal{F}$

        \textbf{Свойства}:

        \begin{enumerate}
            \item $\emptyset \in \mathcal{F}$, так как $\Omega \in \mathcal{F} \Longrightarrow \overline{\Omega} = \emptyset \in \mathcal{F}$

            \item $A_1, A_2, \dots \in \mathcal{F} \Longrightarrow \bigcap_{i = 1}^\infty A_i \in \mathcal{F}$

            \begin{tcolorbox}
                $\Box \quad$ $A_1, A_2, \dots \in \mathcal{F} \Longrightarrow
                \overline{A}_1, \overline{A}_2, \dots \in \mathcal{F} \Longrightarrow
                \bigunion_{i = 1}^\infty \overline{A}_i \in \mathcal{F} \Longrightarrow
                \overline{\bigunion_{i = 1}^\infty \overline{A_i}} = \bigcap_{i = 1}^\infty A_i \in \mathcal{F} \quad \Box$
            \end{tcolorbox}

            \item $A, B \in \mathcal{F} \Longrightarrow A \setminus B \in \mathcal{F}$

            \begin{tcolorbox}
                $\Box \quad$ $A, B \in \mathcal{F} \Longrightarrow A, \overline{B} \in \mathcal{F} \Longrightarrow A \setminus B = A \cdot \overline{B} \in \mathcal{F}$ $\quad \Box$
            \end{tcolorbox}

        \end{enumerate}

        \ExNs{1} $\mathcal{F} = \{\emptyset, \Omega\}$

        \ExNs{2} $\mathcal{F} = \{\emptyset, \Omega, A, \overline{A}\}$

        \ExNs{3} \Defs Борелевская $\sigma$-алгебра $\mathcal{B}(\Real)$ - минимальная $\sigma$-алгебра, содержащая все возможные интервалы на прямой

        % на плоскости борелевская сигма-алгебра - прямоугольники

        \item \Defs $\letsymbol\ \Omega$ - пространство элементарных исходов, $\mathcal{F}$ - его $\sigma$-алгебра событий.
        \textit{Вероятностью} на $(\Omega, \mathcal{F})$ называется функция $P: \mathcal{F} \to \Real$ со свойствами:

        \begin{enumerate}
            \item $P(A) \geq 0 \quad \forall A \in \mathcal{F}$ (неотрицательность)

            \item Если $A_1, A_2, \dots, A_n, \dots \in \mathcal{F}$ - несовместное, то $P(\sum_{i = 1}^\infty A_i) = \sum_{i = 1}^\infty P(A_i)$ (свойство счетной аддитивности)

            \item $P(\Omega) = 1$ (условие нормированности)
        \end{enumerate}

    \end{enumerate}

    \Def Из этого тройка $(\Omega, \mathcal{F}, P)$ называется вероятностным пространством

    \subsection{Свойства вероятности}

    \begin{enumerate}

        \item Так как $\emptyset$ и $\Omega$ - несовместные, то $1 = P(\Omega) = P(\Omega + \emptyset) = 1 + P(\emptyset) \Longrightarrow P(\emptyset) = 0$

        \item Формула обратной вероятности: $P(A) = 1 - P(\overline{A})$

        \begin{tcolorbox}
            $\Box \quad$ $A$ и $\overline{A}$ - несовместные и $A + \overline{A} = \Omega \Longrightarrow P(A + \overline{A}) = P(\Omega) = 1$ $\quad \Box$
        \end{tcolorbox}

        \item $P(A) = 1 - P(\overline{A}) \leq 1$

    \end{enumerate}

    \subsection{Аксиома непрерывности}

    Пусть имеется убывающая цепочка событий $A_1 \supset A_2 \supset A_3 \supset \dots \supset A_n \supset \dots$ и $\bigcap_{i = 1}^\infty A_n = \emptyset$

    Тогда $P(A_n) \underset{n \to \infty}{\to} 0$

    При непрерывном изменении области $A \subset \Omega \subset \Real^n$ соответствующая вероятность $P(A)$ также должна изменятся непрерывно

    \Th Аксиома непрерывности следует из аксиомы счетной аддитивности

    \begin{tcolorbox}
        $\Box$

        Ясно, что $A_n = \sum_{i = n}^\infty A_i \overline{A}_{i + 1} + \prod_{i = n}^\infty A_i$

        $\prod_{i = n}^\infty A_i = A_n \cdot \prod_{i = n + 1}^\infty A_i = \prod_{i = 1}^n
        \cdot \prod_{i = n + 1}^\infty A_i = \prod_{i = 1}^\infty = \emptyset \Longrightarrow
        A_n = \sum_{i = n}^\infty A_n \overline{A_{n + 1}}$ и так как эти события несовместны,
        то по свойству счетной аддитивности $P(A_n) = \sum_{i = n}^\infty P(A_i \overline{A_{i + 1}})$ - это остаток (хвост) сходящегося ряда

        $P(A_1) = \sum_{i = 1}^\infty P(A_i \overline{A_{i + 1}}) = \sum_{i = 1}^{n - 1} P(A_i \overline{A_{i + 1}}) + P(A_n)$ и $P(A_n) \underset{n \to \infty}{\to} 0$ по необходимому признаку сходимости

        $\Box$
    \end{tcolorbox}

    \Nota Аксиому счетной аддитивности можно вывести из конечной аддитивности и аксиомы счетной непрерывности

    \textbf{Свойства операций сложения и умножения}

    1. Свойство дистрибутивности: $A \cdot (B + C) = AB + AC$

    2. Формула сложения: если $A$ и $B$ несовместны, то $P(A + B) = P(A) + P(B)$

    3. Формула сложения вероятностей: $P(A + B) = P(A) + P(B) - P(AB)$

    \begin{tcolorbox}
        $\Box$

        $A + B = A\overline{B} + AB + \overline{A}B$ - несовместные события $\Longrightarrow P(A + B) = P(A\overline{B}) + P(AB) + P(\overline{A}B) =
        (P(A\overline{B}) + P(\overline{A}B)) - P(AB) = P(A) + P(B) - P(AB)$

        $\Box$
    \end{tcolorbox}

    \Ex Из колоды в 36 карт достали одну карту. Какова вероятность того, что будет дама или пика

    Пусть Д - дама, П - пика, $P(\text{Д} + \text{П}) = P(\text{Д}) + P(\text{П}) - P(\text{Д}\text{П}) = \frac{4}{36} + \frac{9}{36} - \frac{1}{36} = \frac{1}{3}$

    Формула сложения при $N = 3$: $P(A_1 + A_2 + A_3) = P(A_1) + P(A_2) + P(A_3) - P(A_2 A_3) - P(A_1 A_3) - P(A_1 A_2) + P(A_1 A_2 A_3)$

    Общий случай: $P(A_1 + A_2 + \dots + A_n) =  \sum_{i = 1}^n P(A_i) - \sum_{i < j} P(A_i A_j) + \sum_{i < j < k} P(A_i A_j A_k) + (-1)^{n - 1} \cdot P(A_1 A_2 \dots A_n)$ - формула включения и исключения

    \Ex $n$ писем случайно раскладывается по $n$ конвертам. Найти вероятность того, что хотя бы одно письмо окажется в своем конверте

    $\letsymbol A_i$ - $i$-ое письмо в своем конверте

    $P(A_i) = \frac{1}{n}; P(A_i A_j) = \frac{1}{A^2_n}; P(A_i A_j A_k) = \frac{1}{A^3_n}; P(A_1 A_2 \dots A_n) = \frac{1}{n!}$

    Слагаемых вида $A_i$ - $n$ штук; $A_i A_j$ - $C^2_n$; $A_i A_j A_k$ - $C^3_n$; $A_1 A_2 \dots A_n$ - 1 штука

    $P(A) = P(A_1 + A_2 + \dots + A_n) = n \cdot \frac{1}{n} - C^2_n \frac{1}{A^2_n} + C^3_n \frac{1}{A^3_n} - \dots + (-1)^{n - 1} \frac{1}{n!} = 1 - \frac{1}{2} + \frac{1}{3!} - \dots + (-1)^{n - 1} \frac{1}{n!}$

    Так как $e^{-1} = 1 - 1 + \frac{1}{2} - \frac{1}{3!} + \dots$, то при $n \to \infty \quad P(A) \underset{n \to \infty}{\to} 1 - e^{-1} \approx 0.63$

    Независимые события

    Под независимыми событиями логично подразумевать события, не связанные причинно-следственной связью (то есть когда факт наступления одного не влияет на оценку вероятности другого)

    $\letsymbol |\Omega| = n; |A| = m_1; |B| = m_2$

    Проведем пару независимых испытаний. Тогда получаем пространство элементарных исходов $\Omega \times \Omega$ и $|\Omega \times \Omega| = n^2$

    По основному принципу комбинаторики $|A \cdot B| = m_1 \cdot m_2$

    $P(AB) = \frac{|A \cdot B|}{|\Omega \times \Omega|} = \frac{m_1 m_2}{n^2} = P(A) \cdot P(B)$

    \Def События $A$ и $B$ называются независимыми, если $P(A \cdot B) = P(A) \cdot P(B)$

    \Lab $\letsymbol P(A), P(B) \neq 0$, доказать, что если $A$ и $B$ несовместны, то они зависимы

    Свойство: Если $A$ и $B$ независимы, то независимы $\overline{A}$ и $\overline{B}$, $A$ и $\overline{B}$, $\overline{A}$ и $B$

    Доказательство: $A = A \cdot (B + \overline{B}) = AB + A\overline{B}$ - несовместные события $\Longrightarrow P(A) = P(AB) + P(A\overline{B}) \Longrightarrow P(A\overline{B}) = P(A) - P(AB) =
    P(A) - P(A) \cdot P(B) = P(A) (1 - P(B)) = P(A) P(\overline{B}) \Longrightarrow$ независимы

    \Def События $A_1, A_2, \dots A_n$ - независимы в совокупности, если для любого набора $i_1, i_2, \dots, i_k \ (2 \leq k \leq n)$
    $P(A_{i_1} \cdot A_{i_2} \cdot \dots \cdot A_{i_k}) = P(A_{i_1}) \cdot P(A_{i_2}) \cdot \dots \cdot P(A_{i_k})$

    \Nota Из независимости в совокупности при $k = 2$ получаем попарную независимость. Обратное утверждение неверно

    \Ex (С. Бернштейн)

    Пусть имеется правильный тетраэдр, одна грань окрашена в красный, вторая в синий, третья в зеленый, а четвертая во все эти три цвета.

    Подбросили тетраэдр, $\letsymbol A$ - грань, которая содержит красный цвет, $B$ - синий, $C$ - зеленый.

    $P(A) = P(B) = P(C) = \frac{2}{4} = \frac{1}{2}$

    Так как $P(AB) = P(AC) = P(BC) = \frac{1}{4}$

    $P(AB) = \frac{1}{4} = \frac{1}{2} \cdot \frac{1}{2} = P(A) P(B)$ - попарная независимость

    $P(ABC) = \frac{1}{4} \neq P(A) P(B) P(C)$ - но вот независимость в совокупности не соблюдается

    \Ex (Шевалье де Мере, Паскаль, Ферма, $\approx$ 1650 г.)

    Какова вероятность того, что при 4 бросании кости выпадет одна шестерка

    $A_1$ - при первом броске шестерка, $A_2$ - при втором, $A_3$ - при третьем, $A_4$ - при четвертом

    $B$ - выпала хотя бы одна шестерка при 4 бросках

    $B = A_1 + A_2 + A_3 + A_4$ - совместные события, но независимые

    Найдем обратную вероятность: $\overline{B}$ - ни разу не выпала шестерка

    $\overline{B} = \overline{A_1} \cdot \overline{A_2} \cdot \overline{A_3} \cdot \overline{A_4}$

    $P(\overline{A_1}) = P(\overline{A_2}) = P(\overline{A_3}) = P(\overline{A_4}) = \frac{5}{6}$

    $\overline{B} = P(\overline{A_1}) P(\overline{A_2}) P(\overline{A_3}) P(\overline{A_4}) = \left(\frac{5}{6}\right)^4 \approx 0.482$

    $P(B) = 1 - P(\overline{B}) \approx 0.52$
% end probtheory_2024_09_10.tex

% begin probtheory_2024_09_17.tex

    \section{Лекция 3}

    \subsection{Условная вероятность}

    Условная вероятность $P(A|B)$ (или $P_B(A)$) - вероятность события $A$, вычисленная в предположении, что событие $B$ уже произошло

    \Ex Бросается кость один раз, известно, что выпало больше 3 очков. Найти вероятность того, что выпало четное число очков

    \smallvspace

    \begin{minipage}{\linewidth}
        \begin{wrapfigure}{r}{0pt}
            \begin{center}
                \includegraphics[height=5cm]{probtheory/images/probtheory_2024_09_17_1}
            \end{center}
        \end{wrapfigure}

        $A$ - выпало четное число очков

        $B$ - выпало больше трех очков

        $\Omega = \{1, 2, 3, 4, 5, 6\}; |\Omega| = 6; A = \{2, 4, 6\}; B = \{4, 5, 6\}$

        $P(A|B) = \frac{2}{3} = \frac{\frac{2}{6}}{\frac{3}{6}} = \frac{P(AB)}{P(B)}$


        Интерпретация с помощью геометрической вероятности:

        $P(A|B) = \frac{S_{AB}}{S_B} = \frac{\frac{S_{AB}}{S_\Omega}}{\frac{S_B}{S_\Omega}}$
    \end{minipage}

    \Def Условной вероятностью события $A$ при условии, что имело место событие $B$, называется величина $P(A|B) = \frac{P(AB)}{P(B)}$

    \Ex Известно, что среди населения 1\% воров. В комнате, где находилось 10 гостей, у хозяина пропал кошелек. Какова вероятность того, что произвольный гость является вором.

    $A$ - гость является вором $P(A) = 0.01$

    $B$ - пропал кошелек (хотя бы один вор среди гостей есть)

    $P(A|B) = \frac{P(AB)}{P(B)} = \frac{P(AB)}{1 - P(\overline{B})} = \frac{P(A)}{1 - 0.99^{10}} = \frac{0.01}{1 - 0.99^{10}} = 0.105$

    Формула умножения:

    В качестве следствия условной вероятности получаем:

    $P(A|B) = \frac{P(AB)}{P(B)} \Longrightarrow P(AB) = P(B) \cdot P(A|B) = P(A) \cdot P(B|A)$

    Общий случай:

    $P(A_1 A_2 A_3 \dots A_n) = P(A_1) P(A_2 | A_1) P(P_3 | A_1 A_2) \dots P(A_n | A_1 A_2 \dots A_{n - 1})$

    \begin{tcolorbox}
        $\Box$

        База индукции $P(AB) = P(B) P(A|B)$

        Шаг индукции: пусть верно при $n - 1$:

        $P(A_1 A_2 A_3 \dots A_{n - 1}) = P(A_1) P(A_2 | A_1) P(P_3 | A_1 A_2) \dots P(A_n | A_1 A_2 \dots A_{n - 2})$

        $P(A_1 A_2 A_3 \dots A_n) = P(A_1 A_2 A_3 \dots A_{n - 1}) \cdot P(A_n | A_1 A_2 \dots A_{n - 1}) = \\
        P(A_1) P(A_2 | A_1) P(P_3 | A_1 A_2) \dots P(A_n | A_1 A_2 \dots A_{n - 1})$

        $\Box$
    \end{tcolorbox}


    \Ex Студент выучил 1 билет из $n$, в группе $n$ студентов. Каким по очереди ему нужно зайти, чтобы вероятность сдать экзамен была наибольшей

    Пусть $A_i$ - билет, вытянутый на $i$-ом шаге ($1 \leq i \leq n$)

    $A$ - студент сдал экзамен

    $P(A) = P(\overline{A_1} \cdot \overline{A_2} \cdot \dots \cdot \overline{A_{i - 1}} \cdot A_i) = \frac{n - 1}{n} \cdot \frac{n - 2}{n - 1} \cdot \dots \cdot \frac{n - (i - 1)}{n - (i - 2)} \cdot \frac{1}{n - (i - 1)} = \frac{1}{n}$

    \subsection{Полная группа событий}

    \Def События $H_1, H_2, \dots, H_n, \dots$ образуют полную группу событий, если они попарно несовместны и содержат все возможные элементарные исходы

    $H_i \cap H_j = \emptyset \ \forall i, j \quad\quad\quad \bigunion_{i = 1}^{\infty} H_i = \Omega$

    Следствие: $\sum_{i = 1}^\infty P(H_i) = 1$

    \ThN{Формула полной вероятности} $\letsymbol H_1, H_2, \dots, H_n, \dots$ - полная группа событий. Тогда $P(A) = \sum_{i = 1}^\infty P(H_i) P(A | H_i)$

    \begin{tcolorbox}
        $\Box$

        $P(A) = P(\Omega A) = P((H_1 + H_2 + H_3 + \dots) A) = P(H_1 A + H_2 A + H_3 A + \dots) = [H_i \cdot A \cdot H_j \cdot A = \emptyset \cdot A] = P(H_1 A) + P(H_2 A) + \dots =
        P(H_1) P(A | H_1) + P(H_2) P(A | H_2) + \dots$

        $\Box$
    \end{tcolorbox}

    \ThN{Формула Байеса} $\letsymbol H_1, H_2, \dots, H_n$ - полная группа событий, и известно, что событие $A$ уже произошло

    Тогда $P(H_k | A) = \frac{P(H_k) P(A | H_k)}{\sum_{i = 1}^\infty P(H_i) P(A | H_i)}$

    \begin{tcolorbox}
        $\Box$

        $P(H_k | A) = \frac{P(H_k A)}{P(A)} = \frac{P(H_k) P(A|H_k)}{\sum_{i = 1}^\infty P(H_i) P(A|H_i)}$

        $\Box$
    \end{tcolorbox}

    \ExN{1} В первой коробке 4 белых и 2 черных шара, во второй 1 белый и 2 черных. Из первой коробки во вторую переложили 2 шара, затем из второй коробки достали шар. Какова
    вероятность того, что он оказался белым

    $\letsymbol H_1$ - переложили 2 белых
    $H_2$ - 2 черных

    $H_3$ - разного цвета

    $A$ - из второй коробки достали белый шар

    $P(H_1) = \frac{4}{6} \cdot \frac{3}{5} = \frac{6}{15}$

    $P(H_2) = \frac{2}{6} \cdot \frac{1}{5} = \frac{1}{15}$

    $P(H_3) = \frac{4}{6} \cdot \frac{2}{5} + \frac{2}{6} \cdot \frac{4}{5} = \frac{4}{15} + \frac{4}{15} = \frac{8}{15}$

    $P(A) = P(H_1) \cdot P(A|H_1) + P(H_2) \cdot P(A|H_2) + P(H_3) \cdot P(A|H_3) = \frac{6}{15} \cdot \frac{3}{5} + \frac{1}{15} \cdot \frac{1}{5} + \frac{8}{15} \cdot \frac{2}{5} =
    \frac{18}{75} + \frac{1}{75} + \frac{16}{75} = \frac{35}{75} = \frac{7}{15}$

    \ExN{2} Вероятность попадания первого стрелка в цель $0.9$, а второго $0.3$. Наугад вызванный стрелок попал в цель. Какова вероятность того, что это бы первый стрелок?

    $H_1$ - вызван первый стрелок

    $H_2$ - вызван второй стрелок

    $A$ - стрелок попал

    $P(H_1) = P(H_2) = \frac{1}{2}$

    $P(A|H_1) = 0.9 \quad\quad P(A|H_2) = 0.3$

    $P(H_1 | A) = \frac{P(H_1) P(A|H_1)}{P(H_1) P(A|H_1) + P(H_2) | P(A | H_2)} = \frac{\frac{1}{2} 0.9}{\frac{1}{2} 0.9 + \frac{1}{2} 0.3} = \frac{9}{9 + 3} = 0.75$

    \ExN{3} По статистике раком болеет 1\% населения. Тест дает правильный результат в 99\% случаев. Тест оказался положительный. Найти вероятность того, что человек болен.

    $H_1$ - человек болен

    $H_2$ - человек здоров

    $A$ - анализ положительный

    $P(H_1) = 0.01$

    $P(H_2) = 0.99$

    $P(A|H_1) = 0.99$

    $P(A|H_2) = 0.01$

    $P(H_1 | A) = \frac{P(H_1)P(A | H_1)}{P(H_1) P(A | H_1) + P(H_2) P(A | H_2)} = \frac{0.01 \cdot 0.99}{0.01 \cdot 0.99 + 0.99 \cdot 0.01} = \frac{1}{2} = 0.5$

    Допустим, что второй независимый с первым анализ также оказался положительным. Найти вероятность того, что человек болен.

    $P(H_1) = 0.01 \quad\quad P(H_2) = 0.99$

    $P(AA|H_1) = 0.99^2 \quad\quad P(AA|H_2) = 0.01^2$

    $P(H_1 | AA) = \frac{0.01 \cdot 0.99^2}{0.01 \cdot 0.99^2 + 0.99 \cdot 0.01^2} = \frac{0.99}{0.99 + 0.01} = 0.99$

    Интуитивно вероятность $\frac{1}{2}$ может поддаваться непониманию, однако можно рассуждать так:
    пусть в городе живут 10000 человек, из них 100 болеют, а у 99 из них положительный анализ; у других 9900 положительный анализ всего лишь у 99, отсюда выходит $\frac{1}{2}$

    \ExN{4} В телевизионной студии 3 двери {\Large 🚪🚪🚪}, за одной из них приз {\Large 🚗}.
    Игрок выбрал наугад одну из 3 дверей, после чего ведущий открывает одну из двух оставшихся дверей и показывает, что там приза нет {\Large 🛴}. После чего
    предлагает игроку поменять свой выбор. Стоит ли игроку соглашаться?

    $H_1$ - игрок угадал

    $H_2$ - игрок не угадал

    $A$ - ведущий открыл дверь без приза

    $P(H_1) = \frac{1}{3} \quad\quad P(H_2) = \frac{2}{3}$

    $P(A|H_1) = 1 \quad\quad P(A|H_2) = \frac{1}{2}$

    $P(H_1|A) = \frac{\frac{1}{3} \cdot 1}{\frac{1}{3} \cdot 1 + \frac{1}{3} \cdot \frac{1}{2}} = \frac{1}{2}$

    Но это неправильно, так как действия ведущего неслучайны - он всегда откроет дверь без приза

    В этом случае, если мы гипотетически выберем 300 дверей, в 100 случаях мы отгадаем, ведущий откроет любую дверь без приза;
    но в 200 случаях мы не отгадаем, ведущий откроет вторую дверь без приза, и в этом случае мы сможем поменяться на дверь с призом,
    отсюда шанс $\frac{2}{3}$, если мы поменяем свой выбор \hfill


    \ExN{5} Вероятность того, что в семье с детьми ровно $k$ детей, равна $\frac{1}{2^k}$, $k = 1, 2, \dots$
    Какова вероятность того, что в семье один мальчик, если известно, что нет
    девочки? Рождения мальчиков и девочек равновероятны.


    $H_i$ - в семье $i$ детей ($1 \leq i < \infty$)

    $P(H_i) = \frac{1}{2^i}$

    $A$ - в семье нет девочки

    $P(A|H_1) = \frac{1}{2}$

    $P(A|H_2) = \frac{1}{4}$

    $P(A|H_i) = \frac{1}{2^i}$

    $P(H_1 | A) = \frac{\frac{1}{2} \frac{1}{2}}{\sum_{i = 1}^\infty \frac{1}{2^i} \cdot \frac{1}{2^i}} = \frac{\frac{1}{4}}{\frac{\frac{1}{4}}{1 - \frac{1}{4}}} = \frac{3}{4} = 0.75$
% end probtheory_2024_09_17.tex

% begin probtheory_2024_09_24.tex

    \section{Лекция 4}

    \subsection{Серия испытаний Бернулли}

    Схемой Бернулли - называется серия одинаковых независимых экспериментов, каждый из которых имеет 2 исхода: произошло интересующее нас событие или нет

    $p = p(A)$ - вероятность успеха при одном испытании

    $q = 1 - p$ - вероятность неудачи

    $v_n$ - число успехов в серии из $n$ испытаний

    $p(v_n = k) = p_n(k)$

    Из этого получаем формулу Бернулли:

    \begin{MyTheorem}
        \Ths Вероятность того, что при $n$ испытаниях произойдет ровно $k$ успехов, равна

        $p_n(k) = C_n^k p^k q^{n - k}$
    \end{MyTheorem}

    \begin{MyProof}
        $\Box$

        Рассмотрим один из элементарных исходов, благоприятных данному событию:

        $A_n = \underset{k}{\underbrace{\text{УУУ}\dots\text{У}}}\underset{n - k}{\underbrace{\text{Н}\dots\text{ННН}}}$ - $k$ успехов, $n - k$ неудачи

        $p(\text{У}) = p, p(\text{Н}) = q$

        Так как испытания независимы, то $p(A_n) = p^k q^{n - k}$

        Остальные элементарные исходы имеют ту же вероятность, перебираем все расстановки исходов, получаем $C_n^k$, в итоге, получаем формулу Бернулли

        $\Box$
    \end{MyProof}

    \Ex Вероятность попадания стрелка при одном выстреле - $0.8$. Какова вероятность того, что из пяти выстрелов точными будут три

    $n = 5 \quad p = 0.8 \quad q = 1 - p = 0.2 \quad k = 3$

    $p_5(3) = C^3_5 p^3 q^2 = 0.2048$

    \subsection{Наиболее вероятное число успехов}

    Выясним, при каком значении $k$ вероятность предшествующего числа успехов $k - 1$ будет не более, чем вероятность $k$ успехов

    $p_n(k - 1) \leq p_n(k)$

    $C_n^{k - 1} p^{k - 1} q^{n - k + 1} \leq C^k_n p^k q^{n - k}$

    $\frac{n!}{(k - 1)! (n - k + 1)!} q \leq \frac{n!}{(k)! (n - k)!} p$

    $\frac{q}{(k - 1)! (n - k + 1)!} \leq \frac{p}{(k)! (n - k)!}$

    $\frac{q}{n - k + 1} \leq \frac{p}{k}$

    $k(1 - p) \leq p(n - k + 1)$

    $k \leq np + p$

    Отсюда $np + p - 1 \leq k \leq  np + p$

    Рассмотрим 3 ситуации:

    1) $np$ - целое, тогда $np + p$ - нецелое, и $k = np$ - наиболее вероятное

    2) $np + p$ - нецелое, тогда $k = \lfloor np + p \rfloor$

    3) $np + p$ - целое, тогда $np + p - 1$ - целое, и 2 наиболее вероятных числа успеха

    Геометрическая интерпретация:

    \begin{center}
        \includegraphics[width=6cm]{probtheory/images/probtheory_2024_09_24_1}
    \end{center}

    При увеличении числа $n$ точки превращаются в кривую Гаусса

    \begin{center}
        \includegraphics[width=6cm]{probtheory/images/probtheory_2024_09_24_2}
    \end{center}

    При увеличении числа испытаний $n$ формула Бернулли вырождается в следующие асимптотические формы (применяем, если требуется найти вероятность точного числа успеха)

    1) локальная формула Муавра-Лапласа

    $p_n(k) \underset{n \to \infty}{\longrightarrow} \frac{1}{\sqrt{npq}} \varphi(x)$, где $\varphi(x) = \frac{1}{\sqrt{2\pi}} e^{-\frac{x^2}{2}}$ - функция гаусса

    $x = \frac{k - np}{\sqrt{npq}}$

    Свойства $\varphi(x)$:

    1. $\varphi(x) = \varphi(-x)$ - функция четная

    2. при $x > 5 \quad \varphi(x) \approx 0$

    2) Интегральная формула Муавра-Лапласа (если требуется найти вероятность того, что число успехов в данном диапазоне)

    $p_n(k_1 \leq k \leq k_2) \underset{n \to \infty}{\longrightarrow} \Phi(x_2) - \Phi(x_1)$, где $\Phi(x) = \frac{1}{\sqrt{2\pi}} \int_0^x e^{-\frac{z^2}{2}} dz$ - функция Лапласа

    $x_1 = \frac{k_1 - np}{\sqrt{npq}}$ - отклонение от левой границы, $x_2 = \frac{k_2 - np}{\sqrt{npq}}$ - отклонение от правой

    Свойства $\Phi(x)$

    1. $\Phi(-x) = -\Phi(x)$ - функция нечетная

    2. при $x > 5 \quad \Phi(x) \approx 0.5$

    \Nota Эти формулы обычно можно применять при $n \geq 100$ и $0.1 \leq p \leq 0.9$

    \Nota В некоторых источниках под функцией Лапласа подразумевают другую функцию: $F_0(x) = \frac{1}{\sqrt{2\pi}} \int_{-\infty}^x e^{-\frac{t^2}{2}}dt$ - стандартное отклонение. Эта функция
    отличается от $F_0 = \frac{1}{\sqrt{2\pi}} \int_{-\infty}^0 e^{-\frac{t^2}{2}}dt + \Phi(x) = \frac{1}{2} + \Phi(x)$

    Так как $\int_{-\infty}^{\infty} e^{-x^2} dx = \sqrt{\pi}$ - интеграл Пуасона

    \Ex Вероятность попадания стрелка в цель $0.8$, стрелок сделал 400 выстрелов. Найти вероятность того, что:

    а) произошло ровно 330 попаданий

    б) произошло от 312 до 336 попаданий

    а) $x = \frac{k - np}{\sqrt{npq}} = \frac{330 - 400 \cdot 0.8}{\sqrt{400 \cdot 0.8 \cdot 0.2}} = \frac{330 - 320}{8} = 1.25$

    $p_{400}(330) \approx \frac{1}{\sqrt{npq}} \varphi(1.25) = \frac{1}{8} \varphi(1.25) \approx \frac{1}{8} \cdot 0.1826 \approx 0.0228$

    б) $x_1 = \frac{312 - 320}{8} = -1$, $x_2 = \frac{336 - 320}{8} = 2$

    $p_{400}(312 \leq k \leq 336) \approx \Phi(2) - \Phi(-1) = \Phi(2) + \Phi(1) \approx 0.4772 + 0.3413 = 0.8185$

    \subsection{Статистическое понятие вероятности}

    Пусть проводим $n$ реальных экспериментов, $n_A$ - число появления события $A$, $\frac{n_A}{n}$ - относительная частота события $A$.

    Эксперименты с монетой показали, что при больших $n$, $\frac{n_A}{n} \approx p(A)$ - явление стабилизации

    Вероятность отклонения относительной частоты от вероятности события

    $n$ - число испытаний, $p = p(A), \frac{n_A}{n}$ - экспериментальная частота

    $p\left(|\frac{n_A}{n} - p| \leq \varepsilon\right) = p\left(-\varepsilon \leq \frac{n_A}{n} - p \leq \varepsilon\right) = p(-n\varepsilon \leq n_A - np \leq n\varepsilon) = p(np - n\varepsilon \leq n_A \leq n\varepsilon + np) \underset{n \to \infty}{\longrightarrow}$ [по интегральной формуле Лапласа] $\underset{n \to \infty}{\longrightarrow} \Phi\left(\frac{n\varepsilon}{\sqrt{npq}}\right) - \Phi\left(-\frac{n\varepsilon}{\sqrt{npq}}\right)$

    $ = \Phi\left(\frac{\sqrt{n}\varepsilon}{\sqrt{pq}}\right) - \Phi\left(-\frac{\sqrt{n}\varepsilon}{\sqrt{pq}}\right)$

    $ = 2\Phi\left(\frac{\sqrt{n}\varepsilon}{\sqrt{pq}}\right)$

    Итак, получили, что нужная нам вероятность $p\left(|\frac{n_A}{n} - p| \leq \varepsilon\right) \approx 2\Phi\left(\frac{\sqrt{n}\varepsilon}{\sqrt{pq}}\right)$

    \subsection{Закон больших чисел Бернулли}

    Итак, $p\left(|\frac{n_A}{n} - p| \leq \varepsilon\right) \underset{n \to \infty}{\longrightarrow} 2 \Phi\left(\frac{\varepsilon}{\sqrt{pq}}\sqrt{n}\right)$

    при $n \to \infty, \sqrt{n} \to \infty, \frac{\varepsilon}{\sqrt{pq}} \sqrt{n} \to \infty, \Phi\left(\frac{\varepsilon}{\sqrt{pq}}\sqrt{n}\right) \to 0.5, p\left(|\frac{n_A}{n} - p| \leq \varepsilon\right) \to 2 \cdot 0.5 = 1$ - закон больших чисел показывает, что вероятность попадания относительной частоты в $\varepsilon$-трубу вероятность события приближается к 1

    $\lim_{n \to \infty} p\left(|\frac{n_A}{n} - p| \leq \varepsilon\right) = 1$ или $\frac{n_A}{n} \underset{n \to \infty}{\longrightarrow} p$ - сходимость по вероятности

    \Ex Для оценки доли $p$ курящих людей берется выборка объема $n$, и делается оценка доли курящих людей по формуле $p^* = \frac{n_A}{n}$.
    Каким должен быть объем $n$, чтобы с вероятностью $\gamma = 0.95$ данная оценка отличалась от истинного значения не более, чем на $\varepsilon = 0.01$

    По формуле вероятности отклонения частоты от вероятности $p(|p^* - p| \leq \varepsilon) = p\left(|\frac{n_A}{n} - p| \leq \varepsilon\right) \approx 2\Phi\left(\frac{\varepsilon}{\sqrt{pq}}\sqrt{n}\right) = 0.95$

    $\Phi\left(\frac{\varepsilon}{\sqrt{pq}}\sqrt{n}\right) = 0.475$

    $\frac{\varepsilon}{\sqrt{pq}}\sqrt{n} = 1.96$

    $\frac{1}{\sqrt{pq}}\sqrt{n} = 196$

    $\frac{n}{pq} = 38416$

    $n \geq 38416 pq$

    В самое худшей ситуации $pq \leq 0.5^2 = \frac{1}{4}$

    $n \geq \frac{38416}{4} = 9604$
% end probtheory_2024_09_24.tex

% begin probtheory_2024_10_01.tex

    \section{Лекция 5}

    \subsection{Схема испытаний и соответствующее распределение}

    Введем обозначения:

    $n$ - число испытаний

    $p$ - вероятность успеха при одном испытании

    $q = 1 - p$ - вероятность неудачи

    \subsubsection{I. Схема Бернулли}

    $\letsymbol v_n$ - число успехов в серии из $n$ испытаний

    $P_n(v_n = k) = C^k_n p^k q^{n - k}, \quad\quad k = 0, 1, \dots, n$

    \Def Соответствие $k \rightarrow C^k_n p^k q^{n - k}, \quad k = 0, \dots, n$ называется биномиальным распределением
    (обозначается $B_{n,p}$ или $B(n, p)$)

    \subsubsection{II. Схема до первого успешного испытания}

    Пусть проводится бесконечная серия испытаний, которая заканчивается после первого успешного испытания
    под номером $\tau$

    \begin{MyTheorem}
        \Ths $P(\tau = k) = q^{k - 1} p, \quad\quad k = 1, 2, \dots$
    \end{MyTheorem}

    \begin{MyProof}
        $\Box$

        $P(\tau = k) = P(\underset{k - 1}{\underbrace{\text{Н}\dots\text{Н}}}\text{У}) = q^{k - 1}p$

        $\Box$
    \end{MyProof}

    \Def Соответствие $k \rightarrow q^{k - 1} p, k \in \Natural$ называется геометрическим
    распределение вероятности (обозначается $G_p$ или $G(p)$)

    \Nota Геометрическое распределение обладает свойством \enqoute{нестарения} или свойством отсутствия
    последействия

    \begin{MyTheorem}
        \Ths $\letsymbol P(\tau = k) = q^{k - 1} p, k \in \Natural$. Тогда $\forall n, k \geq 0 \quad P(\tau > n + k \ | \ \tau > n) = P(\tau > k)$
    \end{MyTheorem}

    \begin{MyProof}
        $\Box$

        Заметим, что $P(\tau > m) = q^m$, первые $m$ - неудачи

        $P(\tau > n + k | \tau > n) = \frac{P(\tau > n + k, \tau > n)}{P(\tau > n)} = \frac{P(\tau > n + k)}{P(\tau > n)} = \frac{q^{n + k}}{q^n} = q^k$

        $\Box$
    \end{MyProof}

    \Nota $P(\tau = n + k \ | \ \tau > n) = p(\tau = k)$ - \Lab доказать

    \subsubsection{III. Схема испытаний с несколькими исходами}

    Пусть при $n$ независимых испытаний могут произойти $m$ исходов (несовместных)

    $p_i$ - вероятность $i$-ого исхода при одном испытании

    \begin{MyTheorem}
        \Ths Вероятность того, что при $n$ испытаниях первый исход появится $n_1$ раз, второй - $n_2$ раз, $m$-ый - $n_m$ ($\sum_{i = 1}^m n_i = n$)
        равно

        $P_n(n_1, n_2, \dots, n_m) = \frac{n!}{n_1! n_2! \dots n_m!} p_1^{n_1} p_2^{n_2} \dots p_m^{n_m}$
    \end{MyTheorem}

    При $m = 2$ получаем формулу Бернулли

    \begin{MyProof}
        $\Box$

        Рассмотрим следующий благоприятный исход, обозначим $A_1$

        $A_1 = \underset{n_1}{\underbrace{11\dots1}}\underset{n_2}{\underbrace{22\dots2}}\dots\underset{n_m}{\underbrace{mm\dots m}}$

        $p(A_1) = p_1^{n_1} p_2^{n_2} \dots p_m^{n_m}$

        Все остальные благоприятные исходы имеют ту же вероятность и отличаются лишь расположением $i$-ых исходов на $n$ позициях,
        получаем мультиномиальную теорему: $\frac{n!}{n_1! n_2! \dots n_m!}$

        В итоге получаем требуемую формулу

        $\Box$
    \end{MyProof}

    \Ex Два одинаковых сильных шахматиста играют шесть партий

    Вероятность ничьи в партии - $0.5$. Какова вероятность того, что второй игрок выиграет две партии, а еще три сведет к ничьей

    1-ый исход - выиграл 1 игрок

    2-ой исход - выиграл 2 игрок

    3-ий исход - ничья

    $n = 6; \quad p_3 = 0.5; \quad p_1 = p_2 = \frac{1 - p_3}{2} = 0.25$

    $P_6(1; 2; 3) = \frac{6!}{1!2!3!} \left(\frac{1}{4}\right)^1 \left(\frac{1}{4}\right)^2 \left(\frac{1}{2}\right)^3 = \frac{4 \cdot 5 \cdot 6}{2} \frac{1}{2^9} \approx 0.12$

    \subsubsection{IV. Урновая схема}

    В урне $N$ шаров, из которых $K$ шаров белые, $N - K$ - черные

    Из урны вынимаем (без учета порядка) $n$ шаров. Найти вероятность, что из них $k$ белых

    а) Схема с возвратом (после каждого раза кладем шар обратно). В этом случае вероятность вынуть белый шар одинакова и
    равна $\frac{K}{N}$. Получаем схему Бернулли: $P_n(k) = C^k_n \left(\frac{K}{N}\right)^k \left(1 - \frac{K}{N}\right)^{n - k}$

    б) Схема без возврата - вынутый шар мы выбрасываем

    $P_{N, K} (n, k) = \frac{C^k_K C^{n - k}_{N - K}}{C^n_N}$

    \Def Соответствие $k \rightarrow \frac{C^k_K C^{n - k}_{N - K}}{C^n_N}, k = 0, \dots, n$ называется гипергеометрическим
    распределением

    \Nota Если $K, N \to \infty$ так, что $\frac{K}{N} \approx p$ (не меняется), а $n$ и $k$ зафиксировать, то после выбора
    $n$ шаров пропорции состава шаров не сильно изменятся, поэтому логично предположить, что гипергеометрическое распределение
    будет сходиться к биномиальному

    \begin{MyTheorem}
        \Ths Если $K, N \to \infty$ таким образом, что $\frac{K}{N} \to p \in (0;1)$, а $n$ и $0 \leq k \leq n$ фиксированы, то
        вероятность при гипергеометрическом распределении будет стремиться к биномиальному:

        $P_{N,K} (n, k) = \frac{C^k_K C^{n - k}_{N - K}}{C^n_N} \rightarrow C^k_n \left(\frac{K}{N}\right)^k \left(1 - \frac{K}{N}\right)^{n - k}$
    \end{MyTheorem}

    Воспользуемся леммой: $C^k_n \sim \frac{n^k}{k!}$ при $n \to \infty$ и фиксированном $k$

    Доказательство леммы: $C_n^k = \frac{n!}{k!(n - k)!} = \frac{n(n - 1) \dots (n - k + 1)}{n^k} \frac{n^k}{k!} = 1 \left(1 - \frac{1}{n}\right) \dots \left(1 - \frac{k - 1}{n}\right) \frac{n^k}{k!} \sim \frac{n^k}{k!}$

    \begin{MyProof}
        $\Box$

        $P_{N,K} (n, k) = \frac{C^k_K C_{N - K}^{n - k}}{C^n_N} \sim \frac{K^k}{k!} \frac{(N - K)^{n - k}}{N^n} \frac{n!}{N^n} =
        \frac{n!}{k!} \frac{(N - K)^{n - k}}{N^n} \frac{K^k}{N^n} = C^k_n \left(\frac{K}{N}\right)^k \left(1 - \frac{K}{N}\right)^{n - k} \to C^k_n \left(\frac{K}{N}\right)^k \left(1 - \frac{K}{N}\right)^{n - k} $

        $\Box$
    \end{MyProof}

    \subsubsection{V. Схема Пуассона. Теорема Пуассона для схемы Бернулли}

    \Nota Если вероятность успеха $p$ в схеме Бернулли мала или близка к 1, то предельная формула Лапласа при недостаточно большом
    числе испытаний дает достаточно большую погрешность. В этой ситуации следует использовать формулу Пуасоона (формула редких событий)

    Схема: вероятность числа успеха при одном испытании $p_n$ зависит от числа испытаний $n$, причем таким образом, что $n p_n \approx \lambda = const$

    $\lambda$ - интенсивность появления редких событий в единицу времени в потоке испытаний

    \begin{MyTheorem}
        \ThNs{1} (формула Пуассона) Пусть $n \to \infty, p_n \to 0$ таким образом, что $n p_n \to \lambda = const > 0$

        Тогда вероятность $k$ успехов при $n$ испытаниях: $P_n(k) = C^k_n p_n^k (1 - p_n)^{n - k} \underset{n \to \infty}{\rightarrow} = \frac{\lambda^k}{k!} e^{-\lambda}$
    \end{MyTheorem}

    \begin{MyProof}
        $\Box$

        Обозначим $\lambda_n = n p_n$. Тогда $p_n = \frac{\lambda_n}{n}$ и

        $P_n(k) = C^k_n \left(\frac{\lambda_n}{n}\right)^k \left(1 - \frac{\lambda_n}{n}\right)^{n - k} \underset{n \to \infty}{\rightarrow} \frac{n^k}{k!} \frac{\lambda^k_n}{n^k} \left(1 - \frac{\lambda_n}{n}\right)^n \cancelto{1}{\left(1 - \frac{\lambda_n}{n}\right)^{-k}}
        \underset{n \to \infty}{\rightarrow} \frac{\lambda_n^k}{k!} \left(1 - \frac{\lambda_n}{n}\right) = \frac{\lambda_n^k}{k!} \left(\left(1 - \frac{\lambda_n}{n}\right)^{-\frac{n}{\lambda_n}}\right)^{-\frac{\lambda_n}{n}n} \underset{n \to \infty}{\rightarrow} \frac{\lambda_n^k}{k!} e^{-\lambda_n} \underset{n \to \infty}{\rightarrow} \frac{\lambda^k}{k!} e^{-\lambda}$

        $\Box$
    \end{MyProof}

    \begin{MyTheorem}
        \ThNs{2} (оценка погрешности в формуле Пуассона) Пусть $v_n$ - число успехов при $n$ испытаниях в схеме Бернулли

        $p$ - вероятность успеха при одном испытании, $\lambda = np$, $A \subset \{0, 1, \dots, n\}$ - произвольное подмножество чисел

        Тогда $|P_n (v_n \in A) - \sum_{k \in A} \frac{\lambda^k}{k!} e^{-\lambda}| \leq \min (p, np^2) = \min (p, p\lambda)$

        (без доказательства)
    \end{MyTheorem}

    \Def Соответствие $k \to \frac{\lambda^k}{k!} e^{-\lambda}, k = 0, 1, \dots$ называется распределением Пуассона
    с параметром $\lambda > 0$ (обозначается $\Pi_\lambda$)

    \Ex Прибор состоит из 1000 элементов, вероятность отказа каждого элемента равна $0.001$. Какова вероятность отказа больше двух элементов

    $P_n(k) \approx \frac{\lambda^k}{k!} e^{-\lambda}$

    $n = 1000, p = 0.001, \lambda = 1$

    $P_n(k > 2) = 1 - P_n (k \leq 2) = 1 - P(0) - P(1) - P(2) \approx 1 - \left(\frac{1^0}{0!} e^{-1} + \frac{1^1}{1!} e^{-1} + \frac{1^2}{2!} e^{-1}\right) =
    1 - \left(1 + 1 + \frac{1}{2}\right) e^{-1} \approx 0.0803$
% end probtheory_2024_10_01.tex

% begin probtheory_2024_10_08.tex

    \section{Лекция 6}

    \subsection{Случайные величины}

    Примеры случайных величин:

    \ExN{1} Бросаем кость, может выпасть 6 граней, здесь случайная величина $\xi$ - число выпавших очков

    \ExNs{2} $\xi$ - время работы микросхемы, в этом случае время может быть:

    а) дискретным - $\xi \in \{0, 1, 2, \dots\}$

    б) непрерывным - $\xi \in [0; \infty)$

    \ExNs{3} Температура за окном: $\xi \in (-50, +50)$

    \Def На вероятностном пространстве $(\Omega, \mathcal{F}, p)$ функция $\xi \ : \ \Omega \to \Real$ называется
    $\mathcal{F}$-измеримой, если $\forall x \in \Real \ \{\omega \in \Omega \ | \ \xi(\omega) < x\} \in \mathcal{F}$
    (то есть $\xi^{-1}(y) \in \mathcal{F}$, где $y \in (-\infty; x)$)

    \Def Случайной величиной, заданной на вероятностном пространстве $(\Omega, \mathcal{F}, p)$, называется
    $\mathcal{F}$-измеримая функция $\xi \ : \Omega \to \Real$, которая сопоставляет каждому элементарному исходу \
    некоторое вещественное число

    \Nota Не все функции являются $\mathcal{F}$-измеримыми

    \Exs Кость: $\Omega = \{1, 2, 3, 4, 5, 6\}; \mathcal{F} = \{\emptyset, \Omega, \{2, 4, 6\}, \{1, 3, 5\}\}$

    Пусть $\xi(\omega) = i$ - число выпавших очков. Тогда при $x = 4: \{\omega \in \Omega \ | \ \xi (\omega) < 4\} = \{1, 2, 3\} \notin \mathcal{F} \Longrightarrow$ случайная величина не является $\mathcal{F}$-измеримой

    В данном случае следует сделать $\xi$ таким, что $\xi(2) = \xi(4) = \xi(6) = 1$, $\xi(1) = \xi(3) = \xi(5) = 0$

    \Nota Смысл измеримости: если задана случайная величина $\xi$, то мы можем задать вероятность попадания случайной
    величины в интервал $(-\infty; x)$: $p(\xi \in (-\infty; x)) = p(\{\omega \in \Omega \ | \ \xi(\omega) < x\})$

    А из интервалов $(-\infty; x)$ с помощью операций пересечения, объединения и дополнения можно получить все другие
    интервалы (включая точки) и также приписать им вероятности

    Из матанализа известно, что мера из интервалов однозначно продолжается до меры на всей Борелевской $\sigma$-алгебры на $\Real$
    и, таким образом, с помощью случайной величины каждому Борелевскому множеству $B$ также приписывается вероятность $p(\xi \in B)$

    Итак, пусть $\xi$ задана на вероятностном пространстве $(\Omega, \mathcal{F}, p)$, с помощью нее получаем новой вероятностное
    пространство $(\Real, \mathcal{B}(\Real), p_\xi)$

    Получая новое вероятностное пространство, мы упрощаем и формализуем работу, так как можем не учитывать природу и структуру исходного пространства

    \Def Функция $p(B), B \in \mathcal{B}(\Real)$, ставящая в соответствие каждому Борелевскому множеству вероятность,
    называется распределением случайной величины $\xi$

    \subsection{Основные типы распределения}

    \begin{enumerate}[label=\alph*) ]
        \item Дискретное

        \item Абсолютно непрерывное

        \item Сингулярное

        \item Смешанное
    \end{enumerate}

    \subsection{Дискретная случайная величина}

    \Def Случайная величина $\xi$ имеет дискретное рапределение, если она принимает не более, чем счетное число значений.
    То есть существует конечный или счетный набор чисел $\{x_1, x_2, \dots, x_n, \dots\}$ такой, что $p(\xi = x_i) = p_i > 0$ и $\sum_{i = 0}^\infty p_i = 1$

    Таким образом, дискретная случайная величина (ДСВ) задается законом распределения:

    доска

    \smallvspace

    \begin{tabular}{c|c|c|c|c|cl}
        $\xi$ & $x_1$ & $x_1$ & \dots & $x_n$ & \dots & \text{\qquad   - значения случайной величины} \\
        \cline{1-6}
        $p$   & $p_1$ & $p_1$ & \dots & $p_n$ & \dots & \text{\qquad   - вероятности этих значений}
    \end{tabular}

    \smallvspace


    ($\sum_{i = 0}^\infty p_i = 1$ - условие нормировки)

    \ExN{1} кость, $\xi(\omega) = i$ - число выпавших очков

    \smallvspace

    %nodisplay

    \begin{tabular}{c|c|c|c|c|c|c}
        $\xi$ & $1$           & $2$           & $3$           & $4$           & $5$           & $6$           \\
        \hline
        $p$   & $\frac{1}{6}$ & $\frac{1}{6}$ & $\frac{1}{6}$ & $\frac{1}{6}$ & $\frac{1}{6}$ & $\frac{1}{6}$
    \end{tabular}

    %yesdisplay

    \smallvspace

    \ExNs{2} все распределения из предыдущих лекций (биномиальное, геометрическое, гипергеометрическое, Пуассона)

    \ExNs{3} индикатор события $A$: $I_A (\omega) = \begin{cases}
                                                        0, \quad \omega \notin A \text{ - событие } A \text{ не происходит} \\ 1, \quad \omega \in A \text{ - событие } A \text{ происходит}
    \end{cases}$

    \subsection{Числовые характеристики дискретных случайных величин}

    \subsubsection{I. Математическое ожидание (среднее значение, полезность)}

    \Defs Математическим ожиданием $E\xi$ случайной величины $\xi$ называется число

    \[ E\xi = \sum_{i = 1}^\infty x_i p_i \]

    при условии, что данный ряд сходится абсолютно

    \Nota Если $E\xi = \sum_{i = 1}^\infty x_i p_i = \infty$, то говорят, что матожидание не существует

    При условной сходимости ряда при перестановке членов сумма изменяется, поэтому необходима абсолютная

    \textbf{Физический смысл}: Среднее значение - число, вокруг которого группируются значения случайной величины, центр тяжести точек $x_i$ с весами $p_i$

    \textbf{Статистический смысл}: среднее арифметическое наблюдаемых значений случайной величины при
    большом числе реальных экспериментов

    \subsubsection{II. Дисперсия}

    \Defs Дисперсией $D\xi$ случайной величины $\xi$ называют среднее квадратов ее отклонения от математического ожидания:

    $D\xi = E (\xi - E\xi)^2$ или $D\xi = \sum_{i = 0}^\infty (x_i - E\xi)^2 p_i$ при условии, что данный ряд сходится

    В противном случае говорится, что дисперсии не существует

    \Nota Дисперсию обычно удобно считать по формуле $D\xi = E\xi^2 - (E\xi)^2 = \sum_{i = 1}^n x^2_i p_i - E\xi^2$

    \textbf{Смысл} - квадрат среднего разброса (рассеивания) значения случайной величины относительно ее математического
    ожидания

    \subsubsection{III. Среднее квадратическое отклонение}

    \Defs Среднее квадратическое отклонение (СКО) $\sigma_\xi$ называется величина $\sigma_\xi = \sqrt{D\xi}$

    \textbf{Смысл} - средний разброс

    \ExN{1} Кость

    \smallvspace

    %nodisplay

    \begin{tabular}{c|c|c|c|c|c|c}
        $\xi$ & $1$           & $2$           & $3$           & $4$           & $5$           & $6$           \\
        \hline
        $p$   & $\frac{1}{6}$ & $\frac{1}{6}$ & $\frac{1}{6}$ & $\frac{1}{6}$ & $\frac{1}{6}$ & $\frac{1}{6}$
    \end{tabular}

    %yesdisplay

    $E\xi = \sum_{i = 1}^6 x_i p_i = 3.5$ (в данном случае ср. арифм.)

    $D\xi = \sum_{i = 1}^6 (x_i - E\xi)^2 p_i = 1^2 \cdot \frac{1}{6} + 2^2 \cdot \frac{1}{6} + 3^2 \cdot \frac{1}{6} + 4^2 \cdot \frac{1}{6} + 5^2 \cdot \frac{1}{6} + 6^2 \cdot \frac{1}{6} - 3.5^2 = \frac{35}{12} $

    $\sigma_\xi = \sqrt{D\xi} \approx 1.79$

    \ExN{2} Индикатор события $A$: $I_A (\omega) = \begin{cases}
                                                       0, \omega \notin A \text{ - событие } A \text{ не происходит} \\ 1, \omega \in A \text{ - событие } A \text{ происходит}
    \end{cases}$

    \begin{tabular}{c|c|c}
        $\xi$ & $0$        & $1$    \\
        \hline
        $p$   & $1 - P(A)$ & $P(A)$
    \end{tabular}

    $E\xi = 0 \cdot (1 - P(A)) + 1 \cdot P(A) = P(A)$

    $D\xi = 0^2 \cdot (1 - P(A)) + 1^2 P(A) - P(A)^2 = P(A) (1 - P(A)) = pq$

    $\sigma_\xi = \sqrt{pq}$

    \subsection{Свойства матожидания и дисперсии}

    \begin{MyTheorem}
        \ThNs{1} Случайная величина $\xi$ имеет вырожденное распределение, если $\xi(\omega) = \mathrm{const} \ \ \forall \omega \in \Omega$

        \begin{tabular}{c|c}
            $\xi$ & $C$ \\
            \hline
            $p$   & $1$
        \end{tabular}

        $E\xi = C \qquad D\xi = 0$
    \end{MyTheorem}

    \begin{MyTheorem}
        \ThNs{2} Свойство сдвига: $E(\xi + C) = E\xi + C; D (\xi + C) = D\xi$
    \end{MyTheorem}

    \begin{MyTheorem}
        \ThNs{3} Свойство растяжения:

        $E(C\xi) = CE\xi$

        $D(C\xi) = C^2 D\xi$
    \end{MyTheorem}

    \Lab 2-3 доказать

    \begin{MyTheorem}
        \ThNs{4} $E(\xi + \eta) = E\xi + E\eta$ (из третьего свойства матожидание - линейная функция)
    \end{MyTheorem}

    \begin{MyProof}
        $\Box$

        $\letsymbol x_i, y_i$ - значения случайных величин $\xi, \eta$, а $p_i$ и $q_i$ - их соответствующие вероятности

        $E(\xi + \eta) = \sum_{i, j} (x_i + y_j) p(\xi = x_i \text{ и } \eta = y_j) = \sum_i x_i \sum_j p(\xi = x_i \text{ и } \eta = y_j) + \sum_j y_j \sum_i p(\xi = x_i \text{ и } \eta = y_j) =
        \sum_i x_i p(\xi = x_i) + \sum_j y_j p(\eta = y_j) = E\xi + E\eta$

        $\Box$
    \end{MyProof}

    \Def Дискретные случайные величины $\xi$ и $\eta$ независимы, если $p(\xi = x_i, \eta = y_i) = p(\xi = x_i) \cdot p(\eta = y_i) \ \forall i, j$

    То есть случайные величины принимают свои величины независимо друг от друга

    \begin{MyTheorem}
        \ThNs{5} Если случайные свойства $\xi$ и $\eta$ независимы, то $E(\xi \eta) = E\xi \cdot E\eta$; обратное неверно
    \end{MyTheorem}

    \begin{MyProof}
        $\Box$

        $E(\xi\eta) = \sum_{i, j} x_i y_i p(\xi = x_i, \eta = y_j) = \sum_i x_i \sum_j y_j p(\xi = x_i, \eta = y_j) =
        \sum_i x_i \sum_j y_j p(\xi = x_i) p(\eta = y_j) = \sum_i x_i p(\xi = x_i) \sum_j y_j p(\eta = y_j) = E\xi \cdot E\eta$

        $\Box$
    \end{MyProof}

    \begin{MyTheorem}
        \ThNs{6} $D\xi = E\xi^2 - (E\xi)^2$
    \end{MyTheorem}

    \begin{MyProof}
        $\Box$

        $D\xi = E(\xi - E\xi)^2 = E(\xi^2 - 2\xi E\xi + (E\xi)^2) = E\xi^2 - 2E\xi E\xi + E((E\xi)^2) =
        E\xi^2 - 2(E\xi)^2 + (E\xi)^2 = E\xi^2 - (E\xi)^2$

        $\Box$
    \end{MyProof}

    \Def $D(\xi + \eta) = D\xi + D\eta + 2\mathrm{cov} (\xi, \eta)$,
    где $\mathrm{cov}(\xi, \eta) = E(\xi\eta) - E\xi E\eta$ - ковариация случайных величин (равна 0 при независимых величинах) - индикатор наличия связи между случайными величинами

    \begin{MyProof}
        $\Box$

    $D(\xi + \eta) = E(\xi + \eta)^2 - (E(\xi + \eta))^2 = E\xi^2 + 2E\xi E\eta + E\eta^2 - (E\xi + E\eta)^2 =
    E\xi^2 + E\eta^2 + 2E(\xi\eta) - (E\xi)^2 - (E\eta)^2 - 2E\xi E\eta = D\xi + D\eta + 2\mathrm{cov}(\xi, \eta)$

        $\Box$
    \end{MyProof}


    \begin{MyTheorem}
        \ThNs{7} Если случайные величины $\xi$ и $\eta$ независимы, то $D(\xi + \eta) = D\xi + D\eta$
    \end{MyTheorem}

    \begin{MyProof}
        $\Box$

        Если $\xi$ и $\eta$ независимы, то $\mathrm{cov}(\xi, \eta) = 0$ и $D(\xi + \eta) = D\xi + D\eta$

        $\Box$
    \end{MyProof}

    \begin{MyTheorem}
        \ThNs{8} Общая формула дисперсии суммы: $D(\xi_1 + \xi_2 + \dots + \xi_n) = \sum_{i = 1}^n D \xi_i + 2\sum_{i, j (i \neq j)} \mathrm{cov} (\xi_i, \xi_j)$
    \end{MyTheorem}

    \subsection{Другие числовые характеристики}

    Моменты старших порядков

    а) $m_k = E\xi^k$ - момент $k$-ого порядка случайной величины $\xi$

    б) $\mu_k = E(\xi - E\xi)^k$ - центральный момент $k$-ого порядка

    $E\xi = m_1$ - момент первого порядка

    $E\xi^2 = m_2$ - момент второго порядка

    $D\xi = E(\xi - E\xi)^2$ - центральный момент второго порядка

    \Nota Центральные моменты можно выразить через обычный момент:

    $\mu_2 = D\xi = E\xi^2 - (E\xi)^2 = m_2^2 - m_1^2$

    $\mu_3 = m_3 - 3m_2 m_1 + 2m^3$

    $\mu_4 = m_4 - 4m_3 m_2 + 6m_2 m_1^2 - 3m_1^4$

    \Ex Разберем \hyperlink{buffonsproblem}{задачу Бюффона} с точки зрения матожидания (для простоты $l$ - ширина доски): пусть $p(A)$ - пересечет стык,
    $\xi = I_A$ - число пересечений. Тогда матожидание $E\xi = E I_A = P(A)$

    Заметим, что при изменении длины иглы с $l$ до $2l$ матожидание пересекаемых стыков увеличивается
    в два раза. Помимо этого можно составить из $k$ игл ломаную, матожидание стыков которой будет равно $kE\xi$

    Заметим, что такое работает и в обратную сторону: при уменьшении иглы в $k$ раз матожидание равно $\frac{E\xi}{k}$

    Теперь сделаем замкнутый многоугольник из игл, получим, что матожидание в таком случае $P\frac{E\xi}{l}$, где $P$ - периметр

    В пределе строим круг диаметра $l$ - он всегда пересечет линии стыка 2 раза, значит матожидание $E_o = P_o\frac{E\xi}{l} = 2$

    Длина окружность $P_o = \pi l$, получаем $E\xi = \frac{2l}{P_o} = \frac{2l}{\pi l} = \frac{2}{\pi}$
% end probtheory_2024_10_08.tex

% begin probtheory_2024_10_15.tex

    \section{Лекция 7}

    \subsection{Стандартное дискретное распределение}

    \subsubsection{I. Распределение Бернулли}

    Распределение Бернулли $B_p$ (с параметром $0 < p < 1$)

    $\xi$ - число успехов при одном испытании, $p$ - вероятность успеха при одном испытании
    
    \smallvspace

    \begin{tabular}{c|c|c}
        $\xi$ & $0$        & $1$    \\
        \hline
        $p$   & $1 - P(A)$ & $P(A)$
    \end{tabular}

    \smallvspace

    Матожидание: $E\xi = p$

    Дисперсия: $D\xi = p(1 - p) = pq$

    \Ex Индикатор события $I_A \in B_p$ как раз имеет распределение Бернулли, где $p = P(A)$

    \subsubsection{II. Биномиальное распределение}

    Биномиальное распределение $B_{n,p}$ (с параметрами $n, p$)

    $\xi$ - число успехов в серии из $n$ испытаний, $p$ - вероятность успеха при одном испытании

    $p(\xi = k) = C^k_n p^k q^{n - k}, \ k = 0, 1, \dots, n \Longleftrightarrow \xi \in B_{n,p}$

    \smallvspace

    \begin{tabular}{c|c|c|c|c|c|c}
        $\xi$ & $0$        & $1$ & \dots & $k$ & \dots & $n$    \\
        \hline
        $p$   & $q^n$ & $n q^{n - 1} p$ & \dots & $C^k_n p^k q^{n - k}$ & \dots & $p_n$
    \end{tabular}

    \smallvspace

    Заметим, что $\xi = \xi_1 + \xi_2 + \dots + \xi_n$, где $\xi_i \in B_p$ - число успехов при $i$-ой испытании

    $E\xi_i = p; \quad D\xi = pq$

    $E\xi = E\xi_1 + \dots + E\xi_n = p + \dots + p = $ \fbox{$np$}

    $D\xi = D\xi_1 + \dots + D\xi_n = pq + \dots + pq = $ \fbox{$npq$}

    
    \subsubsection{III. Геометрическое распределение}

    Геометрическое распределение $G_p$ (с параметром $p$)

    $\xi$ - номер 1-ого успешного испытания в бесконечной серии

    $p(\xi = k) = q^{k - 1}p, \ k = 1, 2, 3, \dots \Longleftrightarrow \xi \in G_p$

    \smallvspace
    
    \begin{tabular}{c|c|c|c|c|c|c}
        $\xi$ & $1$ & $2$ & \dots & $k$ & \dots & $n$    \\
        \hline
        $p$   & $p$ & $qp$ & \dots & $q^{k - 1}p$ & \dots & $p_n$
    \end{tabular}

    \smallvspace

    Матожидание $E\xi = \sum_{k = 1}^\infty k p(\xi = k) = \sum_{k = 1}^\infty k q^{k - 1} p = p \sum_{k = 1}^\infty k q^{k - 1} = 
    p \sum_{k = 1}^\infty (q^k)^\prime = p \left(\sum_{k = 1}^\infty (q^k)\right)^\prime = p \left(\frac{1}{1 - q}\right)^\prime = 
    \frac{p}{p^2} = \frac{1}{p}$

    $E\xi^2 = \sum_{k = 1}^\infty k^2 q_{k - 1} p = p \sum_{k = 1}^\infty k(k - 1)q^{k - 1} = pq \sum_{k = 1}^\infty k(k - 1)q^{k - 2} + E\xi = 
    pq (\sum_{k = 1}^\infty q^k)^{\prime\prime} + \frac{1}{p} = pq \left(\frac{1}{1 - q}\right)^{\prime\prime} + \frac{1}{p} = 
    2pq \frac{1}{(1 - q)^3} + \frac{1}{p} = 2pq \frac{1}{p^3} + \frac{1}{p} = \frac{2q}{p^2} + \frac{1}{p}$
    
    $D\xi = E\xi^2 - (E\xi)^2 = \frac{2q}{p^2} + \frac{1}{p} - \frac{1}{p^2} = \frac{q}{p^2}$

    
    \subsubsection{IV. Распределение Пуассона}

    Распределение Пуассона $\Pi_\lambda$ (с параметром $\lambda > 0$)

    \Def Случайная величина $\xi$ имеет распределение Пуассона с параметром $\lambda > 0$, если $p(\xi = k) = \frac{\lambda^k}{k!}e^{-\lambda}, \ k = 0, 1, 2, \dots$

    \smallvspace

    \begin{tabular}{c|c|c|c|c|c|c}
        $\xi$ & $0$ & $1$ & \dots & $k$ & \dots & $n$    \\
        \hline
        $p$   & $e^{-\lambda}$ & $\lambda e^{-\lambda}$ & \dots & $\frac{\lambda^k}{k!}e^{-\lambda}$ & \dots & $p_n$
    \end{tabular}

    \smallvspace

    Покажем корректность определения - докажем, что сумма нижней строки равна 1:

    $\sum_{k = 0}^\infty p_k = \sum_{k = 0}^\infty \frac{\lambda^k}{k!} e^{-\lambda} = e^{-\lambda} \underset{\substack{\text{ряд Тейлора}\\ \text{для } e^x}}{\underbrace{\sum_{k = 0}^\infty \frac{\lambda_k}{k!}}} = e^{-\lambda} e^\lambda = 1$

    $E\xi = \sum_{k = 0}^\infty k \cdot \frac{\lambda^k}{k!}e^{-\lambda} = e^{-\lambda} \sum_{k = 0}^\infty \frac{\lambda^k}{(k - 1)!} = 
    \lambda e^{-\lambda} \sum_{k = 0}^\infty \frac{\lambda^{k - 1}}{(k - 1)!} = \lambda e^{-\lambda} e^\lambda = \lambda = np$

    
    $E\xi^2 = \sum_{k = 0}^\infty k^2 \cdot \frac{\lambda^k}{k!}e^{-\lambda} = e^{-\lambda} \sum_{k = 2}^\infty k(k - 1) \frac{\lambda^k}{k!} + 
    e^{-\lambda} \sum_{k = 1}^\infty k \frac{\lambda^k}{k!} = \lambda^2 e^{-\lambda} \sum_{k = 2}^\infty \frac{\lambda^{k - 2}}{(k - 2)!} + 
    \lambda e^{-\lambda} \sum_{k = 1}^\infty \frac{\lambda^{k - 1}}{(k - 1)!} = \lambda^2 e^{-\lambda} e^\lambda + \lambda e^{-\lambda} e^\lambda = \lambda^2 + \lambda$

    $D\xi = E\xi^2 - (E\xi)^2 = \lambda^2 + \lambda - \lambda^2 = \lambda$

    \clearpage

    \subsection{Задача о разорении игрока}

    Постановка задачи: играют 2 игрока, вероятность выигрыша первого игрока в одной игре равна $p$, $q = 1 - p$ - вероятность его проигрыша (выигрыш второго)

    В каждой игре разыгрывается 1 биткоин. Капитал первого игрока - $k$ биткоинов, $m - k$ биткоинов - капитал второго

    Найти вероятность разорения первого игрока

    Траектория капитала первого игрока будет выглядить как-то так:

    \begin{center}
        \includegraphics[height=7cm]{probtheory/images/probtheory_2024_10_15_1}
    \end{center}

    Пусть $r_k$ - интересующая нас вероятность разорение игрока при капитале $k$ (то есть достижения оси абсцисс на графике)

    $r_k = p \cdot r_{k + 1} + q r_{k - 1}$

    $pr_{k + 1} - r_k + (1 - p) r_{k - 1} = 0, \quad r_0 = 1, r_{m} = 0$

    $p\lambda^2 - \lambda + (1 - p) = 0$

    $D = 1 - 4p(1 - p) = 4p^2 - 4p + 1 = (2p - 1)^2$

    $\lambda_{1, 2} = \frac{1 \pm (2p - 1)}{2p}; \quad \lambda_1 = 1; \lambda_2 = \frac{2 - 2p}{2p} = \frac{q}{p}$

    Обозначим $\lambda = \frac{q}{p}$

    Рассмотрим два случая: 

    \begin{itemize}
        \item $p \neq \frac{1}{2}$

        Тогда общее решение: $r_k = C_1 \lambda_1^k + C_2 \lambda_2^k = C_1 + C_2 \lambda^k$

        Найдем частное решение:

        $\begin{cases}
            1 = C_1 + C_2 \\
            0 = C_1 + C_2 \lambda^m
        \end{cases} \Longleftrightarrow \begin{cases}
            C_1 = 1 - C_2 \\
            1 - C_2 + C_2 \lambda_m = 0
        \end{cases} \Longleftrightarrow \begin{cases}
            C_1 = 1 - C_2 \\
            C_2 (1 - \lambda_m) = 1
        \end{cases} \Longleftrightarrow \begin{cases}
            C_1 = 1 - \frac{1}{1 - \lambda^m} = \frac{-\lambda^m}{1 - \lambda^m} \\
            C_2 = \frac{1}{1 - \lambda^m}
        \end{cases}$

        $r_k = \frac{-\lambda^m}{1 - \lambda^m} + \frac{1}{1 - \lambda^m} \lambda^k = \frac{\lambda^k - \lambda^m}{1 - \lambda^m}$

        Посмотрим, что будет происходит при бесконечной игре (то есть когда $m \to \infty$ - капитал неограничен)

        1) $p < q$, то есть $\lambda > 1$. Тогда $\lambda^m \to \infty$, $r_k = \frac{\lambda^k - \lambda^m}{1 - \lambda^m} = \frac{\frac{\lambda^k}{\lambda_m} - 1}{\frac{1}{\lambda^m} - 1} \underset{n \to \infty}{\longrightarrow} 1$ - 
        то есть первый игрок гарантированно разорится

        2) $p > q$, то есть $\lambda < 1$. Тогда $\lambda^m \to 0$, $r_k = \frac{\lambda^k - \lambda^m}{1 - \lambda^m} \underset{n \to \infty}{\longrightarrow} \lambda^k$ - 
        то есть $r_k = \left(\frac{q}{p}\right)^k$

        \item $p = \frac{1}{2} \Longrightarrow D = 0$ 

        Тогда $\lambda_1 = \lambda_2 = 1$

        Общее решение: $r_k = C_1 \lambda^k + C_2 k \lambda_k = C_1 + C_2 k$

        Частное решение: 
        
        $\begin{cases}
            1 = C_1 \\
            0 = C_1 + C_2 m
        \end{cases} \Longleftrightarrow \begin{cases}
            1 = C_1 \\
            -1 = C_2 m
        \end{cases} \Longleftrightarrow \begin{cases}
            1 = C_1 \\
            C_2 = -\frac{1}{m}
        \end{cases}$

        $r_k = 1 - \frac{k}{m}$

        При бесконечной игре:

        $r_k = 1 - \frac{k}{m} \underset{m \to \infty}{\longrightarrow} 1$ - то есть при равной игре игрок неминуемо разорится 

    \end{itemize}

    \subsection{Случайное блуждание на прямой}

    Пусть в начальный момент времени находимся в начале координат. С вероятностью $p$ идем на единицу вправо, с вероятностью $q$ - влево

    При $p = \frac{1}{2}$ мы рано или поздно попадем в любую точку числовой прямой

    Можно привести аналогию с орлянкой: рано или поздно каждый игрок будет при сколь угодно большом выигрыше

    Посмотрим на орлянку как на распределение Бернулли:

    \smallvspace

    \begin{tabular}{c|c|c}
        $\xi$ & $-1$        & $1$    \\
        \hline
        $p$   & $\frac{1}{2}$ & $\frac{1}{2}$
    \end{tabular}

    \smallvspace

    $E\xi = 0; \quad D\xi = 1$

    Пусть $\xi$ - выигрыш первого после $n$ игр.

    $E\xi = \sum_{i = 1}^n E\xi_i = 0$

    $D\xi = \sum_{i = 1}^n D\xi_i = n$

    $\sigma_\xi = \sqrt{n}$ - среднее квадратическое отклонение

    Это означает, что при большом $n$ СКО поглотит всю числовую прямую

    $\frac{S_n}{n} \to E\xi$

    Закон больших чисел в этой ситуации говорит, что точка останется у 0, однако в то же время она может оказаться на любой точке на числовой прямой

    \Ex По $n$ конвертам случайным образом раскладывается $m$ писем. Случайная величина $\xi$ - число писем в своих конвертах

    $\letsymbol A_i$ - число $i$ письма в своем конверте, $\xi_i = I_A = \begin{cases}0, \quad i\text{-ое письмо в не своем конверте} \\ 1, \quad i\text{-ое письмо в своем конверте} \end{cases}$

    $\xi = \sum_{i = 1}^n \xi_i$

    $E\xi_i = P(A_i) = \frac{1}{n}$

    $D\xi_i = pq = \frac{1}{n} (1 - \frac{1}{n}) = \frac{n - 1}{n^2}$

    $E\xi = \sum_{i = 1}^n E\xi_i = 1 \frac{1}{n} = 1$ - в среднем будет одно письмо в своем конверте

    $D\xi = D(\xi_1 + \dots + \xi_n) = \sum_{i = 1}^n D\xi_i + 2\sum_{i < j} \mathrm{cov} (\xi_i, \xi_j)$

    Найдем ковариацию:

    $\mathrm{cov}(\xi_i, \xi_j) = E\xi_i \xi_j - E\xi_i E\xi_j = \frac{1}{n(n - 1)} - \frac{1}{n}\frac{1}{n} = \frac{n - (n - 1)}{n^2(n - 1)} = \frac{1}{n^2(n - 1)}$

    Заметим, что для любых $i, j, i < j$: $\xi_i \xi_j = \begin{cases}0, \quad \text{если хотя бы одно не в своем} \\ 1, \quad \text{если оба в своем}\end{cases}$
    
    То есть $\xi_i\xi_j \in B_p$ и $E\xi_i \xi_j = P(\text{оба письма в своих}) = \frac{1}{n(n - 1)}$

    Получаем: $D\xi = n \frac{n - 1}{n^2} + 2\frac{n(n - 1)}{2}\frac{1}{n^2(n - 1)} = \frac{n - 1}{n} + \frac{1}{n} = 1$

    



    
% end probtheory_2024_10_15.tex

% begin probtheory_2024_10_22.tex

    \section{Лекция 8}

    \subsection{Функция распределения}

    \Def Функция распределения $F_\xi(x)$ случайной величины $\xi$ называется функция $F_\xi(x) = P(\xi < x)$

    \includegraphics[height=4cm]{probtheory/images/probtheory_2024_10_22_2}

    \begin{minipage}{\textwidth}
        \begin{wrapfigure}{r}{0pt}
            \includegraphics[height=4.5cm]{probtheory/images/probtheory_2024_10_22_3}
        \end{wrapfigure}

        \Ex $\xi \in B_p$ \qquad \begin{tabular}{c|c|c}
            $\xi$ & 0 & 1 \\ \hline
            $p$ & $1 - p$ & $p$ \\
        \end{tabular}
        
        $F_\xi(x) = \begin{cases}0 \quad x \leq 0, \\ 1 - p \quad 0 < x \leq 1, \\ 1 \quad x > 1\end{cases}$

    \end{minipage}

    \subsubsection{Свойства функции распределения}

    1) $F(x)$ ограничена $0 \leq F(x) \leq 1$

    2) $F(x)$ неубывающая функция: $x_1 < x_2 \Longrightarrow F(x_1) \leq F(x_2)$ 

    \begin{MyProof}
        $x_1 < x_2 \Longrightarrow \{\xi < x_1\} \subset \{\xi < x_2\} \Longrightarrow p(\xi < x_1) \leq p(\xi < x_2)$, то есть $F(x_1) \leq F(x_2)$
    \end{MyProof}

    3) $p(\alpha \leq \xi < \beta) = F(\beta) - F(\alpha)$

    \includegraphics[height=6cm]{probtheory/images/probtheory_2024_10_22_4}

    \begin{MyProof}
        $p(\xi < \beta) = p(\xi < \alpha) + p(\alpha \leq \xi < \beta) \Longrightarrow F(\beta) = F(\alpha) + p(\alpha \leq \xi < \beta)$
    \end{MyProof}
    
    \Notas Функция распределения $F(x)$ - вероятность попадания в интервал $(-\infty; x)$. Так как Борелевская $\sigma$-алгебра порождается такими интервалами,
    то распределение полностью задается этой функцией

    4) $\lim_{x \to -\infty} F(x) = 0; \quad \lim_{x \to +\infty} F(x) = 1$

    \begin{MyProof}
        Так как $F(x)$ монотонна и ограничена, то эти пределы существуют. Поэтому достаточно доказать эти пределы для некоторой последовательности $x_n \to \pm \infty$

        $\letsymbol A_n = \{n - 1 \leq \xi < n, n \in v\}$ - несовместные события, так как $\Real = \bigunion_{n = -\infty}^\infty A_n$, то
        по аксиоме счетной аддитивности, вероятность $p(\xi \in \Real) = 1 = \sum_{n = -\infty}^\infty P(A_n) = \lim_{N \to \infty} \sum_{n = -N}^N p(n - 1 \leq \xi < n) = 
        \lim_{N \to \infty} \sum_{n = -N}^N (F(n) - F(n - 1)) = \lim_{N \to \infty} (F(N) - F(-N - 1)) = \lim_{N \to \infty} F(N) - \lim_{N \to -\infty} F(N) = 1$

        $\Longrightarrow \lim_{N \to \infty} F(N) = 1 + \lim_{N \to -\infty} F(N) $

        Так как $\lim_{N \to \infty} F(N) \leq 1$ и $\lim_{N \to -\infty} F(N) \geq 0$, то $\lim_{N \to \infty} F(N) = 1$ и $\lim_{N \to -\infty} = 0$
    \end{MyProof}
    
    5) $F(x)$ непрерывна слева: $F(x_0 - 0) = F(x_0)$

    \begin{MyProof}
        Этот предел существует в силу монотонности и ограниченности функции, поэтому рассмотрим последовательность событий $B_n = \{x_0 - \frac{1}{n} \leq \xi < x_0, n \in \mathrm{Z}\}$

        Так как $B_1 \supset B_2 \supset \dots \supset B_n \supset \dots$ и $\bigcap_{n = 1}^\infty B_n = \emptyset$

        То по аксиоме непрерывности $p(B_n) \to 0$

        $P(B_n) = F(x_0) - F(x_0 - \frac{1}{n}) \rightarrow 0$

        $F(x_0 - \frac{1}{n}) \to F(x_0)$

        $\lim_{x \to x_0 - 0} F(x) = F(x_0)$
    \end{MyProof}
    
    6) Скачок в точке $x_0$ равен вероятности попадания в данную точку: $F(x_0 + 0) - F(x_0) = p(\xi = x_0)$ или $F(x_0 + 0) = p(\xi = x_0) + p(\xi < x_0) = p(\xi \leq x_0)$

    
    \begin{MyProof}
        Этот предел существует в силу монотонности и ограниченности функции, поэтому рассмотрим последовательность событий $C_n = \{x_0 \leq \xi < x_0 + \frac{1}{n}, n \in \mathrm{Z}\}$

        Так как $C_1 \supset C_2 \supset \dots \supset C_n \supset \dots$ и $\bigcap_{n = 1}^\infty C_n = \emptyset$

        То по аксиоме непрерывности $p(C_n) \to 0$

        $P(C_n) = F(x_0 + \frac{1}{n}) - F(x_0) \rightarrow 0$

        $p(x_0 \leq \xi < x_0 + \frac{1}{n}) + p(\xi = x_0) \rightarrow p(\xi = x_0)$

        $F(x_0 + \frac{1}{n}) - F(x_0) \to p(\xi = x_0)$

        $F(x_0 + 0) - F(x_0) \to p(\xi = x_0)$
    \end{MyProof}
    
    7) Если функция распределения непрерывна в точке $x = x_0$, то очевидно, что вероятность попадания в эту точка $p(\xi = x_0) = 0$ (следствие из 6 пункта)
    
    8) Если $F(x)$ непрерывна $\forall x \in \Real$, то $p(\alpha \leq \xi < \beta) = p(\alpha < \xi < \beta) = p(\alpha \leq \xi \leq \beta) = p(\alpha < \xi \leq \beta) = F(\beta) - F(\alpha)$
    
    \begin{MyTheorem}
        \Ths Случайная величина $\xi$ имеет дискретное распределение тогда и только тогда, когда ее функция распределения имеет ступенчатый вид
    \end{MyTheorem}

    \subsection{Абсолютно непрерывное распределение}

    \Def Случайная величина $\xi$ имеет абсолютно непрерывное распределение, если существует $f_\xi(x)$ такая, что $\forall B \in \mathcal{B}(\Real)
    \ p(\xi \in B) = \int_B f_\xi(x)dx$

    Функция $f_\xi$ называется плотностью распределения случайной величины

    (в определении использует интеграл Лебега, так как $B$ может быть не просто интервалом на $\Real$)

    \subsubsection{Свойства плотности и функции распределения абсолютно непрерывного распределения}

    1) Вероятносто-геометрический смысл плотности: $p(\alpha \leq \xi < \beta) = \int_{\alpha}^\beta f_\xi(x) dx$

    2) Условие нормировки: $\int_{-\infty}^{+\infty} f_\xi(x)dx = 1$

    \begin{MyProof}
        Из определения, если $B = \Real$
    \end{MyProof}

    3) $F_\xi(x) = \int_B f_\xi(x)dx$

    \begin{MyProof}
        Если $B = (-\infty; x)$, то $F_\xi(x) = p(\xi \in (-\infty; x)) = \int_{-\infty}^x f_\xi(x)dx$
    \end{MyProof}

    4) $F_\xi(x)$ непрерывна 
    
    \begin{MyProof}
        Из свойства непрерывности интеграла с верхним переменным пределом
    \end{MyProof}

    5) $F_\xi(x)$ дифференцируема почти везде и $f_\xi(x) = F^\prime_\xi(x)$ для почти всех $x$
    
    \begin{MyProof}
        По теореме Барроу
    \end{MyProof}

    6) $f_\xi(x) \geq 0$ по определению и как производная неубывающей $F_\xi(x)$

    7) $p(\xi = x) = 0 \ \forall x \in \Real$ - так как $F_\xi(x)$ непрерывна

    8) $p(\alpha \leq \xi < \beta) = p(\alpha < \xi < \beta) = p(\alpha \leq \xi \leq \beta) = p(\alpha < \xi \leq \beta) = F(\beta) - F(\alpha)$

    9) \Ths Если $f(x) \leq 0$ и $\int_{-\infty}^{\infty} f(x)dx$ (выполнены свойства 2 и 6), то $f(x)$ - плотность некоторого распределения

    \subsubsection{Числовые характеристики}

    \Def Математическим ожиданием $E\xi$ случайной абсолютно непрерывной величины $\xi$ называется величина $E\xi = \int_{-\infty}^{\infty} xf_\xi(x) dx$ 
    при условии, что данный интеграл сходится абсолютно, то есть $\int_{-\infty}^\infty |x|f_\xi(x)dx < \infty$

    \Def Дисперсией $D\xi$ случайной величины $\xi$ называется величина $D\xi = E(\xi - E\xi)^2 = \int_{-\infty}^\infty (x - E\xi)^2 f_\xi(x) dx$ при условии,
    что данный интеграл сходится

    \Notas Вычислять удобно по формуле $D\xi = E\xi^2 - (E\xi)^2 = \int_{-\infty}^\infty x^2 f_\xi(x)dx - (E\xi)^2$

    \Def Среднее квадратическое отклонение $\sigma_\xi = \sqrt{D\xi}$ определяется, как корень дисперсии

    Смысл этих величин такой же, как и при дискретном распределении. Также свойства аналогичны тем, что и при дискретном распределении

    \subsubsection{Другие числовые характеристики}

    $m_k = E\xi^k = \int_{-\infty}^\infty x^k f_\xi(x)dx$ - момент $k$-ого порядка

    $\mu_k = E(\xi - E\xi)^k = \int_{-\infty}^\infty (x - E\xi)^k f_\xi(x)dx$ - центральный момент $k$-ого порядка

    \Def Медианой $Me$ абсолютно непрерывной случайной величины $\xi$ называется значение случайной величины $\xi$, такое что $p(\xi < Me) = p(\xi > Me) = \frac{1}{2}$
    
    \includegraphics[height=4cm]{probtheory/images/probtheory_2024_10_22_5}

    \Def Модой $Mo$ случайной величины $\xi$ называется точка локального максимума плотности

    \includegraphics[height=4cm]{probtheory/images/probtheory_2024_10_22_6}

    \subsection{Сингулярное распределение}

    \Def Случайная величина $\xi$ имеет сингулярное распределение, если $\exists B$ - Борелевское множество с нулевой мерой Лебега $\lambda(B) = 0$, такое что $p(\xi \in B) \in 1$, но $P(\xi = x) = 0 \ \, \forall x \in B$

    \Nota Такое Борелевское множество состоит из несчетного множества точек, так как в протичном случае по аксиоме счетной аддитивности $p(\xi \in B) = 0$. То есть 
    при сингулярном распределении случайная величина $\xi$ распределена на несчетном множестве меры 0

    \Notas Так как $p(\xi = x) = 0 \  \forall x$, $F_\xi$ непрерывна.

    \smallvspace
    
    \begin{minipage}{\textwidth}
        \begin{wrapfigure}{r}{0pt}
            \includegraphics[width=7cm]{probtheory/images/probtheory_2024_10_22_1}
        \end{wrapfigure}

        % import matplotlib.pyplot as plt
        % import numpy

        % cache = {}

        % def f(x, c=0):
        %     if c >= 100:
        %         return x

        %     if x not in cache:
        %         cache[x] = 0 if x <= 0 else \
        %             1/2 * f(3 * x, c+1) if x <= 1/3 else \
        %             1/2 if x <= 2/3 else \
        %             1/2 + 1/2 * f(3 * x - 2, c+1) if x <= 1 else \
        %             1 

        %     return cache[x]

        % x_domain = numpy.arange(0, 1.001, 0.0001)
        % plt.plot(x_domain, [f(i) for i in x_domain])
        % plt.title("Канторова лестница")
        % plt.grid(True)
        % plt.show()

        \Exs Сингулярное распределение получим, если возьмем случайную величину, функция распределения которой - 
        лестница Кантора
    
        $F_\xi(x) = \begin{cases}0 \quad x \leq 0, \\ \frac{1}{2}F(3x) \quad 0 < x \leq \frac{1}{3}, \\ \frac{1}{2} \quad \frac{1}{3} < x \leq \frac{2}{3}, \\ \frac{1}{2} + \frac{1}{2}F(3x - 2) \quad \frac{2}{3} < x \leq 1, \\ 1 \quad x > 1\end{cases}$
    \end{minipage}

    \begin{MyTheorem}
        \ThNs{Лебега}

        $\letsymbol F_\xi(x)$ - функция распределения $\xi$. Тогда $F_\xi(x) = p_1 F_1(x) + p_2 F_2(x) + p_3 F_3(x)$, где $p_1 + p_2 + p_3 = 1$

        $F_1$ - функция дискретного распределения

        $F_2$ - функция абсолютно непрерывного распределения

        $F_3$ - функция сингулярного распределения

        То есть существуют только дискретное, абсолютно непрерывное, сингулярное распределения и их смеси
    \end{MyTheorem}
% end probtheory_2024_10_22.tex

% begin probtheory_2024_10_29.tex

    \section{Лекция 9}

    \subsection{Стандартное абсолютно непрерывное распределение}

    \subsubsection{I. Равномерное распределение}

    \Defs Случайная величина $\xi$ имеет равномерное распределение $\xi \in u(a, b)$, если ее плотность
    на этом отрезке постоянна

    Получаем функцию плотности $f_\xi(x) = \begin{cases}0, \quad x < a \\ \frac{1}{b - a}, \quad a \leq x < b \\ 0 \quad x \geq b\end{cases}$ \hfill {\scriptsize $\frac{1}{b - a}$ из усл. нормировки}

    \includegraphics[height=4cm]{probtheory/images/probtheory_2024_10_29_1}

    Из этого функция распределения $F(x) = \int_{-\infty}^\infty f(x)dx = \begin{cases}0, \quad x < a \\ \frac{x - a}{b - a}, \quad a \leq x < b \\ 1 \quad x \geq b\end{cases}$

    \includegraphics[height=4cm]{probtheory/images/probtheory_2024_10_29_2}

    \textbf{Числовые характеристики}:

    $E\xi = \int_{-\infty}^\infty x f(x) dx = \int_a^b x \frac{1}{b - a} dx = \frac{1}{b - a} \frac{x^2}{2} \Big|_a^b = \frac{b^2 - a^2}{2(b - a)} = \frac{a + b}{2}$

    $E\xi^2 = \int_{-\infty}^\infty x^2 f(x) dx = \int_a^b x^2 \frac{1}{b - a} dx = \frac{1}{b - a} \frac{x^3}{3} \Big|_a^b = \frac{b^3 - a^3}{3(b - a)} = \frac{b^2 + ab + a^2}{3}$

    $D\xi = E\xi^2 - (E\xi)^2 = \frac{b^2 + ab + a^2}{3} - \left(\frac{a + b}{2}\right)^2 = \frac{b^2 - 2ab + a^2}{12} = \frac{(b - a)^2}{12}$
    
    $\sigma = \sqrt{D\xi} = \frac{b - a}{2\sqrt{3}}$

    $p(\alpha < \xi < \beta) = \frac{\beta - \alpha}{b - a}$ при условии, что $\alpha, \beta \in [a, b]$

    \Nota Примеры равномерного распределения: задача со временем, датчики случайных чисел имеют стандартное равномерное распределение $u(0, 1)$

    \subsubsection{II. Показательное распределение}

    \Defs Случайная величина $\xi$ имеет показательное (или экспоненциальное) распределение с параметром $\alpha > 0$ (обозн. $\xi \in E_\alpha$),
    если ее плотность имеет вид:

    \[f_\xi(x) = \begin{cases}0, \quad x < 0 \\ \alpha e^{-\alpha x}, \quad x \geq 0\end{cases}\]

    \includegraphics[height=4cm]{probtheory/images/probtheory_2024_10_29_3}

    Функция распределения $F_\xi(x) = \begin{cases}0, \quad x < 0 \\ \int_0^x \alpha e^{-\alpha x} = 1 - e^{-\alpha x}, \quad x \geq 0\end{cases}$

    \includegraphics[height=4cm]{probtheory/images/probtheory_2024_10_29_4}

    \textbf{Числовые характеристики}:

    $E\xi = \int_{-\infty}^\infty x f(x) dx = \int_0^\infty x \alpha e^{-\alpha x} dx = \left[\begin{matrix}u = x & du = dx \\ dv = \alpha e^{-\alpha x} \alpha x & v = -e^{-\alpha x}\end{matrix}\right] = -xe^{-\alpha x} \Big|_0^\infty + \int_0^\infty e^{-\alpha x} dx = 
    -\lim_{x \to \infty} \frac{x}{e^{\alpha x}} - \frac{1}{\alpha} e^{-\alpha x} \Big|_0^\infty = -\lim_{x \to \infty} \frac{1}{\alpha e^{\alpha x}} - \frac{1}{\alpha} (\lim_{x \to \infty} e^{-\alpha x} - 1) = \frac{1}{\alpha}$


    $E\xi^2 = \int_{-\infty}^\infty x^2 f(x) dx = \int_0^\infty x^2 \alpha e^{-\alpha x} dx = \left[\begin{matrix}u = x^2 & du = 2xdx \\ dv = \alpha e^{-\alpha x} \alpha x & v = -e^{-\alpha x}\end{matrix}\right] = -x^2 e^{-\alpha x} \Big|_0^\infty + 2\int_0^\infty x e^{-\alpha x} dx = 
    \frac{2}{\alpha} \int_0^\infty \alpha x e^{-\alpha x} = \frac{2}{\alpha} E\xi = \frac{2}{\alpha^2}$

    $D\xi = E\xi^2 - (E\xi)^2 = \frac{2}{\alpha^2} - \left(\frac{1}{\alpha}\right)^2 = \frac{1}{\alpha^2}$
    
    $\sigma = \sqrt{D\xi} = \frac{1}{\alpha}$

    $p(\alpha < \xi < \beta) = F(b) - F(a) = e^{-a\alpha} - e^{-b\alpha} \quad\quad\quad a, b \geq 0$

    \Nota Из непрерывных случайных величин только показательная обладает свойством нестарения

    \begin{MyTheorem}
        \Ths $\letsymbol \xi \in E_\alpha$. Тогда $p(\xi < x + y \ | \ \xi > x) = p(\xi > y) \quad\quad \forall x, y > 0$
    \end{MyTheorem}

    \begin{MyProof}
        $\Box$

        $p(\xi < x + y \ | \ \xi > x) = p(\xi > y) = \frac{p(\xi > x + y, \xi > x)}{p(\xi > x)} = \frac{1 - p(\xi < x + y)}{1 - p(\xi < x)} = 
        \frac{1 - F(x + y)}{1 - F(x)} = \frac{e^{-\alpha(x + y)}}{e^{-\alpha x}} = e^{-\alpha y} = 1 - (1 - e^{-\alpha y}) = 1 - p(\xi < y) = p(\xi > y)$

        $\Box$
    \end{MyProof}

    \ExNs{1} Время работы надежной микросхемы до поломки

    \ExNs{2} Время между появлениями двух редких событий (через схему Пуассона)

    \Notas Применется в системах массового обслуживания, теория надежности


    \subsubsection{III. Нормальное распределение (Гауссовское)}

    \Def Случайная величина $\xi$ имеет нормальное распределение с параметрами $a$ и $\sigma^2$ (обозн. $\xi \in N(a, \sigma^2)$), если
    ее плотность имеет вид:

    \[f(x) = \frac{1}{\sigma \sqrt{2\pi}} e^{-\frac{(x - a)^2}{2\sigma^2}}, \quad -\infty < x < \infty\]

    \includegraphics[height=4cm]{probtheory/images/probtheory_2024_10_29_5}

    Смысл параметров распределения: $a = E\xi$ - матожидание и медиана, $\sigma$ - СКО, а $D\xi = \sigma^2$

    Функция распределения: $F(x) = \frac{1}{\sigma \sqrt{2\pi}} \int_{-\infty}^x e^{-\frac{(t - a)^2}{2\sigma^2}} dt$

    \includegraphics[height=4cm]{probtheory/images/probtheory_2024_10_29_6}

    Проверим корректность определения - условие нормировки. Покажем, что $\int_{-\infty}^\infty f(x)dx = 1$

    $\int_{-\infty}^\infty \frac{1}{\sigma \sqrt{2\pi}} e^{-\frac{(x - a)^2}{2\sigma^2}} dx = \left[\begin{matrix}t = \frac{x - a}{\sigma \sqrt{2}} & dt = \frac{dx}{\sigma\sqrt{2}} \\ t (\pm \infty) = \pm \infty & dx = \sigma\sqrt{2}dt\end{matrix}\right] = 
    \int_{-\infty}^\infty \frac{1}{\sigma\sqrt{2\pi}} e^{-t^2} \sigma\sqrt{2} dt = \frac{1}{\sqrt{\pi}} \int_{-\infty}^{\infty} e^{-t^2} dt = \frac{1}{\sqrt{\pi}} \sqrt{\pi} = 1$ - верно

    Ясно, что $m_k = \int_{-\infty}^\infty x^k f(x) dx = \int_{-\infty}^\infty x^k \frac{1}{\sigma \sqrt{2\pi}} e^{-\frac{(x - a)^2}{2\sigma^2}} dx$ - интеграл сходится абсолютно для любого $k$ (степень $e$ задавит полином)

    $E\xi = m_1 = \int_{-\infty}^\infty x^k \frac{1}{\sigma \sqrt{2\pi}} e^{-\frac{(x - a)^2}{2\sigma^2}} dx = a$ в силу симметрии

    Найдем дисперсию при помощи дифференцирования интеграла по параметру: 
    
    Из условия нормировки $\int_{-\infty}^\infty \frac{1}{\sigma \sqrt{2\pi}} e^{-\frac{(x - a)^2}{2\sigma^2}} dx = 1$

    $\frac{1}{\sqrt{2\pi}} \int_{-\infty}^\infty e^{-\frac{(x - a)^2}{2\sigma^2}} dx = \sigma$
    
    $\frac{1}{\sqrt{2\pi}} \int_{\infty}^\infty e^{-\frac{(x - a)^2}{2\sigma^2}} \left(-\frac{(x - a)^2}{2} (-\sigma^{-3})\right) dx = 1$
    
    $\frac{1}{\sigma\sqrt{2\pi}} \int_{\infty}^\infty (x - a)^2 e^{-\frac{(x - a)^2}{2\sigma^2}} dx = \sigma^2 = D\xi$, получаем, что $\sigma$ - СКО

    \mediumvspace

    \textbf{Стандартное нормальное распределение}

    \Def Стандартным нормальным распределением называется нормальное распределение с параметрами $a = 0, \sigma^2 = 1$: $\xi \in N(0, 1)$

    Плотность: $\phi(x) = \frac{1}{\sqrt{2\pi}} e^{-\frac{x^2}{2}}$ - функция Гаусса

    $E\xi = 0; \ D\xi = 1$

    Распределение: $F_0(x) = \frac{1}{\sqrt{2\pi}} \int_{-\infty}^x e^{-\frac{z^2}{2}} dz$ - функция стандартного нормального распределения

    Заметим, что $F_0(x) = \frac{1}{\sqrt{2\pi}} \int_{-\infty}^0 e^{-\frac{z^2}{2}} dz + \frac{1}{\sqrt{2\pi}} \int_0^x e^{-\frac{z^2}{2}} dz = \frac{1}{2} + \Phi(x)$, где $\Phi(x)$ - функция Лапласа

    Функция Лапласа нечетная и из соображения симметрии легко вычисляется для отрицательных $x$, однако большинство ПО используют $F_0(x)$
    
    \subsubsection{Связь между нормальным и стандартным нормальным распределениями}

    1) $\letsymbol \xi \in N(a, \sigma^2)$. Тогда $F_\xi(x) = F_0\left(\frac{x - a}{\sigma}\right)$

    \begin{MyProof}
        $\Box$
        
        $F_\xi(x) = \frac{1}{\sigma\sqrt{2\pi}} \int_{-\infty}^x e^{-\frac{(t - a)^2}{2\sigma^2}} dt = \left[\begin{matrix}z = \frac{t - a}{\sigma} & t = \sigma z + a & dt = \sigma dz \\ z (-\infty) = -\infty & z(x) = \frac{x - a}{\sigma} & \end{matrix}\right] = 
        \frac{1}{\sigma\sqrt{2\pi}} \int_{-\infty}^\frac{x - a}{\sigma} e^{-\frac{z^2}{2}} \sigma dz = \frac{1}{\sqrt{2\pi}} \int_{-\infty}^\frac{x - a}{\sigma} e^{-\frac{z^2}{2}} dz = F_0\left(\frac{x - a}{\sigma}\right)$
        
        $\Box$
    \end{MyProof}

    2) Если $\xi \in N(a, \sigma^2)$, то $\eta = \frac{\xi - a}{\sigma} \in N(0, 1)$ (процесс $\xi \to \eta$ называется стандартизацией)

    \begin{MyProof}
        $\Box$
        
        $F_\eta(x) = p(\eta < x) = p\left(\frac{\xi - a}{\sigma} < x\right) = p(\xi < \sigma x + a) = F_\xi(\sigma x + a) = F_0\left(\frac{\sigma x + a - a}{\sigma}\right) = F_0(x)$, так как $F_\eta(x) = F_0(x)$, то $\eta \in N(0, 1)$
        
        $\Box$
    \end{MyProof}

    3) $\letsymbol \xi \in N(a, \sigma^2)$. Тогда $p(\alpha < \xi < \beta) = \Phi\left(\frac{\beta - a}{\sigma}\right) - \Phi\left(\frac{\alpha - a}{\sigma}\right)$

    \begin{MyProof}
        $p(\alpha < \xi < \beta) = F_\xi(\beta) - F_\xi(\alpha) = F_0\left(\frac{\beta - a}{\sigma}\right) - F_0\left(\frac{\alpha - a}{\sigma}\right) = \Phi\left(\frac{\beta - a}{\sigma}\right) - \Phi\left(\frac{\alpha - a}{\sigma}\right)$
    \end{MyProof}

    4) Вероятность попадания в симметричный интервал (вероятность отклонения случайной величины от матожидания) 
    $p(|\xi - a| < t) = 2\Phi\left(\frac{t}{\sigma}\right)$

    \begin{MyProof}
        $p(|\xi - a| < t) = p(-t < \xi - a < t) = p(a - t < \xi < a + t) = \Phi\left(\frac{a + t - a}{\sigma}\right) - \Phi\left(\frac{a - t - a}{\sigma}\right) = \Phi\left(\frac{t}{\sigma}\right) - \Phi\left(-\frac{t}{\sigma}\right) = 2\Phi\left(\frac{t}{\sigma}\right)$
    \end{MyProof}

    \Notas Если через $F_0(x)$, то $p(|\xi - a| < t) = 2\F_0\left(\frac{t}{\sigma}\right) - 1$

    5) Правило 3 \enquote{сигм}: $p(|\xi - a| < 3\sigma) \approx 0.9973$ - попадание случайной величины нормального распределения в интервал $(a - 3\sigma, a + 3\sigma)$ близко к 1

    \begin{MyProof}
        $p(|\xi - a| < 3\sigma) = 2\Phi\left(\frac{3\sigma}{\sigma}\right) = 2\Phi(3) = 2 \cdot 0.49685 = 0.9973$
    \end{MyProof}

    6) Свойство линейности: если случайная величина $\xi \in N(a, \sigma^2)$, то $\eta = \gamma \xi + b \in N(a \gamma + b, \gamma^2 \sigma^2)$ (можем доказать при помощи свойств ранее, но мы докажем позже, используя другие методы)

    7) Устойчивость относительно суммирования: если случайные величины $\xi_1 \in N(a_1, \sigma_1^2), \xi_2 \in N(a_2, \sigma_2^2)$, и они независимы, то $\xi_1 + \xi_2 \in N(a_1 + a_2, \sigma^2_1 + \sigma^2_2)$

    \subsubsection{Коэффициенты асимметрии и эксцесса}

    \DefN{1} Асимметрией распределения называется число $A_s = E\left(\frac{\xi - a}{\sigma}\right)^3 = \frac{\mu^3}{\sigma^3}$

    \DefNs{2} Эксцессом распределения называется число $E_s = E\left(\frac{\xi - a}{\sigma}\right)^4 - 3 = \frac{\mu^4}{\sigma^4} - 3$

    \Notas Если случайная величина $\xi \in N(a, \sigma^2)$, то $A_s = E_s = 0$, таким образом, отличие этих характеристик от нуля характеризирует 
    степень отклонения распределения. Благодаря этим и другим параметрам, можно проверять на практике, является ли распределение нормальным 
% end probtheory_2024_10_29.tex

% begin probtheory_2024_11_05.tex

    \section{Лекция 10}

    \subsection{Преобразование случайных величин}

    \subsubsection{Стандартизация случайной величины}

    \Def Пусть имеется случайная величина $\xi$. Соответствующей ей стандартной величиной называется
    случайная величина $\eta = \frac{\xi - E\xi}{\sigma}$

    \textbf{Свойства}:

    $E\eta = 0; D\eta = 1$

    \begin{MyProof}
        $E\eta = E \frac{\xi - E\xi}{\sigma} = \frac{1}{\sigma} (E\xi - E\xi) = 0$

        $D\eta = D \frac{\xi - E\xi}{\sigma} = \frac{1}{\sigma^2} D\xi = 1$
    \end{MyProof}

    Стандартизованная случайная величина не имеет единиц измерения, таким образом, ее свойства от них не зависят

    \mediumvspace

    \underline{Задача}: пусть имеется функция $g(x)$ и случайная величина $\xi$, $\eta = g(\xi)$. Определить ее характеристики

    \Nota Если $\xi$ - дискретная случайная величина, то ее законы распределения находятся просто: значения $x_i$ в верхней строке заменяем $g(x_i)$, вероятности остаются прежние.
    Поэтому будем рассматривать непрерывной случайной величины $\xi$

    \Notas Возможна ситуация, когда $\xi$ - абсолютно непрерывная случайная величина, $g(x)$ - непрерывна, но $g(\xi)$ имеет дискретное распределение

    \subsubsection{Линейное преобразование}

    \begin{MyTheorem}
        \Ths Пусть $\xi$ имеет плотность $f_\xi(x)$, тогда $\eta = a\xi + b$, где $a \neq 0$, имеет плотность $f_\eta(x) = \frac{1}{|a|}f_\xi(\frac{x - b}{a})$
    \end{MyTheorem}

    \begin{MyProof}
        Пусть $a > 0$, тогда $F_\eta(x) = p(\eta < x) = p(a\xi + b < x) = p(\xi < \frac{x - b}{a}) = \int_{-\infty}^{\frac{x - b}{a}} f_\xi(t) dt = 
        \left[\begin{matrix}t = \frac{y - b}{a} & dt = \frac{1}{a} dy & y = at + b \\ y(-\infty) = -\infty & y(\frac{x - b}{a}) = x\end{matrix}\right] = 
        \int_{-\infty}^x \frac{1}{a} f_\xi(\frac{y - b}{a}) dy \Longrightarrow \eta = \frac{1}{|a|} f_\xi (\frac{x - b}{a})$

        Пусть $a < 0$, тогда $F_\eta(x) = p(\eta < x) = p(a\xi + b < x) = p(\xi > \frac{x - b}{a}) = \int_{\frac{x - b}{a}}^{\infty} f_\xi(t) dt = 
        \left[\begin{matrix}t = \frac{y - b}{a} & dt = \frac{1}{a} dy & y = at + b \\ y(\infty) = -\infty & y(\frac{x - b}{a}) = x\end{matrix}\right] = 
        -\int_{-\infty}^x \frac{1}{a} f_\xi(\frac{y - b}{a}) dy \Longrightarrow \eta = \frac{1}{|a|} f_\xi (\frac{x - b}{a})$
    \end{MyProof}

    \underline{Следствие}

    1) Если $\xi \in N(a, \sigma^2)$, то $\eta = \gamma \xi + b \in N(a\gamma + b; \gamma^2 \sigma^2)$

    \begin{MyProof}
        Так как $\xi \in N(a, \sigma^2)$, то $f_\xi(x) = \frac{1}{\sigma\sqrt{2\pi}} e^{-\frac{(x - a)}{2\sigma^2}}, x \in \Real$

        Тогда $f_\eta(x) = \frac{1}{|\gamma|} \cdot \frac{1}{\sigma\sqrt{2\pi}} e^{-\frac{(\frac{x - b}{\gamma} - a)^2}{2\sigma^2}} = \frac{1}{|\gamma|\sigma\sqrt{2\pi}} e^{-\frac{(x - b - a\gamma)^2}{2\sigma^2\gamma^2}} \Longrightarrow \eta \in N(b + a\gamma; \sigma^2\gamma^2)$
    \end{MyProof}

    2) Если $\eta \in N(0, 1)$ - стандартное нормальное распределение, то $\xi = \sigma \eta + a \in N(a, \sigma^2)$

    3) Если $\eta \in U(0, 1)$ - стандартное равномерное распределение и $a > 0$, то $\xi = a\eta + b \in U(b, a + b)$

    4) Если $\xi \in E_\alpha$, то $\alpha \xi \in E_1$

    \subsubsection{Монотонное преобразование}

    \begin{MyTheorem}
        \Ths Пусть $f_\xi(x)$ - плотность случайной величины $\xi$, $g(x)$ - строго монотонная функция. Тогда 
        случайная величина $\eta = g(\xi)$ имеет плотность

        \[f_\eta(x) = |h^\prime(x)| f_\xi(h(x)), \qquad\qquad \text{где } h(g(x)) = x\]
    \end{MyTheorem}

    Если $g(x)$ не является монотонное функцией, то поступаем следующим образом: разбиваем $g(x)$ на интервалы монотонности, 
    для каждого $i$-ого интервала находимся $h_i(x)$ и плотность случайной величины ищем по \textit{формуле Смирнова}: 
    $f_\eta(x) = \sum_{i = 0}^n |h_i^\prime(x)| f_\xi(h_i(x))$
    
    \subsubsection{Квантильное преобразование}

    \begin{MyTheorem}
        \ThNs{1} Пусть функция распределения случайной величины $\xi$ $F_\xi(x)$ - непрерывная функция. 
        Тогда $\eta = F(\xi) \in U(0, 1)$ - стандартное равномерное распределение
    \end{MyTheorem}

    \begin{MyProof}
        Ясно, что $0 \leq \eta \leq 1$

        a) $F(x)$ - строго возрастающая функция. Тогда $\exists F^{-1}(x)$ - обратная, $F_\eta(x) = p(\eta < x) = p(F(\xi) < x) = 
        \begin{cases}0, & x < 0 \\ p(\xi < F^{-1}(x)) = F(F^{-1}(x)) = x, & 0 \leq x \leq 1 \text{ - функция распределения } U(0, 1) \\ 1, & x > 1 \end{cases}$

        б) $F(x)$ - не является строго возрастающей функцией - то есть существуют участки постоянства, в этом случае
        определим $F^{-1}$ как $F^{-1}(x) = \min_{t} (t \ | \ F(t) = x)$ - то есть берем самую левую точку такого интервала

        Тогда снова будет при $0 \leq x \leq 1 \ F_\eta(x) = p(\eta < x) = p(F(\xi) < x) = F(F^{-1}(x)) = x$

    \end{MyProof}

    Сформулируем обратную теорему: пусть $F(x)$ - функция распределения (необязательно непрерывная) случайной величины $\xi$,
    обозначим $F^{-1}(x) = \inf_t (t \ | F(t) \geq x)$.

    В случае непрерывной $F(x)$ это определение совпадет с предыдущем

    \begin{MyTheorem}
        \ThNs{2} Пусть $\eta \in U(0, 1)$ - стандартное равномерное распределение, $F(x)$ - произвольная функция распределения. 
        Тогда $\xi = F^{-1}(\eta)$ имеет функцию распределения $F(x)$
    \end{MyTheorem}

    Данное преобразование $\xi = F^{-1}(\eta)$ называют квантильным

    Доказательство аналогично предыдущей теореме
    
    Смысл: датчики случайных чисел имеют стандартное равномерное распределение, из теоремы следует, что при помощи
    датчика случайных чисел и квантильного преобразования мы сможем смоделировать любое нужно распределение

    \ExN{1} Смоделируем показательное распределение $E_\alpha: \ F_\alpha(x) = \begin{cases}0, & x < 0 \\ 1 - e^{-\alpha x}, & x \geq 0\end{cases}$

    $\eta = 1 - e^{-\alpha x}, \ e^{-\alpha x} = 1 - \eta, \ x = -\frac{1}{\alpha} \ln(1 - \eta)$ - функция, обратная к $F_\alpha(x)$

    Если $\eta \in U(0, 1)$, то $\xi = -\frac{1}{\alpha} \ln(1 - \eta) \in E_\alpha$

    \ExN{2} $\xi \in N(0, 1), \ F_0(x) = \frac{1}{\sqrt{2\pi}} \int_{-\infty}^x e^{-z^2}{2} dz$

    Пусть $F_0^{-1}(x)$ - функция обратная к $F_0(x)$

    Если $\eta \in U(0, 1)$, то $F_0^{-1}(\eta) \in N(0, 1)$

    \subsection{Характеристики преобразованной случайной величины}

    \begin{MyTheorem}
        \Ths Если $\xi$ - дискретная случайная величина, то $Eg(\xi) = \sum_{i = 1}^\infty g(x_i) \cdot p_i = \sum_{i = 1}^\infty g(x_i) \cdot p(\xi = x_i)$

        Для непрерывной случайной величины $Eg(\xi) = \int_{-\infty}^{\infty} g(x) f_\xi(x) dx$
    \end{MyTheorem}

    \subsubsection{Свойства моментов}

    1) Если $\xi \geq 0$, то $E\xi \geq 0$

    2) Если $\xi \leq \eta$, то $E\xi \leq E\eta$

    \begin{MyProof}
        $\xi \leq \eta \Longrightarrow \eta - \xi \geq 0 \Longrightarrow E(\eta - \xi) \geq 0 \Longrightarrow E\eta - E\xi \geq 0 \Longrightarrow E\eta \geq E\xi$
    \end{MyProof}

    3) Если $|\xi| \leq |\eta|$, то $E|\xi|^k \leq E|\eta|^k$

    4) Если существует момент $m_t$ случайной величины $\xi$, то существует $m_s$ при $s < t$ (при условии, что интеграл/сумма сходятся)

    \begin{MyProof}
        Пусть $s < t$. Тогда $|x|^s \leq \min(1, |x|^t) \leq 1 + |x|^t$, так как при $|x| < 1, \ |x|^s \leq 1$ и при $|x| \geq 1, \ |x|^s \leq |x|^t$
    
        $E|\xi|^s \leq E|\xi|^t + 1$ и если $E|\xi|^t$ существует (конечно), то $\exists E|\xi|^s$
    
    \end{MyProof}

    \begin{MyTheorem}
        \ThNs{Неравенство Йенсена} Пусть функция $g(x)$ выпукла вниз, тогда для любой случайной величины $\xi$

        \[Eg(\xi) \geq g(E\xi)\]
    \end{MyTheorem}

    \Nota Если $g(x)$ выпукла вверх, знак неравенства меняется
    
    \begin{MyProof}
        Если $g(x)$ выпукла вниз, то в любой ее точке, можно провести прямую, лежащую не выше графика функции. То есть для 
        любой $x_0$ существует $k(x_0)$ такой, что $g(x) \geq g(x_0) + k(x_0) (x - x_0)$

        Пусть $x_0 = E\xi$, $g(E\xi) \geq g(E\xi) + k(E\xi) (x - E\xi)$

        $Eg(\xi) \geq Eg(E\xi) + k(E\xi) \underset{= 0}{(E\xi - E\xi)}$

        $Eg(\xi) \geq g(E\xi)$
    \end{MyProof}

    % ниче не понял, maybe wrong

    Следствие:

    $Ee^\xi \geq e^{E\xi}, \quad E\xi^2 \geq (E\xi)^2, \quad E|q| \geq |Eq|, \quad E\ln(\xi) \leq \ln(E\xi), \quad E\frac{1}{\xi} \geq \frac{1}{E\xi}$ при $\xi > 0$


    \begin{minipage}{\textwidth}
        \begin{wrapfigure}{r}{0pt}
            \includegraphics[width=7cm]{probtheory/images/probtheory_2024_11_05_1}
        \end{wrapfigure}

        \Ex на формулу Смирнова: дана плотность распределения
    
        $f_\xi(x) = \begin{cases}0, & x < 1 \\ \frac{4}{3x^2}, & 1 \leq x \leq 4, 0 & x > 4\end{cases}$
    
        Найти $f_\eta$ для $\eta = |\xi - 2|$

        \underline{Решение}

        $\xi \in [1, 4], \quad \eta \in [0, 2]$
    \end{minipage}

    \begin{cases}
        0 \leq \eta \leq 1 \Longrightarrow h_1(\eta) = \eta + 2 \text{ и } h_2(\eta) = 2 - \eta & \text{\qquad - 2 ветви} \\ 
        1 < \eta \leq 2 \Longrightarrow h_1(\eta) = \eta + 2 & \text{\qquad - 1 ветвь}
    \end{cases}

    $h_1^\prime(\eta) = 1, h_2^\prime(\eta) = -1 \qquad |h_1^\prime(\eta)| = |h_2^\prime(\eta)| = 1$

    $f_\eta(x) = \sum_i |h_i^\prime(x)| f_\xi(h_i(x))$

    $f_\eta(x) = \begin{cases}0, & x < 0 \\ \frac{4}{3}\left(\frac{1}{(x + 2)^2} + \frac{1}{(2 - x)^2}\right), & 0 \leq x \leq 1 \\ \frac{4}{3}\frac{1}{(x + 2)^2}, & 1 < x \leq 2 \\ 0 & x > 2\end{cases}$
% end probtheory_2024_11_05.tex

% begin probtheory_2024_11_12.tex

    \section{Лекция 11}

    \subsection{Сходимость случайных величин}

    Рассмотрим 3 вида сходимости:

    \begin{itemize}
        \item Сходимость \enquote{почти наверное}

        \Defs Случайная величина $\xi$ имеет свойство $\mathrm{Cond}$ \enquote{почти наверное}, если вероятность $p(\xi \text{ имеет свойство } \mathrm{Cond}) = 1$
    
        \Nota То есть $p(\xi \text{ не имеет свойство } \mathrm{Cond}) = 0$

        $p(\omega \in \Omega \ | \ \xi(\omega) \text{ не имеет св-во } \mathrm{Cond})$

        \Def Последовательность случайных величин $\{\xi_n\}$ сходится \enquote{почти наверное} к случайной величине $\xi$ при $n \to \infty$ ($\xi_n \overset{\text{п. н.}}{\longrightarrow} \xi$), 
        если $p(\omega \in \Omega \ | \ \xi_n(\omega) \underset{n \to \infty}{\longrightarrow} \xi(\omega)) = 1$

        \item Сходимость по вероятности

        \Defs Последовательность случайных величин $\{\xi_n\}$ сходится по вероятности к случайной величине $\xi$ при $n \to \infty$
        ($\xi_n \overset{p}{\longrightarrow} \xi$), если $\forall \varepsilon > 0 \quad p(|\xi_n - \xi| < \varepsilon) \underset{n \to \infty}{\longrightarrow} 1$
        
        \Nota Не надо думать, что из сходимости по вероятности следует сходимости математического ожидания $\xi_n \overset{p}{\longrightarrow} \xi \centernot{\Longrightarrow} E\xi_n \longrightarrow E\xi$

        \begin{MyTheorem}
            \Ths Пусть $|\xi_n| \leq C = \mathrm{const} \quad \forall n$

            Тогда $\xi_n \overset{p}{\longrightarrow} \xi \Longrightarrow E\xi_n \longrightarrow E\xi$
        \end{MyTheorem}

        \item Слабая сходимость

        \Defs Последовательность случайных величин $\xi_n$ слабо сходится к случайной величине $\xi$ при $n \to \infty$
        ($\xi_n \rightrightarrows \xi$), если $F_{\xi_n}(x) \longrightarrow F_\xi(x) \forall x$, где $F_\xi(x)$ - непрерывна
    
    \end{itemize}

    \subsubsection{Связь между видами сходимости}

    \begin{MyTheorem}
        \Ths $\xi_n \overset{\text{п. н.}}{\longrightarrow} \xi \Longrightarrow \xi_n \overset{p}{\longrightarrow} \xi \Longrightarrow \xi_n \rightrightarrows \xi$
    \end{MyTheorem}

    \begin{MyTheorem}
        \Ths Если $\xi_n C = \mathrm{const}$, то $\xi_n \overset{p}{\longrightarrow} C$
    \end{MyTheorem}

    \begin{MyProof}
        Если $\xi_n \rightrightarrows C$, то по определению $F_{\xi_n}(x) \longrightarrow F_C(x) = \begin{cases}0, & x \leq C \\ 1, & x > C\end{cases} \quad \forall x \neq C$

        $\forall \varepsilon > 0 \quad p(|\xi_n - C| < \varepsilon) = p(-\varepsilon < \xi_n - C < \varepsilon) = 
        p(C - \varepsilon < \xi_n < C + \varepsilon) \geq p\left(C - \frac{\varepsilon}{2} < \xi_n < C + \varepsilon\right) =
        F_{\xi_n}(C + \varepsilon) - F_{\xi_n}\left(C - \frac{\varepsilon}{2}\right) = 1 - 0 = 1$

        Так как $p(|\xi_n - C| < \varepsilon) \leq 1$, то по теореме о 2 милиционерах $p(|\xi_n - C| < \varepsilon) \underset{n \to \infty}{\longrightarrow} 1$
        то есть по определению $\xi_n \overset{p}{\longrightarrow} C$
    \end{MyProof}

    \Nota В общем случае не только из слабой сходимости не следует сходимость по вероятности, но и бессмысленно говорить
    об этом, так как слабая сходимость - это сходимость не случайных величин, а их распределений

    \Ex $\letsymbol \xi_n \rightrightarrows \xi \in N(0, 1)$, тогда $\eta = -\xi \in N(0, 1)$, но ясно, что $\xi_n \overset{p}{\longrightarrow} \eta = -\xi$ - неверно 
    
    \subsection{Ключевые неравенства}

    В дальнейшем будем считать, что у случайных величин первый момент существует

    \subsubsection{I. Неравенство Маркова}

    \begin{MyTheorem}
        \Ths $p(|\xi| \geq \varepsilon) \leq \frac{E|\xi|}{\varepsilon} \quad \forall \varepsilon > 0$
    \end{MyTheorem}

    \begin{MyProof}
        $I_A(\omega) = \begin{cases}0, & \omega \notin A \quad - A\text{ нет} \\ 1, & \omega \in A \quad - A\text{ есть}\end{cases}$

        $EI_A = p(A)$

        $|\xi| \geq |\xi| \cdot I(|\xi| \geq \varepsilon) \geq \varepsilon I(|\xi| \geq \varepsilon)$

        $E|\xi| \geq E(\varepsilon \cdot I(|\xi| \geq \varepsilon))$

        $E|\xi| \geq \varepsilon \cdot E(\varepsilon I(|\xi| \geq \varepsilon)) = \varepsilon \cdot p(|\xi| \geq \varepsilon) 
        \Longrightarrow p(|\xi| \geq \varepsilon) \leq \frac{E|\xi|}{\varepsilon}$
    \end{MyProof}

    \subsubsection{II. Неравенство Чебышева}

    \begin{MyTheorem}
        \Ths $P(|\xi - E\xi| \geq \varepsilon) \leq \frac{D\xi}{\varepsilon^2}$
    \end{MyTheorem}

    \begin{MyProof}
        $p(|\xi - E\xi| \geq \varepsilon) = p((\xi - E\xi)^2 \geq \varepsilon^2) \leq \frac{E(\xi - E\xi)^2}{\varepsilon^2} = \frac{D\xi}{\varepsilon}$
    \end{MyProof}

    \subsubsection{III. Правило \enquote{трех сигм}}
    
    \begin{MyTheorem}
        \Ths $P(|\xi - E\xi| \geq 3\sigma) \leq \frac{1}{9}$
    \end{MyTheorem}

    \begin{MyProof}
        По неравенству Чебышева $P(|\xi - E\xi| \geq 3\sigma) \leq \frac{D\xi}{(3\sigma)^2} = \frac{D\xi}{9\sigma^2} = \frac{1}{9}$
    \end{MyProof}

    \subsection{Среднее арифмитическое независимых одинаково распределенных случайных величин}

    Пусть $\xi_1, \xi_2, \dots, \xi_n$ - независимые одинаково распределенные случайные величины с конечным вторым моментом

    Обозначим $a = E\xi_i, d = D\xi_i, \sigma = \sigma_{\xi_i}, \quad 1 \leq i \leq n$

    $S_n = \xi_1 + \dots + \xi_n$ - их сумма

    $\frac{S_n}{n} = \frac{\xi_1 + \dots + \xi_n}{n}$ - среднее арифмитическое

    $E\left(\frac{S_n}{n}\right) = \frac{1}{n} (E\xi_1 + \dots + E\xi_n) = \frac{1}{n} na = a = E\xi_1$ - математическое ожидание не меняется

    $D\left(\frac{S_n}{n}\right) = \frac{1}{n^2} (D\xi_1 + \dots + D\xi_n) = \frac{1}{n^2} nd = \frac{d}{n} = \frac{D\xi_1}{n}$ - дисперсия уменьшилась в $n$ раз

    $\sigma\left(\frac{S_n}{n}\right) = \frac{\sigma}{\sqrt{n}}$ - СКО уменьшилось в $\sqrt{n}$ раз

    \subsection{Законы больших чисел}

    \subsubsection{I. Закон больших чисел Чебышева}

    \begin{MyTheorem}
        \Ths Пусть $\xi_1, \dots, \xi_n, \dots$ - последовательность независимых одинаково распределенных с конечным вторым моментом,
        тогда $\frac{\xi_1 + \dots + \xi_n}{n} \overset{p}{\underset{n \to \infty}{\longrightarrow}} E\xi_1$
    \end{MyTheorem}

    \begin{MyProof}
        Обозначим $a = E\xi_i, d = D\xi_i, \sigma = \sigma_{\xi_i}, \quad 1 \leq i \leq n$

        $S_n = \sum_{i = 1}^n \xi_i$

        Тогда по неравенству Чебышева $p\left(\left|\frac{S_n}{n} - a\right| \geq \varepsilon\right) = p\left(\left|\frac{S_n}{n} - E\left(\frac{S_n}{n}\right)\right| \geq \varepsilon\right) \leq
        \frac{D\left(\frac{S_n}{n}\right)}{\varepsilon^2} = \frac{d}{n\varepsilon^2} \underset{n \to \infty}{\longrightarrow} 0 \Longrightarrow p\left(|\frac{S_n}{n} - a| < \varepsilon\right) \underset{n \to \infty}{\longrightarrow} 1$,
        то есть $\frac{S_n}{n} \overset{p}{\longrightarrow} a$
    \end{MyProof}

    Среднее арифмитическое большое числа независимых одинаковых случайных величин \enquote{стабилизируется} около математического ожидания,
    \enquote{при $n \to \infty$ случайность переходит в закономерность}

    \underline{Статистический смысл}: при большом объеме $n$ статистических данных среднее арифмитическое данных
    дает достаточно точную оценку теоретического математического ожидания

    \Nota При доказательстве получили полезную, хотя и грубую оценку: $p\left(\left|\frac{S_n}{n} - a\right| \geq \varepsilon\right) \leq \frac{D\xi_i}{n\varepsilon^2}$

    \subsubsection{II. Закон больших чисел Бернулли}

    \begin{MyTheorem}
        \Ths Пусть $v_n$ - число успехов из $n$ независимых испытаний, $p = P(A)$ - вероятность успеха при одном испытании.
        Тогда $\frac{v_n}{n} \overset{p}{\longrightarrow} P(A)$
    \end{MyTheorem}

    При этом $P\left(\left|\frac{v_n}{n} - p\right| \leq \varepsilon\right) \leq \frac{p(1 - p)}{n\varepsilon^2}$

    \begin{MyProof}
        $v_n = \xi_1 + \dots + \xi_n$, где $\xi_i \in B_p$ - число успехов при $i$-ом испытании

        $E\xi_i = p; D\xi_i = pq$

        $\frac{v_n}{n} \overset{p}{\longrightarrow} E\xi_1 = p$

        $p\left(\left|\frac{v_n}{n} - p\right| \geq \varepsilon\right) \leq \frac{D\xi_1}{n\varepsilon^2} = \frac{pq}{n\varepsilon^2}$
    \end{MyProof}

    \subsubsection{III. Закон больших чисел Хинчина}

    \begin{MyTheorem}
        \Ths $v_n = \xi_1 + \dots + \xi_n$ последовательность независимых одинаково распределенных случайных величин с конечным первым моментом, тогда
        $\frac{\xi_1 + \dots + \xi_n}{n} \overset{p}{\longrightarrow} E\xi_i$
    \end{MyTheorem}

    \subsubsection{IV. Усиленный закон больших чисел Колмогорова}

    В условиях теоремы Хинчина $\frac{\xi_1 + \dots + \xi_n}{n} \overset{\text{п.н.}}{\longrightarrow} E\xi_1$

    \subsubsection{V. Закон больших чисел Маркова}

    \begin{MyTheorem}
        \Ths Пусть имеется последовательность случайных величин $\xi_1, \dots, \xi_n, \dots$ с конечными вторыми моментами, таких 
        что $D(S_n) = o(n^2)$. Тогда $\frac{S_n}{n} \overset{p}{\longrightarrow} E\left(\frac{S_n}{n}\right)$ или $\frac{\xi_1 + \dots + \xi_n}{n} \overset{p}{\longrightarrow} \frac{1}{n} (E\xi_1 + \dots + E\xi_n)$
    \end{MyTheorem}

    \begin{MyProof}
        По неравенству Чебышева $p\left(\left|\frac{S_n}{n} - E\left(\frac{S_n}{n}\right)\right| \geq \varepsilon\right) \leq \frac{D\left(\frac{S_n}{n}\right)}{\varepsilon^2} = \frac{D(S_n)}{n^2 \varepsilon^2} =
        \frac{1}{\varepsilon^2} \frac{o(n^2)}{n^2} \longrightarrow 0 \Longrightarrow p\left(\left|\frac{S_n}{n} - E\left(\frac{S_n}{n}\right)\right| \leq \varepsilon\right) \longrightarrow 1$
    \end{MyProof}

    \subsection{Центральная предельная теорема}

    \begin{MyTheorem}
        \Ths Центральная предельная теорема (ЦПТ Ляпунова, $\approx$1901 год)

        Пусть $\xi_1, \dots, \xi_n, \dots$ - последовательность независимых одинаково распределенных случайных величин
        с конечной дисперсией ($D\xi_1 < \infty$) и $S_n = \sum_{i = 1}^n \xi_i$. Тогда имеет место слабая сходимость:

        \[\frac{S_n - nE\xi_1}{\sqrt{nD\xi_1}} \rightrightarrows N(0, 1)\]
    \end{MyTheorem}

    Теорема показывает, что стандартизованная сумма слабо сходится к стандартному нормальному распределению

    \Nota Можно представить в ином виде: $\letsymbol a = E\xi_i, \sigma = \sigma_{\xi_i}$, тогда $E\left(\frac{S_n}{n}\right) = a, \sigma\left(\frac{S_n}{n}\right) = \frac{\sigma}{\sqrt{n}}$, а $\frac{\frac{S_n}{n} - a}{\sigma \sqrt{n}} \rightrightarrows N(0, 1)$

    \Nota Другая, грубая, формулировка: $\frac{S_n}{n} \rightrightarrows N\left(a, \frac{\sigma^2}{n}\right)$
% end probtheory_2024_11_12.tex

% begin probtheory_2024_11_19.tex

    \section{Лекция 12}

    \subsection{Совместное распределение случайных величин}

    Пусть $\xi_1, \xi_2, \dots, \xi_n$ заданы на одном и том же вероятностном пространстве $(\Omega, \mathcal{F}, p)$

    \Def Случайным вектором $\vec{\xi} = (\xi_1, \xi_2, \dots, \xi_n)$ называется упорядоченный набор случайных величин, заданных
    на одном вероятностном пространстве

    Случайный вектор задает отображение $(\xi_1, \dots, \xi_n) (\omega) : \Omega \longrightarrow \Real^n$

    Поэтому случайный вектор еще называют многомерной случайной величиной, 
    а соответствующее ей распределение многомерным распределением: 

    $\forall B \in \mathcal{B}(\Real^n) \qquad P(B) = P(\omega \in \Omega \ | \ (\xi_1, \dots, \xi_n) \in B)$

    Таким образом, получили новое вероятностное пространство. В качестве элементарных исходов берем точки многомерного пространства, 
    а $\sigma$-алгебра - многомерное Борелевская $\sigma$-алгебра

    $(\Real^n, \mathcal{B}(\Real^n), P(B))$

    \subsubsection{Функция распределения}

        
    \Def Функцией совместного распределения случайных величин $\xi_1, \xi_2, \dots, \xi_n$ называется функция 
    $F_{\xi_1, \xi_2, \dots, \xi_n}(x_1, x_2, \dots, x_n) = P(\xi_1 < x_1, \xi_2 < x_2, \dots, \xi_n < x_n)$

    \Notas Распределение полностью задается функцией распределения

    \Nota В дальнейшем, в основном, будем рассматривать системы из 2 случайных величин. Функция распределения в данном случае $F_{\xi, \eta}(x, y) = P(\xi < x, \eta < y)$ - вероятность попадания в эту область.

    \begin{center}
        \includegraphics[width=0.55\textwidth]{probtheory/images/probtheory_2024_11_19_1}
    \end{center}


    \subsubsection{Свойства функции распределения}

    \begin{enumerate}
        \item $0 \leq F_{\xi, \eta}(x, y) \leq 1$
        \item $F_{\xi, \eta}(x, y)$ - неубывающая по каждому аргументу
        \item $\lim_{x \to -\infty} F_{\xi, \eta}(x, y) = \lim_{y \to -\infty} F_{\xi, \eta}(x, y) = 0, $
        $\lim_{\substack{x \to \infty \\ y \to \infty}} F_{\xi, \eta}(x, y) = 1$

        \item Восстановление маргинального (частного) распределения: 
        $\lim_{x \to \infty} F_{\xi, \eta}(x, y) = F_\eta(y)$, и наоборот - $\lim_{y \to \infty} F_{\xi, \eta}(x, y) = F_\xi(x)$

        \item $F_{\xi, \eta}(x, y)$ - непрерывна слева по каждому аргументу

        \item $P(x_1 \leq \xi < x_2, y_1 \leq \eta < y_2) = F_{\xi, \eta}(x_2, y_2) - F_{\xi, \eta}(x_2, y_1) - F_{\xi, \eta}(x_1, y_2) + F_{\xi, \eta}(x_1, y_1)$
    \end{enumerate}

    \subsection{Независимость случайных величин}

    \Def Случайные величины $\xi_1, \dots, \xi_n$ независимы в совокупности, если для любого набора Борелевских множеств из
    $\mathcal{B}(\Real^n)$, $B_1, B_2, \dots, B_n$

    $p(\xi_1 \in B_1, \xi_2 \in B_2, \dots, \xi_n \in B_n) = p(\xi_1 \in B_1) \cdot p(\xi_2 \in B_2) \cdot \dots \cdot p(\xi_n \in B_n)$

    \Def Случайные величины $\xi_1, \xi_2, \dots, \xi_n$ попарно независимы, если независимы любые две из них

    \Notas Из независимости в совокупности следует попарная независимость: 

    $\xi_1, \dots, \xi_n$ независимы в совокупности, тогда покажем $\forall i, j \ \xi_i$ и $\xi_j$ - независимы

    Возьмем набор $B_i, B_j \in \mathcal{B}(\Real^n)$, при $k \neq i, j \ B_k = \Real$ \hfill $P(\xi_k \in B_k) = 1$

    Тогда $p(\xi_1 \in B_1, \dots, \xi_n \in B_n) = P(\xi_i \in B_i, \xi_j \in B_j) = P(\xi_i \in B_i) \cdot P(\xi_j \in B_j)$

    \Nota Из попарной независимости не следует независимость в совокупности, как видно из примера Берншейна

    Под независимыми величинами будем понимать независимые в совокупности

    \subsection{Дискретная система двух случайных величин}

    \Def Случайные величины $\xi, \eta$ имеют совместное дискретное распределение, если случайный вектор $(\xi, \eta)$
    принимает не более, чем счетное число значений, то есть существует конечный или счетный набор пар чисел $(x_i, y_i)$, 
    таких что $P(\xi = x_i, \eta = y_i) > 0, \sum_{i, j} P(\xi = x_i, \eta = y_i) = 1$

    Таким образом двумерная дискретная случайная величина задается законом распределения - таблице вероятностей

    \begin{tabular}{c|c|c|c|c}
        $\xi \backslash \eta$ & $y_1$ & $y_2$ & $\dots$ & $y_m$ \\
        \hline
        $x_1$ & $p_{11}$ & $p_{12}$ & $\dots$ & $p_{1m}$ \\
        \hline
        $x_2$ & $p_{21}$ & $p_{22}$ & $\dots$ & $p_{2m}$ \\
        \hline
        $\vdots$ & $\vdots$ & $\vdots$ & $\ddots$ & $\vdots$ \\
        \hline
        $x_n$ & $p_{n1}$ & $p_{n2}$ & $\dots$ & $p_{nm}$ \\
    \end{tabular}

    Условие нормировки: $\sum_{i, j} p_{i, j} = 1$

    Зная общий закон распределения, можно восстановить частное (маргинальное) распределение по формулам: 

    $p_i = \sum_{j = 1}^m p_{i, j} \qquad q_j = \sum_{i = 1}^n p_{i, j}$

    \Def Дискретные случайные величины $\xi_1, \xi_2, \dots, \xi_n$ независимы, если для любых $x_1, x_2, \dots, x_n$ 
    $p(\xi_1 = x_1, \xi_2 = x_2, \dots, \xi_n = x_n) = p(\xi_1 = x_1) \cdot p(\xi_2 = x_2) \cdot \dots \cdot p(\xi_n = x_n)$

    При $n = 2$: $p_{i, j} = p_i \cdot q_j \ \forall i, j$

    \Ex

    \begin{tabular}{c|c|c|c|c}
        $\xi \backslash \eta$ & $-1$ & $0$ & $1$ & $p_i$ \\
        \hline
        $-1$ & $0.1$ & $0.2$ & $0.1$ & $0.4$ \\
        \hline
        $2$ & $0.2$ & $0.3$ & $0.1$ & $0.6$ \\
        \hline
        $q_j$ & $0.3$ & $0.5$ & $0.2$ & $\Sigma = 1$ \\
    \end{tabular}

    Найти маргинальное распределение и проверить независимость случайных величин


    \begin{tabular}{c|c|c}
        $\xi$ & $-1$ & $2$ \\
        \hline
        $p_i$ & $0.4$ & $0.6$  \\
    \end{tabular}

    \begin{tabular}{c|c|c|c}
        $\eta$ & $-1$ & $0$ & $1$ \\
        \hline
        $q_j$ & $0.3$ & $0.5$ & $0.2$  \\
    \end{tabular}

    $p_{11} = 0.1 \neq 0.12 = p_1 \cdot q_1 \qquad \Longrightarrow \xi, \eta$ - зависимы

    \subsection{Абсолютно непрерывная система двух случайных величин}

    \Def Случайные величины $\xi$ и $\eta$ имеют абсолютно непрерывное совместное распределение, если
    $\exists f_{\xi, \eta}(x, y)$, такая что $\forall B \in \mathcal{B}(\Real^2) \ P((\xi, \eta) \in B) = \iint_B f_{\xi, \eta}(x, y) dxdy$

    Функцию $f_{\xi, \eta}(x, y)$ будем называть функцией плотности совместного распределения случайных величин $\xi$ и $\eta$

    \underline{Геометрический смысл} плотности: 

    % картиночка

    \underline{Свойства} плотности:

    \begin{enumerate}
        \item $f_{\xi, \eta}(x, y) \leq 0$
        \item Условие нормировки: $\iint_{\Real^2} f_{\xi, \eta}(x, y) dxdy = 1$
        \item $F_{\xi, \eta} = \int_{-\infty}^x \int_{-\infty}^y f_{\xi, \eta}(x, y) dydx$

        \item $f_{\xi, \eta}(x, y) = \frac{\partial^2 F_{\xi, \eta}(x, y)}{\partial x \partial y}$
        
        \item Если случайные величины $\xi, \eta$ имеют абсолютно непрерывное совместное распределение с плотностью $f(x, y)$, 
        то маргинальное распределение величин $\xi, \eta$ также имеют абсолютно непрерывное распределение
        с плотностями $f_\xi(x) = \int_{-\infty}^\infty f_{\xi, \eta}(x, y) dy, f_\eta(y) = \int_{-\infty}^\infty f_{\xi, \eta}(x, y) dx$

        \begin{MyProof}
            $F_{\xi}(x) = \lim_{y \to \infty} F_{\xi, \eta}(x, y) = \int_{-\infty}^x \int_{-\infty}^\infty f(x, y) dydx$

            Из этого $\int_{-\infty}^\infty f(x, y) = f_\xi(x)$
        \end{MyProof}

        \item Так как вероятность попадания в Борелевские множества полностью задается функцией распределения, 
        то условие независимости случайных величин эквивалентно следующему:

        $\xi_1, \xi_2, \dots, \xi_n$ независимы, если функция общего распределения распадается в произведение 
        отдельных функцию распределения
    
        $F_{\xi_1, \xi_2, \dots, \xi_n}(x_1, x_2, \dots, x_n) = F_{\xi_1}(x_1) \cdot F_{\xi_2}(x_2) \cdot \dots \cdot F_{\xi_n}(x_n)$

        \item \textit{Равносильное определение}: абсолютно непрерывные случайные величины $\xi_1, \dots, \xi_n$ независимы в совокупности тогда и только тогда, 
        когда плотность совместного распределения $f_{\xi_1, \xi_2, \dots, \xi_n}(x_1, x_2, \dots, x_n) = f_{\xi_1}(x_1) \cdot f_{\xi_2}(x_2) \cdot \dots \cdot f_{\xi_n}(x_n)$
    
        \begin{MyProof}
            При $n = 2$ случайные величины $\xi$ и $\eta$ независимы $\Longleftrightarrow F_{\xi, \eta}(x, y) = F_\xi(x) \cdot F_\eta(y) = \int_{-\infty}^x f_\xi(x) dx \cdot \int_{-\infty}^y f_\eta(y) dy = \int_{-\infty}^x \int_{-\infty}^y f_\xi(x) \cdot f_\eta(y) dxdy \Longrightarrow f_{\xi,\eta}(x, y) = f_\xi(x)f_\eta(y)$

            Аналогично для высших размерностей
        \end{MyProof}

    \end{enumerate}

    \Nota Совместное распределение абсолютно непрерывных случайных величин не обязано быть абсолютно непрерывным, оно может быть сингулярным

    \Exs Бросаем точку на отрезок прямой $y = x$ ($0 \leq x, y \leq 1$), $\xi$ - абсцисса точки, $\eta$ - ордината точки

    Случайный вектор $(\xi, \eta)$ имеют сингулярное распределение (непрерывное с нулевой областью) - 
    так как число элементарных исходов несчетно, но мера Лебега в $\Real^2$ отрезка равна 0

    \Nota Совместное распределение $\xi_1, \xi_2, \dots, \xi_n$ будет сингулярным, если одна из координат является функцией других (наблюдается функциональная зависимость)

    \subsection{Многомерное равномерное распределение}

    \Def $\letsymbol D \subset \Real^n$ - Борелевское множество в $\Real^n$ с конечной мерой Лебега ($0 < \lambda(D) < \infty$),
    случайный вектор $(\xi_1, \dots, \xi_n)$ имеет равномерное распределение, если плотность совместного распределения 
    постоянна в данной области и равна нулю вне данной области

    $f_{\xi_1, \dots, \xi_n}(x_1, \dots, \x_n) = \begin{cases}\frac{1}{\lambda(D)}, & \text{если } (x_1, \dots, x_n) \in D \\ 0, & \text{если } (x_1, \dots, x_n) \not\in D\end{cases}$
% end probtheory_2024_11_19.tex

% begin probtheory_2024_11_26.tex

    \section{Лекция 13}

    \subsection{Математическое ожидание и дисперсия случайного вектора}
    
    $\letsymbol \vec{\xi} = (\xi_1, \dots, \xi_n)$ - случайный вектор, 
    $\forall 1 \leq i \leq n \ \xi_i$ - случайная величина

    \Def Математическим ожиданием случайного вектора называется вектор с координатами 
    из математических ожиданий его компонент: $E\vec{\xi} = (E\xi_1, \dots, E\xi_n)$

    \Def Дисперсией (или матрицей ковариаций) случайного вектора называется матрица
    $D\vec{\xi} = E(\vec{\xi} - E\vec{\xi})^T \cdot (\vec{\xi} - E\vec{\xi})$, состоящая
    из элементов $d_{i, j} = \cov (\xi_i, \xi_j)$. В частности $d_{i, i} = \cov (\xi_i, \xi_i) = D\xi_i$

    \subsection{Функции от двух случайных величин}

    \begin{MyTheorem}
        \Ths Пусть $\xi_1, \xi_2$ - случайные величины с общем плотностью $f_{\xi_1, \xi_2}(x, y)$, и есть функция
        $g(x, y) \ : \ \Real^2 \rightarrow \Real$. Тогда случайная величина $\eta = g(\xi_1, \xi_2)$ имеет
        функцию распределения $F_{\eta}(z) = \iint_{D_z} f(x, y)dxdy$, 
        где $D_z = \{(x, y) \in \Real^2 \ | \ g(x, y) < z\}$
    \end{MyTheorem}

    \begin{MyProof}
        $F_\eta = p(\eta < z) = p(g(\xi_1, \xi_2) < z) = p((\xi_1, \xi_2) \in D_z) = \iint_{D_z} f(x, y) dxdy$
    \end{MyProof}

    \begin{minipage}{\textwidth}
        \begin{wrapfigure}{r}{0pt}
            \includegraphics[width=5cm]{probtheory/images/probtheory_2024_11_26_1}
        \end{wrapfigure}

        \ExN{Задача о встрече} двое договорились встретится между 12:00 и 13:00. Случайная величина $\eta$ - 
        время ожидания. Найти функцию распределения

        $\xi_1$ - время прихода первого, $\xi_2$ - второго; $\xi_1, \xi_2 \in U(0, 1)$, они независимы, 
        $\forall x, y \in [0, 1] \ f_{\xi_1}(x) = 1, f_{\xi_2}(y) = 1$

        Поэтому $f_{\xi_1, \xi_2}(x, y) = f_{\xi_1}(x) f_{\xi_2}(y) = 1, (x, y) \in [0, 1] \times [0, 1]$

        $\eta = |\xi_1 - \xi_2| \Longrightarrow D_z = \{(x, y) \in \Real^2 \ | \ |x - y| < z\}$

        $F_\eta = \iint_{D_z} f_{\xi_1, \xi_2}(x, y) dxdy = \iint_{D_z} dxdy = 1 - 2 \cdot \frac{1}{2} (1 - z)^2 = 
        2z - z^2, \ z \in [0, 1]$
    \end{minipage}

    \begin{MyTheorem}
        \Ths $\letsymbol \xi_1, \xi_2$ - независимые абсолютно непрерывные случайные величины с плотностями
        $f_{\xi_1}(x)$ и $f_{\xi_2}(y)$

        Тогда плотность суммы $\xi_1 + \xi_2$ равна $f_{\xi_1 + \xi_2}(t) = \int_{-\infty}^\infty 
        \underset{\text{т. н. свертка}}{\underbrace{f_{\xi_1}(x) f_{\xi_2}(t - x)}} dx$
    \end{MyTheorem}

    \begin{MyProof}
        \begin{minipage}{\textwidth}
            \begin{wrapfigure}{r}{0pt}
                \includegraphics[width=5cm]{probtheory/images/probtheory_2024_11_26_2}
            \end{wrapfigure}
            
            Так как случайные величины $\xi_1$ и $\xi_2$ независимы, то $f_{\xi_1, \xi_2}(x, y) = f_{\xi_1}(x) f_{\xi_2}(y)$

            И согласно предыдущей теореме $F_{\xi_1 + \xi_2}(z) = \iint_{D_z} f_{\xi_1, \xi_2}(x, y) dxdy = 
            \iint_{D_z} f_{\xi_1}(x)f_{\xi_2}(y) dxdy$, где $D_z = \{(x, y) \in \Real^2 \ | \ x + y < z\}$

            $F_{\xi_1 + \xi_2}(z) = \int_{-\infty}^\infty dx \int_{-\infty}^{z - x} f_{\xi_1}(x)f_{\xi_2}(y) dy = \\
            \left[\begin{matrix}y = t - x; & dy = dt; & t = y + x \\ t(-\infty) = -\infty; & t(z - x) = z & \end{matrix}\right] = \\
            \int_{-\infty}^{\infty} f_{\xi_1}(x) dx \int_{-\infty}^z f_{\xi_2}(t - x) dt = \\
            \int_{-\infty}^z \left(\int_{-\infty}^\infty f_{\xi_1}(x)f_{\xi_2}(t - x) dx\right) dt \Longrightarrow
            f_{\xi_1 + \xi_2}(t) = \int_{-\infty}^\infty f_{\xi_1}(x)f_{\xi_2}(t - x) dx$
        \end{minipage}
    \end{MyProof}

    Следствие: сумма двух независимых абсолютно непрерывных случайных величин также имеет абсолютно 
    непрерывное распределение

    \Notas Условие независимости существенно, контр-пример: $\xi_1; \xi_2 = -\xi_1$, тогда $\xi_1 + \xi_2 \equiv 0$

    \subsection{Сумма стандартных распределений. Устойчивость относительно суммирования}

    \Def Если сумма двух независимых случайных величин одного типа распределения также будет этого же типа,
    то говорят, что распределение устойчиво относительно суммирования

    \ExN{1} $\xi \in B_{n, p}; \eta \in B_{m, p}$. Тогда ясно, что $\xi + \eta \in B_{n + m, p}$ 
    (по определению биномиального распределения $B_{n, p}$ - число успехов из $n$ испытаний, где $p$ - вероятность успеха)

    \ExN{2} $\xi \in \Pi_{\lambda}, \eta \in \Pi_{\mu}$, они независимы. Тогда $\xi + \eta \in \Pi_{\lambda + \mu}$

    \begin{MyProof}
        $\xi + \eta = 0, 1, 2, 3, \dots \quad \letsymbol k \geq 0$. 
        Тогда $p(\xi + \eta = k) = \sum^k_{i = 0} P(\xi = i, \eta = k - i) = 
        \sum^k_{i = 0} P(\xi = i) P(\eta = k - i) = \sum_{i = 0}^k \frac{\lambda^i}{i!} e^{-\lambda} \frac{\mu^{k - i}}{(k - i)!} e^{-\mu} = 
        e^{-\lambda - \mu} \sum_{i = 0}^k \frac{\lambda^i \mu^{k - i}}{i! (k - i)!} = 
        e^{-\lambda - \mu} \frac{1}{k!} \sum_{i = 0}^k \frac{\lambda^i \mu^{k - i} k!}{i! (k - i)!} = 
        e^{-\lambda - \mu} \frac{1}{k!} \sum_{i = 0}^k \lambda^i \mu^{k - i}C_k^i = 
        e^{-\lambda - \mu} \frac{(\lambda + \mu)^k}{k!} \Longrightarrow \xi + \eta \in \Pi_{\lambda + \mu}$
    \end{MyProof}

    \ExN{3} $\xi, \eta \in N(0, 1)$ и независимы. Тогда $\xi + \eta \in N(0, 2)$

    \begin{MyProof}
        $f_{\xi}(x) = \frac{1}{\sqrt{2\pi}} e^{-\frac{x^2}{2}}; f_\eta(y) = \frac{1}{\sqrt{2\pi}} e^{-\frac{y^2}{2}}$

        По формуле свертки $f_{\xi + \eta}(t) = \int_{-\infty}^{\infty} \frac{1}{\sqrt{2\pi}} e^{-\frac{x^2}{2}} \frac{1}{\sqrt{2\pi}} e^{-\frac{(t - x)^2}{2}} = 
        \frac{1}{2\pi} \int e^{-(x^2 - tx + \frac{t^2}{2})} = \frac{1}{2\pi} e^{-\frac{t^2}{4}} \int_{-\infty}^\infty e^{-(x^2 - tx + \frac{t^2}{4})} dx = 
        \frac{1}{2\pi} e^{-\frac{t^2}{4}} \int_{-\infty}^\infty e^{-(x - \frac{t}{2})^2} d(x - \frac{t}{2}) = 
        \frac{1}{2\pi} e^{-\frac{t^2}{4}} \sqrt{\pi} = \frac{1}{\sqrt{2}\sqrt{2\pi}} e^{-\frac{t^2}{2(\sqrt{2})^2}} 
        \Longrightarrow \xi + \eta \in N(0, 2)$
    \end{MyProof}

    \ExN{4} В общности для независимых $\xi \in N(a_1, \sigma^2_1), \eta \in N(a_2, \sigma_2^2) \ \xi + \eta \in N(a_1 + a_2, \sigma_1^2 + \sigma_2^2)$ 

    \ExN{5} Равномерное распределение неустойчиво относительно суммирования, контрпример:

    $\xi, \eta \in U(0, 1)$ - независимы

    $\forall x, y \in [0, 1] \ f_{\xi}(x) = 1, f_\eta(y) = 1$ и $f_{\xi, \eta}(x, y) = 1$

    По первой теореме $F_{\xi, \eta}(x, y) = \iint_{D_z} f_{\xi, \eta}(x, y) dxdy = \iint_{D_z} dxdy = S_{D_z}$, где $D_z = \{(x, y) \ | \ x + y < z\}$

    \mediumvspace

    \begin{multicols}{2}
        \begin{center}
            \includegraphics[height=6cm]{probtheory/images/probtheory_2024_11_26_3}

            а) $0 < z \leq 1$
        \end{center}

        \begin{center}
            \includegraphics[height=6cm]{probtheory/images/probtheory_2024_11_26_4}

            б) $1 < z \leq 2$
        \end{center}
    \end{multicols}

    \smallvspace

    $S_{D_z} = \begin{cases}0, & z < 0 \\ \frac{z^2}{2}, & 0 \leq z \leq 2 \\ 1 - \frac{1}{2}(2 - z)^2 & 1 \leq z \leq 2 \\ 0, & z > 2\end{cases}$

    $f_{\xi + \eta}(z) = \begin{cases}0, & z < 0 \\ z, & 0 \leq z \leq 2 \\ 2 - z & 1 \leq z \leq 2 \\ 0, & z > 2\end{cases} \not\equiv C \Longrightarrow \xi + \eta$ не имеют равномерное распределение

    \Nota FUN FACT: сумма нескольких величин с равномерным распределением приближается к нормальному распределению

    \subsection{Условное распределение}

    \Def Условным распределением случайной величины из системы случайных величин $(\xi, \eta)$ 
    называется ее распределение, найденное при условии, что другая случайная величина приняла 
    определенное значение. Обозначается $\xi | \eta = y$

    \DefN{A}: Условным математическим ожиданием (обозначается $E(\xi | \eta = y)$) называется 
    математическим ожиданием случайной величины $\xi$ при соответствующем условном распределении

    \subsubsection{I. Условное распределение в дискретной системе двух случайных величин}

    Пусть $(\xi, \eta)$ задана законом распределения:

    \begin{tabular}{c|c|c|c|c}
        $\xi \backslash \eta$ & $y_1$ & $y_2$ & $\dots$ & $y_m$ \\
        \hline
        $x_1$ & $p_{11}$ & $p_{12}$ & $\dots$ & $p_{1m}$ \\
        \hline
        $x_2$ & $p_{21}$ & $p_{22}$ & $\dots$ & $p_{2m}$ \\
        \hline
        $\vdots$ & $\vdots$ & $\vdots$ & $\ddots$ & $\vdots$ \\
        \hline
        $x_n$ & $p_{n1}$ & $p_{n2}$ & $\dots$ & $p_{nm}$ \\
    \end{tabular}

    Формула условной вероятности: $P(A | B) = \frac{P(AB)}{P(B)}$

    Вероятности условных распределений считаем по формулам:

    $\xi | \eta = y_j$: $p_i = p(\xi = x_i \ | \ \eta = y_j) = \frac{p(\xi = x_i, \eta = y_j)}{p(\eta = y_j)} = \frac{p_{ij}}{q_j}$

    $\eta | \xi = x_i$: $q_j = p(\eta = y_j \ | \ \xi = x_i) = \frac{p(\xi = x_i, \eta = y_j)}{p(\xi = x_i)} = \frac{p_{ij}}{p_i}$

    То есть вероятность в соответствующем столбце делим на 

    \subsubsection{II. Условное распределение в непрерывной системе двух случайных величин}

    Пусть $(\xi, \eta)$ задана плотностью $f_{\xi, \eta}(x, y)$ совместного распределения, тогда плотность 
    условного распределения $\xi | \eta = y$: 

    $f(x | y) = \frac{f_{\xi, \eta}(x, y)}{\int_\Real f_{\xi, \eta}(x, y)dx} = \frac{f_{\xi, \eta}(x, y)}{f_{\eta}(y)}$

    \Def Функция $f(x | y) = \frac{f_{\xi, \eta}(x, y)}{f_{\eta}(y)}$ называется условной плотностью

    \Def Условное математические ожидание вычисляется по формуле $E(\xi | \eta = y) = \int_{-\infty}^\infty xf(x | y)dx$

    Аналогично $E(\eta | \xi = x) = \int_{-\infty}^\infty yf(y | x)dy$

    \Nota При фиксированном значении $x$ $f(y | x)$ зависит только от $y$, а $E(\eta | \xi = x) \in \Real$. 
    Если рассматривать $x$ как переменную, то условное математическое ожидание $E(\eta | \xi = x)$ является
    функцией от $x$ и называется функцией регрессии $\eta$ на $\xi$. График такой функции называют линией регрессии

    \Nota Так как значение $x$ - значение случайной величины $\xi$, то условное матожидание $E(\eta | \xi = x)$ 
    можно рассматривать как случайную величину
% end probtheory_2024_11_26.tex

% begin probtheory_2024_12_03.tex





\section{Лекция 14}

\subsection{Пространство случайных величин}

\begin{minipage}{\textwidth}
    \begin{wrapfigure}{r}{0pt}
        \includegraphics[width=5cm]{probtheory/images/probtheory_2024_12_03_1}
    \end{wrapfigure}

    \Nota Если две случайных величин $\xi \overset{\text{п.н.}}{=} \eta$, то считаем, что $\xi = \eta$

    Пусть имеется вероятностное пространство $(\Omega, \mathcal{F}, P)$

    Введем пространство $L_2 (\Omega, \mathcal{F}, P) = \{\xi \ | \ D\xi < \infty\}$ - множество случайных величин 
    на данном пространстве с конечной дисперсией

    Ясно, что $L_2$ - линейное пространство. Введем на нем скалярное произведение

    \Def Скалярным произведением случайных величин $\xi$ и $\eta$ из $L_2(\Omega, \mathcal{F}, P)$ 
    называется число $(\xi, \eta) = E(\xi\eta)$

    \Nota Если $(\xi, \eta)$ - дискретная система случайных величин ($p(\xi = x_i, \eta = y_i) = p_{ij}$), 
    то $E(\xi\eta) = \sum_{i, j} x_i y_j p_{ij}$
\end{minipage}

Если же $(\xi, \eta)$ - непрерывная система с плотностью $f_{\xi, \eta}(x, y)$, 
то $E(\xi\eta) = \iint_{\Real^2} xy f_{\xi, \eta}(x, y) dxdy$

\mediumvspace

\textbf{Свойства}:

\begin{enumerate}
    \item $(\xi, \eta) = (\eta, \xi)$

    \item $(C\xi, \eta) = C(\xi, \eta)$

    \item $(\xi_1 + \xi_2, \eta) = (\xi_1, \eta) + (\xi_2, \eta)$

    \item $(\xi, \xi) \geq 0$

    \item $(\xi, \xi) = 0 \Longrightarrow \xi = 0$ п.н.
\end{enumerate}

То есть это действительно скалярное произведение

\Def Норма вектора равна числу $\|\xi\| = \sqrt{(\xi, \xi)}$

\Def Метрикой (расстоянием) между случайными величинами называют число $d(\xi, \eta) = \|\xi - \eta\|$

\begin{MyTheorem}
    \Ths Неравенство Коши-Буняковского-Шварца

    Пусть случайные величины $\xi$ и $\eta$ имеют конечный второй момент, тогда 
    $|E(\xi, \eta)| \leq \sqrt{E\xi^2 \cdot E\eta^2}$ (или $|(\xi, \eta)| \leq \|\xi\|\cdot\|\eta\|$)

    Причем $|E(\xi, \eta)| = \sqrt{E\xi^2 \cdot E\eta^2} \Longleftrightarrow \eta = C\xi$, где $C = \mathrm{const}$
\end{MyTheorem}

\begin{MyProof}
    $P_2(x) = E(x\xi - \eta)^2 = x^2 E\xi^2 - 2xE(\xi\eta) + E\eta^2 \geq 0 \Longrightarrow D = 4(E(\xi\eta))^2 - 
    4 E\xi^2 - E\eta^2 \leq 0 \Longrightarrow |E(\xi\eta)| \leq \sqrt{E\xi^2 \cdot E\eta^2}$
    
    $|E(\xi, \eta)| = \sqrt{E\xi^2 - E\eta^2} \Longrightarrow D = 0 \Longrightarrow \exists$ какая-либо точка касания $C$, 
    из этого $E(C\xi - \eta)^2 = 0 \Longrightarrow C\xi - \eta = 0 \Longleftrightarrow \eta = C\xi \text{ п.н. }$
\end{MyProof}

\subsection{Условное математическое ожидание}

В $L_2(\Omega, \mathcal{F}, P)$ возьмем линейное подпространство $L(\eta) = \{g(\eta) \ | \ Dg(\eta) < \infty\}$

\DefN{B} Условное математическое ожидание (УМО, обозначается $E(\xi|\eta) = \hat{\xi}$) случайной величины $\xi$
относительно случайной величины $\eta$ называется ортогональная проекция случайной величины $\xi$ на $L(\eta)$ 

\mediumvspace

\textbf{Свойства}:

\begin{enumerate}

    \item Тождество ортопроекций: $\letsymbol \hat{\xi} \in L(\eta)$, тогда $\hat{\xi} = E(\xi|\eta) \Longleftrightarrow E(\xi\cdot g(\eta)) = E(\hat{\xi}\cdot g(\eta)) \ \forall g(\eta) \in L(\eta)$

    \begin{MyProof}
        $\hat{\xi} = E(\xi|\eta) \Longleftrightarrow (\xi - \hat{\xi}) \perp L(\eta) \Longleftrightarrow 
        (\xi - \hat{\xi}, g(\eta)) = 0 \ \forall g(n) \in L(\eta) \Longleftrightarrow E(\xi\cdot g(\eta)) = E(\hat{\xi}\cdot g(\eta))$
    \end{MyProof}

    \item Формула полного математического ожидания

    $E\xi = E(E(\xi|\eta))$ или $E\xi = E\hat{\xi}$

    \Nota При распределении Бернулли получаем обычную формулу полной вероятности

    \begin{MyProof}
        Верно из тождества ортопроекций при $g(\eta) = 1$
    \end{MyProof}

    \item Линейность: $E(C_1\xi_1 + C_2\xi_2 \ | \ \eta) = C_1 E(\xi_1|\eta) + C_2 E(\xi_2|\eta)$

    \item Если $\xi$ и $\eta$ независимы, то $E(\xi|\eta) = E\xi$

    \begin{MyProof}
        $\xi, \eta$ независимы $\Longrightarrow \xi$ и $g(\eta)$ независимы

        Из этого $E(\xi \cdot g(\eta)) = E\xi \cdot E(g(\eta)) = E(E\xi \cdot g(\eta)) \Longrightarrow E\xi = \hat{\xi}$
    \end{MyProof}

    \item Если $\xi$ и $\eta$ независимы, то $(\xi - E\xi) \perp g(\eta) \ \forall g(\eta) \in L(\eta)$, 
в частности $(\xi - E\xi) \perp \eta$

\end{enumerate}

Докажем, что \DefN{A} согласуется c \DefNs{B}

По \DefNs{A} $E(\xi|\eta) = h(\eta)$, где $h(y) = E(\xi|\eta = y)$

Рассмотрим случай абсолютно непрерывной системы $(\xi, \eta)$ с плотностью $f_{\xi,\eta}(x, y)$.
Тогда $h(y) = \int_{-\infty}^\infty xf(x|y)dx$, где $f(x|y) = \frac{f_{\xi,\eta}(x, y)}{f_\eta(y)}$

Следует доказать, что функция $h(y)$ удовлетворяет тождеству ортопроекций $E(\xi g(\eta)) = E(h(\eta)g(\eta)) \ \forall g(\eta) \in L(\eta)$

$E(\xi\cdot g(\eta)) = \iint_{\Real^2} xg(y) f_{\xi,\eta}(x, y)dxdy$

$E(h(\eta)g(\eta)) = \int_{-\infty}^{\infty} h(y) g(y) f_{\eta}(y) dy = \int_{-\infty}^\infty \left(\int_{-\infty}^\infty x\frac{f_{\xi,\eta}(x, y)}{f_\eta(y)}dx\right) g(y) f_\eta(y) = dy = \iint_{\Real^2} xg(y)f_{\xi,\eta}(x, y)dxdy = E(\xi g(\eta))$

\subsection{Числовые характеристики. Зависимости случайных величин}

\Mem Если случайные величины $\xi$ и $\eta$, то $E(\xi\eta) = E\xi E\eta \Longrightarrow E(\xi\eta) - E\xi E\eta = 0$

Поэтому в качестве индикатора наличия связи берем величину $E(\xi\eta) - E\xi E\eta = \mathrm{cov}(\xi, \eta)$

\Def Ковариацией $\mathrm{\cov}(\xi, \eta)$ называется величина $\mathrm{cov}(\xi, \eta) = E((\xi - E\xi)(\eta - E\eta))$

\mediumvspace

\textbf{Свойства}:

\begin{enumerate}
    \item $\mathrm{cov} (\xi, \eta) = E(\xi\eta) - E\xi E\eta$

    \begin{MyProof}
        $\mathrm{cov}(\xi, \eta) = E((\xi - E\xi)(\eta - E\eta)) = E(\xi\eta - \eta E\xi - \xi E\eta + E\xi E\eta) = E(\xi\eta) - E\xi E\eta$
    \end{MyProof}

    \item $\mathrm{cov} (\xi, \xi) = D\xi$

    \begin{MyProof}
        $\mathrm{cov}(\xi, \xi) = E\xi^2 - (E\xi)^2$
    \end{MyProof}

    \item $\mathrm{cov}(\xi, \eta) = \mathrm{cov}(\eta, \xi)$

    \item $\mathrm{cov}(C_1 \xi_1 + C_2 \xi_2, \eta) = C_1 \mathrm{cov}(\xi_1, \eta) + C_2 \mathrm{cov}(\xi_2, \eta)$

    \item $D(\xi + \eta) = D\xi + D\eta + 2\mathrm{cov}(\xi, \eta)$

    \item $D(\xi_1 + \dots + \xi_n) = \sum_{i = 1}^n D\xi_i + 2\sum_{i < j} \mathrm{cov}(\xi_i, \xi_j) = \sum_{i, j = 1}^{n} \mathrm{cov}(\xi_i, \xi_j)$

    \item \begin{enumerate}
        \item Если $\xi$ и $\eta$ - независимы, то $\mathrm{cov}(\xi, \eta) = 0$

        \item Если $\mathrm{cov}(\xi, \eta) \neq 0$, то $\xi$ и $\eta$ - зависимы

        \item Если $\mathrm{cov}(\xi, \eta) = 0$, то неясно
    \end{enumerate}

    \item Если $\mathrm{cov}(\xi, \eta) > 0$, то зависимость прямая, если $\mathrm{cov}(\xi, \eta) < 0$, то обратная
\end{enumerate}

\Nota Ковариация зависит от единиц измерения случайных величин, поэтому по ее величине нельзя судить о силе зависимости

\subsection{Коэффициент линейной корреляции}

\Def Коэффициентом корреляции случайных величин $\xi$ и $\eta$ с конечными вторыми моментами,
называется величина $r_{\xi,\eta} = \frac{\mathrm{cov(\xi, \eta)}}{\sqrt{D\xi} \sqrt{D\eta}} = \frac{E(\xi\eta) - E\xiE\eta}{\sigma_\xi \sigma_\eta}$

Можно записать в другой форме: $r_{\xi,\eta} = \frac{E((\xi - E\xi)(\eta - E\eta))}{\sqrt{E(\xi - E\xi)^2}\sqrt{E(\eta - E\eta)^2}} = 
\frac{(\xi - E\xi, \eta - E\eta)}{\|\xi - E\xi\|\|\eta - E\eta\|} = \cos(\widehat{\xi - E\xi, \eta - E\eta})$ - косинус угла между величинами (грубая интерпретация)

\mediumvspace

\textbf{Свойства}:

\begin{enumerate}
    \item $r_{\xi, \eta} = r_{\eta, \xi}$

    \item \begin{enumerate}
        \item Если $\xi$ и $\eta$ - независимы, то $r_{\xi,\eta} = 0$

        \item Если $r_{\xi,\eta} \neq 0$, то $\xi$ и $\eta$ - зависимы

        \item Если $r_{\xi,\eta} = 0$, то неясно
    \end{enumerate}

    \item $|r_{\xi,\eta}| \leq 1$

    \begin{MyProof}
        По неравенству Коши-Буняковского-Шварца $|E((\xi - E\xi)(\eta - E\eta))| \leq \sqrt{E(\xi - E\xi)^2 E(\eta - E\eta)^2}$
    \end{MyProof}

    \item $|r_{\xi,\eta}| = 1 \Longleftrightarrow \eta = a \xi + b$ п.н.

    \begin{MyProof}
        По неравенству Коши-Буняковского-Шварца $|r_{\xi,\eta}| = 1 \Longleftrightarrow 
        |E((\xi - E\xi)(\eta - E\eta))| = \sqrt{E(\xi - E\xi)^2 E(\eta - E\eta)^2} \Longrightarrow \eta - E\eta = C(\xi - E\xi) \Longrightarrow \eta = C\xi + (E\eta - CE\xi)$ п.н.
    \end{MyProof}

    \item \begin{enumerate} 
        \item Если $r_{\xi,\eta} = 1$, то $\eta = a\xi + b$ и $a > 0$ (прямая линейная зависимость)

        \item Если $r_{\xi,\eta} = -1$, то $\eta = a\xi + b$ и $a < 0$ (обратная линейная зависимость)
    \end{enumerate}

    \begin{MyProof}
        Так как $|r_{\xi,\eta}| = 1$, то по свойству 4) $\eta = a\xi + b$ и $r_{\xi,\eta} = \frac{E(\xi\eta) - E\xi E\eta}{\sigma_\xi \sigma_\eta} = 
        \frac{E(\xi(a\xi + b)) - E\xi E(a\xi + b)}{\sqrt{D\xi D(a\xi + b)}} = \frac{aE\xi^2 + bE\xi - a(E\xi)^2 - bE\xi}{\sqrt{D\xi a^2 D\xi}} = \frac{a(E\xi^2 - (E\xi)^2)}{|a|D\xi} = \frac{a}{|a|} = \mathrm{sign} \,a$
    \end{MyProof}
\end{enumerate}

\Def Если $r_{\xi,\eta} \neq 0$, то говорят, что случайные величины коррелированы друг с другом. Если $r_{\xi,\eta} > 0$, 
то имеет прямая корреляция, если $r_{\xi,\eta} < 0$ - обратная

\Nota Корреляция не транзитивна: $r_{\xi_1,\xi_2} > 0 \land r_{\xi_2,\xi_3} > 0 \centernot\Longrightarrow r_{\xi_1,\xi_3} > 0$


% end probtheory_2024_12_03.tex

% begin probtheory_2024_12_10.tex





\section{Лекция 15}

\subsection{Характеристические функции}

\Mem $i$ - комплексная единица

\Mems $e^{it} = \cos t + i \sin t$

Пусть $\xi + i\eta$ - комплексная случайная величина, где $\xi$ - вещественная часть, а $\eta$ - мнимая часть

\Def $E(\xi + i\eta) = E\xi + iE\eta$

\Def Характеристической функций случайной величины $\xi$ называется функция 

\[\varphi_\xi(t) = Ee^{it\xi}, t \in \Real\]

\underline{Свойства}:

\begin{enumerate}
    \item Любая случайная величина $\xi$ имеет характеристическую функцию, причем $|\varphi_\xi(t)| \leq 1$

    \begin{MyProof}
        Характеристическая функция существует по теореме об абсолютной сходимости интеграла от произведения ограниченной и 
        нормированной функций

        Докажем неравенство:

        $|\varphi_\xi(t)|^2 = |Ee^{it\xi}|^2 = |E\cos t\xi + iE\sin t\xi|^2 = (E\cos\xi t)^2 + (E \sin\xi t)^2 \leq [\text{по неравенству Йенсена}] \leq
        E\cos^2 \xi t + E\sin^2 \xi t = E(\cos^2 \xi t + \sin^2 \xi t) = E 1 = 1$
    \end{MyProof}

    \item Пусть $\varphi_\xi(t)$ - характеристическая функция случайной величины $\xi$. Тогда характеристическая функция
    случайной величины $a + b\xi$ равна $\varphi_{a + b\xi}(t) = e^{ita} \varphi_{\xi}(bt)$

    \begin{MyProof}
        $\varphi_{a + b\xi}(t) = Ee^{it(a + b\xi)} = E(e^{ita} \cdot e^{itb\xi}) = e^{ita}Ee^{itb\xi} = e^{ita} \varphi_{\xi}(bt)$
    \end{MyProof}

    \item Характеристическая функция суммы независимых случайных величин равна произведению их характеристических функций

    \begin{MyProof}
        Пусть случайные величины $\xi$ и $\eta$ - независимы. Тогда 

        $\varphi_{\xi + \eta}(t) = E(e^{it\xi} \cdot e^{it\eta}) = [\text{так как они независимы}] = Ee^{it\xi} \cdot Ee^{it\eta} = \varphi_\xi(t) \cdot \varphi_\eta(t)$

        Аналогично для большего числа величин
    \end{MyProof}

    \item Пусть $E\xi^k < \infty$. Тогда 

    \[\varphi_\xi(t) = 1 + it E\xi - \frac{t^2}{2}E\xi^2 + \dots + \frac{(it)^k}{k!} E\xi^k + o(|t|^k)\]

    \begin{MyProof}
        $\varphi_\xi(t) = Ee^{it\xi} = E(1 + it\xi + \frac{(it\xi)^2}{2!} + \dots + \frac{(it\xi)^k}{k!} + o(|t|^k)) = 
        1 + it E\xi \frac{i^2 t^2}{2}E\xi^2 + \dots + \frac{(it)^k}{k!} E\xi^k + o(|t|^k)$
    \end{MyProof}

    \item Пусть $E\xi^k < \infty$. Тогда $\varphi_\xi^{(k)}(0) = i^k E\xi^k$

    \begin{MyProof}
        $E\xi^k < \infty \Longrightarrow$ существует $k$ членов разложения в ряд Маклорена: 
        $\frac{\varphi_\xi^{(k)}(0)}{k!}t^k = \frac{i^k E\xi^k}{k!} t^k$; $\frac{\varphi_\xi^{(k)}(0)}{k!}t^k = i^k E\xi^k$
    \end{MyProof}

    \item Существует взаимно-однозначное соответствие между распределениями и характеристическими функциями.
    Зная характеристическую функцию можно восстановить распределение.

    \Ex Если распределение абсолютно непрерывное, то его можно восстановить по преобразованию Фурье

    $f_\xi(x) = \frac{1}{\sqrt{2\pi}} \int_{-\infty}^{\infty} e^{-itx} \varphi_\xi(t) dt$

    \item Теорема о непрерывном соответствии
    
    \begin{MyTheorem}
        \Ths Последовательность случайных величин $\{\xi_n\}$ слабо сходится к $\xi$ тогда и только тогда, когда
        соответствующая последовательность характеристических функций сходится поточечно к $\varphi_\xi(t)$

        $\{\xi_n\} \rightrightarrows \xi \Longleftrightarrow \varphi_{\xi_n}(t) \longrightarrow \varphi_\xi(t) \forall t \in \Real$
    \end{MyTheorem}

\end{enumerate}

\subsection{Характеристические функции стандартных распределений}

\begin{itemize}
    \item Распределение Бернулли

    \smallvspace

    \begin{tabular}{c|c|c}
        $\xi$ & $0$     & $1$    \\
        \hline
        $p$   & $1 - p$ & $p$
    \end{tabular}

    \smallvspace

    $\varphi_\xi(t) = Ee^{i\xi t} = e^{i \cdot 0 \cdot t} p(\xi = 0) + e^{i \cdot 1 \cdot t} p(\xi = 1) = 1 - p + p e^{it}$

    \item Биномиальное распределение

    $P(\xi = k) = C_n^k p^k q^{n - k}, \quad k = 0, 1, \dots, n$

    Если $t \in B_{n,p}$, то $\xi = \xi_1 + \xi_2 + \xi_3 + \dots + \xi_n$, где $\xi_i \in B_p$ - независимы

    $\varphi_\xi(t) = (\varphi_{\xi_n}(t))^n = (1 - p + p e^{it})^n$

    \item Распределение Пуассона

    $P(\xi = k) = \frac{\lambda^k}{k!} e^{-\lambda}, \quad k = 0, 1, \dots, n$

    $\varphi_\xi(t) = Ee^{it\xi} = \sum_{k = 0}^\infty e^{itk} p(\xi = k) = \sum_{k = 0}^\infty e^{itk} \frac{\lambda^k}{k!} e^{-\lambda} = 
    e^{-\lambda} \sum_{k = 0}^\infty \frac{(\lambda e^{it})^k}{k!} = e^{-\lambda} e^{\lambda e^{it}} = e^{\lambda (e^{it} - 1)}$

    \underline{Следствие}: распределение Пуассона устойчиво относительно суммирования

    \begin{MyTheorem}
        $\letsymbol \xi \in \Pi_\lambda, \eta \in \Pi_\mu$, они независимы. Тогда $\xi + \eta \in \Pi_{\lambda + \mu}$
    \end{MyTheorem}

    \begin{MyProof}
        По третьему свойству $\varphi_{\xi + \eta}(t) = \varphi_\xi(t) \cdot \varphi_\eta(t) = e^{\lambda(e^{it} - 1)} e^{\mu(e^{it} - 1)} = e^{(\lambda + \mu)(e^{it} - 1)}$ - характеристическая функция распределения Пуассона $\Pi_{\lambda + \mu}$
    \end{MyProof}

    \item Стандартное нормальное распределение

    $f_\xi(x) = \frac{1}{2\pi} e^{-\frac{x^2}{2}}$

    $\varphi_\xi(t) = Ee^{it\xi} = \int_{-\infty}^{\infty} e^{itx} f_\xi(x) dx = \int_{-\infty}^\infty e^{itx} \frac{1}{\sqrt{2\pi}} e^{-\frac{x^2}{2}} dx = 
    \frac{1}{\sqrt{2\pi}} \int_{-\infty}^\infty e^{-\frac{1}{2}(x^2 - 2itx)} dx = \\
    \frac{1}{\sqrt{2\pi}} \int_{-\infty}^{\infty} e^{-\frac{1}{2}(x^2 - 2itx - t^2) e^{-\frac{t^2}{2}}} dx = 
    \frac{1}{\sqrt{2\pi}} \int_{-\infty}^{\infty} e^{-\frac{(x - it)^2}{2}} d(x - it) = \frac{1}{\sqrt{2\pi}} e^{-\frac{t^2}{2}} \sqrt{2\pi} = e^{-\frac{t^2}{2}}$

    \item Нормальное распределение

    $\xi \in N(a, \sigma^2)$

    Если $\eta \in N(0, 1)$, то $\xi = a + \sigma \eta \in N(a, \sigma^2)$

    По второму свойству $\varphi_\xi(t) = e^{ita} \varphi_\eta(\sigma t) = e^{ita - \frac{\sigma^2t^2}{2}}$

    \underline{Следствие}: нормальное распределение устойчиво относительно суммирования

    \begin{MyTheorem}
        Если $\xi \in N(a_1, \sigma_1^2), \eta \in N(a_2, \sigma^2_2)$ и они независимы, то $\xi + \eta \in N(a_1 + a_2, \sigma_1^2 + \sigma_2^2)$
    \end{MyTheorem}

    \begin{MyProof}
        $\varphi_{\xi + \eta}(t) = \varphi_\xi(t) \varphi_\eta(t) = e^{ita_1 - \frac{\sigma_1^2t^2}{2}} e^{ita_2 - \frac{\sigma_2^2t^2}{2}} = e^{it(a_1 + a_2) - \frac{(\sigma_1^2 + \sigma_2^2)t^2}{2}}$ - 
        характеристическая функция $N(a_1 + a_2, \sigma_1^2 + \sigma_2^2)$
    \end{MyProof}
\end{itemize}

\mediumvspace

\subsection{Доказательства теорем через свойства характеристических функций}

Докажем некоторые теоремы с помощью характеристических функций

\subsubsection{Закон больших чисел Хинчина}

Для доказательства закона больших чисел Хинчина докажем такую лемму: 

$\left(1 + \frac{x}{n} + o\left(\frac{1}{n}\right)\right)^n \underset{n \to \infty}{\longrightarrow} e^x$

\begin{MyProof}
    $\left(1 + \frac{x}{n} + o\left(\frac{1}{n}\right)\right)^n = e^{n \ln\left(1 + \frac{x}{n} + o\left(\frac{1}{n}\right)\right)} = 
    e^{n \left(\frac{x}{n} + o\left(\frac{1}{n}\right) + o\left(\frac{x}{n} + o\left(\frac{1}{n}\right)\right)\right)} = e^{n\left(\frac{x}{n} + o\left(\frac{1}{n}\right) + o\left(\frac{1}{n}\right)\right)} = e^{x + n o\left(\frac{1}{n}\right)} \underset{n \to \infty}{\longrightarrow} e^x$
\end{MyProof}

\begin{MyTheorem}
    \Ths Закон больших чисел Хинчина

    Пусть $\xi_1, \xi_2, \dots, \xi_n$ - последовательность независимых одинаково распределенных случайных величин с конечным матожиданием.
    Тогда $\frac{S_n}{n} = \frac{\xi_1 + \dots + \xi_n}{n} \overset{p}{\longrightarrow} E\xi_1$
\end{MyTheorem}

\begin{MyProof}
    Обозначим $a = E\xi_1$

    Ранее было доказано, что сходимость по вероятности к константе эквивалентно к слабой сходимости. Поэтому достаточно доказать, что $\frac{S_n}{n} \rightrightarrows a$

    По теореме о непрерывном соответствии остается доказать, что $\varphi_{\frac{S_n}{n}}(t) \longrightarrow \varphi_a(t) = e^{ita}$

    По четвертому свойству $\varphi_{\xi_1}(t) = 1 + itE\xi_1 + o(|t|) = 1 + ita + o(|t|)$

    $\varphi_{\frac{S_n}{n}}(t) = [\text{по второму свойству}] = \varphi_{S_n}\left(\frac{t}{n}\right) = \left(\varphi_{\xi_1}\left(\frac{t}{n}\right)\right)^n = \left(1 + ia\frac{t}{n} + o\left(\left|\frac{t}{n}\right|\right)\right)^n \underset{\text{по лемме}}{\longrightarrow}
    e^{ita} = \varphi_a(t)$
\end{MyProof}

\subsubsection{Центральная предельная теорема}

\begin{MyTheorem}
    \Ths Центральная предельная теорема Ляпунова, 1901 г.

    Пусть $\xi_1, \xi_2, \dots, \xi_n$ - последовательность независимых одинаково распределенных случайных величин с конечным вторым моментом ($D\xi_1 < \infty$)

    Обозначим $a = E\xi_1, \sigma^2 = D\xi_1$. Тогда 

    \[\frac{S_n - na}{\sigma\sqrt{n}} \rightrightarrows N(0, 1)\]
\end{MyTheorem}

\begin{MyProof}
    Пусть $\eta_i = \frac{\xi_i - a}{\sigma}$ - стандартизованная случайная величина

    $E\eta_i = 0, D\eta_i = 1$

    Обозначим $Z_n = \eta_1 + \dots + \eta_n = \frac{(\xi_1 + \dots + \xi_n) - na}{\sigma} = \frac{S_n - na}{\sigma}$

    Надо доказать, что если $\frac{Z_n}{\sqrt{n}} \rightrightarrows N(0, 1)$

    По четвертому свойству $\varphi_{\eta_1}(t) = 1 + itE\eta_1 - \frac{t^2}{2} E\eta_1^2 + o(t^2) = 1 - \frac{t^2}{2} + o(t^2)$

    $\varphi_{\frac{Z_n}{\sqrt{n}}} = \varphi_{Z_n}\left(\frac{t}{\sqrt{n}}\right) = \left(\varphi_{\eta_1}\left(\frac{t}{\sqrt{n}}\right)\right)^n = 
    \left(1 - \frac{\left(\frac{t}{\sqrt{n}}\right)^2}{2} + o\left(\left(\frac{t}{\sqrt{n}}\right)^2\right)\right)^n =
    \left(1 - \frac{t^2}{2n} + o\left(\left(\frac{t}{\sqrt{n}}\right)^2\right)\right)^n \underset{n \to \infty}{\longrightarrow} e^{-\frac{t^2}{2}}$ - 
    характеристическая функция $N(0, 1)$
\end{MyProof}

\subsubsection{Предельная теорема Муавра-Лапласа}

\begin{MyTheorem}
    \Ths Пусть $v_n(A)$ - число появления события $A$ при $n$ независимых испытаний, $p$ - вероятность успеха при одном испытании, $q = 1 - p$.
    Тогда $\frac{v_n(A) - np}{\sqrt{npq}} \rightrightarrows N(0, 1)$
\end{MyTheorem}

\begin{MyProof}
    $v_n(A) = \xi_1 + \xi_2 + \dots + \xi_n = S_n$, где $\xi_i \in B_p$ и независимы, $E\xi_1 = p, D\xi_1 = pq$

    По ЦПТ $\frac{v_n(A) - np}{\sqrt{npq}} = \frac{S_n - nE\xi_1}{\sqrt{nD\xi_1}} \rightrightarrows N(0, 1)$
\end{MyProof}

\underline{Следствие}. Интегральная формула Лапласа:

$p(k_1 \leq v_n \leq k_2) = p\left(\frac{k_1 - np}{\sqrt{npq}} \leq \frac{v_n - np}{\sqrt{npq}} \leq \frac{k_2 - np}{\sqrt{npq}}\right)$. Обозначим $\eta = \frac{v_n - np}{\sqrt{npq}}$

$p\left(\frac{k_1 - np}{\sqrt{npq}} \leq \frac{v_n - np}{\sqrt{npq}} \leq \frac{k_2 - np}{\sqrt{npq}}\right) = F_\eta\left(\frac{k_2 - np}{\sqrt{npq}}\right) - F_\eta\left(\frac{k_1 - np}{\sqrt{npq}}\right) \underset{n \to \infty}{\longrightarrow} F_0\left(\frac{k_2 - np}{\sqrt{npq}}\right) - F_0\left(\frac{k_1 - np}{\sqrt{npq}}\right)$,
где $F_0(x) = \frac{1}{\sqrt{2\pi}} \int_{-\infty}^x e^{-\frac{t^2}{2}} dt$

\Nota Аналогичным образом ЦПТ применяется для приближенного вычисления вероятностей, связанных с суммами большого числа независимых одинаковых случайных величин, заменяя стандартизованную сумму на стандартное нормальное распределение.
Возникает вопрос: какова погрешность данного вычисления?

\begin{MyTheorem}
    \Ths Неравенство Берри-Эссеена

    В условиях ЦПТ для $\xi_1$ с конечным третьим моментом можно оценить так:

    $\left|p\left(\frac{S_n - nE\xi_1}{\sqrt{nD\xi_1}} < x\right) - F_0(x)\right| \leq C\frac{E|\xi_1 - E\xi_1|^3}{\sqrt{n(D\xi_1)^3}} \forall x \in \Real$
\end{MyTheorem}

\Nota На практике берут $C = 0.4$, точная оценка сверху $C < 0.77$

% end probtheory_2024_12_10.tex

% begin probtheory_2024_12_17.tex





\section{Лекция 16}

\subsection{Условная дисперсия}

\Def Условной дисперсией случайной величины $\xi$ относительно случайной величины $\eta$ называется случайная величина 
$D(\xi | \eta) = E((\xi - E(\xi | \eta))^2 | \eta)$

\Nota То есть дисперсия соответствующего условного распределения

\underline{Свойства}

\begin{enumerate}
    \item $D(\xi | \eta) = E(\xi^2 | \eta) - E^2(\xi | \eta)$

    \item Закон полной дисперсии

    \begin{MyTheorem}
        \Ths $D\xi = E(D(\xi | \eta)) + D(E(\xi | \eta))$
    \end{MyTheorem}

    \begin{MyProof}
        Из первого свойства $E(\xi^2 | \eta) = D(\xi | \eta) + E^2(\xi | \eta)$

        $D\xi = E\xi^2 - (E\xi)^2 = E(E\xi^2 | \eta) - E^2(E(\xi | \eta)) = E(D(\xi | \eta) + E^2(\xi | \eta)) - E^2(E(\xo | \eta)) = E(D(\xi | \eta)) + E(E^2(\xi | \eta)) - E^2(E(\xi | \eta)) = E(D(\xi | \eta)) + D(E(\xi | \eta))$
    \end{MyProof}

    \underline{Следствие и смысл}: 
    
    \begin{itemize}
        \item Если $\xi$ и $\eta$ независимы (некоррелированы), то $D(E(\xi | \eta)) = D(E\xi) = 0$ и $D\xi = E(D(\xi | \eta))$

        \item Если имеется функциональная зависимость (то есть $\xi = g(\eta)$), то $D(E(\xi | \eta)) = D(E(g(\eta) | \eta)) = 
        D(g(\eta)) = D\xi$

        \item Таким образом по величине $R^2 = \frac{D(E(\xi | \eta))}{D\xi}$ ($0 \leq R^2 \leq 1$) можно судить о силе корреляционной зависимости.
        Такая величина называется корреляционным отношением
    \end{itemize}
\end{enumerate}

\subsection{Энтропия}

Пусть $\xi$ - результат эксперимента с исходами $A_1, A_2, \dots, A_N$, вероятности которых $p_1, p_2, \dots, p_N$

\Def Энтропией эксперимента называется величина $H(\xi) = -\sum_{i = 1}^N p_i \cdot \log_2 p_i$

\underline{Свойства энтропии}:

\begin{enumerate}
    \item Очевидно, что $H(\xi) \geq 0$, так как $p \geq 0$, а $\log_2 p_i \leq 0$
    
    \item $H(\xi) = 0 \Longleftrightarrow \exists i$, такой что $p_i = 1, p_j = 0 \forall j \neq i$ - то есть эксперимент заканчивается всегда одним исходом, нет неопределенности

    \item Максимум $H(\xi) = \log_2 N = H_0$ достигается при $p_1 = p_2 = \dots = \frac{1}{N}$ - то есть когда все вероятности одинаковы, ни одному исходу нельзя отдать предпочтение, и результат эксперимента получается максимально неопределенным

    \begin{MyProof}
        Рассмотрим $\varphi(x) = x \log_2 x$. Так как $\varphi^{\prime\prime}(x) = \frac{1}{x\ln 2} > 0$ при $x > 0$, следовательно $\varphi(x)$ выпукла вниз

        Рассмотрим случайную величину $\eta$

        \begin{tabular}{c|c|c|c|c}
            $\eta$ & $p_1$ & $p_2$ & \dots & $p_n$ \\
            \cline{1-6}
            $p$   & $\frac{1}{N}$ & $\frac{1}{N}$ & \dots & $\frac{1}{N}$
        \end{tabular}

        По неравенству Йенсена $\varphi(E\eta) = \varphi(\sum_{i = 1}^N \frac{p_i}{N}) = \varphi(\frac{1}{N} \sum_{i = 1}^N p_i) = 
        \varphi(\frac{1}{N}) = \frac{\log_2 \frac{1}{N}}{N} \leq E(\varphi(\eta)) = 
        \frac{1}{N} \sum_{i = 1}^N p_i \log_2 p_i = -\frac{1}{N} H(\eta)$

        Получаем $\frac{\log_2 \frac{1}{N}}{N} \leq -\frac{1}{N} H(\eta)$, то есть $H \leq \log_2 N$
    \end{MyProof}

    \underline{Следствие}: Энтропию можно рассматривать как меру неопределенности эксперимента
\end{enumerate}

\Ex $\xi \in B_{p}$

\begin{tabular}{c|c|c}
    $\xi$ & $0$ & $1$ \\
    \cline{1-6}
    $p$   & $1 - p$ & $p$
\end{tabular}

$H(\xi) = -(1 - p) \log_2 (1 - p) - p \log_2 p$

% график

\ExN{1} Психолог Р. Хайман проводил такой эксперимент: перед человеком загорались с некоторой частотой лампочки, замерялась время реакции на загоревшуюся лампочку. 
Если лампочки загорались с одинаковой частотой, то энтропия была пропорциональна $H_0$

\ExN{2} Также с помощью энтропии определен второй закон термодинамики

\ExN{3} Теория кодирования информации

Если алфавит сообщения состоит из $N$ символов, то каждому символу присваиваем последовательность одинаковой длины из 0 и 1, 
причем ее длина будет $\lceil\log_2 N\rceil$

Для передачи $n$ символов потребуется последовательность длиной $n\lceil\log_2 N\rceil$

Цель: сократить длину последовательности

Для больших по объему сообщений можно заметно уменьшить эту величину, используя, что разные символы встречаются с разными частотами.

Если $p_1, p_2, \dots, p_N$ - эти частоты, то в сообщении длиной $N$ $i$-ый символ появляется $v_i \approx n p_i$ раз 

\Def Сообщение длины $N$ называется типичным с параметрами $n$ и $\delta$, если $|v_i - n p_i| < \delta \ \forall 1 \leq i \leq N$

Пусть $M_{n, \delta}$ - число таких сообщений

\begin{MyTheorem}
    \Ths (частный случай теоремы Макмиллана)

    $\frac{1}{n} \log_2 M_{n, \delta} \underset{n \to \infty}{\longrightarrow} H = -\sum_{i = 1}^N p_i \log_2 p_i$
\end{MyTheorem}

\underline{Следствие}: существует $\varepsilon > 0 \ | \ \frac{1}{n} \log_2 M_{n,\delta} < H + \varepsilon$ (или $M_{n, \delta} < 2^{n(H + \varepsilon)}$)

Если можно занумеровать эти типичные сообщения, то для них потребуется число символов $\log_2 2^{n(H + \varepsilon)} = n \cdot (H + \varepsilon)$

И поэтому с вероятностью приблизительно 1 можно сократить длины сообщение с коэффициентом сжатия $\gamma \approx \frac{nH}{nH_0} = \frac{H}{H_0}$, где $H_0 = \log_2 N$

Если все символы встречаются независимо, то дальнейшее сжатие невозможно, но так как буквы встречаются в определенных сочетаниях, то можно сжать 
информации дальше, используя этот факт

Пусть $\gamma_\infty$ - коэффициент итогового сжатия

% здесь не понял, что он точно сказал

В русском языке $\gamma \approx 0.87$. Если считать слова символами нашего алфавита, 
то получится $\gamma_\infty \approx 0.24$ для литературного языка и
$\gamma_\infty \approx 0.17$ для делового языка

\Def $1 - \gamma_\infty$ называют коэффициентом избыточности языка

\subsection{Энтропия при непрерывном распределении}

\Def Пусть $\xi$ абсолютно непрерывная случайная величина с плотностью $f(x)$ и носителем $A = \{x \ | \ f(x) > 0\}$. Энтропией $H(\xi)$ называется
величина $-\int_A f(x) \log_2 f(x) dx$

\begin{MyTheorem}
    \Ths Следующие распределения имеют наибольшую энтропию:

    \begin{enumerate}
        \item Если $A = [0, 1]$, то $U(0, 1)$

        \item Если $A = [0, \infty)$ и $E\xi = 1$, то показательное $E_1$

        \item Если $A = \Real$ и $E\xi = 0$, а $D\xi = 1$, то $N(0, 1)$
    \end{enumerate}
\end{MyTheorem}


% end probtheory_2024_12_17.tex

% begin probtheory_exam_list.tex
\clearpage

\section{X. Программа экзамена в 2024/2025}

\begin{enumerate}
    \item Пространство элементарных исходов. Случайные события. Операции над  событиями.
    \item Статистическое определение вероятности. Классическое определение вероятности.
    \item Геометрическое определение вероятности. Задача Бюффона об игле.
    \item Аксиоматическое определение вероятности. Вероятностное пространство. Свойства вероятности.
    \item Аксиома непрерывности. Ее смысл и вывод.
    \item Свойства операций сложения и умножения. Формула сложения вероятностей.
    \item Независимость событий. Независимые события в совокупности и попарно.                   Пример Бернштейна. 
    \item Условная вероятность. Формула умножения событий.
    \item Полная группа событий. Формула полной вероятности. Формула Байеса.
    \item Последовательность независимых испытаний. Формула Бернулли. Наиболее вероятное число успехов в схеме Бернулли.
    \item Локальная и интегральная формулы Муавра-Лапласа (без док-ва).
    \item Вероятность отклонения относительной частоты от вероятности события.                            Закон больших чисел Бернулли.
    \item Схемы испытаний: Бернулли, до первого успеха. Биномиальное и геометрическое распределения. Свойство отсутствия последействия.
    \item Урновая схема с возвратом и без возврата. Гипергеометрическое распределение. Теорема об его асимптотическом приближении к биномиальному.
    \item Схема Пуассона. Формула Пуассона. Оценка погрешности в формуле Пуассона.
    \item Случайные величины, определение. Измеримость функции, ее смысл. Вероятностное пространство (R, B, P). Распределение случайной величины.
    \item Дискретные случайные величины. Определение, закон распределения, числовые характеристики.
    \item Свойства математического ожидания и дисперсии дискретной случайной величины.
    \item Стандартные дискретные распределения и их числовые характеристики (Бернулли, биномиальное, геометрическое, Пуассона).
    \item Функция распределения и ее свойства (в свойствах 4, 5, 6 достаточно привести одно из доказательств).
    \item Абсолютно непрерывные случайные величины. Плотность и ее свойства.
    \item Числовые характеристики абсолютно непрерывной случайной величины, их свойства.
    \item Равномерное распределение. 
    \item Показательное распределение. Свойство нестарения.
    \item Нормальное распределение. Стандартное нормальное распределение, его числовые характеристики.
    \item Связь между стандартным нормальным и нормальным распределениями. Следствия.
    \item Сингулярные распределения. Теорема Лебега (без док-ва).
    \item Преобразования случайных величин. Стандартизация случайной величины. 
    \item Теорема о монотонном преобразовании. Линейное преобразование случайной величины. (без док-ва).
    \item Квантильное преобразование. Моделирование случайной величины с помощью датчика случайных чисел.
    \item Виды сходимостей случайных величин, связь между ними. Теорема об эквивалентности сходимостей к константе (все без док-ва).
    \item Математическое ожидание преобразованной случайной величины. Свойства моментов.
    \item Неравенство Йенсена, следствие.
    \item Неравенства Маркова, Чебышева, правило трех сигм.
    \item Среднее арифметическое одинаковых независимых случайных величин. Закон больших чисел Чебышева.
    \item Вывод закона больших чисел Бернулли из закона больших чисел Чебышева. Законы больших чисел Хинчина и Колмогорова (только формулировки).
    \item Совместные распределения случайных величин. Функция совместного распределения, ее свойства. Независимость случайных величин.
    \item Дискретная система двух случайных величин. Закон совместного распределения. Маргинальные распределения.
    \item Абсолютно непрерывная система двух случайных величин. Плотность совместного распределения, ее свойства.
    \item Функции от двух случайных величин. Теорема о функции распределения. Формула свертки.
    \item Суммы стандартных распределений, устойчивость по суммированию (биномиальное, Пуассона, стандартное нормальное).
    \item Условные распределения и условные математические ожидания. Случаи дискретной и абсолютно непрерывной систем двух случайных величин.
    \item Пространство случайных величин. Скалярное произведение, неравенство Коши-Буняковского-Шварца. 
    \item Условное математическое ожидание как случайная величина, его свойства. Формула полного математического ожидания.
    \item Условная дисперсия. Закон полной дисперсии. Смысл второго слагаемого в разложении дисперсии.
    \item Числовые характеристики зависимости случайных величин. Ковариация, ее свойства. Коэффициент корреляции, его свойства. Корреляция случайных величин.
    \item Характеристическая функция случайной величины, ее свойства. Теорема о непрерывном соответствии (формулировка).
    \item Характеристические функции стандартных распределений (Бернулли, биномиальное, Пуассона, нормальное). Следствия.
    \item Доказательство закона больших чисел Хинчина.
    \item Центральная предельная теорема. Вывод из нее предельной теоремы Муавра-Лапласа. Неравенство Берри-Ессеена (формулировка). 
\end{enumerate}

% end probtheory_exam_list.tex



\end{document}

