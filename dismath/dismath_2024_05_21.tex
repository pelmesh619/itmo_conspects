\documentclass[12pt]{article}
\usepackage{preamble}

\pagestyle{fancy}
\fancyhead[LO,LE]{$\mathcal{D}$искретная математика}
\fancyhead[CO,CE]{21.05.2024}
\fancyhead[RO,RE]{Лекции Чухарева К. И.}



\begin{document}
    \begin{itemize}
        \hypertarget{recurrencerelations}{}

        \item \textbf{Рекуррентные соотношения} (Recurrence relations)

        \underline{Решить рекуррентное соотношение} -- найти закрытую (то есть нерекуррентную) формулу

        \Ex Арифметическая прогрессия

        $a_n = \begin{cases}a_0 = const \quad n = 0 \\ a_{n - 1} + d, \quad n > 0\end{cases}$

        Решение: $a_n = a_0 + nd$ -- анзац (Ansatz, догадка)

        Проверка: $a_n = a_0 + nd = a_{n - 1} + d = a_0 + (n - 1)d + d = a_0 + nd \quad$ - {\Large👍👍}

        \item Метод характеристического уравнения

        \substack{\text{Рекуррентное} \\ \text{соотношение}} \ $\stackrel{a_n \to r^n}{\rightsquigarrow}$ \ \substack{\text{Характеристическое} \\ \text{уравнение}} \ $\stackrel{\text{решение}}{\rightsquigarrow}$ Корни $\stackrel{\text{магия}}{\rightsquigarrow}$ Решение $\rightsquigarrow$ Проверка

        \Ex $a_n = a_{n - 1} + 6a_{n - 2}$

        $r^n - r^{n - 1} - 6r^{n - 2} = 0$

        $r^{n-  2} (r^2 - r - 6) = 0$

        $r_{1,2} = -2, 3$

        \fbox{\begin{tabular}{l}
            Если $r_1 \neq r_2$, то $a_n = ar_1^n + br_2^n$ -- общее решение \\
            Если $r_1 = r_2 = r$, то $a_n = ar^n + bnr^n$
        \end{tabular}}

        \vspace{3mm}

        $a_n = a(-2)^n + b(3)^n$

        Пусть $\begin{cases}a_0 = 1 = a + b \\ a_1 = 8 = -2a + 3b\end{cases}$

        $-5a = 5 \Longrightarrow \begin{cases}a = -1 \\ b = 2\end{cases} \Longrightarrow a_n = -(-2)^n + 2 \cdot 3^n$

        \item \textbf{Разделяй и властвуй} (Divide-and-Conquer)

        $T(n) = \underset{\text{работа рекурсии}}{\undergroup{2T\left(\frac{n}{2}\right)}} + \underset{\text{работа разделения/слияния}}{\undergroup{\theta(n)}}$

        \hypertarget{mastertheorem}{}

        \item \textbf{Основная теорема о рекуррентных соотношениях} (Master Theorem)
        \hfill\href{https://ru.wikipedia.org/wiki/%D0%9E%D1%81%D0%BD%D0%BE%D0%B2%D0%BD%D0%B0%D1%8F_%D1%82%D0%B5%D0%BE%D1%80%D0%B5%D0%BC%D0%B0_%D0%BE_%D1%80%D0%B5%D0%BA%D1%83%D1%80%D1%80%D0%B5%D0%BD%D1%82%D0%BD%D1%8B%D1%85_%D1%81%D0%BE%D0%BE%D1%82%D0%BD%D0%BE%D1%88%D0%B5%D0%BD%D0%B8%D1%8F%D1%85}{*тык*}


        $T(n) = aT\left(\frac{n}{b}\right) + f(n)$

        Из этого, $c_{crit} = \log_b a$

        \mediumvspace

        \underline{I случай}: слияние $<$ рекурсия

        $f(n) \in O(n^c)$, где $c < c_{crit} \Longrightarrow T(n) \in \Theta(n^{c_{crit}})$

        $f(n) \in O(n^c) \Longleftrightarrow f(n) \in o(n^{c_{crit}})$

        \mediumvspace

        \underline{II случай}: слияние $\approx$ рекурсия

        $f(n) \in \Theta(n^{c_{crit}} \log^k n) \Longrightarrow T(n) \in \Theta(n^{c_{crit}} \log^{k + 1} n)$

        Здесь $k \geq 0$. В общем случае см. википедию

        \mediumvspace

        \underline{III случай}: слияние $>$ рекурсия

        $f(n) \in \Omega(n^c)$, где $c > c_{crit} \Longrightarrow T(n) \in \Theta(f(n))$

        \hypertarget{akrabazzimethod}{}

        \item \textbf{Метод Акра-Бацци} (Akra-Bazzi method)
        \hfill\href{https://en.wikipedia.org/wiki/Akra%E2%80%93Bazzi_method}{*тык*}


        $T(n) = f(n) + \sum_{i = 1}^k a_i T(b_i n + h_i(n)) \Longrightarrow T(n) \in \Theta\left(n^p \cdot \left(1 + \int_1^n \frac{f(x)}{x^{p + 1}} dx\right)\right)$, где $p$ -- решение для $\sum_{i = 1}^k a_i b_i^p = 1$

        \begin{cases}
            k = const \\
            a_i > 0 \\
            0 < b_i < 1 \\
            h_1(n) \in O\left(\frac{n}{\log^2 n}\right) \text{ -- малые возмущения}
        \end{cases}

        \Ex $T(n) = T\left(\left\lfloor\frac{n}{2}\right\rfloor\right) + T\left(\left\lceil\frac{n}{2}\right\rceil\right) + n$ -- асимптотика сортировки слиянием

        $T(n) = T\left(\frac{n}{2} + O(1)\right) + T\left(\frac{n}{2} - O(1)\right) + \theta(n)$

        Здесь $b_i = \frac{1}{2}, \quad h = \pm O(1) \in O\left(\frac{n}{\log^2 n}\right)$

        \Ex $T(n) = T\left(\frac{3n}{4}\right) + T\left(\frac{n}{4}\right) + n$

        $a_1 = 1, b_1 = \frac{3}{4}, a_2 = 1, b_2 = \frac{1}{4}, f(n) = n$

        $\left(\frac{3}{4}\right)^p + \left(\frac{1}{4}\right)^p = 1$

        $p = 1$

        $\int_1^n \frac{x}{x^{1 + 1}}dx = \int_1^n \frac{dx}{x} = \ln x \Big|_1^n = \ln n$

        $T(n) \in \Theta(n \cdot (1 + \ln n))$

        $T(n) \in \Theta(n \ln n)$

        \item Решить рекуррентное соотношение $a_n = 3a_{n-1} - 2a_{n-1}$, где $a_0 = 1, a_1 = 3$

        Используем производящие функции:

        $A(x) = \frac{1}{1 - 3x + 2x^2} = \frac{1}{(1 - x)(1 - 2x)} = \frac{-1}{1 - x} + \frac{2}{1 - 2x} \to 2^{n + 1} - 1$

    \end{itemize}


\end{document}
