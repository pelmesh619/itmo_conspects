\documentclass[12pt]{article}
\usepackage{preamble}

\pagestyle{fancy}

\begin{document}
    \clearpage


    \section{X. Программа экзамена в 2023/2024}


    \subsection{5. Теория графов.}

    \begin{enumerate}
        \item \textbf{Ориентированные и неориентированные графы} (\textit{Directed and undirected graphs})

        Граф - множество вершин $V$ и множество ребер $E$ (в общем случае), соединяющие какие-либо две вершины: $G(V, E)$

        По виду ребер различают:

        \smallvspace

        \begin{minipage}{0.45\textwidth}
            \begin{center}
                неориентированный граф

                \begin{tikzpicture}
                    \node[circle, draw=black!60, thick, minimum size=0.5cm] (v1) {$v_1$};
                    \node[circle, draw=black!60, thick, minimum size=0.5cm, above right=1cm and 1cm of v1] (v2) {$v_2$};
                    \node[circle, draw=black!60, thick, minimum size=0.5cm, below right=1cm and 1cm of v2] (v3) {$v_3$};
                    \node[circle, draw=black!60, thick, minimum size=0.5cm, below right=1cm and 1cm of v1] (v4) {$v_4$};
                    \node[circle, draw=black!60, thick, minimum size=0.5cm, right=1cm of v3] (v5) {$v_5$};
                    \path[-, ultra thick]
                    (v1) edge [] node {} (v2)
                    (v1) edge [] node {} (v3)
                    (v2) edge [] node {} (v3)
                    (v3) edge [] node {} (v4)
                    (v3) edge [] node {} (v5)
                    ;
                \end{tikzpicture}

                - ребра не имеют направлений
            \end{center}


        \end{minipage}%
        \hfill
        \begin{minipage}{0.45\textwidth}

            \begin{center}
                ориентированный граф

                \begin{tikzpicture}
                    \node[circle, draw=black!60, thick, minimum size=0.5cm] (v1) {$v_1$};
                    \node[circle, draw=black!60, thick, minimum size=0.5cm, above right=1cm and 1cm of v1] (v2) {$v_2$};
                    \node[circle, draw=black!60, thick, minimum size=0.5cm, below right=1cm and 1cm of v2] (v3) {$v_3$};
                    \node[circle, draw=black!60, thick, minimum size=0.5cm, below right=1cm and 1cm of v1] (v4) {$v_4$};
                    \node[circle, draw=black!60, thick, minimum size=0.5cm, right=1cm of v3] (v5) {$v_5$};
                    \path[->, ultra thick]
                    (v1) edge [] node {} (v2)
                    (v3) edge [] node {} (v1)
                    (v2) edge [] node {} (v3)
                    (v3) edge [] node {} (v2)
                    (v3) edge [] node {} (v4)
                    (v3) edge [] node {} (v5)
                    ;
                \end{tikzpicture}

                - ребра имеют направления

            \end{center}
        \end{minipage}

        \smallvspace


        \item \textbf{Простые графы и псевдографы} (\textit{Simple graphs and pseudographs})

        \smallvspace

        \begin{minipage}{\linewidth}
            \begin{wrapfigure}{r}{0pt}
                \begin{tikzpicture}
                    \node[circle, draw=black!60, thick, minimum size=0.5cm] (v1) {$v_1$};
                    \node[below=0.3cm of v1] (v2) {\small Петля};
                    \draw
                    (v1) edge [loop above] node {} (v1)
                    ;
                \end{tikzpicture}
            \end{wrapfigure}

            Простой граф $G(V, E)$ - граф, в котором
            \begin{itemize}
                \item $V \neq \emptyset$ - граф не пустой
                \item $E \subseteq V \times V$ - ребра представлены как множество пар вершин
                \item $\forall v \in V \ \Pair{v, v} \notin E$ - граф не содержит петлей - ребер, соединяющих одну вершину с собой же

            \end{itemize}

            Псевдограф $G(V, E)$ - простой граф, в котором разрешены петли

        \end{minipage}

        \smallvspace

        \item \textbf{Мультиребра и мультиграфы} (\textit{Multiedges and multigraphs})

        \begin{minipage}{\linewidth}
            \begin{wrapfigure}{r}{0pt}
                \begin{tikzpicture}
                    \node[circle, draw=black!60, thick] (v1) {};
                    \node[circle, draw=black!60, thick, above right=0.4cm and 0.4cm of v1] (v2) {};
                    \node[circle, draw=black!60, thick, below right=0.4cm and 0.4cm of v2] (v3) {};
                    \node[circle, draw=black!60, thick, below right=0.4cm and 0.4cm of v1] (v4) {};
                    \node[circle, draw=black!60, thick, right=0.4cm of v3] (v5) {};
                    \path[-]
                    (v1) edge [] node {} (v2)
                    (v3) edge [] node {} (v1)
                    (v3) edge [] node {} (v2)
                    (v3) edge [bend right] node {} (v2)
                    (v1) edge [bend right] node {} (v4)
                    (v1) edge [bend left] node {} (v4)
                    (v3) edge [] node {} (v5)
                    (v3) edge [bend right] node {} (v5)
                    (v3) edge [bend left] node {} (v5)
                    ;
                \end{tikzpicture}
            \end{wrapfigure}

            \smallvspace


            Мультиребра - ребра, соединяющие одну и ту же пару вершин больше одного раза

            Мультиграфы - графы, содержащие мультиребра. В этом случае $E$ - мультимножество
        \end{minipage}

        \begin{minipage}{\linewidth}
            \begin{wrapfigure}{R}{-10pt}
                \begin{tikzpicture}
                    \node[circle, draw=black!60, thick] (v1) {};
                    \node[circle, draw=black!60, thick, right=0.4cm of v1] (v2) {};
                    \node[circle, draw=black!60, thick, right=0.4cm of v2] (v3) {};
                    \node[circle, draw=black!60, thick, right=0.4cm of v3] (v4) {};
                    \node[circle, draw=black!60, thick, below right=0.4cm and 0.2cm of v2] (v5) {};
                    \draw[rounded corners] ([xshift=-0.2cm,yshift=-0.3cm]v1.west) rectangle ([xshift=0.2cm,yshift=0.3cm]v4.east) {};
                    \node[above right=0.2cm and -0.15cm of v1] (vcc) {гиперребро};
                    \path[-]
                    (v3) edge [bend left] node {} (v5)
                    (v1) edge [bend right] node {} (v5)
                    ;
                \end{tikzpicture}
            \end{wrapfigure}

            \smallvspace

            \item \textbf{Гиперграфы} (\textit{Hypergraphs})

            Гиперребро - ребро, соединяющее несколько вершин

            Гиперграф - граф, содержащий гиперребро
        \end{minipage}

        \bigvspace

        \item \textbf{Нуль-граф, пустой граф и синглтон} (\textit{Null, empty, singleton graphs})

        Нуль-граф - граф, не содержащий вершин (и ребер)

        Пустой граф - граф, не содержащий ребер

        Синглтон - граф, содержащий из одной вершины

        \smallvspace

        \item \textbf{Полный граф} (\textit{Complete graph})

        Полный граф $K_n$ - простой граф из $n$ вершин, в которой все вершин соединены друг с другом

        \begin{center}
            \begin{tikzpicture}
                \node[circle, draw=black!60, thick] (v1) {};
                \node[circle, draw=black!60, thick, right=0.5cm of v1] (v2) {};
                \node[below right=1.2cm and -0.25cm of v1] (vcc) {$n = 2$};
                \path[-]
                (v1) edge [] node {} (v2)
                ;
                \node[circle, draw=black!60, thick, right=2.5cm of v2] (p1) {};
                \node[circle, draw=black!60, thick, below right=0.5cm and 0.25cm of p1] (p2) {};
                \node[circle, draw=black!60, thick, below left=0.5cm and 0.25cm of p1] (p3) {};
                \node[below right=0.4cm and -0.25cm of p3] (pcc) {$n = 3$};
                \path[-]
                (p1) edge [] node {} (p2)
                (p1) edge [] node {} (p3)
                (p3) edge [] node {} (p2)
                ;
                \node[circle, draw=black!60, thick, right=2.5cm of p1] (q1) {};
                \node[circle, draw=black!60, thick, right=0.5cm of q1] (q2) {};
                \node[circle, draw=black!60, thick, below=0.5cm of q1] (q3) {};
                \node[circle, draw=black!60, thick, right=0.5cm of q3] (q4) {};
                \node[below right=0.3cm and -0.25cm of q3] (qcc) {$n = 4$};
                \path[-]
                (q1) edge [] node {} (q2)
                (q1) edge [] node {} (q3)
                (q1) edge [] node {} (q4)
                (q2) edge [] node {} (q3)
                (q2) edge [] node {} (q4)
                (q3) edge [] node {} (q4)
                ;
            \end{tikzpicture}
        \end{center}

        \item \textbf{Взвешенный граф} (\textit{Weighted graph})

        Взвешенный граф - граф, в котором ребра (и/или вершины) имеют числовой вес.
        Иначе говоря, определена функция $w \ : \ E \to \Real$

        \item \textbf{Планарный граф} (\textit{Planar graphs})

        Планарный граф - граф, который можно изобразить на плоскости без пересечений рёбер

        По теореме Понтрягина-Куратовского граф планарен \textit{тогда и только тогда, когда}
        он не содержит подграфов, гомеоморфных полному графу из пяти вершин K\textsubscript{5} или графу «домики и колодцы» K\textsubscript{3,3}


        \item \textbf{Подграф} (\textit{Subgraph})

        Подграф графа $G(V, E)$ - граф $G^\prime(V^\prime, E^\prime)$
        такой, что $V^\prime \subseteq V, E^\prime \subseteq E$

        \item \textbf{Остовный подграф} (\textit{Spanning subgraph})

        Остовный подграф графа $G(V, E)$ - такой подграф $G^\prime(V, E^\prime)$, содержащий все вершины исходного


        \item \textbf{Порожденный подграф} (\textit{Induced subgraph})

        Порожденный подграф $G[S]$ графа $G(V, E)$ - подграф $G^\prime(S, E^\prime)$, который содержит все ребра, соединяющие вершины из $S$ в исходном графе

        \item \textbf{Отношение смежности} (\textit{Adjacency relation})

        Отношение смежности - отношение $A$ между вершинами, соединенными ребром: $A = \Set{\Pair{u, v} \ | \ \Pair{u, v} \in E}$

        \item \textbf{Матрица смежности} (\textit{Adjacency matrix})

        Матрица смежности - матрица $A_{V\timesV}$, выражающее отношение смежности

        % TODO picture

        \item \textbf{Отношение инцидентности} (\textit{Incidence relation})

        Отношение инцидентности - отношение $B$ между вершиной и соединяющей ее ребром: $B = \Set{\Pair{u, e} \ | \ u \in V \land e \in E \land \exists v \in V \ | \ (\Pair{u, v} \in E \lor \Pair{v, u} \in E)}$

        \item \textbf{Матрица инцидентности} (\textit{Incidence matrix})

        Матрица инцидентности - матрица $A_{V\timesV}$, выражающее отношение инцидентности

        % TODO picture

        \item \textbf{Степень вершины} (\textit{Vertex degree})

        Степень $\deg(v)$ вершины $v$ - количество и ребер из этой вершины (петли считаются дважды)

        Назовем $\delta(G)$ - минимальная степень вершины в графе, $\Delta(G)$ - максимальная степень вершины в графе

        \item \textbf{Регулярный граф} (\textit{Regular graph})

        $r$-регулярный граф - граф, все степени вершин которого равны $r$ - $\forall v \in V \ \deg(v) = r$

        \item \textbf{Лемма о рукопожатиях} (\textit{Handshaking lemma})

        $\sum_{v \in V} \deg(v) = 2|E|$ - сумма степеней всех вершин равна удвоенному количеству ребер

        \item \textbf{Изоморфизм графов} (\textit{Graph isomorphism})

        Графы $G(V, E)$ и $H(U, F)$ называются изоморфными, если существует биекция $f \ | \ V \to U$ такая, что
        если вершины $v$ и $u$ графа $G$ смежны, то и вершины $f(v)$ и $f(u)$ графа $H$ тоже смежны

        \begin{minipage}{\linewidth}
            \begin{wrapfigure}{r}{0.4\linewidth}
                \begin{center}
                    \includegraphics[width=\linewidth]{dismath/images/dismath_exam_list_graph_homomorphism}

                    Гомоморфизм

                    \smallvspace

                    \includegraphics[width=\linewidth]{dismath/images/dismath_exam_list_graph_homeomorphism}

                    Гомеоморфизм
                \end{center}
            \end{wrapfigure}

            \smallvspace

            \item \textbf{Гомоморфизм графов} (\textit{Graph homomorphism})

            Гомоморфизм графов - отображение вершин графа $G$ в вершины графа $H$ такое,
            что смежные вершины графа $G$ отображаются в смежные вершины графа $H$


            \item \textbf{Гомеоморфизм графов} (\textit{Graph homeomorphism})

            Деление (Subdivision) ребра $\Pair{u, v}$ - операция, добавляющее верщину $w$, ребра $\Pair{u, w}$ и $\Pair{w, v}$ и удаляющее ребро $\Pair{u, v}$

            Исключение (Smoothing) вершины $w$ (степени 2) - операция, обратная делению - исключение вершины $w$ и ребер $\Pair{u, w}$ и $\Pair{w, v}$ и добавление ребра $\Pair{u, v}$

            Графы $G$ и $H$ гомеоморфны, если граф $H$ можно получить в результате деления или исключения графа $G$

        \end{minipage}

        \smallvspace

        \item \textbf{Пути и циклы} (\textit{Walks, paths, trails, cycles})

        Путь (Walk) - последовательность из вершин и ребер, соединяющих соседние вершины: $l = (v_0, e_0, v_1, e_1, \dots, e_{n - 1}, v_n)$

        Цепь (Trail) - путь (walk), все ребра которого различны

        Простая цепь (Path) - путь (walk), все вершины (и соответственно ребра) которого различны

        Замкнутый путь (Closed walk) - путь (walk), начальная вершина которого является конечной

        Контур (Circuit) - цепь (trail), являющаяся замкнутым

        Цикл (Cycle) - простая цепь (path), являющаяся замкнутым

        (*терминология из Википедии)


        \item \textbf{Эйлеровы путь, цикл, граф} (\textit{Eulerian path, cycle, graph})

        Эйлеров путь - путь, содержащий все ребра графа

        Эйлеров цикл - замкнутый путь, содержащий все ребра графа

        Граф называют эйлеровым, если в нем есть эйлеров цикл. Граф называют полуэйлеровым, если в ней есть эйлеров путь.

        \item \textbf{Теорема Эйлера для графов} (\textit{Euler's theorem for graphs})

        Граф эйлеров, если все степени вершин четные, а ребра принадлежат одной компоненте связности

        Граф полуэйлеров, если ровно 2 вершины имеют нечетную степень, а ребра принадлежат одной компоненте связности

        \item \textbf{Гамильтоновы путь, цикл, граф} (\textit{Hamiltonian path, cycle, graph})

        Гамильтонов путь - путь, содержащий все вершины графа

        Гамильтонов цикл - замкнутый путь, содержащий все вершины графа

        Граф называют гамильтоновым, если в нем есть гамильтонов цикл. Граф называют полугамильтоновым, если в ней есть гамильтонов путь.

        \item \textbf{Теорема Оре} (\textit{Ore’s theorem})

        Теорема Оре - достаточное условие существования гамильтонова цикла: если в графе $G(V, E)$ для любых $u, v \in V$ $\deg u + \deg v \geq |V|$, то граф $G$ гамильтонов

        \item \textbf{Теорема Дирака} (\textit{Dirac’s theorem})

        Теорема Дирака - достаточное условие существования гамильтонова цикла: если в графе $G(V, E)$ для любой $u \in V$ $\deg u \geq \frac{|V|}{2}$, то граф $G$ гамильтонов


        \item \textbf{Эксцентриситет вершины} (\textit{Eccentricity of a vertex})

        Расстояние $\text{dist}(u, v)$ - длина (количество ребер) кратчайшего пути между $u$ и $v$

        Эксцентриситет $\varepsilon(v)$ - наибольшая длина кратчайшего пути от этой вершины до другой в этом графе: $\varepsilon(v) = \max_{u \in V} \text{dist}(v, u)$

        \item \textbf{Радиус и диаметр графа} (\textit{Radius and diameter of a graph})

        Радиус графа $\text{rad}(G)$ - наименьший эксцентриситет вершины из графа: $\text{rad}(G) = \min_{v \in V} \varepsilon(v)$

        Диаметр графа $\text{diam}(G)$ - наибольший эксцентриситет вершины из графа: $\text{diam}(G) = \max_{v \in V} \varepsilon(v)$

        \item \textbf{Центр графа} (\textit{Center of a graph})

        Центр графа - вершина (вершины), эксцентриситет которой равен радиусу графа: $\text{center}(G) = \Set{v \in V \ | \ \varepsilon(v) = \text{rad}(G)}$

        \item \textbf{Центроид дерева} (\textit{Centroid of a tree})

        Центроид дерева - вершина (или 2 вершины), удаление которой приведет к распаду на поддеревья, каждое из которое имеет не больше $\frac{|V|}{2}$ вершин

        Очевидно, что только деревья, состоящие из четного количества вершин, могут иметь 2 центроида

        % TODO picture

        \item \textbf{Клика} (\textit{Clique})

        Клика графа - порожденный подграф, который является полный графом. $1$-клика - вершина, $2$-клика - 2 вершины и ребро, $3$-клика - треугольник, $n$-клика - граф $K_n$

        \item \textbf{Независимое (стабильное множество)} (\textit{Independent set})

        Независимое (стабильное) множество - множество вершин, каждая из которых не соединена ребром с другой вершиной из множества

        \item \textbf{Паросочетание} (\textit{Matching})

        Паросочетание (независимое множество ребер) - множество ребер, каждые из которые не соединяют одну и ту же вершину

        \item \textbf{Идеальное паросочетание} (\textit{Perfect matching})

        Идеальное паросочетание - паросочетание, ребра которого инцидентны ко всем вершинам графа (то есть паросочетание, являющееся реберным покрытием)

        \item \textbf{Вершинное покрытие} (\textit{Vertex cover})

        Вершинное покрытие - множество вершин, к которым инцидентны все ребра графа

        \item \textbf{Реберное покрытие} (\textit{Edge cover})

        Реберное покрытие - множество ребер, которые инцидентны ко всем вершинам

        \item \textbf{Дерево} (\textit{Tree})

        Дерево - связный ацикличный граф

        \item \textbf{Лес} (\textit{Forest})

        Лес - несвязный граф, каждая компонента которого не имеет циклов (граф, состоящий из деревьев)

        \item \textbf{Минимальное остовное дерево} (\textit{Minimum spanning tree})

        Минимальное остовное дерево взвешенного графа $G(V, E, w)$ - дерево $T(V, E^\prime)$, сумма весов ребер которого имеет наименьшее значение

        \item \textbf{Код Прюфера} (\textit{Prüfer code})

        Код Прюфера - алгоритм кодировки маркированного дерева размера $n$ в последовательность чисел

        \textbf{Кодировка}:

        1. Делаем биекцию между названиями вершин и числа из диапазона $[1;n]$ (если необходимо)

        2. Берем лист с наименьшим значением, удаляем его, записываем в последовательность номер его родителя

        3. Повторяем 2. до тех пор, пока не останется 2 вершины - их кодировка тривиальна и не нуждается в хранении

        \textbf{Декодировка}:

        1. Создаем $n$ вершин, и множество вершин $W$, которых нет в последовательности

        2. Читаем номер вершины из последовательности

        3. Соединяем эту вершину с вершиной из $W$ с минимальным номером, удалив ее

        4. Добавляем вершину из последовательности в $W$

        5. Повторяем 2.-4.

        6. Соединяем 2 оставшиеся вершины из $W$


        \item \textbf{Двудольный граф} (\textit{Bipartite graph})

        Двудольный граф $K_{n,m}$ - граф, вершины которого можно разбить на две части размеров $n$ и $m$ таким образом, что вершины из одной части не смежны друг с другом

        \item \textbf{Теорема баланса регулярных двудольных графов} (\textit{Theorem on the balance of regular bipartite graphs})

        Если двудольный граф $K_{n,m}$ регулярный, то $n = m$

        $\Box$ Граф регулярный $\Longrightarrow \forall v \in V \ \deg v = r \in \Natural \Longrightarrow $ левая доля имеет $nr$ исходящих ребер, а правая доля имеет $mr$ входящих ребер, но так как вершины в долях не соединены ребрами, $nr = mr$ $\Box$

        \item \textbf{Теорема существования идеального паросочетания регулярного двудольного графа} (\textit{Theorem on the existence of a perfect matching in a regular bipartite graph})

        Теорема: у любого $r$-регулярного двудольного графа ($r > 0$) существует идеальное паросочетание

        $\Box$

        Пусть $G(V, E)$ - граф, вершины разбиваются на две доли $X \xor Y = V$

        Пусть $N(A) = \Set{y \in Y \ | \ \exists x \in X \ \Pair{x, y} \in E}$ - соседи (смежные вершины) вершин из множества $A \subseteq X$

        Докажем от противного: пусть идеального паросочетания не существует, тогда по теореме Холла $\exists S \subset X \ | \ |S| > |N(S)|$,
        но тогда кол-во ребер, выходящих из $S$, равно $r|S|$, но кол-во ребер, выходящих из $N(S)$, равно $r|N(S)|$

        Из этого $r|S| > r|N(S)|$, что невозможно, так как $N(S)$ - соседи $S$ - противоречие

        $\Box$


        \item \textbf{Теорема Холла} (\textit{Hall's theorem (on the existence of an X-perfect matching in a bipartite graph)})

        Пусть $G(V, E)$ - граф, вершины разбиваются на две доли $X \xor Y = V$

        Тогда в графе $G(V, E)$ существует $X$-идеальное паросочетание (паросочетание, покрывающее все вершины $X$) тогда и только тогда,
        когда для любого $A \subset X \ |A| \leq |N(A)|$

        $\Box$ Если существует такое $A$, что $|A| > |N(A)|$, то какой-либо вершине из $A$ не найдется противоположная вершина из $N(A)$ и $X$-идеального паросочетания не выйдет $\Box$


        \item \textbf{Связность в неориентированных графах} (\textit{Connectivity in undirected graphs})

        Компонента связность графа - максимальный подграф, в котором от каждой вершины до любой другой существует путь

        Граф считается связным, если он представляет собой одну компоненту связности

        \item \textbf{Сильная и слабая связность в ориентированных графах} (\textit{Strong and weak connectivity in directed graphs})

        Компонента сильной связности - максимальный подграф, в котором для любых вершин $u, v$ существует пути $u \rightsquigarrow v$ и $v \rightsquigarrow u$

        Компонента слабой связности - максимальный подграф, который является компонентой связности в неориентированном графе, полученном при удалении ориентации ребер у исходного

        % TODO picture

        \item \textbf{Конденсация ориентированного графа} (\textit{Condensation of a directed graph})

        Конденсация графа - сжатие сильно связных компонент графа до вершин с целью получения упрощенного и ациклического графа

        \item \textbf{Вершинная связность} (\textit{Vertex connectivity})

        Вершинная связность $\kappa(G)$ графа $G$ - минимальное число вершин, которое нужно удалить в графе, чтобы он стал несвязным или синглтоном

        \item \textbf{Реберная связность} (\textit{Edge connectivity})

        Реберная связность $\lambda(G)$ графа $G$ - минимальное число ребер, которое нужно удалить в графе, чтобы он стал несвязным

        \item \textbf{Теорема Уитни} (\textit{Whitney's theorem})

        Для любого графа $\kappa(G) \leq \lambda(G) \leq \delta(G)$

        $\Box$

        Допустим, что $\kappa(G) > \lambda(G)$, тогда после удаления $\lambda(G)$ ребер будет $k \leq \lambda(G)$ вершин со одной стороны и $m \leq \lambda(G)$ с другой.
        Но мы их тоже можем удалить, и граф распадется, значит $\lambda < \kappa(G) = \min (k, m) \leq \lambda(G)$ - противоречие

        Допустим, что $\lambda(G) > \delta(G)$, тогда мы можем найти в графе вершину с наименьшей степенью $\delta(G)$,
        при удалении $\delta(G)$ ребер граф распадется, значит $\lambda(G) = \delta(G)$ - противоречие

        $\Box$


        \item \textbf{$k$-связный граф} (\textit{k-connected graph})

        $k$-вершинно-связный граф - граф, остающийся связным после удаления $k$ вершин ($\kappa(G) \geq k$).

        НО: синглтон имеет $\kappa(G) = 0$, он не $1$-вершинно-связный, при этом он связный; $K_2$ имеет $\kappa(G) = 1$,
        поэтому он не $2$-вершинно-связный, но $K_2$ может быть блоком

        $k$-реберно-связный граф - граф, остающийся связным после удаления $k$ ребер ($\lambda(G) \geq k$)

        НО: у синглтона $\lambda(G) = 0$, он не $1$-реберно-связный, при этом синглтон - компонента реберной двусвязности

        \item \textbf{Теорема Менгера} (\textit{Menger's theorem})

        Теорема (Менгера о реберной двойственности в ориентированном графе):

        Между вершинами $u$ и $v$ существует $L$ реберно непересекающихся путей тогда и только тогда,
        когда после удаления любых $(L - 1)$ ребер существует путь из $u$ в $v$.

        Теорема (Менгера о вершинной двойственности в ориентированном графе):

        Между вершинами $u$ и $v$ существует $L$ вершинно непересекающихся путей тогда и только тогда,
        когда после удаления любых $(L - 1)$ вершин существует путь из $u$ в $v$.

        \href{https://neerc.ifmo.ru/wiki/index.php?title=%D0%A2%D0%B5%D0%BE%D1%80%D0%B5%D0%BC%D0%B0_%D0%9C%D0%B5%D0%BD%D0%B3%D0%B5%D1%80%D0%B0#.D0.A2.D0.B5.D0.BE.D1.80.D0.B5.D0.BC.D0.B0}{Доказательства}


        \item \textbf{Двусвязность} (\textit{Biconnectivity})

        Двусвязность (вершинная) определяется как отношение эквивалентности 2 ребер, между концами которых существуют 2 вершинно-различных пути  % a little strange

        Компонента (вершинной) двусвязности (также блок) - подграф, который включает все двусвязные ребра (класс эквивалентности двусвязности).

        Реберная двусвязность определяется как отношение эквивалентности 2 вершины, между которыми существуют 2 реберно-различных пути

        Компонента реберной двусвязности - подграф, который включает все двусвязные вершины (класс эквивалентности двусвязности).

        \item \textbf{Точка сочленения} (\textit{Articulation point})

        Точка сочленения - вершина, принадлежащая нескольким компонентам (вершинной) двусвязности

        \item \textbf{Мост} (\textit{Bridge})

        Мост - ребро, соединяющее две компоненты реберной двусвязности

        \item \textbf{Блок} (\textit{Blocks})

        Блок - компонента вершинной двусвязности

        \item \textbf{Дерево блоков и точек сочленений} (\textit{Block-cut tree})

        Дерево блоков и точек сочленений графа - дерево, в котором каждая вершина представляет собой либо точку сочленения, либо блок,
        при этом вершина точки сочленения соединена только с вершиной блока и наоборот

        % TODO bc-tree picture

    \end{enumerate}

    \subsection{6. Теория автоматов.}

    \begin{enumerate}
        \item \textbf{Детерминированный конечный автомат} (\textit{Deterministic Finite Automaton (DFA)})

        Детерминированный конечный автомат $A = (Q, \Sigma, \delta, q_0, F)$ - объект, представляющий собой множество состояний $Q$, множество входных символов $\Sigma$,
        функция переходов $\delta \ : \ Q \times \Sigma \to Q$, начальное состояние $q_0$ и множество конечных состояний $F$

        Автомат принимает какую-то цепочку символов из $\Sigma^*$ и решает, принадлежит ли она соответствующему автомату регулярному языку $L$

        Для простоты обычно выбирают $\Sigma = \Set{0, 1}$

        Автомат можно представить как орграф

        \tikz[myautomatonstyle]{
            \node[state, initial] (s0) {$q_0$};
            \node[below=0.75cm of s0] (s0cc) {начальное};
            \node[state, right=3cm of s0] (s1) {$q_2$};
            \node[state, accepting, right=3cm of s1] (s2) {$q_1$};
            \node[below=0.75cm of s2] (s0cc) {конечное};
            \path[->]
            (s0) edge [above] node {$0$} (s1)
            (s0) edge [loop above, above] node {$1$} (s0)
            (s1) edge [above] node {$1$} (s2)
            (s1) edge [loop above, above] node {$0$} (s1)
            (s2) edge [loop above, above] node {$0,1$} (s2)
            ;
        }

        Или как таблицу функции переходов

        \begin{tabular}{c|cc}
            & $0$   & $1$   \\
            \hline
            $\to q_0$ & $q_2$ & $q_0$ \\
            \hline
            $*q_1$    & $q_1$ & $q_1$ \\
            \hline
            $q_2$     & $q_2$ & $q_1$
        \end{tabular}

        \item \textbf{Недетерминированный конечный автомат (НКА)} (\textit{Non-deterministic Finite Automaton (NFA)})

        Недетерминированный конечный автомат $A = (Q, \Sigma, \delta, q_0, F)$ - объект, представляющий собой множество состояний $Q$, множество входных символов $\Sigma$,
        функция переходов $\delta \ : \ P(Q) \times \Sigma \to \mathcal{P}(Q)$, начальное состояние $q_0$ и множество конечных состояний $F$

        Главное отличие НКА от ДКА: от одного состояния в НКА можно перейти сразу к нескольким другим или к ни одному

        Пример:

        \begin{multicols}{2}
            \tikz[myautomatonstyle]{
                \node[state, initial] (s0) {$q_0$};
                \node[state, right=3cm of s0] (s1) {$q_1$};
                \node[state, accepting, right=3cm of s1] (s2) {$q_2$};
                \path[->]
                (s0) edge [loop above, above] node {$0,1$} (s0)
                (s0) edge [above] node {$0$} (s1)
                (s1) edge [above] node {$1$} (s2)
                ;
            }

            \begin{tabular}{c|cc}
                & $0$              & $1$         \\
                \hline
                $\to q_0$ & $\Set{q_0, q_1}$ & $\Set{q_0}$ \\
                \hline
                $q_1$     & $\emptyset$      & $\Set{q_1}$ \\
                \hline
                $*q_2$    & $\emptyset$      & $\emptyset$
            \end{tabular}

        \end{multicols}

        \item \textbf{Формальные языки} (\textit{Formal languages})

        Формальный язык $L$ - множество конечных слов над конечным алфавитом символов $\Sigma$

        \item \textbf{Операции над формальными языками (конкатенация, объединение, замыкание Клини)} (\textit{Operations on formal languages (concat, union, Kleene closure)})

        Конкатенация $LM$ языков $L$ и $M$ - множество слов, состоящих из записанных подряд слова из $L$ и слова из $M$: $LM = \Set{uw \ | \ u \in L \land w \in M}$

        Объединения $L \union M$ языков $L$ и $M$ - множество слов, которые содержатся в $L$ или/и в $M$: $L \union M = \Set{w \ | \ w \in L \lor w \in M}$

        Замыкание Клини $L^*$ языка $L$ - множество слов, которые могут быть получены в результате конкатенации слов из $L$: $L^* = \Set{w_1 w_2 \dots w_n \forall n \geq 0 \ | \ w_i \in L}$ (включая пустое слово $\varepsilon$)

        \item \textbf{Регулярные языки} (\textit{Regular languages})

        Регулярный язык - формальный язык, который задается некоторым автоматом

        Также регулярный язык задается индуктивно:

        1. Пустое множество $\emptyset$ и множество из пустой строки $\Set{\varepsilon}$ являются регулярными языками

        2. Множество из однобуквенного слова $\Set{a}$, где $a \in \Sigma$ является регулярным языком

        3. Для регулярных языков $\alpha$ и $\beta$ объединение $\alpha \union \beta$, конкатенация $\alpha\beta$ и замыкание Клини $\alpha^*$ - тоже регулярные языки

        4. Других регулярных языков нет

        \item \textbf{Регулярное выражение} (\textit{Regular expression})

        Регулярное выражение - способ описания регулярного языка

        \begin{tabular}{c|c}
            Регулярное выражение & Язык, который оно описывает \\
            \hline
            & $\emptyset$                 \\
            $\varepsilon$        & $\Set{\varepsilon}$         \\
            $a$ (какое-либо РВ)  & $\alpha$                    \\
            $b$ (какое-либо РВ)  & $\beta$                     \\
            $(a)$                & $\alpha$                    \\
            $ab$                 & $\alpha\beta$               \\
            $a + b$              & $\alpha \union \beta$       \\
            $a*$                 & $\alpha^*$                  \\

        \end{tabular}

        \item \textbf{Теорема Клини} (\textit{Kleene's theorem})

        Для любого регулярного выражения существует конечный автомат, и они описывают равные регулярные языки

        \item \textbf{Конструкция подмножеств (ДКА из НКА)} (\textit{Powerset construction (DFA from NFA)})

        Из состояний $Q$ НКА построим ДКА с состояниями, каждое из которых представляет собой подмножество $Q$.
        Далее при помощи магии умным образом строим переходы

        \begin{multicols}{2}
            \tikz[myautomatonstyle]{
                \node[state, initial] (s0) {$q_0$};
                \node[state, right=2cm of s0] (s1) {$q_1$};
                \node[state, accepting, right=2cm of s1] (s2) {$q_2$};
                \path[->]
                (s0) edge [loop above, above] node {$0,1$} (s0)
                (s0) edge [above] node {$0$} (s1)
                (s1) edge [above] node {$1$} (s2)
                ;
            }

            \smallvspace

            \begin{tabular}{c|cc}
                & $0$              & $1$              \\
                \hline
                $\emptyset$            & $\emptyset$      & $\emptyset$      \\
                \hline
                $\to\Set{q_0}$         & $\Set{q_0, q_1}$ & $\Set{q_0}$      \\
                \hline
                $\Set{q_1}$            & $\emptyset$      & $\Set{q_2}$      \\
                \hline
                $\Set{q_0, q_1}$       & $\Set{q_0, q_1}$ & $\Set{q_0, q_2}$ \\
                \hline
                $*\Set{q_2}$           & $\emptyset$      & $\emptyset$      \\
                \hline
                $*\Set{q_2, q_0}$      & $\Set{q_0, q_1}$ & $\Set{q_0}$      \\
                \hline
                $*\Set{q_2, q_1}$      & $\emptyset$      & $\Set{q_2}$      \\
                \hline
                $*\Set{q_2, q_1, q_0}$ & $\Set{q_0, q_1}$ & $\Set{q_0, q_2}$ \\
            \end{tabular}

            \tikz[myautomatonstyle]{
                \node[state, initial] (s0) {\scriptsize $\Set{q_0}$};
                \node[state, right=2cm of s0] (s01) {\scriptsize $\Set{q_0, q_1}$};
                \node[state, accepting, right=2cm of s01] (s02) {\scriptsize $\Set{q_0, q_2}$};
                \node[state, accepting, right=2cm of s02] (s012) {\scriptsize $\Set{q_0, q_1, q_2}$};
                \node[state, below=2cm of s0] (s) {\scriptsize $\emptyset$};
                \node[state, right=2cm of s] (s1) {\scriptsize $\Set{q_1}$};
                \node[state, accepting, right=2cm of s1] (s2) {\scriptsize $\Set{q_2}$};
                \node[state, accepting, right=2cm of s2] (s12) {\scriptsize $\Set{q_1, q_2}$};
                \path[->]
                (s) edge [loop left, left] node {$0,1$} (s)
                (s0) edge [loop above, above] node {$1$} (s0)
                (s0) edge [above] node {$0$} (s01)
                (s1) edge [above] node {$1$} (s2)
                (s1) edge [above] node {$0$} (s)
                (s01) edge [loop above,above] node {$0$} (s01)
                (s01) edge [bend left,above] node {$1$} (s02)
                (s02) edge [bend left,below] node {$0$} (s01)
                (s02) edge [out=-135,in=-45,above] node {$1$} (s0)
                (s012) edge [out=135,in=45,above] node {$0$} (s01)
                (s012) edge [above] node {$1$} (s02)
                (s12) edge [above] node {$1$} (s2)
                (s12) edge [out=-135,in=-45, below] node {$0$} (s)
                (s2) edge [out=-135,in=-45, above] node {$0,1$} (s)
                ;
            }
        \end{multicols}

        Как можем видеть, 5 состояний являются недостижимыми, поэтому их мы можем удалить.
        В итоге в ДКА остается 3 состояния (зачастую количество состояний не $2^{|Q|}$, а чуть больше $|Q|$)

        \tikz[myautomatonstyle]{
            \node[state, initial] (s0) {\scriptsize $\Set{q_0}$};
            \node[state, right=2cm of s0] (s01) {\scriptsize $\Set{q_0, q_1}$};
            \node[state, accepting, right=2cm of s01] (s02) {\scriptsize $\Set{q_0, q_2}$};
            \path[->]
            (s0) edge [loop above, above] node {$1$} (s0)
            (s0) edge [above] node {$0$} (s01)
            (s01) edge [loop above,above] node {$0$} (s01)
            (s01) edge [bend left,above] node {$1$} (s02)
            (s02) edge [bend left,below] node {$0$} (s01)
            (s02) edge [out=-135,in=-45,below] node {$1$} (s0)
            ;
        }

        \item \textbf{$\varepsilon$-НКА} (\textit{$\varepsilon$-NFA})

        $\varepsilon$-НКА $A = (Q, \Sigma, \delta, q_0, F)$ - НКА, допускающий $\varepsilon$ переходы (переходы по пустым строчкам)
        Тогда $\delta \ : \ Q \times (\Sigma \union \Set{\varepsilon}) \to \mathcal{P}(Q)$

        Пример - автомат, допускающий цепочки $01(01)*$:

        \tikz[myautomatonstyle]{
            \node[state, initial] (s0) {$q_0$};
            \node[state, right=2cm of s0] (s1) {$q_1$};
            \node[state, accepting, right=2cm of s1] (s2) {$q_2$};
            \path[->]
            (s0) edge [above] node {$0$} (s1)
            (s1) edge [above] node {$1$} (s2)
            (s2) edge [out=-135,in=-45,below] node {$\varepsilon$} (s0)
            ;
        }

        \item \textbf{Конструкция НКА из $\varepsilon$-НКА} (\textit{NFA construction from $\varepsilon$-NFA})

        Алгоритм:

        1. Транзитивное замыкание: если из состояния $u$ мы можем сделать больше одного $\varepsilon$-перехода в состояние $w$, то мы можем сделать сразу $\varepsilon$-переход из $u$ в $w$

        2. Добавление допускающих состояний: если есть $\varepsilon$-переход из $u$ в $w$, причем $w$ - допускающее состояние, то $u$ можно сделать тоже допускающем

        3. Добавление ребер: если есть переходы $\delta(u, \varepsilon) = v$ и $\delta(v, c) = w$, то сделаем равное ребро $\delta(u, c) = w$

        4. Удаление $\varepsilon$-переходов

        \item \textbf{Конструкция Томпсона ($\varepsilon$-НКА из регулярного выражения)} (\textit{Thompson’s construction ($\varepsilon$-NFA from regular expression)})


        \begin{tabular}{m{0.12\linewidth}|m{0.12\linewidth}|m{0.76\linewidth}}
            Регулярное выражение & Язык, который оно описывает & Автомат \\
            \hline

            & $\emptyset$ & \smallvspace \tikz[myautomatonstyle]{
                \node[state, initial] (s0) {};
            } \\
            $\varepsilon$ & $\Set{\varepsilon}$ & \tikz[myautomatonstyle]{
                \node[state, initial] (s0) {};
                \node[state, accepting, right=2cm of s0] (s1) {};
                \path[->] (s0) edge [above] node {$\varepsilon$} (s1);
            } \\
            $c$ (символ) & $\Set{c}$ & \tikz[myautomatonstyle]{
                \node[state, initial] (s0) {};
                \node[state, accepting, right=2cm of s0] (s1) {};
                \path[->] (s0) edge [above] node {$c$} (s1);
            } \\
            $ab$ & $\alpha\beta$ & \tikz[myautomatonstyle]{
                \node[state, initial] (s0) {};
                % \node[above right=2cm and 2cm of s0] (r1) [draw, rounded rectangle] {rectangle \\ oppopoopo};
                \node[state, right=1.5cm of s0] (r1) {};
                \node[right=1.5cm of r1] (rcap) {Автомат $\alpha$};
                \node[state, right=3cm of r1] (r2) {};
                \node[state, right=1.5cm of r2] (l1) {};
                \node[right=1.5cm of l1] (lcap) {Автомат $\beta$};
                \node[state, right=3cm of l1] (l2) {};
                \node[state, accepting, right=1.5cm of l2] (s1) {};
                \draw[rounded corners] ([xshift=-0.1cm,yshift=-0.5cm]r1.west) rectangle ([xshift=0.1cm,yshift=0.5cm]r2.east) {};
                \draw[rounded corners] ([xshift=-0.1cm,yshift=-0.5cm]l1.west) rectangle ([xshift=0.1cm,yshift=0.5cm]l2.east) {};
                \path[->]
                (s0) edge [bend left, above] node {$\varepsilon$} (r1)
                (r2) edge [bend right, above] node {$\varepsilon$} (l1)
                (l2) edge [bend left, above] node {$\varepsilon$} (s1)
                ;
            } \\
            $a + b$ & $\alpha \union \beta$ & \tikz[myautomatonstyle]{
                \node[state, initial] (s0) {};
                % \node[above right=2cm and 2cm of s0] (r1) [draw, rounded rectangle] {rectangle \\ oppopoopo};
                \node[state, above right=0.65cm and 2cm of s0] (r1) {};
                \node[right=1.5cm of r1] (rcap) {Автомат $\alpha$};
                \node[state, right=3cm of r1] (r2) {};
                \node[state, below right=0.65cm and 2cm of s0] (l1) {};
                \node[right=1.5cm of l1] (lcap) {Автомат $\beta$};
                \node[state, right=3cm of l1] (l2) {};
                \node[state, accepting, right=7cm of s0] (s1) {};
                \draw[rounded corners] ([xshift=-0.1cm,yshift=-0.5cm]r1.west) rectangle ([xshift=0.1cm,yshift=0.5cm]r2.east) {};
                \draw[rounded corners] ([xshift=-0.1cm,yshift=-0.5cm]l1.west) rectangle ([xshift=0.1cm,yshift=0.5cm]l2.east) {};
                \path[->]
                (s0) edge [bend left, above] node {$\varepsilon$} (r1)
                edge [bend right, above] node {$\varepsilon$} (l1)
                (l2) edge [bend right, above] node {$\varepsilon$} (s1)
                (r2) edge [bend left, above] node {$\varepsilon$} (s1)

            } \\
            $a*$ & $\alpha^*$ & \tikz[myautomatonstyle]{
                \node[state, initial] (s0) {};
                \node[state, above right=1cm and 2cm of s0] (r1) {};
                \node[right=1.5cm of r1] (rcap) {Автомат $\alpha$};
                \node[state, right=3cm of r1] (r2) {};
                \node[state, accepting, right=7cm of s0] (s1) {};
                \draw[rounded corners] ([xshift=-0.1cm,yshift=-0.5cm]r1.west) rectangle ([xshift=0.1cm,yshift=0.5cm]r2.east) {};
                \path[->]
                (s0) edge [bend left, above] node {$\varepsilon$} (r1)
                edge [bend right, below] node {$\varepsilon$} (s1)
                (r2) edge [bend left, above] node {$\varepsilon$} (s1)
                ;
                \draw[->] (r2) edge [out=-130,in=-50,below] node {$\varepsilon$} (r1)
            } \\


        \end{tabular}

        Пользуясь этими преобразованиям, можно построить $\varepsilon$-НКА


        \item \textbf{Алгоритм Клини} (\textit{Kleene’s algorithm})

        Алгоритм Клини - алгоритм для превращения ДКА в регулярное выражение

        Пусть ДКА $(Q, \Sigma, \delta, q_0, F)$, а $Q = \Set{q_0, \dots, q_n}$, $F = \Set{q_i \ | \ i \in \Natural_F \subset \Natural}$

        Определим $R^{-1}_{ij} = a_1 + \dots + a_m$, где $q_j \in \delta(q_i, a_k)$ для $k$ - другими словами все символы, по которым можно перейти из $q_i$ в $q_j$.
        Для $i = j$ $R^{-1}_{ii} = a_1 + \dots + a_m + \varepsilon$

        Далее для каждого $k$ от $0$ до $n$ итеративно определяем

        $R^k_{ij} = R^{k - 1}_{ik} (R^{k - 1}_{kk})* R^{k-1}_{kj} | R^{k-1}_{ij}$

        Таким образом, ответом будет регулярное выражение $\bigunion_{i \in \Natural_F} R^n_{0i}$

        \item \textbf{Лемма о накачке для регулярных языков} (\textit{Pumping lemma for regular languages})

        Если $L$ - регулярный язык, то существует константа $p \geq 1$, зависящая от $L$, такая, что любая строка $w \in L (|w| \geq p)$
        может быть записана $w = xyz$ так, что удовлетворены условия:

        1. $|y| \geq 1$

        2. $|xy| \leq p$

        3. Для любого $n \geq 0$ $xy^n z \in L$

        \item \textbf{Свойства замыкания регулярных языков} (\textit{Closure properties of regular languages})

        Для регулярных языков $L$ и $M$:

        1. $L^*$ (замыкание Клини) - регулярный язык

        2. $L \union M$ (объединение) - регулярный язык

        3. $LM$ (конкатенация) - регулярный язык

        4. $L \cap M$ (пересечение) - регулярный язык

        5. $\overline{L}$ (дополнение - $\Sigma^* \setminus L = \overline{L}$) - регулярный язык

        6. $L^R$ (инверсия - $abac \to caba$) - регулярный язык

        7. $L \setminus M$ (разность) - регулярный язык

        8. $h(L)$ (гомоморфизм $h \ | \ \Sigma \to \Sigma^*$, например $h(0) = ab, h(1) = ba$) - регулярный язык

        9. $h^{-1}(L)$ (обратный гомоморфизм $h^{-1} \ | \ \Sigma^* \to \Sigma$, например $h^{-1}(01) = a, h^{-1}(10) = b$) - регулярный язык

        \smallvspace

        \begin{minipage}{\linewidth}
            \begin{wrapfigure}{R}{0.4\linewidth}
                \tikz[myautomatonstyle]{
                    \node[state, initial] (s0) {$q_0$};
                    \node[state, right=2cm of s0] (s1) {$q_1$};
                    \node[state, right=2cm of s1] (s2) {$q_2$};
                    \path[->]
                    (s0) edge [bend left, above] node {$0/0$} (s1)
                    (s0) edge [bend right, below] node {$1/1$} (s1)
                    (s1) edge [bend left, above] node {$0/0$} (s2)
                    (s1) edge [bend right, below] node {$1/1$} (s2)
                    (s2) edge [out=-130,in=-50, below] node {$0/1$} (s0)
                    (s2) edge [out=130,in=50, above] node {$1/0$} (s0)
                    ;
                }

                \footnotesize
                \textit{Этот автомат Мили преобразует каждый 3-ий символ с 0 на 1 и наоборот:} $\mathtt{100101} \to \mathtt{101100}$

            \end{wrapfigure}

            \item \textbf{Автомат Мили} (\textit{Mealy machine})

            Автомат Мили $M_{\text{Mealy}} = (Q, \Sigma, \Omega, q_0, \delta, \lambda)$ - автомат, выводящий последовательность, которая зависит от входной последовательности

            Здесь $\Omega$ - алфавит выходящей последовательности, а
            $\lambda \ : \ Q \times \Sigma \to \Omega$ - функция выходов, зависящая от состояния и входного символа

            Значение функции $\lambda$ на ребре графа обозначают после переходного символа


        \end{minipage}

        \begin{minipage}{\linewidth}
            \begin{wrapfigure}{R}{0.35\linewidth}
                \tikz[myautomatonstyle]{
                    \node[state, initial] (s0) {$q_0/${\Large 👍}};
                    \node[state, right=2cm of s0] (s1) {$q_1/${\Large 💀}};
                    \node[state, right=2cm of s1] (s2) {$q_2/${\Large 💀}};
                    \path[->]
                    (s0) edge [loop above, above] node {$0$} (s0)
                    (s0) edge [bend left, above] node {$1$} (s1)
                    (s1) edge [bend left, above] node {$0$} (s2)
                    (s1) edge [bend left, below] node {$1$} (s0)
                    (s2) edge [bend left, below] node {$0$} (s1)
                    (s2) edge [loop above, above] node {$1$} (s0)
                    ;
                }

                \footnotesize
                \textit{Этот автомат Мура выдает {\Large 👍}, если двоичное число делится на 3, иначе {\Large 💀}}


            \end{wrapfigure}

            \item \textbf{Автомат Мура} (\textit{Moore machine})

            Автомат Мура $M_{\text{Moore}} = (Q, \Sigma, \Omega, q_0, \delta, \lambda)$ - автомат, выводящий последовательность, зависящую от входной последовательности

            Как и в автомате Мили, в автомате Мура $\Omega$ - алфавит выходящей последовательности, но
            $\lambda \ : \ Q \to \Omega$ - функция выходов, зависящая от текущего состояния

            Значение функции $\lambda$ на графе обозначают в вершине состояния

        \end{minipage}

        \smallvspace


        \item \textbf{Пустота языка конечного автомата} (\textit{Emptiness of finite automaton language})

        Язык автомата $L$ считается пустым в том случае, если язык не содержит никаких цепочек (в том числе пустых) - $L = \emptyset$

        По конечному автомату можно понять, является ли язык пустым: если какое-либо допускающее состояние можно достигнуть из начального,
        то язык автомата не является пустым (это можно определить при помощи обхода графа)

        \item \textbf{Конечность языка конечного автомата} (\textit{Finiteness of finite automaton language})

        Язык автомата $L$ считается конечным, если он содержит конечное множество цепочек

        Конечность языка можно определить так: если есть такое состояние $v$, что к нему можно прийти из начального состояния,
        от него можно прийти к какому-либо допускающему состоянию, а из $v$ можно каким-либо образом прийти в $v$, то язык бесконечный -
        мы можем сколь угодно раз зацикливаться по $v$ и получать бесконечное количество цепочек

        \item \textbf{Эквивалентность конечных автоматов} (\textit{Equivalence of finite automata})

        Автоматы эквивалентны, если они допускают одно и то же множество цепочек.

        Пусть автомат $A = (Q, \Sigma, \delta, q_0, F)$. Введем функцию $\lambda \ : \ Q \to \Set{0, 1}$, возвращающую $\mathtt{1}$, если состояние допускающее, иначе $\mathtt{0}$

        Введем такое отношение эквивалентности $R_0 \subset Q \times Q$ между состояниями. Определим, что $q \, R_0 \, p$ в том случае, если $\lambda(q) = \lambda(p)$

        Теперь определим $R_1$ как отношение состояний $q$ и $p$, для которых $\lambda(q) = \lambda(p)$ и $\lambda(\delta(q, c)) = \lambda(\delta(p, c))$ для любого символа $c \in \Sigma$

        Теперь определим $R_2$ как отношение состояний $q$ и $p$, для которых $\lambda(\hat\delta(q, w)) = \lambda(\hat\delta(p, w))$ для любой цепочки $w \in \Sigma^*$ длины не больше 2

        Окончательно определим $R$ как отношение состояний $q$ и $p$, для которых $\lambda(\hat\delta(q, w)) = \lambda(\hat\delta(p, w))$ для любой цепочки $w \in \Sigma^*$


        \begin{minipage}{\linewidth}
            \begin{wrapfigure}{R}{0.35\linewidth}
                \tikz[myautomatonstyle]{
                    \node[state, initial] (m0) {$q_1$};
                    \node[state, accepting, right=3.5cm of m0] (m1) {$F_1$};
                    \node[right=0.1cm of m0.east] (mcap) {Автомат $M$};
                    \draw[rounded corners] ([xshift=-0.3cm,yshift=-0.6cm]m0.west) rectangle
                        ([xshift=0.3cm,yshift=0.6cm]m1.east) {};


                    \node[state, below=2cm of m0] (n0) {$q_2$};
                    \node[state, accepting, right=3.5cm of n0] (n1) {$F_2$};
                    \node[right=0.1cm of n0.east] (ncap) {Автомат $N$};
                    \draw[rounded corners] ([xshift=-0.3cm,yshift=-0.6cm]n0.west) rectangle
                        ([xshift=0.3cm,yshift=0.6cm]n1.east) {};

                    \node[below right=0.3cm and 0.5cm of n0.south] (acap) {Автомат $A$};
                    \draw[rounded corners] ([xshift=-0.6cm,yshift=-1.6cm]n0.west) rectangle
                        ([xshift=0.6cm,yshift=1.2cm]m1.east) {};
                }

            \end{wrapfigure}

            Пусть даны автоматы $M = (Q_1, \Sigma, \delta_1, q_1, F_1)$ и $N = (Q_2, \Sigma, \delta_2, q_2, F_2)$.

            Теперь построим такой автомат $A = (Q, \Sigma, \delta, q_1, F)$, выбрав какое-либо начальное состояние, объединив множества состояний и множества допускающих состояний и расширив функцию переходов

            Автоматы $M$ и $N$ эквивалентны, если состояния $q_1$ и $q_2$ принадлежат одному классу эквивалентности, то есть $q_1 \, R \, q_2$

        \end{minipage}

        \smallvspace

        \item \textbf{Теорема Майхилла-Нероуда} (\textit{Myhill-Nerode theorem})

        На языке $L$ определим различимое расширение как строку $z$, которой можно расширить строки $x$ и $y$ до строк $xz$ и $yz$ так, что
        только одна из этих строк принадлежит языку $L$

        Определим отношение эквивалентности $\sim_L$ на языке $L$ как отношение между такими строками $x$ и $y$,
        что не существует никакого различимого расширения $z$ (то есть либо строки $xz, yz$ принадлежат языку, либо не принадлежат).
        Отношение $\sim_L$ разделяет цепочки на классы эквивалентности

        Теорема Майхилла-Нероуда гласит:

        1) Язык $L$ регулярен тогда и только тогда, когда количество классов эквивалентности конечно

        2) Минимальный ДКА, допускающий язык $L$, имеет столько состояний, сколько классов эквивалентности

        3) Любой минимальный ДКА, допускающий $L$, изоморфен следующему:
        пусть каждый класс эквивалентности $[x]$ для строки $x$ будет соотнесен к состоянию, причем существуют переходы $[x] \to [xa]$ для $a \in \Sigma$,
        начальным состоянием будет состояние класса $[\varepsilon]$, а допускающими состояниями будут состояния классов $[s]$ для $s \in L$


    \end{enumerate}

    \subsection{7. Комбинаторика.}

    \begin{enumerate}
        \item \textbf{} (\textit{Ordered arrangements})

        \item \textbf{} (\textit{Permutations})

        \item \textbf{} (\textit{k-permutations})

        \item \textbf{} (\textit{Cyclic permutations})

        \item \textbf{} (\textit{Unordered arrangements})

        \item \textbf{} (\textit{k-combinations})

        \item \textbf{} (\textit{Multisets})

        \item \textbf{} (\textit{Permutations of multisets})

        \item \textbf{} (\textit{Combinations of infinite multisets})

        \item \textbf{} (\textit{Compositions})

        \item \textbf{} (\textit{Set partitions})

        \item \textbf{} (\textit{Stirling numbers of the second kind})

        \item \textbf{} (\textit{Integer partitions})

        \item \textbf{} (\textit{Principle of Inclusion-Exclusion})

    \end{enumerate}

    \subsection{8. Рекуррентности и производящие функции.}


    \begin{enumerate}
        \item \textbf{} (\textit{Recurrence relations})

        \item \textbf{} (\textit{Solving recurrence relations using characteristic equations})

        \item \textbf{} (\textit{Generating functions})

        \item \textbf{} (\textit{Power series})

        \item \textbf{} (\textit{Solving linear recurrences using generating functions})

        \item \textbf{} (\textit{Solving combinatorial problems using generating functions})

        \item \textbf{} (\textit{Operators and annihilators})

        \item \textbf{} (\textit{Solving linear recurrences using annihilators})

        \item \textbf{} (\textit{Catalan numbers})

        \item \textbf{} (\textit{Divide-and-Conquer algorithms analysis using recursion trees})

        \item \textbf{} (\textit{Master theorem})

        \item \textbf{} (\textit{Akra-Bazzi method})

    \end{enumerate}


\end{document}