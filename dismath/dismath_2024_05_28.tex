\documentclass[12pt]{article}
\usepackage{preamble}

\pagestyle{fancy}
\fancyhead[LO,LE]{$\mathcal{D}$искретная математика}
\fancyhead[CO,CE]{28.05.2024}
\fancyhead[RO,RE]{Лекции Чухарева К. И.}



\begin{document}
    \begin{itemize}
        \item \textbf{Линейные рекуррентности} (Linear recurrences)

        $\underset{\text{линейная комб. рекуррентных членов}}{\undergroup{k_1 a_n + k_2 a_{n - 1} + k_3 a_{n - 2} + \dots}} =
        \underset{\text{функция от }n}{\undergroup{f(n)}}$

        Линейное рекуррентное соотношение -- $\begin{cases}f = 0 \Longrightarrow \text{гомогенное (однородное)} \\ f \neq 0 \Longrightarrow \text{негомогенное (неоднородное)}\end{cases}$

        \Ex Последовательность Фибоначчи:

        $F(n) = \begin{cases}0, \quad n = 0 \\ 1, \quad n = 1 \\ F(n - 1) + F(n - 2)\end{cases}$

        $F(n) - F(n - 1) - F(n - 2) = 0$ -- однородное

        \vspace{5mm}

        \item Операторы:

        Сумма: $(f + g)(n) = f(n) + g(n)$

        Умножение на число: $(\alpha \cdot f)(n) = \alpha f(n)$

        Сдвиг: $(Ef)(n) = f(n + 1)$

        \Ex $E(f - 3(g - h)) = Ef + (-3)Eg + 3Eh$

        Составные операторы:

        $(E - 2) f = Ef + (-2)f = f(n + 1) - 2f(n)$

        $E^2 f = E(Ef) = f(n + 2)$

        \Ex $f(n) = 2^n$

        $2f = 2 \cdot 2^n$

        $Ef = 2^{n + 1}$

        $(E^2 - 1)f(n) = E^2 f(n) - f(n) = 2^{n + 2} - 2^n = 3 \cdot 2^n$

        \vspace{5mm}

        \item \textbf{Аннигилятор} (Annihilator) -- оператор, который трансформирует $f$ в функцию, тождественную $0$

        \Ex Оператор $(E - 2)$ аннигилирует функцию $f(n) = 2^n$

        \Exs $(E - c)$ аннигилирует $c^n$

        \Exs $(E - 3)(E - 2)$ аннигилирует $2^n + 3^n$

        \Exs $(E - c)^d$ аннигилирует любую функцию формы $p(n) \cdot C^n$, где $p(n)$ - многочлен степени не больше $d - 1$

        \Nota Любой составной оператор аннигилирует класс функций

        \Notas Любая функция, составленная из полинома и экспоненты, имеет свой единственный аннигилятор

        Если $X$ аннигилирует $f$, то $X$ также аннигилирует $Ef$

        Если $X$ аннигилирует $f$ и $Y$ аннигилирует $g$, то $XY$ аннигилирует $f \pm g$

        \vspace{5mm}

        \item Аннигилирование рекуррентностей:

        1. Запишите рекуррентное соотношение в форме операторов

        2. Выделите аннигилятор для соотношения

        3. Разложите на множители (если понадобится)

        4. Выделите общее решение из аннигилятора

        5. Найдите коэффициенты используя базовые случаи (если даны)

        \Ex $r(n) = 5r(n - 1), r(0) = 3$

        1. $r(n + 1) - 5r(n) = 0 \quad (E - 5)r(n) = 0$

        2. $(E - 5)$ аннигилирует $r(n)$

        3. $(E - 5)$ уже разложен

        4. $r(n) = \alpha \cdot 5^n$

        5. $r(0) = 3 \Longrightarrow \alpha = 3$

        \Ex $T(n) = 2T(n - 1) + 1, \quad T(0) = 0$

        1. $(E - 2)T(n) = 1$

        2. $(E - 2)$ не аннигилирует $T(n)$, остается $1$. Тогда добавим аннигилятор $(E - 1)$, получим, что $(E - 1)(E - 2)$ аннигилирует $T(n)$

        3. Разложение не требуется

        4. $T(n) = \alpha \cdot 2^n + \beta$ - общее решение

        5. $T(0) = 0 = \alpha \cdot 2^0 + \beta$

        $T(1) = 1 = \alpha \cdot 2^1 + \beta$

        $\alpha = 1, \beta = -1$

        \vspace{5mm}

        \item \textbf{Псевдонелинейные уравнения} (Pseudo-non-linear equations)

        \Ex $a_n = 3a_{n - 1}^2, a_0 = 1$

        $\log_2 a_n = \log_2 (3a_{n - 1}^2)$

        Пусть $b_n = \log_2 a_n$

        $b_n = 2b_{n - 1} + \log_2 3, b_0 = 0$

        $b_n = (2^n - 1)\log_2 3$

        $a_n = 2^{(2^n - 1)\log_2 3} = 3^{2^n - 1}$

    \end{itemize}

\end{document}
