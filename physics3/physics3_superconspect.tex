\documentclass[12pt]{article}
\usepackage{preamble}

\pagestyle{fancy}
\fancyhead[LO,LE]{Физические основы компьютерных \\ и сетевых технологий}
\fancyhead[RO,RE]{Лекции Зинчика А. А.}

\fancyfoot[L]{\scriptsize исходники найдутся тут: \\ \url{https://github.com/pelmesh619/itmo_conspects} \Cat}

\renewcommand{\thesection}{}

\begin{document}

    \tableofcontents
    \clearpage

    % begin physics3_2025_09_01.tex





\section{Лекция 1. Поляризация}

В прошлом семестре мы говорили о плоских бесконечных волнах. В реальности волны не бесконечные -- о них говорят, как о импульсе, одиночном, кратковременном возмущении

Свет излучается атомами за конечное время, порядка наносекунд. Получаем конечный световой импульс, длину распространения которого можно посчитать -- $l = c \cdot t$, а значит мы можем говорить о световом импульсе, который локализован, как о частице. Здесь появляется понятие кванта: атом не может излучить меньше одного фотона, поэтому фотон -- это квант, неделимая часть

Из прошлого семестра мы знаем, что электрон может преодолеть потенциальный барьер, действуя как волна, из-за своего размера. Следствием этого является ограничением на размер транзистора

Такой эффект не сходится с представлениями классической физики. В классической физике (в том числе в механике Ньютона) рассматриваются более высокие порядки размеров и на более низких скоростях, чем скорость света.
В механике Гамильтона, основывающейся на концепции гамильтониана (оператора полной энергии) отпадает понятие траектории

\mediumvspace

Будем говорить, что волна представляет $E(z, t) = \RE(E_0 e^{i(\omega t - kz)})$

Если волна не лежит в системе координат, то добавляют матрицу поворота: $E(z, t) = \RE\left(E_0 \begin{pmatrix}\cos \theta \\ \sin \theta \end{pmatrix} e^{i(\omega t - kz)}\right)$

\smallvspace

Свет считается \textbf{поляризованным}, если направления колебания светового вектора $\vec E$ упорядочены каким-либо образом

% TODO картинка

В простом случае поляризация бывает линейной (или плоской) -- в этом случае вектор напряженности движется в одной плоскости

Большинство бытовых источников света излучают неполяризованные волны -- в них колебания разных направлений быстро и беспорядочно сменяют друг друга. С помощью устройства с названием \textbf{поляризатор} можно получить поляризованный свет, поглощая другие. Поляризатор лишь частично задерживающий колебания, перпендикулярные к его плоскости, называется несовершенным. Качество поляризатора зависит от толщины и материала

С помощью другого прибора -- монохроматора -- можно получить монохроматическую волну. Так как свет с разной длиной волны имеет разные коэффициенты преломления, то монохроматор способен пропускать свет с нужной длиной волны

Если свет поляризован плохо, то его называют \textbf{частично поляризованным}

% TODO картинка

Если пропустить частично поляризованный свет через поляризатор, прибора вокруг направления луча интенсивность прошедшего света будет изменяться от $I_\min$ до $I_\max$. Причем, так как поляризатор симметричен, то угол между $I_\min$ и $I_\max$ равен $\frac{\pi}{2}$

Степенью поляризации $P = \frac{I_{\max} - I_\min}{I_{\max} + I_\min}$ можно выразить, насколько сильно поляризован свет 

\mediumvspace

Однако, так как поляризатор не пропускает лучи в неправильном направлении, то интенсивность света уменьшиться. \textbf{Закон Малюса} гласит, что доля интенсивность выходящего света от интенсивность входящего равна $\cos^2 \varphi$, где $\varphi$ -- угол между плоскостью поляризатора и плоскостью колебания $\vec E$

\[I = I_0 \cos^2 \varphi\]

Если пропустить естественный свет через поляризатор, то интенсивность выходящего света равна $I = \frac{1}{2} I_0$. Это объясняется тем, что в естественном свете волны направлены во все стороны равновероятно, а среднее значение $\cos^2 \varphi$ равна $\frac{1}{2}$

\mediumvspace

Существует круговая (или эллиптическая) поляризация, когда вектор $\vec E$ вращается в плоскости, перпендикулярной направлению распространения волны


% end physics3_2025_09_01.tex

% begin physics3_2025_09_08.tex





Всего существуют 3 способа поляризации:

\begin{enumerate}
    \item Поглощение (или дихроизм): свет проходит через вещество с длинными нитевидными молекулами. Проходя вдоль молекулы, свет свободно проходит, а поперек молекул свет не проходит

    Большинство таких линейных поляризаторов (или так называемых поляроидов) состоят из полимерной пленки или частиц кристаллов турмалина или герапатита в нитроцеллюлозной пленке

    \item Преломление: в призме Николя используется двойное лучепреломления света. В ней используется анизотропный кристалл исландского шпата, в котором

    \begin{itemize}
        \item лучи, поляризованные горизонтально, имеют показатель преломления $n_o = 1.66$ -- их называют обыкновенными
        \item лучи, поляризованные вертикально, имеют показатель преломления $n_o = 1.51$ -- их называют необыкновенными
    \end{itemize}

    Призма Николя представляет собой две одинаковые треугольные в сечении призмы. Обыкновенный луч испытывает полное внутреннее отражение от склеивающего слоя с $n = 1.55$ и поглощается, а необыкновенный свободно проходит через него и вторую призму, так как показатели преломления приблизительно равны

    % TODO картинка

    \item Отражение: Столетов предложил сделать поляризатор из стекла. При определенном угле падения $\alpha = \arctg n$ (известном как угол Брюстера) отраженный свет получается поляризованным. Для стекла этот угол равен примерно $59^{\circ}$, однако отраженный свет получается с интенсивностью 4\% от интенсивности входящего света.

    Столетов предложил использовать несколько стеклянных пластин, чтобы увеличить интенсивность -- данное устройство, состоящее из стопки стекла, получило название стопа Столетова

    Угол Брюстера применяется в изготовлении лазеров для получения поляризованных волн 

    % TODO картинка
\end{enumerate}

\section{Лекция 2. Дисперсия света}

Дисперсией света называется зависимость показателя преломления от частоты волны света

Данных эффект был обнаружен Исааком Ньютоном при разложении света в спектр. Тогда Ньютон обнаружил, что для разных частот света (а следовательно для разных волн) показатель преломления разный, поэтому в стекле лучи разных частот двигаются с разной скоростью, на выходе призмы получается радужный спектр 

Благодаря дисперсии существует радуга: лучи Солнца, проходя под определенным углом (42 градуса над горизонтом) через капельки воды в воздухе, раскладываются в спектр и попадают на сетчатку глаза

\begin{wrapfigure}{R}{0pt}
    \includegraphics[width=6cm]{physics3/images/physics3_germanium_refractive_index}
\end{wrapfigure}

На сайте \url{https://refractiveindex.info} можно узнать показатель преломления. Например, металл германий, использующийся в тепловизорах, имеет показатель преломления 3.5-4 в инфракрасном спектре волн, что улучшает разрешение тепловизора при ограниченном объёме устройства

Подобные призмы используются в спектрометрах - приборах, позволяющих разложить свет в спектр и узнать, какие длины волн пресутсвуют в спектре

Разные газы в газоразрядной лампе излучают свет разного цвета (то есть спектр из разных длин волн). Поэтому с помощью спектрометра можно обнаружить, из чего состоит источник света (например, Солнца): зная спектр горения водорода и гелия, можно предположить концентрацию горящего вещества на поверхности Солнца

Более продвинутый прибор -- масс-спектрометр -- используется для изучения состава вещества: вещество нагревают, излученный свет попадает на масс-спектрометр, который определяет интенсивность для разных волн света

% NO ANTI-MASS SPECTROMETER? 😿 
% @@@%%%%%%%%%%%%%%@%@@@@@@@@%%%%%%%%%#%##%%@@@@%%%%%%@@@@@@@%#***#####%%%@@%@@@@@@@@@@@@@%%@@@@@@%@@@
% %%%%%@%%%%%%%%%%%%%%@%@%@@@@%%%%%#%####%@@@@@@@@%%@%@@@@@@@@@@@@%%%%%%%%@@@@@@@@@@@@%%%@@@@%@@@@@@%%
% @%%%%%@%@@%%%%%%%%%%%%%@@@@@%@%%%%%%%##%@%@@@@@@%%@@@@@@@@@@@@@@@%%%%%%%@@@@@@@@@%%%@@@%%%%@@@%@@%@%
% @%%%%%%%%%%%@@%%%%%%%%%%%%%%@%@@%%%%%%@%@%%@@@@%#%@@@@@@@@@@@@@@%###%%@@@@@@@%%%%@@%%%%@@@@@%%%@@%%@
% %%%%%%@@%@%%%%%%%@@%%%%%%%%%%%%@@@@%%%@%@%@@@@@%%%@@@@@@@@@@@@@@@@@@@@%%%%%%%%@%@%@@@@@%%%%@%@%@%%%%
% %%%%%%%@@%%%%%%%%%%%%%@%%%%%%%%%%%%%%@%@@@@@@@@%%%@@@@@@@@@@@@@@@@@@%%%@%@%@@%%@@@%%@%%%%@@%%%%@%%%%
% %%%%%%%%%%@%@%%%%%%%%%%%%%@%%%%%%%%%%@%@%@@@@@@%%%@@@@@@@@@@@@@@@@@@@%%%%%@%@@%%%@%@@%%%%%%%%%%%%%%%
% %@%%%%%%%%%%%%%%%%%%%%%%%%%%%%%@%%%%%@@%@@@@@@@%%%@@@@@@@@@@@@@@@@@@@@@@%%%%@%%%%@%%%%%%%%%%%%%%%@@@
% %%%%%%%%@@@@%%%%%%%%%%%%%%%%%%%%%%%%%@%@%@@@@@@%#%@@@@@@@@@@@@@@@@@@%%@%%%%%%%%%%%%%%%%%%%%%%%@%%%%%
% %%%%%%%%%%%%%%%@@@%%%%%%%%%%%%%%%%%%%@@%@@@@@@@%%%@@@@@@@@@@@@@@@@@@%%%%%%%%%%%%%%%%@@@%%%%%%%%%%%%%
% %%%%%%%%%%%%%%%%%%%%%%%%%%%%%%%%%%%%%@%@%@@@@@@%%%@@@@@@@@@@@@@@@@@@%%%%%%%%%%@@@%%%%%%%%%%%%%%%%%%%
% %%%%%%%%%%%%%%%%%%%%%%%%%%%%%%%%%%%%%@%@%%%%@@@%%%@@@@%%%%%%%%%@@@@%%%@%%@%%%%%%%%%%%%%%%%%%%%%%%%%%
% %%%%%%%%%%%%%%%%%%%%%%%%%%%%%%%%%%%%%@%%%%%%%@@@%#%%@%%%%%%%%%@@@@@%%%%%%%%%%%%%%%%%%%%%%%%%%%%%%%%%
% %%%%%%%%%%%%%%%%%%%%%#%%%%%%%%%%%%%%%%%@%%%%%%%@@%%%%%#####%%@@@%%%%%%%%%%%%%%%%%%%%%%%%%%%%%%%%%%%%
% @@@@%%%%%%%%##%%%%%%%%%%%%%%%%%%%%%%%%%%%%%%%%%%@%%%@%####%@@@%%%%%%%%%%%%%%%%%%%%%%%%%%%%%%%%%%%%%%
% %%%%%%%%%%%%%%%%%%%%%%%%%%%%%%%%%%%%%%%%%%%%%%%%%%#*####%@@%%%%%%%%%%%%%%%%%%%%%##%%%%%%%%%%%%%%%%%%
% %%%%%%%%%%%%%%#####%#*#####%%%%%%%%%%%%%%**#%%%%%%%@@%%%%%####*##%###%%%%%%%%##########%%###%%#%%%%%
% ###%%%%%%%%%#########**#%####%%%%%%%%%#%%***%%%%%%%%%%%@%%####**###*#%#%%%%%%###*######%%###%##%%%%%
% ####%%%%%%%%#########*#*########%%%######===******#%%%*#***##*+*###*#%####%#####*######%%###%#######
% #####%%%%##################%%#####%%#####**+**##%%*==#%%%###***#######%%%%#%%%######################
% #%%%#%%%%####*#%%%##*####*#%%#*##%%%%#*##%%#=*#%%%%=-%%%#*++*%%%######%%%%#%%#*#**#%%#*#*##**#%%%###
% ##%%####%######%%%##***####%%####%%%#######*=*#%%%#=-#%%#***##%#######%%%%#%%%#%%%%%###%####*#%%%###
% #*#%%######%%##%%%##**################*#####*+*####-=*%%%**#%%%#######%%%##%####**#%%%%%%#%%#%%%%%%#
% #%%%%###%%#%%%%#####*****##***#*######***##%#*+*#%*==##%%**###%%%######%#######***#######****#%%####
% ##%%%#######%%######**##**##%%#**##%#**%#*****++*#*==#****+*####%%#*##%%%%%###******##******######%%
% #%%%%####%######*#####*****###***#*#***##*****=*##*==#+***++***#%##**####*#%#****###*#*****######%##
% #############%%%%%%#*******************##***##*#%#+:=*****++**#*%#****###**##*****#%##****#*****####
% #######%##**########********#*+++***************%%*::=*****#######**++++**###******#*******#%%#*####
% #*##%%##*#****####********+++******************+=**:-++***%%%%%%%*****#%%**++***************#%%####*
% #*##%#****#*##**###**+++****%%%***#%%##***++*##+=*=.-=-:-=+++**++**#%%%##%##**+==+**********#####*##
% #*######%###*##**++*****#%%%%%%#%#%####%%%*++*#*=++::=+*#*+*+++==#*#########%##*+=====********###***
% ****#**#*#*#++*******%%%#%%#**#*+#%%+++******#%##==:-:=#@*#@%*+****+#**#***###*#%#**++===+*##%#*##*#
% ***#*#%%@@%@%%%%%#********++++===*%%*===*****%%@%*=:+**#%%%*+==*#%#%#%%%%%%%%%#%%%%%*#***+++++*##***
% ****#%@%%%%%%%%%*+*#******++===++*%%%****##****%%*+:=%*#%***+*#%*#%%%%%%%%%%%%%#%######%%##****++***
% ***%@%%%%%%%%%%*=+#%%%%#***#*+++++%@%%%##%%%%%%@@@#:+****##%%%%***%%%%%*#******%%########%#%%#*****+
% ##%%%%%%%%%%#%*=+*#%%#*******++++*#@@@@%%%%%%%%@@@%*###**#%%%%%@*#%##*+*****====+++***********######
% %@%%%%%%%%%%%*=+*#%%**************#%%%%%%%%%%%%%%%%@%##**#%%%%%@@%%**++++***=====+++*************#*#
% %%%%%%%%%%%%*=+**%#***********+==**%%%%%%%##%@%%%%%%%%#%#%%%%%##%%#****+*+=+==+*%*###**************#
% %%%%%%%%%%%+=+*#%%*****++*******+**##*##*#%%%%%%%%%@@%%@%%%%%%#*=---=*##*+:===+*#*****%#%#%%%@%*****
% %%%%%%%#%%*=++*#%#*********************+++*##*=*####**%@%%%@@%******+==--=*%===#%##*#*%#%%####%%#***
% %%%%%%#%%*=++*#%#**********+*++++++++++++**#+-=*##**##%@@@@@%#*++++***++**+====*#####%%%#%%%%%%%%#**
% %%%%%%%%+=++*#%%***********+++++++++++*****=-=******##%%@@@%%%#*****#***+**+===*#%*#**#%#%%%%%%%%%#*
% %%%%#%%+=++*##%%*******+=====+++++++++**#*=-=*#**+++*##%%%%%%#******+++****+====+**%#%%%@%%#%%%%@%%#
% %%%@%%*=++**#%%%****++===========++=+*##*--=+***++++*##*##%%%******++++++**#*+====+++*%#%%%%%%%%#%%@


Дисперсия возникает как следствие уравнение Максвелла. Допустим для слабопроводящей среды $\sigma, \varepsilon, \mu = \const$ ($\sigma = \frac{1}{\rho}$ - удельная проводимость в сименсах)

По закону индукции Фарадея $\vec\nabla \times \vec E = -\mu\mu_0 \frac{\partial \vec H}{\partial t}$

$\nabla \times (\nabla \times \vec E) = -\mu\mu_0 \frac{\partial}{\partial t} (\nabla \times \vec H)$

$\nabla \times (\nabla \times \vec E) = \nabla (\nabla \vec E)$

$\nabla^2 \times \vec E = -\mu\mu_0 \frac{\partial}{\partial t} (\nabla \times \vec H)$

По теореме о циркуляции магнитного поля $\nabla \times \vec H = \sigma \vec E + \varepsilon \varepsilon_0 \frac{\partial \vec E}{\partial t}$

Получаем $\frac{\partial^2 \vec E}{\partial t^2} + \frac{\sigma}{\varepsilon \varepsilon_0} \frac{\partial \vec E}{\partial t} = v^2 \Delta \vec E$ -- волновое уравнение, где $v^2 = \frac{1}{\varepsilon \varepsilon_0 \mu \mu_0}$

Из этого волнового уравнения для волны, направленной в сторону оси $Ox$, получаем $\frac{\partial^2 E_y}{\partial t^2} = v^2 \Delta E_y - \frac{\sigma}{\varepsilon \varepsilon_0} \frac{\partial E_y}{\partial t}$

Решение его является функция $E_y = E_0 e^{i (\omega t - k x)}$, то есть $\omega^2 = v^2 k^2 - \frac{i \omega \sigma}{\varepsilon \varepsilon_0}$, где $k = \frac{2\pi}{\lambda}$ -- волновое число

Уравнение 

\[k^2 = \frac{omega}{v^2} - \frac{i \omega \sigma}{\varepsilon \varepsilon_0 v^2}\]

называют дисперсионным (то есть зависимость $k(\omega)$). Из него $k = \pm \frac{\omega}{v} \sqrt{1 - \frac{i \sigma}{\varepsilon \varepsilon_0 \omega}}$

Для $\frac{\omega}{\varepsilon\varepsilon_0 \omega} \ll 1$ можем аппроксимировать корень, получаем $k \approx \frac{\omega}{v} (1 - i \frac{\sigma}{2\varepsilon\varepsilon_0 \omega}) = k^\prime - i k^{\prime\prime}$

В ходе вычисления получаем комплексное $k$: вещественная часть волнового числа $k^\prime$ определяет длину волны, мнимая часть $k^{\prime\prime} = $ показывается коэффициент затухания волн, то есть поглощение, получаем $E_y = E_0 e^{i (\omega t - k^{\prime} x) - k^{\prime\prime} x}$

Зависимость фазовой скорость волны (то скорость волны с одной длиной) от частоты в среде $v_\text{фаз}(\omega) = \frac{\omega}{k^{\prime}(\omega)}$ называют дисперсией

Для световых волн дисперсия -- $n(\omega) = \frac{c}{v_{\text{фаз}}(\omega)}$ или $n(\lambda_0) = \frac{c}{v_{\text{фаз}}(\lambda_0)}$

Если $\sigma = 0$, то $v_{\text{фаз}} = \frac{1}{\sqrt{\varepsilon \varepsilon_0 \mu \mu_0}}$

Получаем дисперсию световых волн: $n(\omega) = \frac{c}{v_{фаз}(\omega)}$ или $n(\lambda_0) = \frac{c}{v_{фаз}(\lambda_0)}$

Из этого выходит закон Бугера: пусть свет интенсивности $I_0$ падает на вещество толщины $L$, тогда интенсивность света уменьшается по экспоненциальному закону: $I = I_0 e^{- k L}$

При сложении волн из квазимонохроматического спектра получаем ограниченную в пространстве волну -- так называемый волновой пакет. Длительность волнового пакета $\tau$ пропорциональна обратной разности частот $\frac{1}{\Delta v}$

В среде волны с разными длинами двигаются с разной скорость, поэтому пакет будет деформироваться из-за дисперсии. Из-за этого пакет получает приращение $\Delta t = \frac{L}{v_{мин}} - \frac{L}{v_{макс}} = \frac{L}{c} \Delta n$

При увеличении пропускной способности оптоволокна нужно уменьшить длительности импульса $\tau$. Из этого получаем, что разность частот увеличивается

Если импульс занимает весь видимый диапазон, то $\Delta n \approx 0.03$. При прохождении 1 метра волокна получаем $\Delta t = \frac{1}{3 \cdot 10^8} 0.03 = 10^{-10}$ с. Если длительность пакета меньше $\Delta t$, то импульсы сливаются во время прохождения и на приемнике их становится невозможно различить 

Групповая скорость $v_{\text{гр}} = \frac{d\omega}{dk}$ - это скорость движения волнового пакета. Если среда дисперсионная, то $v_{\text{гр}} \neq v_{\text{фаз}}$

Заметим, что $v_{\text{фаз}} = \frac{\omega}{k}$, тогда $v_{\text{гр}} = \frac{d\omega}{dk} = v_{\text{фаз}} + k \frac{d v_{\text{фаз}}}{dk} = v_{\text{фаз}} + k \frac{d v_{\text{фаз}}}{d\lambda} \frac{d\lambda}{dk}$

Так как $\frac{dk}{d\lambda} = -\frac{2\pi}{\lambda^2}$, то $v_{\text{гр}} = v_{\text{фаз}} - \lambda \frac{d v_{\text{фаз}}}{d\lambda}$

То есть $v_{\text{фаз}} - v_{\text{гр}} = \lambda \frac{dv_{\text{фаз}}}{d\lambda}$

Если дисперсии нет, то $k_1 - k_2 = \frac{\omega_1}{c} - \frac{\omega_2}{c}$, и тогда $v_{\text{гр}} = c$


% end physics3_2025_09_08.tex



\end{document}

