$subject$=Физические основы компьютерных \\ и сетевых технологий
$teacher$=Лекции Зинчика А. А.
$date$=08.12.2025

\mediumvspace

% $p = p_0 e^{- \frac{mgh}{kT}}$ -- давление на произвольной высоте

% Здесь вместо $m$ берут среднюю массу молекул в смеси воздуха, то есть $29$ грамм на моль

% Эта формула используется в барометрических высотомерах

При передачи достаточного числа энергии молекулам, они разбиваются на атомы, а атомы на ядра и электроны

Введем величину плотности квантовых состояний $g(E) = \frac{d \nu(E)}{dE}$, где $d\nu$ -- число квантовых состояний в диапазоне от $E$ до $dE$

Для твердых тел экспериментально получено, что $g(E) = 2 \frac{V}{4\pi^2} \left(\frac{2m}{\hbar^2}\right)^{\frac{3}{2}} \sqrt{E}$

Отсюда число электронов определяется как произведение $dN(E) = f(E) \cdot g(E) \cdot dE = \frac{1}{e^{\frac{E - \mu}{kT}} + 1} \cdot \frac{V}{2\pi^2} \left(\frac{2m}{\hbar^2}\right)^{\frac{3}{2}} \sqrt{E} \cdot dE$

Это означает, что число электронов возрастает пропорционально корню, то есть $\frac{dN}{dE} \sim \sqrt{E}$ при $E < \mu$ и $T \to 0$


Статистике Бозе-Эйнштейна подчиняются частицы с целым спином -- бозоны. Помимо фотона, примером бозона является фонон. Фонон -- это квант колебаний кристаллической решетки, которым описывают распространение упругих волн в кристалле

Статистика Бозе-Эйнштейна равна $f(E_i) = \frac{1}{e^{\frac{E_i - \mu}{kT}} - 1}$, где $f(E_i)$ -- среднее число бозонов, имеющих состояние $E_i$. Для бозонов химический потенциал $\mu \leq 0$, а для фотонов и фононов конкретно $\mu = 0$, потому что фотоны и фононы могут свободно рождаться и поглощаться

Эффект сверхпроводимости основан на том, что при низких температура электроны образуются в так называемые куперовские пары, которые имеют целый спин и ведут себя как бозоны

Газ называется вырожденным, если его свойства отличаются от свойств классического газа -- в этом случае вместо классической статистики Больцмана нужно использовать квантовую статистику

Если $e^{\frac{E_i - \mu}{kT}} \gg 1$, то обе статистики можно записать как $f(E_i) = \frac{1}{e^{\frac{E_i - \mu}{kT}} \pm 1} \approx \frac{1}e^{\frac{E_i - \mu}{kT}} = e^{\frac{\mu}{kT}} e^{-\frac{E_i}{kT}} = A e^{-\frac{E_i}{kT}}$ -- классическая статистика Больцмана

Газ становится вырожденный при температуре ниже температуры вырождения $T_\text{В} \approx \frac{\hbar^2 n^{\frac{2}{3}}}{3mk}$

Температура вырождения зависит от концентрации газа. Если рассмотреть электроны в полупроводниках и металле как газ, то в полупроводнике концентрация свободных электронов равна $n = 10^{18} \text{м}^{-3}$, температура вырождения приблизительно равна $T_{\text{В}} \approx 10^{-4} \text{К}$, поэтому электронный газ считается классическим

В металлах $n = 10^29 \text{м}^{-3}$, $T_{\text{В}} \approx 10^4 \text{К}$, то есть в комнатных условиях газ вырожденный

Рассмотрим газ из фотонов. Так как массы у фотона нет, то температура вырождения стремится к бесконечности, то есть фотонный газ всегда вырожденный и поддается статистике Бозе-Эйнштейна

Также, так как $\mu = 0$ для фотонов, фотонный газ имеет плотность $f = \frac{1}{e^{\frac{h\nu}{kT}} - 1}$ -- формула, очень похожая на формулу Планка для абсолютно черного тела

\mediumvspace

Позднее статистику Ферми-Дирака применили для обоснования электропроводимости металлов

По классической теории (модели Друде-Лоренца) рассматривалось воздействие силы ионов кристаллический решетки $\vec F = e \vec E$ на свободный электрон. Во время пробега электрон движется равноускоренно, приобретая к концу пробега максимальную скорость скорость $\langle u_{\max}\rangle = a t = \frac{e E}{m} t$, где $t$ -- время между двумя соударениями электрона с ионами

Так как после столкновения направление скорости электрона считается случайным, то усреднённая по времени \emph{дрейфовая} скорость электрона равна половине максимальной: $\langle u \rangle = \frac{1}{2} u_{\max} = \frac{eE}{2m} t$

Среднее время между столкновениями можно выразить через среднюю длину пробега $\lambda$ и средней тепловой скорости электронов $v$, получаем $t = \frac{\lambda}{v}$. Отсюда $\langle u \rangle = \frac{1}{2} \langle u_{\max}\rangle = \frac{eE}{2m} \frac{\lambda}{v}$ -- средняя скорость тока

Тогда плотность тока равна $j = n e \langle u \rangle = \frac{n e^2 E}{2m} \frac{\lambda}{v}$, где $n$ -- концентрация электронов

Используя закон Ома в дифференциальной форме ($\vec j = \sigma \vec E$) находим электропроводность $\sigma = \frac{1}{\rho} = \frac{n e^2}{2m} \frac{\lambda}{v}$

В классической статистике средняя тепловая скорость электронов определяется как $\frac{1}{2} m v^2 \sim k_B T$ ($k_B$ -- постоянная Больцмана), то есть $\rho \sim v \sim \sqrt{T}$ -- удельное сопротивление зависит от корня температуры

Однако экспериментально для большинства металлов при не слишком низких температурах была получена такая зависимость удельного сопротивления: $\rho = \rho_0 (1 + \alpha t^\circ) = p_0 \alpha T$, где $\alpha$ -- коэффициент, зависящий от вещества, $t^\circ$ -- температура в градусах Цельсия, $T$ -- температура в Кельвинах

На самом деле электрон в металле движется не в пустоте, а в периодическом поле ионов решётки. Полная сила, действующая на электрон, определяется как $\vec F = m \vec a = -e(\vec E + \vec E_{\text{реш}})$, где $\vec E_{\text{реш}}$ -- внутреннее электрическое поле решётки, $E$ -- внешнее электрическое поле

Пусть $m^* \vec a = (-e) \vec E$, тогда $m^* \vec a = m\vec a (-e) \vec E_\text{реш}$. Здесь $m^*$ называется эффективной массой. В классической теории рассматривалось только действие внешнего поля

Движение электрона в металле описывается волной де Бройля. Такое движение в кристалле отлично описывается волной и дифракционной решеткой. Статистическое распределение электронов по энергиям подчиняется статистике Ферми–Дирака. При температурах, характерных для обычных экспериментов, $k_B T \ll E_F$, где $E_F$ -- энергия Ферми. Поэтому почти все электроны находятся глубоко под уровнем Ферми и не участвуют в переносе тока. В проводимости участвуют только электроны в узком энергетическом слое порядка $k_B T$ около уровня Ферми

В классической теории скорость электронов зависит от температуры, а в квантовой -- от длины свободного пробега $\lambda_F$. С увеличением температуры возрастает рассеяние электронных волн на 
фононах, уменьшается средняя длина свободного пробега электронов, то есть $\sigma \sim \langle \lambda \rangle \sim \frac{1}{T}$

Получаем, что $\rho \sim \frac{1}{\sigma} \sim T$, а из этого $\rho = \rho_0 \alpha T$

При уменьшении температуры сопротивление металлов уменьшается, так как уменьшается рассеяние электронных волн на тепловых колебаниях ионов, и при $T \to 0$ стремится к некоторому значению -- остаточному сопротивлению, связанному с примесями и дефектами кристаллической решётки

\section{10. Устройство полупроводников}

Рассмотрим модель твердого тела

Атомы считаем неподвижными (иначе это будет жидкость, а не твердое тело), также можно упростить модель тем, что рассматривать уравнение Шрёдингера для одного электрона в суммарном электрическом поле

Структуру кристалла тела можно описать как сумму потенциалов от полей ионов. Тогда уравнение Шрёдингера записывается так:

\[
\hat H \Psi = -\frac{\hbar^2}{2m} \Delta \Psi + U(\vec r) \Psi = E \Psi,
\]

где $U(\vec r)$ -- периодический потенциал, создаваемый ионами кристалла

Для решетки натрия расстояние между ядрами равно $0.43$ нм. По статистике Ферми-Дирака при пересечении электронных облаков действует принцип запрета Паули -- в объединенном облако не может быть электронов с одинаковыми квантовыми числами

Энергетической зоной называют непрерывный интервал разрешённых энергий электронов в кристалле. Между такими зонами могут существовать интервалы энергий, в которых стационарные состояния отсутствуют, -- эти интервалы называются запрещёнными зонами

Среди энергетических зон выделяют \textbf{валентную зону} и \textbf{зону проводимости}. Электроны с энергией, находящийся в валентной зоне, привязаны к атомам вещества, а электроны в зоне проводимости свободны от связи с атомами и могут переносить ток

У металлов зона проводимости пересекается с валентной зоной, поэтому, чтобы получить свободные электроны в металле, нужно сообщить малое количество энергии, из-за этого металлы хорошо проводят ток

В полупроводниках при температуре $T=0$ валентная зона полностью заполнена электронами, а зона проводимости пуста. Между ними существует запрещённая зона конечной ширины. Чтобы электрон перешёл в зону проводимости и стал свободным носителем заряда, ему необходимо сообщить энергию не меньшую ширины запрещённой зоны

\mediumvspace

Периодичность кристаллической решётки приводит к тому, что потенциал удовлетворяет условию $U(\vec r) = U(\vec r + \vec R)$, поэтому работает теорема Блоха: для стационарного уравнения Шрёдингера $\hat H \Psi = -\frac{\hbar^2}{2m} \Delta \psi + U(r) \Psi$ справедливо равенство $\Psi_{k}(r) = u_{k}(r) e^{-i k \cdot \vec r}$, где функция $u_{k}(r)$ обладает периодичностью решётки, а $k$ -- волновой вектор электрона

Из этой теоремы возникла модель Кронига-Пенни. В ней для моделирования решетки вместо суммы гипербол используются прямоугольные периодические барьеры. Несмотря на упрощённость, эта модель наглядно демонстрирует возникновение разрешённых и запрещённых энергетических зон.

Функция потенциала будет иметь вид $V = \begin{cases}0, & 0 \leq x \leq b \\ V_0, & -c \leq x \leq 0\end{cases}$, где $b$ -- ширина ямы, а $c$ -- ширина барьера. Обозначим $\alpha^2 = \frac{2 m E}{\hbar^2}$ и $\delta^2 = \frac{2 m (V_0 - E)}{\hbar^2}$

Будем искать решения в виде $\Psi(x) = u(x) e^{ikx}$, где функция $\psi(x)$ имеет период решётки

Тогда получаем два уравнения: 

\[\begin{cases}\frac{d^2 u_1}{dx^2} + 2ik \frac{d u_1}{dx} + (\alpha^2 - k^2) u_1 = 0, & 0 \leq x \leq b \\ \frac{d^2 u_2}{dx^2} + 2ik \frac{d u_2}{dx} + (\delta^2 + k^2) u_2 = 0, & -c \leq x \leq 0\end{cases}\]

И два решения:

\[\begin{cases}u_1(x) = A e^{i \alpha x} + B e^{- i \alpha x}, & 0 \leq x \leq b \\ u_2(x) = C e^{\delta x} + D e^{- \delta x}, & -c \leq x \leq 0\end{cases}\]

Итоговая кусочная функция должна быть гладкой, то есть $\psi_1(0) = \psi_2(0)$ и $\psi^\prime_1(0) = \psi^\prime_2(0)$, получаем:

\[\begin{cases}
    A + B - C - D = 0 \\
    i \alpha A - i \alpha B - \delta C + \delta D \\
    A e^{i \alpha b} + B e^{- i \alpha b} - C e^{ika - \delta c} - D e^{\delta c + i k a} = 0 \\
    i \alpha A e^{i \alpha b} - i \alpha B e^{- i \alpha b} - \delta C e^{ika - \delta c} + \delta D e^{\delta c + i k a} = 0 \\
\end{cases}\]

Решаем СЛАУ, получаем $\cos(k a) = \frac{\delta^2 - \alpha^2}{2 \alpha \delta} \mathrm{sh} (\delta c) \sin(\alpha b) + \mathrm{\delta c} \cos (\alpha b)$

При $c \to 0$, $V_0 \to \infty$, а $c V_0 = \const$ получаем $\cos kb = \frac{P}{b\alpha} \sin b\alpha + \cos b \alpha$, где $P = \frac{m c V_0 b}{\alpha \hbar^2}$, $\alpha = \sqrt{\frac{2 m E}{\hbar^2}}$

Это уравнение связывает энергию $E$ с волновым числом $k$ и имеет решение не всегда, поэтому мы можем найти области, где решения есть 

% https://www.geogebra.org/calculator/xehwrzgj

\begin{center}
    \includegraphics[width=12cm]{physics3/images/physics3_kronig_penney_solution}
\end{center}

Если функция $f(b \alpha) = \frac{P}{b\alpha} \sin b\alpha + \cos b \alpha$ ($\alpha = \sqrt{\frac{2 m E}{\hbar^2}}$) по модулю меньше 1, то $k = \frac{p}{\hbar}$ представляет из себя вещественное число. Фиолетовые зоны - это интервалы, где $k \in \Real$, то есть электрон может иметь такую энергию под этим волновым числом. Для мнимого $k$ волновая функция $\Psi$ экспоненциально затухает, поэтому электрона с такой энергией не может быть

Обычно этот график представляют в виде как $E(k)$:

\begin{multicols}{2}
\begin{center}
    \includegraphics[width=8cm]{physics3/images/physics3_E_k_1}

    \includegraphics[width=8cm]{physics3/images/physics3_E_k_2}
\end{center}
\end{multicols}

Если ширина барьера стремится к 0 (или ширина ямы к периоду решетки) при фиксированном $V_0$, то мы получаем зависимость $E \sim k^2$, что подтверждает формулу $E = \frac{\hbar^2 k^2}{2m}$ для свободного электрона, поэтому справа на картинке вырождается парабола

\mediumvspace

Аналогичные идеи используются и в оптике. В фотонных кристаллах периодически изменяется показатель преломления, и для фотонов также возникают разрешённые и запрещённые зоны. Запрещённая зона для фотонного кристалла означает, что электромагнитные волны с длинами волн из этого диапазона не могут распространяться в кристалле.
