$subject$=Физические основы компьютерных \\ и сетевых технологий
$teacher$=Лекции Зинчика А. А.
$date$=29.09.2025

Разберем примеры длин волн де Бройля:

\begin{itemize}
    \item У частицы массой 1 г и скорость 1 метр в секунду получим $\lambda_{\text{Б}} = \frac{h}{p} = \frac{h}{1} = 6.63 \cdot 10^{-34}$ м

    \item Молекула кислорода со скоростью 500 м/с имеет длину волны $\lambda_{\text{Б}} = \frac{6.63 \cdot 10^{-34}}{5.32 \cdot 10^{-26} \cdot 500} = 0.025 \text{нм}$
\end{itemize}

Опыт Дэвиссона и Джермера подтверждал гипотезу де Бройля:

Узкий пучок моноэнергетических электронов направлялся на поверхность монокристалла никеля и наблюдалось отражение электронов от его поверхности. Атом в кристалле образуют упорядоченную периодическую структуру, поэтому интенсивность отраженного пучка на экране показывала распределение с резкими максимумами, как при дифракции

По закону Вульфа-Брэгга $2d \sin \varphi = n \lambda_{\text{Б}}$ можно узнать длину волны де Бройля и сравнить ее с формулой $\lambda_{\text{Б}} = \frac{h}{p}$

% TODO картинка

Аналогично опыт Томсона и Тартаковского с прохождение электронов через металлическую фольгу показала это

Также можно повторить опыт Юнга с электроном и получить дифракционную картину

% TODO картинка

Если взять одну щель, то распределение электронов на экране будет соответствовать функции $\frac{\sin x}{x}$ -- преобразование Фурье от $t(x) = \begin{cases}0, & |x| < \Delta x / 2 \\ 1, & |x| \geq \Delta x / 2 \end{cases}$, функции пропускания, где $\Delta x$ -- ширина щели

Здесь можно определить, что первый минимум находится на угле $\varphi_1$, для которого выполнено $\Delta x \sin \varphi_1 = \lambda_{\text{Б}}$. Если пучок отклонился, значит появилась проекция импульса на плоскость экрана. При малом угле $\sin \varphi_1 \approx \tg \varphi_1 = \frac{p_x}{p_0}$

Так как $\Delta p_x \approx p_x$, $\tg \varphi_1 = \frac{\Delta p_x}{p_0} \Longrightarrow \Delta x \cdot \Delta p_x = \lambda_{\text{Б}} p_0$

Получим $\Delta x \cdot \Delta p_x \approx h$ -- соотношение неопределенностей Гейзенберга. Оно означает, что, чем точнее измеряется одна характеристика частицы (либо расстояние, либо импульс), тем менее точно можно измерить вторую

По-другому можно соотношение представить как $\Delta t \Delta \nu \approx 1$ -- оптическое соотношение неопределенности

Или как $\Delta x \cdot \Delta k \approx 1$ -- пространственное соотношение, где $\Delta k$ -- неопределенность измерения волнового числа

В макромире соотношение Гейзенберга можно смело игнорировать

С помощью принципа неопределенности можно объяснить природу электронов внутри атома. Чтобы не упасть на ядро под действием силы Кулона, электрон должен иметь скорость. Масса электрона равна $9.1 \cdot 10^{-31}$ кг, поэтому из-за принципа неопределенности нельзя точно узнать положение электрона внутри атома, отсюда появляется понятие электронного облака (или орбитали) -- область внутри атома, внутри которой с какой-то вероятностью находится электрон

\section{5. Строение атома}

В конце XIX века был открыт электрон. Электрон имеет отрицательный заряд, но, так как атом по заряду нейтрален, ядро должно быть положительным. В 1903 году появилась модель Томпсона, которая предполагала, что электроны находились в положительно заряженном атоме, словно изюм в кексе 

По расчета Томпсона размер атома равен приблизительно $10^{-10}$ м

Далее Резерфорд провел такой опыт: альфа-частицы разгонялись на тонкую золотую фольгу, затем отклонялись на экран. Альфа-частицы были обнаружены позади фольги, немного в бок и напротив фольги. Модель Томпсона утверждала, что напряженность атома была равномерно распространена, поэтому ее бы не хватило, чтобы отклонить альфа-частицу на меньший угол

Значит, модель Томпсона оказалась неверной. Потом появилась модель Резерфорда -- в ней в центра атома есть ядро, в котором был заключен весь положительный заряд, а вокруг ядра вращались электроны. Возникает несостыковка: заряженные электроны, вращаясь, создают переменное магнитное поле, перенося энергию, значит, скорость электронов должна уменьшаться, а атом прекращать существование

Тогда Нильс Бор выдвинул гипотезу, что на некоторых орбитах электроны не излучают энергию. Удивительно, что момент импульса $L = m v r$ имеет такую же размерность, что и постоянная планка $h$

Бор предположил, что если момент импульса $L = n \hbar$, где $n \in \Natural$, то орбита считается стабильной (или стационарной)

