$subject$=Физические основы компьютерных \\ и сетевых технологий
$teacher$=Лекции Зинчика А. А.
$date$=24.11.2025

\begin{itemize}
    \item Гейт Z меняет знак у состояния $|1\rangle$, его матрица равна $\begin{pmatrix}1 & 0 \\ 0 & -1\end{pmatrix}$

    Гейт Z изображается так:

    \begin{center}
    \begin{quantikz}
        \lstick{$|0\rangle$} & \gate{Z} & & \rstick{$\longrightarrow |0\rangle$} \\
        \lstick{$a|0\rangle + b|1\rangle$} & \gate{Z} & & \rstick{$\longrightarrow a|0\rangle - b|1\rangle$}
    \end{quantikz}
    \end{center}
    
    Гейт $Z$ не меняет вероятности измерения, а изменяет только относительную фазу компонент состояния

    Из этого последовательное применение гейтов H, Z и H эквивалентно X, а H, X и H -- Z

    \begin{center}
    \begin{quantikz}
        \lstick{$a|0\rangle + b|1\rangle$} & \gate{H} & \gate{Z} & \gate{H} & \rstick{$\longrightarrow a|0\rangle + b|1\rangle$}
    \end{quantikz}
    \end{center}

    \item Гейт Y вращает состояние вокруг оси $OY$ и имеет матрицу $\begin{pmatrix}0 & -i \\ i & 0\end{pmatrix}$

    \begin{center}
    \begin{quantikz}
        \lstick{$a|0\rangle + b|1\rangle$} & \gate{Y} & & \rstick{$\longrightarrow -i b|0\rangle + i a|1\rangle$}
    \end{quantikz}
    \end{center}

    Из этого применения гейтов H, Y и H эквивалентно -Y

    \item Гейт CNOT (от Controlled NOT) применяется на два кубита: один из них управляющий, другой -- управляемый. Если управляющий кубит равен единице, то управляемый бит инвертируется, как после применения гейта X:

    \begin{center}
    \begin{quantikz}
        \lstick{$a|0\rangle + b|1\rangle$} & \ctrl{1} & & \rstick[2]{$ac |00\rangle + ad |01\rangle + bd |10\rangle + bc |11\rangle$} \\
        \lstick{$c|0\rangle + d|1\rangle$} & \gate{X} & & 
    \end{quantikz}
    \end{center}

    Гейт CNOT имеет матрицу $\begin{pmatrix}1 & 0 & 0 & 0 \\ 0 & 1 & 0 & 0 \\ 0 & 0 & 0 & 1 \\ 0 & 0 & 1 & 0\end{pmatrix}$

    Если управляющий бит имеет состояние $\frac{1}{\sqrt{2}} |0\rangle + \frac{1}{\sqrt{2}} |1\rangle$, то управляемый кубит инвертирует значение с вероятностью 50\%

    С помощью этого гейта кубиты можно запутать -- измерение одного из них переводит другой в определенное состояние:

    \begin{center}
    \begin{quantikz}
        \lstick{$a|0\rangle + b|1\rangle$} & \ctrl{1} & & \rstick[2]{$a |00\rangle + b |11\rangle$} \\
        \lstick{$|0\rangle$} & \gate{X} & & 
    \end{quantikz}
    \end{center}

    Аналогично запутывать можно и большее число кубитов
\end{itemize}

Из-за свойства гейта CNOT можно увести кубиты на огромное расстояние, и, измерив один кубит, мы можем узнать значение другого

Из этого можно сделать миллион пар таких кубитов, которые будут передавать информацию. Передачу квантового состояния на расстояние запутанной пары кубитов называют квантовая телепортацией

Допустим Алиса хочет передать кубит с состоянием $|psi\rangle = \alpha |0\rangle + \beta |1\rangle$ Бобу. У Боба есть приемник -- третий кубит, а второй кубит является вспомогательным у Алисы и связан с третьим

Боб с третьим кубитом уезжает на другую планету. 

Получаем трехкубитовую систему: $\frac{1}{\sqrt{2}} (\alpha|000\rangle + \alpha|011\rangle + \beta|011\rangle + \beta |111\rangle)$

После применения еще нескольких гейтов, мы получаем такое состояние: 

\[\frac{1}{2} \left(|00\rangle (\alpha |0\rangle + \beta |1\rangle) + |01\rangle (\alpha |1\rangle + \beta |0\rangle) + |10\rangle (\alpha |0\rangle - \beta |1\rangle) + |11\rangle (\alpha |1\rangle - \beta |0\rangle)\right)\]

Как видно, только в одном случае из четырех третий кубит имеет такое же состояние, что и первый. В остальных случаях Бобу нужно применить соответствующее преобразование (гейтами Z или X), чтобы получить значение с точностью до знака

Если Алиса измерила свои кубиты и получила $00$, то информация доставлена успешна

\begin{center}
\begin{quantikz}
    \lstick{$|\psi\rangle$} & & & \ctrl{1} & \gate{H} & & \ctrl{2} & & \rstick[2]{$\text{Кубиты Алисы}$}\\
    \lstick{$|0\rangle$} & \gate{H} & \ctrl{1} & \gate{X} & & \ctrl{1} & & & \\
    \lstick{$|0\rangle$} & & \gate{X} & & & \gate{X} & \gate{Z} & \meter{} & \rstick{$\text{Кубит Боба}$}
\end{quantikz}
\end{center}

Проблема в том, что третий гейт CNOT должен быть реализован на этом огромном расстоянии. Запутанность не позволяет передавать информацию мгновенно. Хотя измерение одного кубита мгновенно определяет состояние другого, результат измерения является случайным и не может быть использован для передачи сигнала

Помимо этого, неизвестное квантовое состояние невозможно скопировать на другой кубит. Это утверждение известно как теорема о запрете клонирования и следует из линейности квантовой механики

