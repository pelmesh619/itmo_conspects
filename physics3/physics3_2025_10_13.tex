$subject$=Физические основы компьютерных \\ и сетевых технологий
$teacher$=Лекции Зинчика А. А.
$date$=13.10.2025

\section{6. Оператор Гамильтона, уравнение Шрёдингера}

Потребность в квантовой механике возникает тогда, когда характерный размер исследуемого объекта становится сравнимым или меньше длины волны де Бройля: $L < \lambda_{\text{Б}} = \frac{h}{p}$

Это условие означает, что волновые свойства материи начинают играть существенную роль, и классическое описание (через координаты и силы) перестает быть точным

\mediumvspace

В квантовой механике состояние системы описывается \textit{волновой функцией} $\psi(\vec r, t)$, которая содержит полную информацию о системе.  
Физический смысл имеет не сама $\psi$, а ее модуль в квадрате:
\[
|\psi(\vec r, t)|^2 = \psi^*(\vec r, t)\psi(\vec r, t),
\]
который задает \textbf{плотность вероятности} обнаружения частицы в точке $\vec r$ в момент времени $t$

\mediumvspace

В отличие от классической механики Ньютона, где движение описывается через силы и ускорения, квантовая механика оперирует \textit{операторами} физических величин, действующих на волновые функции.  

Для описания различных физических величин вводят соответствующие операторы:

\begin{itemize}
    \item оператор координаты $\hat{x}$
    \item оператор импульса $\hat{p}$
    \item оператор кинетической энергии
    \item оператор потенциальной энергии
    \item оператор полной энергии (гамильтониан) $\hat{H}$
\end{itemize}

Все эти операторы линейные. Это значит, что для любых констант $c_1, c_2$ и функций $\psi_1, \psi_2$ выполняется: $\hat{L}(c_1 \psi_1 + c_2 \psi_2) = c_1 \hat{L}\psi_1 + c_2 \hat{L}\psi_2$


\textbf{Оператор координаты} действует как простое умножение на саму координату: $\hat{x}\psi(x) = x\psi(x)$

\textbf{Оператор потенциальной энергии} также представляет собой умножение на соответствующую функцию потенциала: $\hat{U}\psi(x) = U(x, t)\psi(x)$

\textbf{Оператор импульса}: из соотношений де Бройля $p = \hbar k$ и волнового выражения $\psi(x) \sim e^{ikx}$ следует, что $\frac{d\psi}{dx} = i k \psi = \frac{i p}{\hbar}\psi$

Отсюда, действуя на $\psi$, можно записать оператор импульса как: $\hat{p} = - i \hbar \frac{\partial}{\partial x}$ или в трехмерном случае $\hat{\vec{p}} = - i \hbar \vec{\nabla}$

\textbf{Оператор кинетической энергии} выражается через оператор импульса: $\hat{T} = \frac{\hat{p}^2}{2m} = -\frac{\hbar^2}{2m}\nabla^2$

\mediumvspace

Теперь можно записать уравнение Шрёдингера. В общем (учитывающем время, то есть временном) виде оно имеет вид:

\[
i\hbar \frac{\partial \psi}{\partial t} = -\frac{\hbar^2}{2m}\nabla^2 \psi + U(\vec r, t)\psi
\]

Здесь первый член $-\frac{\hbar^2}{2m}\nabla^2 \psi$ в скобках отвечает за кинетическую энергию, а второй $U(\vec r, t) \psi$ -- за потенциальную. Суммарный оператор называется \textbf{гамильтонианом}: $\hat{H} = -\frac{\hbar^2}{2m}\nabla^2 + U(\vec r, t)$

И уравнение принимает компактный вид:
\[
i\hbar \frac{\partial \psi}{\partial t} = \hat{H}\psi.
\]

\mediumvspace

Решением этого уравнения является волновая функция $\psi$, описывающая эволюцию состояния системы во времени

Если состояние можно описать одной функцией $\psi$, то оно называется чистым состоянием

\mediumvspace

Далее были сформулированы постулаты квантовой механики:

\begin{enumerate}
    \item \textbf{1-ый постулат}: 
    состояние квантовой системы полностью определяется ее волновой функцией $\psi(\vec r, t)$.  
    Квадрат модуля волновой функции $|\psi|^2$ задает плотность вероятности нахождения системы в данном состоянии
    
    \item \textbf{2-ой постулат}:
    каждой физической величине соответствует линейный оператор $\hat{A}$, действующий в пространстве волновых функций
    
    \item \textbf{3-ий постулат}: 
    при измерении физической величины можно получить только одно из собственных значений оператора, соответствующего этой величине
    
    \item \textbf{4-ый постулат}:
    квадрат модуля волновой функции $\psi(\vec r, t)$
    определяет плотность $W$ вероятности того, что в момент времени $t \geq 0$ частица может быть обнаружена в точке пространства $\vec r$ 
\end{enumerate}

\mediumvspace

Если потенциальная энергия не зависит от времени, то волновую функцию можно искать в виде разделения переменных: $\psi(\vec r, t) = \phi(\vec r)e^{-\frac{iEt}{\hbar}}$

Подставив это выражение в уравнение Шрёдингера, получаем стационарное уравнение Шрёдингера:

\[
-\frac{\hbar^2}{2m}\nabla^2 \phi + U(\vec r)\phi = E\phi.
\]

Здесь $E$ -- собственное значение гамильтониана, соответствующее энергии данного стационарного состояния

\mediumvspace

\Ex Одномерный гармонический осциллятор:

Потенциал имеет вид $U(x) = \frac{1}{2}kx^2$ и уравнение Шрёдингера принимает вид:
\[
-\frac{\hbar^2}{2m}\frac{d^2\psi}{dx^2} + \frac{kx^2}{2}\psi = E\psi.
\]

Это уравнение имеет дискретные значения энергии:

\[
E_n = \hbar\omega\left(n + \frac{1}{2}\right), \qquad n = 0, 1, 2, \ldots,
\]

где $\omega = \sqrt{\frac{k}{m}}$ -- собственная частота осциллятора

\mediumvspace

Таким образом, квантовая механика описывает не отдельные траектории частиц, а распределения вероятностей и энергетические уровни, определяемые волновыми функциями и их собственными значениями

Сравним физический величины классической механики и операторы квантовой
\begin{center}
    \begin{tabularx}{\textwidth}{p{0.25\textwidth}|X|X}
        Физическая величина & Классическая & Квантовая \\
        \hline
        Координата & $\vec r = (x, y, z)$ & $\vec r = (x, y, z)$ \\
        \hline
        Импульс & $\vec p = (p_x, p_y, p_z)$ & $-ih \vec \nabla = \left(-ih \frac{\partial}{\partial x}, -ih \frac{\partial}{\partial y}, -ih \frac{\partial}{\partial z}\right)$ \\
        \hline
        Угловой момент & $\vec L = [\vec r \times \vec p] = (y p_x - z p_y, z p_x - x p_z, x p_y - y p_x)$ & 
        \begin{tabular}{@{}c@{}}$
        \hat{\vec L} = -i\hbar [\vec r \times \vec{\nabla}] = \\[3pt]
        \biggl(
          -i\hbar\!\left(y \frac{\partial}{\partial z} - z \frac{\partial}{\partial y}\right), \\[3pt]
          -i\hbar\!\left(z \frac{\partial}{\partial x} - x \frac{\partial}{\partial z}\right), \\[3pt]
          -i\hbar\!\left(x \frac{\partial}{\partial y} - y \frac{\partial}{\partial x}\right)
        \biggr)
        $ \end{tabular}\\
        \hline
        Энергия (в нерелятивистском приближении) & $E = \frac{p^2}{2m} + U(\vec r)$ & $H = -\frac{\hbar^2}{2m} \nabla^2 + U(\vec r)$


    \end{tabularx}

\end{center}

