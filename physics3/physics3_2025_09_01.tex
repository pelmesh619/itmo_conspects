$subject$=Физические основы компьютерных \\ и сетевых технологий
$teacher$=Лекции Зинчика А. А.
$date$=01.09.2025

\section{Поляризация}

В прошлом семестре мы говорили о плоских бесконечных волнах. В реальности волны не бесконечные -- о них говорят, как о импульсе, одиночном, кратковременном возмущении

Свет излучается атомами за конечное время, порядка наносекунд. Получаем конечный световой импульс, длину распространения которого можно посчитать -- $l = c \cdot t$, а значит мы можем говорить о световом импульсе, который локализован, как о частице. Здесь появляется понятие кванта: атом не может излучить меньше одного фотона, поэтому фотон -- это квант, неделимая часть

Из прошлого семестра мы знаем, что электрон может преодолеть потенциальный барьер, действуя как волна, из-за своего размера. Следствием этого является ограничением на размер транзистора

Такой эффект не сходится с представлениями классической физики. В классической физике (в том числе в механике Ньютона) рассматриваются более высокие порядки размеров и на более низких скоростях, чем скорость света.
В механике Гамильтона, основывающейся на концепции гамильтониана (оператора полной энергии) отпадает понятие траектории

\mediumvspace

Будем говорить, что волна представляет $E(z, t) = \RE(E_0 e^{i(\omega t - kz)})$

Если волна не лежит в системе координат, то добавляют матрицу поворота: $E(z, t) = \RE\left(E_0 \begin{pmatrix}\cos \theta \\ \sin \theta \end{pmatrix} e^{i(\omega t - kz)}\right)$

\smallvspace

Свет считается \textbf{поляризованным}, если направления колебания светового вектора $\vec E$ упорядочены каким-либо образом

% TODO картинка

В простом случае поляризация бывает линейной (или плоской) -- в этом случае вектор напряженности движется в одной плоскости

Большинство бытовых источников света излучают неполяризованные волны -- в них колебания разных направлений быстро и беспорядочно сменяют друг друга. С помощью устройства с названием \textbf{поляризатор} можно получить поляризованный свет, поглощая другие. Поляризатор лишь частично задерживающий колебания, перпендикулярные к его плоскости, называется несовершенным. Качество поляризатора зависит от толщины и материала

С помощью другого прибора -- монохроматора -- можно получить монохроматическую волну. Так как свет с разной длиной волны имеет разные коэффициенты преломления, то монохроматор способен пропускать свет с нужной длиной волны

Если свет поляризован плохо, то его называют \textbf{частично поляризованным}

% TODO картинка

Если пропустить частично поляризованный свет через поляризатор, прибора вокруг направления луча интенсивность прошедшего света будет изменяться от $I_\min$ до $I_\max$. Причем, так как поляризатор симметричен, то угол между $I_\min$ и $I_\max$ равен $\frac{\pi}{2}$

Степенью поляризации $P = \frac{I_{\max} - I_\min}{I_{\max} + I_\min}$ можно выразить, насколько сильно поляризован свет 

\mediumvspace

Однако, так как поляризатор не пропускает лучи в неправильном направлении, то интенсивность света уменьшиться. \textbf{Закон Малюса} гласит, что доля интенсивность выходящего света от интенсивность входящего равна $\cos^2 \varphi$, где $\varphi$ -- угол между плоскостью поляризатора и плоскостью колебания $\vec E$

\[I = I_0 \cos^2 \varphi\]

Если пропустить естественный свет через поляризатор, то интенсивность выходящего света равна $I = \frac{1}{2} I_0$. Это объясняется тем, что в естественном свете волны направлены во все стороны равновероятно, а среднее значение $\cos^2 \varphi$ равна $\frac{1}{2}$

\mediumvspace

Существует круговая (или эллиптическая) поляризация, когда вектор $\vec E$ вращается в плоскости, перпендикулярной направлению распространения волны

