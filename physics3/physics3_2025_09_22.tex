$subject$=Физические основы компьютерных \\ и сетевых технологий
$teacher$=Лекции Зинчика А. А.
$date$=22.09.2025

% Увидеть результат фотоэффекта, то есть выбивание электрон, невооруженным глазом сложно. Однако можно соединить пластинки к источнику напряжения и увидеть силу тока чуть-чуть выше из-за фотоэффекта

\section{4. Давление света}

Фотоны, частицы света, имеют массу покоя равную нулю, движутся в вакууме со скоростью $c = 3 \cdot 10^8$, имеют энергию и импульс

Энергия фотона зависит от частоты света: $\varepsilon = h \nu = \hbar \omega = \frac{h c}{\lambda}$

Из специальной теории относительности нам известна формула связи энергии с массой и скоростью: $E^2 = m^2 c^4 + p^2 c^2$. Если тело покоится, то его энергия равна $E^2 = p^2 c^2$. Если масса равна нуля $E^2 = p^2 c^2$ или $E = p c$

Тогда импульс (то есть мера количества движения) фотона равен $p = \frac{\varepsilon}{c} = \frac{h \nu}{c} = \frac{h}{\lambda}$

Рассмотрим такую модель: электромагнитная волна падает на металлическую пластину. Пластина содержит свободные электроны, которые двигаются циклично из-за электрического поля в волне, а магнитное поле создает 

Свет, падая на поверхность, создает давление. В общем случае, часть фотонов отражается от поверхности, а часть поглощается

Давление света определяется импульсом, который передается поверхности фотонами, падающими на поверхность за время наблюдения

\[P = \frac{F}{S} = \frac{\Delta p}{S \Delta t}\]

При отражении импульс фотона меняется на $2 \frac{h}{\lambda}$ (так как направление становится противоположным), а при поглощении -- на $\frac{h}{\lambda}$

Пусть $\alpha$ -- коэффициент отражения, тогда $\alpha N$ фотонов отразится, а $(1 - \alpha) N$ -- поглотится

Полное изменение импульса равно $\Delta p = \alpha 2\frac{h}{\lambda} N + (1 - \alpha) \frac{h}{\lambda} N$ или $\Delta p = (1 + \alpha) \frac{h}{\lambda} N$

Количество фотонов, падающих на поверхность, можно выразить так: $N = n S c \Delta t$

Тогда давление $P = (1 + \alpha) n \frac{h c}{\lambda} = (1 + \alpha) n h \nu$

Введем другую переменную $w = n h \nu$ -- объемная плотность световой энергии, тогда $P = (1 + \alpha) w$

Так как $w = \frac{I}{c}$, $P = (1 + \alpha) \frac{I}{c}$, то есть световое давление определяется энергией (интенсивностью света)

Общее давление солнечных лучей на Землю равно $4.3$ мкПа, поэтому в земных условиях заметить величину светового давления тяжело. Впервые давление света измерил физик Лебедев в 1899 году

\mediumvspace

В 1922 году физик Комптон изучал взаимодействие рентгеновского излучения с парафином и графита и наблюдал дифракционные картины рассеянного излучения

Предполагалось, согласно классической волновой теории рассеяния ЭМИ, что длина волны волны не должна изменяться

Под действием периодического электрического поля электромагнитной волны электрон вещества должен колебаться с частотой поля. Поэтому рассеянные веществом вторичные волны должны иметь ту же частоту, что и первичное излучение

% TODO картинка

Рассеянное рентгеновское излучение состояло не только из компонент с исходной длины волны $\lambda$, но и из компоненты с другой длиной волны $\lambda^\prime$, которые рассеивались под другим углом

Подобное явление получило название эффекта Комптона -- явление упругого рассеяния электромагнитного излучения на свободных электронах вещества, сопровождающееся увеличением длины волны. Дело в том, что при столкновении фотона с электроном фотон теряет часть импульса, которая передается электрону, таким образом, фотон меняет длину волны

Сдвиг волны составляет $\Delta \lambda = \lambda_K (1 - \cos\varphi)$, где $\varphi$ -- угол отклонения вторичной волны, а $\lambda_K = \frac{h}{m_{\text{пок}} c} = 2.426$ пм -- Комптоновская длина волны

\mediumvspace

Таким образом, в разных опытах свет ведет себя по-разному. Явления интерференции, дифракции, поляризации, дисперсии объясняются электромагнитной волновой природой света. В тепловом излучении, фотоэффекте, эффекте Комптона, давления света свет представляется как поток частиц

Поэтому свет обладает двойственностью: он является и частицей и волной. Тогда Луи де Бройль в 1924 году выдвинул гипотезу, что частицы вещества наряду с корпускулярными свойствами обладают свойствами волны

Тогда $\lambda = \frac{h}{p} = \frac{h}{mv}$ (а в релятивистском случае $p = \frac{mv}{\sqrt{1 - \frac{v^2}{c^2}}}$)

А это значит, что маленькие частицы, такие как электроны, протоны, нейтроны, могут обладать длиной волны

С помощью этого можно измерить период решетки: направляя пучок электронов с известной скоростью (а значит известной длиной волны де Бройля) на монокристалл, измерив углы дифракции, можем получить период по формуле $d \sin \varphi = k \lambda$

