$subject$=Физические основы компьютерных \\ и сетевых технологий
$teacher$=Лекции Зинчика А. А.
$date$=01.12.2025

\section{9. Квантово-механическое описание молекул и статистика частиц}

Из курса школьной химии известно, что есть два вида связи между атомами в молекуле:

\begin{itemize}
    \item Ионная (или гетерополярная) -- осуществляется кулоновским электростатическим взаимодействием ионов противоположных знаков, например $\mathrm{Na}\mathrm{Cl} \ (\mathrm{Na}^+\mathrm{Cl}^-)$ или $\mathrm{K}\mathrm{Br} \ (\mathrm{K}^+\mathrm{Br}^-)$. Электрон в этом случае в среднем локализован около одного из атомов, а связь имеет в основном электростатическую природу
    \item Ковалентная (или гомеополярная) -- образуется парами электронов с противоположными спинами в таких молекулах, как $\mathrm{O}_2$, $\mathrm{N}_2$, $\mathrm{CN}$, электроны значительную часть времени проводят в пространстве между атомами и являются \enquote{общими} для обоих ядер
\end{itemize}

Рассмотрим подробнее, как возникает ковалентная связь на примере молекулы водорода. При сближении атомов водорода наступает перекрывание электронных областей и возникает новое состояние, которое не свойственно системе изолированных атомов

Однако на самом деле происходит перераспределение плотности вероятности, и вероятность того, что электрон находится между ядрами, близка к единице. Именно это приводит к уменьшению энергии системы: электроны экранируют кулоновское отталкивание протонов и одновременно притягиваются к обоим ядрам

Здесь принцип Паули утверждает, что в одной области не может быть электронов с теми же квантовыми числами, то есть эти два электрона, находящиеся на одном уровне, должны отличаться спином

Если спины электронов противоположны, полная волновая функция системы оказывается симметричной по пространственным координатам, и плотность вероятности между ядрами велика -- связь образуется. Если же спины электронов одинаковы, пространственная часть волновой функции становится антисимметричной. В этом случае волновая функция обращается в нуль в области между ядрами, вероятность нахождения там электронов резко уменьшается, и энергетически выгодной связи не возникает

Составим уравнение Шрёдингера для ковалентной связи в молекуле $\mathrm{H}_2$

% https://www.geogebra.org/calculator/hkcemf2v

\begin{center}
    \includegraphics[width=10cm]{physics3/images/physics3_h2_model}
\end{center}


Тут $e_1$ и $e_2$ -- электроны, а $A$ и $B$ -- протоны. Функция потенциала будет выглядеть так: $U = \frac{e^2}{4\pi \varepsilon_0} \left(\frac{1}{R} + \frac{1}{r_{12}} - \frac{1}{r_{1a}} - \frac{1}{r_{1b}} - \frac{1}{r_{2a}} - \frac{1}{r_{2b}}\right)$. Здесь $r_{12}$ -- расстояние между электронами, $R$ -- расстояние между протонами, а $r_{1a}$ -- расстояние между электроном и ядром

А уравнение Шрёдингера будет выглядеть так: $\Delta_1 \Psi + \Delta_2 \Psi + \frac{2m}{\hbar^2} (E - U) \Psi = 0$

Здесь волновая функция $\Psi$ зависит от координат обоих электронов. Собственные значения энергии зависят от расстояния между ядрами $R$:

% https://www.geogebra.org/calculator/kj3rqpr8

\begin{center}
    \includegraphics[width=12cm]{physics3/images/physics3_energy_functions_h2}
\end{center}

График $E(R)$ имеет минимум, соответствующий устойчивой молекуле. Глубина этого минимума определяет энергию диссоциации $E_0$ -- энергию, которую необходимо сообщить молекуле, чтобы разорвать связь и получить два свободных атома водорода

3-ье состояние -- это возбужденное состояние, в котором расстояние между ядрами больше, электроны находятся на орбиталях высоких порядков, что позволяет молекуле держатся при большем расстоянии между ядрами

В газе молекулы, помимо электронных состояний, обладают поступательным, вращательным и колебательным движением. Каждому виду движения соответствуют свои квантованные энергетические уровни. Вращательные уровни связаны с квантуемым моментом импульса молекулы, а колебательные — с квантованием движения ядер около положения равновесия. В результате каждому электронному уровню соответствует большое число близко расположенных уровней вращения и колебаний

В адиабатическом приближении волновую функцию молекулы можно представить в виде произведения трёх независимых сомножителей:
$\Psi = \psi_e \cdot \psi_v \cdot \psi_r$

Электронная часть волновой функции $\psi_e$ описывает распределение электронов в поле практически неподвижных ядер. В этой функции межъядерное расстояние $R$ выступает как параметр, а не как динамическая переменная. Решение электронного уравнения Шрёдингера даёт собственные значения энергии $E_e(R)$, которые зависят от $R$ и образуют потенциальную кривую взаимодействия ядер

Колебательная волновая функция $\psi_v(R)$ описывает движение ядер вдоль линии связи, то есть колебания молекулы около равновесного межъядерного расстояния $R_0$, соответствующего минимуму потенциальной кривой $E_e(R)$. Вблизи этого минимума потенциал можно разложить в ряд по $(R - R_0)$ и в первом приближении считать гармоническим. Тогда колебательные уровни энергии имеют вид $E_v = \hbar \omega \left(v + \tfrac{1}{2}\right), v = 0,1,2,\dots$, что соответствует квантовому гармоническому осциллятору

Вращательная волновая функция $\psi_r(\theta,\varphi)$ описывает вращение молекулы как целого вокруг центра масс. Для двухатомной молекулы эта задача эквивалентна вращению жёсткого ротатора, и вращательные уровни энергии определяются выражением $E_r = \frac{\hbar^2}{2I} J(J+1), J = 0,1,2,\dots$, где $I$ — момент инерции молекулы. Собственные функции вращательного движения выражаются через сферические гармоники

Таким образом, полная энергия молекулы в этом приближении представляется суммой электронной, колебательной и вращательной энергий: $E \approx E_e + E_v + E_r$


Переходы между этими уровнями сопровождаются поглощением или испусканием фотонов, что приводит к сложным спектрам поглощения и пропускания, например, переходы электронных энергии дают фотоны с длиной волны в ультрафиолетовом диапазоне и так далее. Анализ таких спектров позволяет определить межъядерные расстояния, моменты инерции молекул, силовые константы связей и массы ядер. Например, спектр пропускания молекулы $^{13}\mathrm{C}^{16}\mathrm{O}_2$ содержит характерные полосы, связанные с колебательно-вращательными переходами

\bigvspace

Описать движение молекул газа с помощью методов классической механики представляет из себя сложной задачей из-за числа уравнений. Поэтому используются статистические методы

В XIX веке физик Максвелл создал теорию того, как распределяется скорость молекул газа:

% https://www.geogebra.org/calculator/mu2u7wbu

\begin{center}
    \includegraphics[width=12cm]{physics3/images/physics3_maxwell_distribution}
\end{center}

Функция плотности распределения выглядит так: $f(v) = A v^2 e^{-\frac{mv^2}{2kT}}$, где $m$ -- масса молекулы, $T$ -- температура, $k$ -- постоянная Больцмана, а $A$ -- коэффициент нормировки

Это распределение показывает, что в газе существует наиболее вероятная скорость, а также медленные и быстрые молекулы

Другой, более общий подход предложил Больцман. Частицы подчиняются разными статистическими закономерностями в зависимости от того, целый ли его спин или полуцелый

Если при перестановке координат двух частиц выполняется $|\Psi(x_1, x_2)|^2 = |\Psi(x_2, x_1)|^2$, то получаем $\Psi(x_1, x_2) = \Psi(x_2, x_1)$ или $\Psi(x_1, x_2) = -\Psi(x_2, x_1)$

То есть волновая функция частиц с целым спином симметрична (то есть первое решение) -- такая статистика называется статистикой Бозе-Эйнштейна, а рассматриваемые частицы -- бозоны. Для бозонов вероятность того, что при добавлении в систему нового бозона он займет $i$-ое состояние, пропорциональна корню из числа заполнения этого состояния $p_i \sim \sqrt{N_i}$

Если частицы имеют полуцелый спин, то она называются фермионами, функция получается антисимметричной, а статистика называется статистикой Ферми-Дирака, функция распределения для нее выглядит так: $n_i(E_i) = \frac{1}{e^{\frac{E_i - \mu}{kT}} + 1}$, где $n_i$ -- среднее число фермионов в системе с энергией $E_i$

Для фермионов справедлив принцип Паули: если две частицы находятся в одинаковом состоянии, то их волновая функция не может менять знак при их перестановке. Из-за этого $0 \leq n_i \leq 1$

% https://www.geogebra.org/calculator/afv5jygp

\begin{center}
    \includegraphics[width=12cm]{physics3/images/physics3_fermi_stats}
\end{center}

Важной величиной в статистической физике является химический потенциал $\mu = \frac{dE}{dN}$, который показывает, на сколько изменяется энергия системы при добавлении одной частицы. Для ферми-газа при температуре $T \to 0$ все состояния с энергией $E < \mu$ заняты ($n_i(E) \to 1$), а состояния с $E > \mu$ свободны ($n_i(E) \to 0$)

При $T = 0$ статистика имеет такой вид:

% https://www.geogebra.org/calculator/afv5jygp

\begin{center}
    \includegraphics[width=12cm]{physics3/images/physics3_fermi_stats_t_0}
\end{center}

Здесь значения разлома функции называется энергией Ферми. До нее состояния считаются занятыми (то есть вероятность найти фермион с $E < E_F$ высока), а после нее свободными (то есть новый фермион свободно может занять его)

Энергия Ферми равна $E_F = \frac{\hbar^2}{8m} \left(\frac{3n}{\pi}\right)^{\frac{2}{3}}$

Если разделить $E_F$ на постоянную Больцмана $k$, получим $T_F = \frac{E_F}{k}$ -- температура Ферми. Если температура системы $T \ll T_F$, то квантовые эффекты в системе доминируют, газ из фермионов (например, электроны в металле) подчиняются статистике Ферми–Дирака, фермионов много. При $T \gg T_F$ фермионы ведут себя почти как классические частицы в газе и подчиняются статистике Максвелла-Больцмана. При $T \sim T_F$ проявляются как квантовые, так и тепловые эффекты

При повышении температуры резкий переход в распределении Ферми–Дирака сглаживается: электроны вблизи уровня Ферми могут переходить на более высокие энергетические уровни за счёт тепловой энергии

Эти квантово-статистические эффекты лежат в основе работы многих современных устройств, в том числе полупроводниковых приборов и светодиодов
