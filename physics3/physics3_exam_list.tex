\clearpage

\section{X. Программа экзамена в 2023/2024}

\begin{enumerate}
    \item Электромагнитная плоская монохроматическая волна волна и ее характеристики: вектора $E$, 
$B$ и $k$, амплитуда, поляризация, фаза, длина волны и фазовая скорость. Длина волны и 
фазовая скорость в вакууме и среде. 
    \item Поляризация. Степень поляризации. Закон Малюса. 
    \item Линейная, круговая и эллиптическая поляризации. Способы получения и преобразования 
одной в другую.  
    \item Сложение двух бегущих, в одном направлении плоских монохроматических волн с близкими 
частотами. Простейший волновой пакет - биения. Фазовая и групповая скорости волн.  
    \item Сложение электромагнитных волн с непрерывным спектром по частотам. Волновой пакет. 
Групповая скорость.  
    \item Волновой пакет. Локализация волнового пакета и его длительность. 
    \item Дисперсия групповой скорости. Расплывание волнового пакета в среде. Оценка времени 
расплывания.  
    \item Фотоэффект. Законы фотоэффекта. Уравнение Эйнштейна. 
    \item Эффект Комптона. 
    \item Корпускулярно волновой дуализм. Волна де-Бройля. Волновая функция и ее физическая 
интерпретация.  
    \item Основные постулаты квантовой механики. Операторы координаты и импульса.  
    \item Коммутационные соотношения координаты - проекции импульса. Соотношения 
неопределенностей.  
    \item Гамильтониан. Оператор полной энергии в координатном представлении.  
    \item Нестационарное и стационарное уравнение Шредингера.  
    \item Прохождение квантовой частицы энергетического барьера. Туннелирование.  
    \item Частицы в потенциальной яме с бесконечными стенками. Энергетические уровни. 
    \item Уравнение Шредингера для частицы в параболическом потенциале. Уровни энергии и 
волновые функции квантового гармонического осциллятора.  
    \item Оператор орбитального момента (момента импульса) в декартовых и сферических 
координатах. Собственные значения $z$-й проекции оператора орбитального движения 
электрона.  
    \item Операторы полного орбитального момента и квадрата орбитального момента. Собственные 
значения операторов. Коммутационные отношения между ними.  
    \item Атом водорода. Построение решения для задачи движения электрона в атоме водорода. 
Разделение волновой функции на радиальную и угловую части.  
    \item Атом водорода без учета спина. Квантовые числа. Уровни энергий. Вырождение энергии по 
орбитальному квантовому числу.  
    \item Опыт Штерна-Герлаха. Гиромагнитное отношение. Магнетон Бора. 
    \item Спин электрона. Оператор спина электрона. Собственный магнитный момент электрона 
    \item Физические системы для квантовой информации (двумерные кубиты). Базисные состояния 
для фотонных и спиновых систем.  
    \item Сфера Блоха и представление кубита на сфере Блоха.  
    \item Запутанные состояния для двух частичных квантовых систем и их отличие состояний двух 
невзаимодействующих систем.  
    \item Статистика Ферми-Дирака. Энергия Ферми. 
    \item Модель Кронига-Пенни. Энергетические зоны. 
    \item Заполнение зон чистого полупроводника. Донорная и акцепторная примесь. Изменения в 
структуре зон. 
    \item pn-переход. Прямое и обратное смещение. Полевой транзистор.
\end{enumerate}
