$subject$=Физические основы компьютерных \\ и сетевых технологий
$teacher$=Лекции Зинчика А. А.
$date$=15.12.2025

\mediumvspace

Материалы можно разделить на три вида:

\begin{itemize}
    \item Проводники

    В них тип связи металлическая, а зона проводимости и валентная зона пересекаются, что позволяет с легкостью выбивать электроны

    \item Полупроводники
    
    В них тип связи ковалентная и ионная, но размер запрещенной зоны достаточно мал, чтобы перенести электрон из валентной зоны в зону проводимости без изменения структуры решетки

    \item Диэлектрики
    
    В них тип связи ковалентная и ионная, но размер запрещенной зоны настолько большой, что при передачи огромной энергии электронам структура диэлектрика ломается из-за высоких температур
\end{itemize}

Полупроводники обладают важным свойством: с ростом температуры вероятность перехода электронов из валентной зоны в зону проводимости возрастает. Это связано с тем, что тепловая энергия $k_B T$ становится сопоставимой с шириной запрещённой зоны, и часть электронов получает возможность преодолеть энергетический зазор. В отличие от металлов, сопротивление полупроводников при нагреве уменьшается

Чтобы полупроводники могли проводить в комнатной температуре, в полупроводники, в основном состоящие из кремния, добавляют примеси -- такой процесс называется легированием (или doping). Атомы примеси могут содержать больше электронов или места для электронов по сравнению с атомами полупроводников

Например, можно добавить в кристалл германия атом сурьмы, которые имеют больше валентных электронов (такие примеси называются донорными, а полупроводники -- n-типа), а примесь индия даст больше дырок или незаполненные энергетические состояния, которые могут быть заняты электронами (такие называются акцепторными, а полупроводники p-типа)

Названия полупроводников n-типа и p-типа происходят от слов negative и positive, несмотря на то, что заряд таких полупроводников нейтрален. В общем случае запрещенная зону у таких полупроводников уменьшается

\begin{center}
    \includegraphics[width=0.9\linewidth]{physics3/images/physics3_p_n_type_semiconductors}
\end{center}

Однако можно добавлять примеси, которые увеличивают запрещенную зону. Благодаря этому можно сделать ее определенной ширины, чтобы переход электрона через запрещенную зону вызывал квант света определенной длины -- так получается цветной светодиод

Или же наоборот, при направлении света на полупроводник происходит внутренний фотоэффект, электронов свободных становится больше, из-за чего сопротивление уменьшается -- получился фоторезистор. Фоторезисторы производят из сульфида кадмия, сульфида цинка и селенида кадмия

Объединяя полупроводники p-типа и n-типа, мы получаем p-n переход. В нем вблизи границы возникает обеднённый (или запирающий) слой, в котором отсутствуют свободные носители заряда и формируется внутреннее электрическое поле. Это поле препятствует дальнейшему перемещению электронов и дырок. 

\begin{center}
    \includegraphics[width=0.9\linewidth]{physics3/images/physics3_p_n_transition}
\end{center}

Если к p-области подключить положительный полюс, а к n-области отрицательный (прямое включение), внешнее поле уменьшает потенциальный барьер. Обеднённый слой сужается, и электроны из n-области могут переходить через границу в p-область, где рекомбинируют с дырками. Аналогично дырки переходят в n-область. В этом режиме через p-n переход течёт большой ток

Если же подключить наоборот -- положительный полюс к n-области, а отрицательный к p-области (обратное включение), внешнее поле увеличивает потенциальный барьер. Обеднённый слой расширяется, и основные носители не могут пройти через переход, а ток практически отсутствует 

\begin{center}
    \includegraphics[width=0.9\linewidth]{physics3/images/physics3_p_n_current}
\end{center}

В результате p-n переход пропускает ток преимущественно в одном направлении, что лежит в основе работы полупроводникового диода. При достаточно большом обратном напряжении возможен пробой, и ток начинает протекать и в обратном направлении

p-n переходы используются не только в диодах, но и в солнечных батареях, где энергия света преобразуется в электрическую энергию за счёт разделения фотогенерированных электронов и дырок внутренним полем перехода

Главным образом на основе p-n переходов создаются и транзисторы. В биполярных транзисторах (Bipolar Junction Transistor, BJT) ток управляется за счёт инжекции носителей через два p-n перехода. 

В полевых транзисторах (Field-Effect Transistor, FET) управление осуществляется электрическим полем, изменяющим концентрацию носителей в проводящем канале

Если длина затвора в полевом транзисторе становится очень малой (порядка нескольких нанометров), начинает существенно проявляться туннельный эффект. Электроны могут проходить через потенциальный барьер затвора с ненулевой вероятностью, даже если классически это запрещено. Это приводит к утечкам тока и нарушению нормальной работы транзистора, и именно по этой причине современная индустрия производства транзисторов пришла в ограничение по уменьшению размера транзистора

Современные транзисторы изготавливаются с применением сложных технологических процессов: выращивание монокристалла кремния, получение тонких пластин, нанесение защитных слоёв, использование фоторезиста, фотолитография, травление, легирование, формирование контактов, разрезание пластин на отдельные чипы и их упаковка

На современном этапе уменьшение размеров транзисторов перестало приводить к их удешевлению. Напротив, дальнейшая миниатюризация требует всё более сложных и дорогих технологий, что приводит к росту стоимости одного транзистора

