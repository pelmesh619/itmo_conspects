$subject$=Физические основы компьютерных \\ и сетевых технологий
$teacher$=Лекции Зинчика А. А.
$date$=06.10.2025

После этого Нильс Бор сформулировал свои постулаты

Первый постулат Бора: атомная система может находиться только в особых стационарных состояниях. Каждому стационарному состоянию соответствует определённая энергия $E_n$

Выбор радиусов стационарных орбит подчиняется условию квантования момента импульса электрона: момент импульса $L = m v r$ электрона в стационарных состояниях принимает дискретные значения

Второй постулат Бора: излучение (или поглощение) электромагнитного излучения атомом происходит при переходе электрона из одного состояния в другой

Тогда энергия излученного фотона равна разности энергий двух стационарных состояний $\varepsilon = h \nu = E_n - E_m$

При поглощении электрон переходит на состояние выше с большей энергией, а при излучении -- на состояние ниже с меньшей энергией

По второму закону Ньютона $F = m a = m \frac{v^2}{r} = \frac{k Z e^2}{r^2} = m \frac{\hbar^2 n^2}{m^2 r^3}$ -- квантование орбит 

Радиус орбиты вычисляется так: $r_n = \frac{\hbar^2 n^2}{k Z m e^2}$, где $Z$ -- порядковый номер атома

Для водорода радиус первой орбиты равен $r_1 = \frac{\hbar^2}{k e^2 m} = 0.529$ нм

Далее, зная радиус первой орбиты, можно вычислить другие орбиты по формуле $r_n = n^2 r_1$

% TODO картинка

\smallvspace

Полная энергия электрона на стационарных орбитах складывается как сумма кинетической и потенциальной: $E = \frac{m v^2}{2} - \frac{k Z e^2}{r}$

Потенциальная энергия притяжения электрона к ядру меньше нуля. Также полную энергию можно выразить как $E = -\frac{k^2 Z^2 e^4 m}{2 \hbar^2 n^2}$

Основным состоянием электрона называется такое, что при $n = 1$ $E = -\frac{k^2 Z^2 m e^4}{2\hbar^2} = -13.6$ эВ 

Все состояния, кроме основного, называются возбужденными. Время жизни в них ограничено и равно $10^8$ с

При $E = 0$ атом ионизуется, то есть электрон покидает пределы атома и становится свободным. Для ионизации требуется сообщить энергию $E_i = 0 - E_1$ (для водорода $13.6$ эВ)

\mediumvspace

Теория Бора привела к количественному согласию с экспериментом для значений частот, излучаемых водородом. Так частоты излучений образуют ряд серий, при которых электрон перемещается из уровня $n$ в уровень $m$ ($n > m$)

Частоту можно вычислить по формуле $\nu = \frac{k^2 Z^2 e^4 m_e}{4 \pi \hbar^3} \left(\frac{1}{m^2} - \frac{1}{n^2}\right)$

Отсюда $R_c = \frac{m_e k^2 e^4}{4 \pi \hbar^3}$, а $R = \frac{R_c}{c}$ -- постоянная Ридберга

Тогда $\frac{1}{\lambda} = R Z^2 \left(\frac{1}{m^2} - \frac{1}{n^2}\right)$ -- формула Ридберга

Излучение фотонов из первой серии или серии Лаймера было открыто в 1906 году. Частоты этой серии относятся к ультрафиолетовой области

Серия Бальмера была открыта в 1885 году, излучение таких фотонов относят к видимой области спектра. Серия Пашнеа была открыта в 1908 году, излучение таких фотонов относят к инфракрасной области спектра

\mediumvspace

Доказательство существование дискретных энергетических уровней у атомов было предоставлено опытом Франка-Герца

В этом опыте электроны с катода переходили на анод через колбу с парами ртути. Ртуть - атом тяжелый, но ее пары можно получить при комнатной температуре. По классическим представления ВАХ должна была быть линейной, однако в реальности на ВАХ были обнаружены локальные максимумы. 

Дело в том, что при определенном напряжении на аноде и катоде, электроны при столкновении с ртутью передавали ровно столько энергии, чтобы электроны в атомах ртути переходили с одного состояния на другой, излучая ультрафиолетовое свечение

\mediumvspace

Теория Бора:

\begin{itemize}
    \item построила количественную теорию спектра атома водорода
    \item согласовала теоретически вычисленные значения частот с экспериментальными значениями;
    \item позволила сделать качественные заключения о водородоподобных атомах
\end{itemize}

Однако теория Бора имеет недостатки:

\begin{itemize}
    \item не удалось создать количественную теорию водородоподобных атомов
    \item не является последовательно классической теорией (электрон -- классическая частица, но его энергия квантуется)
    \item не является последовательно квантовой теорией (электрон движется по круговым орбитам, но для квантовой частицы не применимо понятие траектории)
\end{itemize}

