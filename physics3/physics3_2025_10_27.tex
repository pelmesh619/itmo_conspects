$subject$=Физические основы компьютерных \\ и сетевых технологий
$teacher$=Лекции Зинчика А. А.
$date$=27.10.2025

Рассмотрим стационарное уравнение Шрёдингера: если $U = 0$, то

\[\frac{\hbar^2}{2m} \Delta \psi + i\hbar \frac{\partial}{\partial t} \psi = 0\]

Тогда можно искать решения в виде $\psi(\vec r, t) = e^{\frac{-iEt}{\hbar}} \theta(\vec r)$, то есть 

\[\frac{\hbar^2}{2m} \Delta e^{\frac{-iEt}{\hbar}} \theta + E e^{\frac{-iEt}{\hbar}} \theta = 0\]

Или

\[\frac{\hbar^2}{2m} \Delta \theta + E \theta = 0\]

Получаем, что $\psi(\vec r, t) = e^{\frac{-iEt}{\hbar}} \theta(\vec r) = \left(\cos \frac{-Et}{\hbar} + i \sin \frac{-Et}{\hbar}\right) \theta(\vec r)$

Из этого волновое число $k = \sqrt{\frac{2 m E}{\hbar^2}}$

\mediumvspace

В общем случае для $U \neq 0$, зная, что $-\frac{\hbar^2}{2m} \Delta + U = \hat H$, получаем $\hat H \psi = E \psi$

Так как $\hat H$ -- линейный оператор, то у него есть соответствующая матрица, для которой можно найти собственные числа. Так как $E \in \Complex$, значения энергии $E$ и являются собственными числами, а функции $\psi$ -- собственными векторами


\mediumvspace

Огромное число задач можно свести к такой модели: потенциальная яма $U(x) = \begin{cases}0, & 0 \leq x \leq L \\ U_0, & \text{иначе}\end{cases}$


В ней вероятность встретить частицу за пределами ямы равен 0, то есть $\psi(0) = \psi(L) = 0$. Если стенки бесконечно большие, то волновая функция представляет собой синусоиду $\psi = \psi_0 \sin (\omega x)$, где $\omega = \frac{n \pi}{L}$

После нормировки получаем $\psi(x) = \sqrt{\frac{2}{L}} \sin \frac{n \pi x}{L}$

Энергия для соответствующей функции равна $E = \frac{n^2 \pi^2 \hbar^2}{2 m L^2}$

Если яма шириной в $L = 1$ м, а масса $m = 1$ кг, то минимальная энергия при $n = 1$ имеет вид $\frac{\pi^2 \hbar^2}{2} \approx 4 \cdot 10^{-66}$ -- энергия дискретна, но на каждом шаге изменяется на $10^{-66}$

% https://www.geogebra.org/calculator/ubcv93md

\begin{center}
    \includegraphics[width=10cm]{physics3/images/physics3_potential_pit.png}
\end{center}

\mediumvspace

Также можно представить гиперболу -- модель атома Бора 

% https://www.geogebra.org/calculator/ksvyk29b

\begin{wrapfigure}{R}{0pt}
    \includegraphics[width=7cm]{physics3/images/physics3_potential_hyperbola}
\end{wrapfigure}

Подставив множество атомов, получаем множество гипербол, а в трехмерном пространстве получаем сетку, решив волновое уравнение для которой получаем зонную структуру вещества

Получаем 2 пространства: где решений волнового уравнения нет -- так называемая запрещенная зона энергий; и где решения есть -- разрешенная зона

\mediumvspace

Тривиально уравнение Шрёдингера аналитически решается для модели с потенциальным прямоугольным барьером с конечными длиной и высотой $U(x) = \begin{cases}U_0, & 0 \leq x \leq L \\ 0, & \text{иначе}\end{cases}$

% https://www.geogebra.org/calculator/hwufehuc

\begin{center}
    \includegraphics[width=10cm]{physics3/images/physics3_potential_barrier}
\end{center}

Если $E < U_0$, то существует ненулевая вероятность, что электрон перескочит барьер, а если $E > U_0$, то электрон беспрепятственно проходит

В общем случае любой барьер можно представить как композицию прямоугольных барьеров

Если волновая функция в барьере не успевает экспоненциально спасть, то выходит из барьера

Слева от барьера, где $U = 0$ уравнение имеет вид $\Delta \psi + \frac{2 m E}{\hbar^2} \psi = 0$

Поэтому $\psi_1(x) = A_1 e^{ik_1 x} + B_1 e^{-ik_1 x}$, где $k_1 = \sqrt{\frac{2m}{\hbar^2} E}$ -- волновое число

Здесь часть $A_1 e^{ikx}$ -- волна, идущая вправо на барьер, а часть $B_1 e^{-ikx}$ -- отраженная от нее волна. Если не учитывать отраженную волну, то действительная часть $\RE(\psi_1)$ является косинусоидальной волной $A_1 \cos(k_1 x)$

\mediumvspace

Внутри барьера $U = U_0$, уравнение принимает вид $\Delta \psi + \frac{2 m}{\hbar^2} (E - U_0) \psi = 0$

В рассматриваемой задаче частицы, прошедшие в область барьера, при 
движении в этой области никаких препятствий не встречают, 
поэтому отраженного потока в этой области быть не должно, то есть 
амплитуда отраженной волны в области барьера равна нулю

Поэтому для случая $E < U_0$ получаем $\psi_2(x) = A_2 e^{-\beta x}$, где $\beta = i k_2$ -- экспоненциальный спад

Так как $\psi$ -- непрерывная функция, то $\psi_1(0) = \psi_2(0)$, $\frac{d \psi_1}{dx} \Big|_{x = 0} = \frac{d \psi_2}{dx} \Big|_{x = 0}$

Получаем, что $A_1 + B_1 = A_2$ и $A_1 - B_1 = \frac{k_2}{k_1} A_2$

То есть $A_2 = \frac{2k_1}{k_1 + k_2} A_1$, $B_1 = \frac{k_1 - k_2}{k_1 + k_2} A_1$

Таким образом, коэффициент отражения волны $R = \frac{B^2_1}{A^2_1} = \left(\frac{k_1 - k_2}{k_1 + k_2}\right)^2 = \left(\frac{1 - \sqrt{1 - \frac{U_0}{E}}}{1 + \sqrt{1 - \frac{U_0}{E}}}\right)^2$

\mediumvspace

В третьей области также получаем $\psi_3(x) = A_3 e^{i k_1 x}$

Тут $A_3 = A_1 e^{-\beta L}$

\mediumvspace

В этой задаче возникает туннельный эффект -- преодоление объекта потенциального барьера в случае, когда её полная энергия меньше высоты барьера

Представим такую систему: стеклянная пластинка, а по сторонам от нее прозрачный слой с меньшим показателем преломления. Свет внутри пластинки испытывает внутреннее отражение, что согласуется с геометрической оптикой и законом преломления, так как $\sin \theta_2 = \frac{n_1}{n_2} \sin \theta_1 > 1$, но с точки зрения квантовой механики свет в этом слою испытывает затухание

Если между двумя такими пластинками поместим тонкий слой с меньшим показателем преломления, то свет будет способен проходить через него, несмотря на то, что $\sin \theta_2 > 1$


Также туннельный эффект используется в сканирующем туннельном микроскопе: на конце очень острого электрода создаётся очень маленькое расстояние порядка атомов до поверхности образца: электроны туннелируют между кончиком и образцом, и измеряемый туннельный ток чрезвычайно чувствителен к расстоянию и локальной плотности состояний поверхности

\mediumvspace

Коэффициентом прозрачности барьера считается величиной $D = \left(\frac{A_3}{A_1}\right)^2 = e^{-2\beta l} = e^{-\frac{2 l}{\hbar} \cdot \sqrt{2m(U_0 - E)}}$

Для произвольного барьера $D = e^{-\frac{2}{h} \int_a^b \sqrt{2m (U(x) - E)} dx}$

Из этого можно найти коэффициент отражения $R = 1 - D$

\mediumvspace

Если барьер представляет параболу, то получаем модель колебаний атомов в двухатомной молекуле, или так называемый квантовый гармонический осциллятор $U(x) = \frac{1}{2} m \omega^2 x^2$

% https://www.geogebra.org/calculator/wbmwsaxz

\begin{wrapfigure}{R}{0pt}
    \includegraphics[width=7cm]{physics3/images/physics3_molecule_potential.png}
\end{wrapfigure}

В нем минимальная энергия -- $\frac{1}{2} \hbar \omega$. Последующие задаются формулой $E_n = \hbar \omega \left(n + \frac{1}{2}\right)$

В реальной жизни правая ветвь параболы после порогового значения становятся нулю -- это значит, что атомы нарушили свою связь. Если парабола несимметрична, то говорят, что осциллятор ангармонический

При колебаниях молекула излучает электромагнитные волны в инфракрасном спектре
