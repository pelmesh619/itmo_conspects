\documentclass[12pt]{article}
\usepackage{preamble}
\usepackage{biblatex}

\pagestyle{fancy}
\fancyhead[LO,LE]{Математический анализ}
\fancyhead[CO,CE]{27.03.2024}
\fancyhead[RO,RE]{Лекции Далевской О. П.}


\begin{document}
    \clearpage

    \section{5. Интеграл ФНП}

    \subsection{5.1. Общая схема интегрирования}

    \underline{Постановка задачи.}

    В некоторой области $\Omega$ (дуга кривой, участок поверхности, тело и т. д.)
    распределена или действует непрерывно некоторая функция скалярная $g$ или векторная $\vec{G}$,
    то есть определены $g(M)$ или $\vec{G}$ $\forall M \in \Omega$

    \Ex Область $\Omega$ - дуга кривой $l : y = y(x)$

    Скалярная функция $g(M)$ - плотность в точке $M$

    \Ex Область $\Omega$ - трубка в $\Real^3$
    Из всех векторов $\vec{v}$ (для всех $M \in \Omega$) складывается \enquote{поле жидких скоростей}

    Векторная величина $\overrightarrow{G}(M)$ - скорость жидкой частицы, движущейся по трубке


    \Ex Область $\Omega$ - кривая, по которой движется точка $M$ под действием силы $\overrightarrow{G}(M)$

    Задача интегрирования - найти суммарное содержание скалярной величины или действие векторной величины в области $\Omega$

    \underline{Схема} Величины $g(M)$ и $\overrightarrow{G}(M)$, меняясь от точки к точке заменяются на квазипостоянные на малых (элементарных) участках $d\omega$

    Так как $g(M)$ или $\overrightarrow{G}(M)$ должны быть непрерывны на $\Omega$, то на малом участке $d\omega$ их изменение незначительно и
    значение функции можно считать почти постоянным, приняв за это значение какое-либо среднее $g_{\text{ср.}}(M), \overrightarrow{G_{\text{ср.}}}(M)$

    Тогда элементарное содержание $g(M)$ в $d\omega$ будет отличаться от среднего содержания, то есть $g_{\text{ср.}}d\omega$ на б. м. большего порядка

    \Ex Проиллюстрируем на примере $\int_a^b f(x) dx$

    $S$ - площадь по наибольшей границе, $\sigma$ - площадь по наименьшей границе, $S_{\text{трап.}}$ - \enquote{истинная} площадь

    Т. к. $f(x)$ непр. $\forall x \in [a, b]$, то $\Delta f \stackrel{\Delta x \to 0}{\rightarrow} 0$

    Для простоты рассмотрим монотонно возрастающую $f(x)$

    Хотим доказать, что $S - S_{\text{трап.}}$ - б. м. большего порядка, чем $S_{\text{трап.}}$ или $S$

    \[0 \leq S - S_{\text{трап.}} \leq dx \Delta y\]

    Сравним $\frac{dx \Delta y}{S} = \frac{dx \Delta y}{dx f(x + \Delta x)} = \frac{\Delta y}{\text{огр.}} \stackrel{\Delta x \to 0}{\rightarrow} 0$

    таким образом $S - S_{\text{трап.}} = 0 (S_{\text{трап.}})$

    \underline{Смысл интеграла} в случае векторной функции $\overrightarrow{G}(M)$

    Будем интегрировать только скалярные выражения вида $\overrightarrow{G}(M) \cdot d\overrightarrow{\omega}$ - скал. произведение векторов,
    где $d\overrightarrow{\omega}$ - ориентированный элемент $d\omega$

    \Ex Сила $\overrightarrow{F}(M)$ перемещает точку $M$ вдоль плоской кривой $l$. При этом сила совершает работу по перемещению
    (работа $A$ - скалярная величина)

    Известна формула для $\overrightarrow{F} = const$ и перемещения $\overrightarrow{s}$ по прямой: $A = \overrightarrow{F} \cdot \overrightarrow{s}$

    Разобьем дугу на элементы $dl \approx ds$ и ориентируем их (зададим направление перемещению $ds$)

    $dl = ds + o(dl)$, $d\overrightarrow{s}$ - вектор элем. перемещения, как правило, $ds$ направлен согласовано с $Ox$

    Элемент работы $dA = \overrightarrow{F} \cdot d\overrightarrow{s} = (F_x, F_y) \cdot (dx, dy) \stackrel{\text{обозн.}}{=}
    (P, Q) \cdot (dx, dy) = Pdx + Qdy$ - скаляр. Вся работа равна $A = \int dA$

    \Nota Ориентированный участок поверхности $d\overrightarrow{\sigma}$ - это размер участка $d\sigma$, умноженный на вектор нормали к участку $\overrightarrow{n}$,
    то есть $d\overrightarrow{\sigma} = \overrightarrow{n}d\sigma$

    \underline{Итак.} Схема интегрирования:

    \textbf{1*} Дробление области $\Omega$ на элементы $d\omega$

    \textbf{2*} Выбор постоянного значения функции на $d\omega$, то есть $g_{\text{ср.}}$ или $\overrightarrow{G_\text{ср.}}$

    \textbf{3*} Составление подынтегрального выражения $g_{\text{ср.}}d\omega$ или $\overrightarrow{G_\text{ср.}}d\overrightarrow{\omega}$

    \textbf{4*} \enquote{Суммирование} элементарных величин $\int gd\omega$ или $\int \overrightarrow{G}d\overrightarrow{\omega}$



    \textbf{1* По размерности $\Omega$}

    \begin{tabular}{p{8cm}p{8cm}}
        $n = 1$: * прямая (опред. интеграл $\int_a^b$)    & * кривая (криволинейный интеграл $\int_A^B$)                       \\
    \subsection{5.2. Классификация интегралов}

        $n = 2$: * плоскость (двойной интеграл $\iint_D$) & * поверхность, не криволинейная (поверхностный интеграл $\iint_S$) \\

        $n = 3$: * пространство $\Real^3$  \\
        (тройной $\iiint_V$ или $\iiint_T$)

    \end{tabular}


    \textbf{2* По виду функции}

    \begin{tabular}{p{8cm}p{8cm}}
        скалярная $g(M)$                            & векторная $\overrightarrow{G}(M)$         \\

        $n = 1$: определенный, криволинейный I рода & криволин. II рода (интегралы в проекциях) \\

        $n = 2$: двойной, поверхн. I рода            & поверхн. II рода                          \\

        $n = 3$: тройной
    \end{tabular}


\end{document}
