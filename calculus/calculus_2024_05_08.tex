\documentclass[12pt]{article}
\usepackage{preamble}
\usepackage{stmaryrd}

\pagestyle{fancy}
\fancyhead[LO,LE]{Математический анализ}
\fancyhead[CO,CE]{08.05.2024}
\fancyhead[RO,RE]{Лекции Далевской О. П.}


\begin{document}
    \subsection{5.6. Поверхностные интегралы}

    \hypertarget{surfaceintegraloffirstkind}{}

    \begin{enumerate}[label*=\textbf{\arabic** }]
    \item \textbf{Поверхностные интегралы I рода} (по участку поверхности)

    \underline{Задача}: найти массу поверхности. Дана функция $u = u(x, y, z)$ (ее физический смысл - плотность)

    Элементарная масса: $dm = u_{\text{ср.}}(\xi, \eta, \zeta) d\sigma$, $d\sigma$ -- элемент поверхности

    $M = \iint_S dm = \iint_S u(x, y, z)$ -- поверхностный интеграл I рода

    \begin{enumerate}
        \item Дробление $S$ на элементы $\Delta \sigma_k$ координатными плоскостями $x = x_i, y = y_j$

        \item Определение средней точки $(\xi_k, \eta_k, \zeta_k)$

        \item Интегральная сумма $\nu_n = \sum_{k = 1}^{n} u(\xi_k, \eta_k, \zeta_k) \Delta \sigma_k$

        \item \Defs $\iint_S u(x, y, z) \Delta \sigma = \lim_{\substack{n \to \infty \\ \tau = \max \Delta \sigma_k \to 0}} \nu_n$ -- поверхностный интеграл первого рода
    \end{enumerate}

    \hypertarget{surfaceintegraloffirstkindproperties}{}

    Свойства: смена обхода поверхности $S$ не меняет знака интеграла: $\iint_{S^+} u d\sigma = \iint_{S^-} u d\sigma$

    \mediumvspace


    \underline{Вычисление}

    \Mems Криволинейный интеграл $\int_L f(x, y) dl$ мы вычисляли через параметризацию кривой одной переменной $\begin{cases}x = \varphi(t) \\ y = \psi(t)\end{cases} \quad t \in [\alpha, \beta]$, замену элементарного участка
    $dl = \sqrt{\varphi^{\prime 2}(t) + \psi^{\prime 2}(t)} |dt|$ и функции $f(x, y)$ на $\tilde{f}(t)$. Получаем 
    $\iint_L f(x, y)dl = \int_\alpha^\beta \tilde{f}(t) \sqrt{\varphi^{\prime 2}(t) + \psi^{\prime 2}(t)} |dt|$
    
    \mediumvspace

    \hypertarget{surfaceintegraloffirstkindcalculation}{}

    Аналогично для поверхностного: $\iint_S u(x, y, z) d\sigma$

    \begin{enumerate}
        \item Параметризация $S$: самая частая -- $z = z(x, y), (x, y) \in D$ -- пределы интегрирования

        \item $d\sigma = \sqrt{1 + \left(\frac{\partial z}{\partial x}\right)^2 + \left(\frac{\partial z}{\partial y}\right)^2} |dxdy|$, но так как
        в двойном интеграле договорились, что $dxdy > 0$ (площадь), модуль можно не ставить (область $D$ проходится в направлении против часовой стрелки)

        \item $u(x, y, z) = \tilde{u}(x, y, z(x, y)) = \tilde{u}(x, y)$

        $\iint_S u(x, y, z) d\sigma = \iint_{D^+} \tilde{u}(x, y) \sqrt{1 + z_x^{\prime 2} + z_y^{\prime 2}} dxdy$
    \end{enumerate}

    \Ex $S: \ x^2 + y^2 = z^2, z = 0, z = 1$

    $u(x, y, z) = z$

    $\iint_S zd\sigma =
    \begin{bmatrix}
        S: z = \sqrt{x^2 + y^2} \\
        D: \text{круг}, x^2 + y^2 = 1 \\
        d\sigma = \sqrt{1 + \frac{x^2}{x^2 + y^2} + \frac{y^2}{x^2 + y^2}} dxdy = \sqrt{2} dxdy \\
    \end{bmatrix} =
    \iint_D \sqrt{x^2 + y^2} \sqrt{2} dxdy = \begin{bmatrix}\text{переход в ПСК}\end{bmatrix} = \sqrt{2} \int^{2\pi}_{0} d\varphi \int_{0}^{\rho} \rho \underset{|J|}{\undergroup{\rho}} d\rho = \sqrt{2} 2\pi \frac{\rho^3}{3} \Big|_0^1 = \frac{2\sqrt{2}\pi}{3}$

    \hypertarget{surfaceintegralofsecondkind}{}

    \item \textbf{Поверхностный интеграл II рода.}

    \underline{Задача}: нахождение потока

    Будем говорить о потоке вектора $\vec{F} = (P, Q, R)$ через площадку $S$ в направлении нормали $\vec{n^+}$ или $\vec{n^-}$

    Если задано поле жидких скоростей, то потоком называют количество жидкости, протекающей через $S$ за время $\Delta t$

    В простой ситуации поток $\Pi = FS \ (\vec{F} \perp S, \vec{F} = \const)$

    В общем случаем $\vec{F}$ -- переменная, $S$ -- искривленная и $\angle \vec{F}, S \neq \frac{\pi}{2}$

    Переходим к вычислению элементарного потока $d\Pi$

    $d\sigma$ -- малый элемент поверхности (почти плоский)

    В пределах $d\sigma$ $\vec{F}$ меняется мало, за среднее берем $\vec{F} = (P, Q, R)$, где $P = P(x, y, z), Q = Q(x, y, z), R(x, y, z)$

    Разберемся с наклоном: если площадка перпендикулярна, то $d\Pi = F d\sigma$,
    но в нашем случае высота цилиндра равна $\text{проек.}_{\vec{n}} \vec{F} = (\vec{n}, \vec{F}) = F \cos\varphi$, где $\vec{n}$ -- единичный вектор нормали, $\varphi$ -- угол между нормалью и потоком,
    $d\Pi = (\vec{F}, \vec{n}) d\sigma = F_n d\sigma$

    Пусть $\vec{n} = (\cos\alpha, \cos\beta, \cos\gamma)$, тогда $d\Pi = (\vec{F}, (\cos\alpha, \cos\beta, \cos\gamma)) d\sigma =
    (P\cos\alpha, Q\cos\beta, R\cos\gamma)d\sigma$

    Итак, $\Pi = \iint_{S^{\vec{n}}} d\Pi = \iint_{S^{\vec{n}}} F_n d\sigma = \iint_{S^{\vec{n}}} (\vec{F}, \vec{n})d\sigma = \iint_{S^{\vec{n}}} (P\cos\alpha + Q\cos\beta + R\cos\gamma)d\sigma$

    Но, еще нет координатной записи подынтегрального выражения. Спроектируем $d\sigma$ на координатные плоскости: сначала разрежем поверхность $S$ на элементы плоскостями $x = \const, y = \const$ (и, таким образом, уточним форму $d\sigma$). Так как $d\sigma$ мал, то можно считать его плоским параллелограммом

    Тогда $\cos\gamma d\sigma = \pm dxdy$ ($\gamma$ -- угол между нормалью и осью $Oz$)

    Нашли последнее слагаемое $\iint_{S^{\vec{n}}} R\cos\gamma d\sigma$ в исходном интеграле (I рода, так как по участку $d\sigma$)

    Найдем $\iint_{S^{\vec{n}}} Q\cos\beta d\sigma$, разобьем поверхность на участки $d\sigma$ плоскостями $x = \const, y = \const$

    Аналогично $\cos\beta d\sigma = \pm dxdz$

    Тогда в $\iint_{S^{\vec{n}}} P\cos\alpha d\sigma \quad \cos\alpha d\sigma = \pm dydz$

    \hypertarget{connectionbetweensurfaceintegral}{}

    Окончательно, поток $\Pi = \iint_{S^{\vec{n}}} \pm Pdydz \pm Qdxdz \pm Rdxdy = \iint_{S^{\vec{n}}} (P\cos\alpha + Q\cos\beta + R\cos\gamma) d\sigma$ -- связь интегралов I и II рода

    \Nota Формулу интеграла можно получить еще так: $(\vec{F}, \vec{n})d\sigma = \vec{F}\vec{n}d\sigma = \vec{F}d\vec{\sigma}$, где $d\vec{\sigma} = (\pm dydz, \pm dxdz, \pm dxdy)$

    \hypertarget{surfaceintegralofsecondkindmath}{}

    \DefN{Математическое}

    Определим $I = \iint_{S^{\overrightarrow{n}}} f(x, y, z) dxdy$

    $I = \lim_{\substack{n \to \infty \\ \tau = \max \Delta s_k \to 0}} \sum_{k=1}^n f(\xi_k, \eta_k, \zeta_k) \Delta s_k$ - поверхностный интеграл второго рода
    ($\Delta s_k = \Delta x\Delta y$ - любого знака, согласованного с обходом)

    \hypertarget{surfaceintegralofsecondkindproperties}{}

    Свойства: Меняет знак при смене обхода с $\overrightarrow{n}^+$ на $\overrightarrow{n}^-$

    \hypertarget{surfaceintegralofsecondkindcalculation}{}

    \underline{Вычисление}

    1) Параметризация $S$ \quad для $\iint Rdxdy \quad z = z(x, y)$, для $\iint Qdxdz \quad y = y(x, z)$,

    для $\iint Pdydz \quad x = x(y, z)$

    Пределы интегрирования $D_{xy} = \text{пр.}_{Oxy} S$ и т. д.

    2) $dxdy \to \pm dxdy$, если обход $D_{xy}$ в направлении против часовой стрелки ($+dxdy$, если угол между $\overrightarrow{n}$ и $Oz$ острый, иначе $-dxdy$)

    3) $R(x, y, z) = \tilde{R}(x, y, z(x, y)), P(x, y, z) = \tilde{P}(y, z), Q(x, y, z) = \tilde{Q}(x, z)$

    4) $\iint_{S^{\overrightarrow{n}}} f(x, y, z) dxdy = \iint_{D_{xy}} \pm \tilde{P}dydz \pm \tilde{Q}dxdz \pm \tilde{R}dxdy$
    \end{enumerate}

\end{document}




