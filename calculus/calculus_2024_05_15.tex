\documentclass[12pt]{article}
\usepackage{preamble}

\pagestyle{fancy}
\fancyhead[LO,LE]{Математический анализ}
\fancyhead[CO,CE]{15.05.2024}
\fancyhead[RO,RE]{Лекции Далевской О. П.}


\begin{document}

    Разберем пример поверхностного интеграла:

    \Exs $S_1:\ x^2 + y^2 = 1, \quad S_2: z = 0, \quad S_3: z = 1$

    $S = \bigunion_{i = 1}^3 S_i$ -- цилиндр

    $\vec{F} = (P, Q, R) = (x, y, z)$

    $\iint_{S_{\text{внешн.}}} x dy dz + y dx dz + z dx dy = \iint_{S_1} + \iint_{S_2} + \iint_{S_3}$

    Так как проекции $S_2$ и $S_3$ на $Oxz$ и $Oyz$ -- отрезки, то $dxdz = 0$, $dydz = 0$:

    $\iint_{S_2} xdydz + ydxdz + zdxdy = \iint_{S_2} zdxdy = 0$

    $\iint_{S_3} zdxdy \stackrel{z |_{S_3} = 1}{=} \iint_{S_3} dxdy \stackrel{\text{с \enquote{+}, так как }\vec{n_3} \uparrow\uparrow Oz}{=} \iint_{D_{xy}} dxdy = \pi$

    $\iint_{S_1} xdydz + ydxdz = \iint_{D^+_{yz}: x = \sqrt{1 - y^2}} xdydz + \left(-\iint_{D^-_{yz}: x = -\sqrt{1 - y^2}} xdydz\right) + \iint_{D^+_{xz}} ydxdz + \left(-\iint_{D^-_{xz}} ydxdz\right) = \left(\frac{\pi}{2} + \frac{\pi}{2}\right) + \left(\frac{\pi}{2} + \frac{\pi}{2}\right) = 2\pi$

    $\iint_S = 3\pi$


    \subsection{5.7. Связь поверхностных интегралов с другими}

    \hypertarget{theoremGaussOstrogradskyy}{}

    \begin{MyTheorem}
        \ThNs{Гаусса-Остроградского}

        $S_1\ : \ z = z_1(x, y),\ S_3\ :\ z = z_3(x, y),\ S_2\ : \ f(x, y) = 0$ (проекция на $Oxy$ -- кривая)

        $S = \bigunion_{i = 1}^3 S_i$ - замкнута и ограничивает тело $T$ ($S_2$ -- цилиндр, $S_1$ -- шапочка сверху, $S_3$ -- шапочка снизу)

        $P = P(x, y, z), Q = Q(x, y, z), R = R(x, y, z)$ -- непрерывно дифференцируемы, действуют в области $\Omega \supset T$

        Тогда $\oiint_{S_{\text{внешн.}}} Pdydz + Qdxdz + Rdxdy = \iiint_T \left(\frac{\partial P}{\partial x} + \frac{\partial Q}{\partial y} + \frac{\partial R}{\partial z}\right) dxdydz$
    \end{MyTheorem}

    \Mem Формула Грина: $\oint_K Pdx + Qdy = \iint_{D_{xy}} \left(\frac{\partial Q}{\partial x} - \frac{\partial Q}{\partial y}\right) dxdy$

    \begin{MyProof}
        Вычислим почленно $\iiint_T \left(\frac{\partial P}{\partial x} + \frac{\partial Q}{\partial y} + \frac{\partial R}{\partial z}\right) dv$

        $\iiint_T \left(\frac{\partial R(x, y, z)}{\partial z} dz\right) dxdy = \iint_{D_{xy}} R(x, y, z) \Big|_{z = z_1 (x, y)}^{z = z_3(x, y)} dxdy = \iint_{D_{xy}} (R(x, y, z_3(x, y)) - R(x, y, z_1(x, y))) dxdy =
        \underset{\text{двойной}}{\iint_{D_{xy}} R(x, y, z_3) dxdy} - \iint_{D_{xy}} R(x, y, z_1(x, y)) dxdy = \underset{\text{поверхностный}}{\iint_{S_3} R(x, y, z) dxdy} + \iint_{S_1} R(x, y, z) dxdy +
        \underset{\text{равен } 0\text{, т.к. } dxdy\ |_{S_2} = 0}{\iint_{S_2} R(x, y, z) dxdy} = \iint_{S_{\text{внешн.}}} Rdxdy$

        \smallvspace

        Аналогично остальные члены:

        $\iiint_T \frac{\partial Q}{\partial y} dxdydz = \iint_{S_{\text{внешн.}}} Qdxdz, \iiint_T \frac{\partial P}{\partial y} dxdydz = \iint_{S_{\text{внешн.}}} Pdxdz$
    \end{MyProof}

    \Nota Если считаем поток через внутреннюю поверхность, то $\iint_{S_{\text{внутр}}} = - \iiint_T$

    \Nota С учетом связи поверхностных интегралов $\iiint_T \left(\frac{\partial P}{\partial x} + \frac{\partial Q}{\partial y} + \frac{\partial R}{\partial z}\right) dv =
    \iint_S (P\cos\alpha + Q\cos\beta + R\cos\gamma) dv$

    \hypertarget{theoremStokes}{}

    \begin{MyTheorem}
        \ThNs{Стокса}

        Пусть $S : z = z(x, y)$ -- незамкнутая поверхность, $L$ -- контур, на которую она опирается

        $\text{пр}_{Oxy} L = K_{xy}, \quad \text{пр}_{Oxy} S = D_{xy}$

        В области $\Omega \supset S$ действуют функции $P, Q, R$, непрерывно дифференцируемые

        Тогда $\oint_{L^+} Pdx + Qdy + Rdz = \iint_{S^+} \left(\left(\frac{\partial R}{\partial y} - \frac{\partial Q}{\partial z}\right)\cos\alpha +
        \left(\frac{\partial P}{\partial z} - \frac{\partial R}{\partial x}\right)\cos\beta + \left(\frac{\partial Q}{\partial x} - \frac{\partial P}{\partial y}\right)\cos\gamma\right) d\sigma$
    \end{MyTheorem}

    \begin{MyProof}
        Найдем слагаемое $\oint_L P(x, y, z) dx \stackrel{\text{на } L\ : \ z = z(x, y)}{=\joinrel=\joinrel=\joinrel=}
        \oint_{K^+_{xy}} \tilde{P}(x, y, z(x, y)) dx = \oint_{K_{xy}} \tilde{P}dx + \tilde{Q}dy =
        \iint_{D_{xy}} \left(\frac{\partial \tilde{Q}}{\partial x} - \frac{\partial \tilde{P}}{\partial y}\right) dxdy =
        -\iint_{D_{xy}} \frac{\partial \tilde{P}(x, y)}{\partial y} dxdy =
        -\iint_{S^+} \frac{\partial P(x, y, z)}{\partial y} dxdy =
        -\iint_{S^+} \left(\frac{\partial P}{\partial y} + \frac{\partial P}{\partial z} \frac{\partial z}{\partial y}\right) dxdy =
        -\iint_{S^+} \left(\frac{\partial P}{\partial y}\cos\gamma + \frac{\partial P}{\partial z} (-\cos\beta)\right) d\sigma$

        $\vec{n} = \left(\frac{-\frac{\partial z}{\partial x}}{\sqrt{1 + z_x^{\prime 2} + z_y^{\prime 2}}}\right)$

        $\cos\gamma = \frac{1}{\sqrt{1 + z_x^{\prime 2} + z_y^{\prime 2}}}$

        Аналогично $\oint_L Qdy = \iint_{S^+} \left(\frac{\partial Q}{\partial x}\cos\gamma - \frac{\partial Q}{\partial z}\cos\alpha\right) d\sigma,
        \oint_L Rdz = \iint_{S^+} \left(\frac{\partial R}{\partial y}\cos\alpha - \frac{\partial R}{\partial x}\cos\beta\right) d\sigma$

        Остается сложить интегралы
    \end{MyProof}

    \Nota Формула Грина является частным случаем теоремы Стокса при $\cos \alpha = \cos \beta = 0$ и $\cos \gamma = 1$ -- элементарная площадка на плоскости $Oxy$ всегда сонаправлена оси $Oz$

    \ExN{1} Возьмем пример выше: $S_1:\ x^2 + y^2 = 1, \quad S_2: z = 0, \quad S_3: z = 1$, $S = \bigunion_{i = 1}^3 S_i$ -- замкнутый цилиндр, $\vec F = (P, Q, R) = (x, y, z)$. Получаем по теореме Гаусса-Остроградского:

    $\iint_{S_{\text{внешн}}} xdydz + ydxdz + zdxdy = \iiint_T \left(\frac{\partial x}{\partial x} + \frac{\partial y}{\partial y} + \frac{\partial z}{\partial z}\right) dv = 3V_{\text{цил.}} = 3 \cdot 1 \cdot \pi \cdot 1^2 = 3\pi$

    \ExN{2} Те же $P, Q, R$. По теореме Стокса:

    $\oint_L Pdx + Qdy + Rdz = \iint_{S} \left(\overset{= 0}{\overgroup{\left(\frac{\partial z}{\partial y} - \frac{\partial y}{\partial z}\right)}} \cos\alpha + 0 + 0\right) d\sigma = 0$

    \clearpage


    \section{6. Теория поля}

    \subsection{6.1. Определения}

    \hypertarget{scalarfield}{}

    \DefN{1} Дано многомерное пространство $\Omega \subset \Real^n$. Функция $u \ : \ \Omega \to \Real$ называется скалярным полем в $\Omega$

    \hypertarget{vectorfield}{}

    \DefN{2} Функция $\vec{F} = (F_1(\vec{x}), \dots, F_n(\vec{x})) \ : \ \Omega \to \Real^n$ называется векторным полем

    \Nota Далее будем рассматривать функции в $\Real^3$, то есть $u = u(x, y, z)$ и $\vec{F} = (P(x, y, z), Q(x, y, z), R(x, y, z))$

    \Notas Функции $u$ и $\vec{F}$ могут зависеть от времени $t$. Тогда эти поля называются нестационарными. В противном случае стационарными

    \hypertarget{scalarandvectorfieldgeometric}{}

    \subsection{6.2. Геометрические характеристики полей}

    $u = u(x, y, z)$: $l$ -- линии уровня $u = \const$

    $\vec{F} = (P, Q, R)$: $w$ -- векторная линия, в каждой точке $w$ вектор $\vec{F}$ -- касательная к $w$

    \underline{Векторная трубка} -- совокупность непересекающихся векторных линий

    \Nota Отыскание векторных линий

    Возьмем $\vec{\tau}$ -- элементарный касательный вектор, $\vec{\tau} = (dx, dy, dz)$

    Определение векторной линии: $\vec{\tau} || \vec{F} \quad \frac{dx}{P} = \frac{dy}{Q} = \frac{dz}{R}$ -- система ДУ

    \Ex $\vec{F} = y \vec\imath - x \vec\jmath, M_0 (1, 0)$ -- ищем векторную линию $w \ni M_0$

    Задача Коши:

    $\begin{cases}
        \frac{dx}{y} = -\frac{dy}{x} \\ y(1) = 0
    \end{cases} \Longleftrightarrow \begin{cases}
        xdx = -ydy \\ y(1) = 0
    \end{cases} \Longleftrightarrow \begin{cases}
        x^2 = -y^2 + C \\ y(1) = 0 \Longrightarrow C = +1
    \end{cases} \Longleftrightarrow x^2 + y^2 = 1 $

    \hypertarget{differentialcharacteristics}{}

    \subsection{6.3. Дифференциальные характеристики}

    \Mems $\vec\nabla u = \overrightarrow{\operatorname{grad}} \ u = \left(\frac{\partial u}{\partial x}; \frac{\partial u}{\partial y}; \frac{\partial u}{\partial z}\right)$ -- градиент скалярного поля

    $\vec\nabla = \left(\frac{\partial}{\partial x}; \frac{\partial}{\partial y}; \frac{\partial}{\partial z}\right)$ -- набла-оператор

    \Nota Так как $\vec\nabla$ -- это вектор, то для $\vec\nabla$ определены действия:

    \begin{itemize}
        \item $\vec\nabla \cdot \vec{a} = \frac{\partial a_1}{\partial x} + \frac{\partial a_2}{\partial y} + \frac{\partial a_3}{\partial z}$

        \item $\vec\nabla \times \vec{a} =
        \begin{vmatrix}
            \vec\imath & \vec\jmath & \vec k \\
            \frac{\partial}{\partial x} & \frac{\partial}{\partial y} & \frac{\partial}{\partial z} \\
            a_1 & a_2 & a_3
        \end{vmatrix}$
    \end{itemize}

    Причем:
    
    \begin{itemize}
        \item $\vec\nabla \cdot \vec\nabla = \frac{\partial^2}{\partial x^2} + \frac{\partial^2}{\partial y^2} + \frac{\partial^2}{\partial z^2} = \Delta$ -- лапласиан, оператор Лапласа

        \item $\vec\nabla \times \vec\nabla = 
        \begin{vmatrix}
            \vec\imath & \vec\jmath & \vec k \\
            \frac{\partial}{\partial x} & \frac{\partial}{\partial y} & \frac{\partial}{\partial z} \\
            \frac{\partial}{\partial x} & \frac{\partial}{\partial y} & \frac{\partial}{\partial z} \\
        \end{vmatrix} = 0$ -- повторяющиеся строки в определителе
    \end{itemize}
    
    \Nota $\Delta u = \underset{\substack{\text{часть волнового}\\ \text{уравнения матфизики}}}{\frac{\partial^2 u}{\partial x^2} + \frac{\partial^2 u}{\partial y^2} + \frac{\partial^2 u}{\partial z^2}} = 0$ -- уравнение, определяющее гармоническую функцию $u(x, y, z)$, уравнение Лапласа

    \hypertarget{divergence}{}

    \DefN{1} Дивергенцией поля (от \textit{divergence} -- расхождение) называется
    $\Div \vec{F} \stackrel{def}{=} \vec\nabla \cdot \vec{F}$

    \hypertarget{rotor}{}

    \DefNs{2} Вихрем (ротором) поля называется $\Rot \vec{F} \stackrel{def}{=} \vec\nabla \times \vec{F}$

    \hypertarget{vectorfieldtypes}{}

    \DefNs{3} Если $\Rot \vec{F} = 0$, то $\vec{F}$ называется безвихревым полем

    \DefNs{4} Если $\Div \vec{F} = 0$, то $\vec{F}$ называется соленоидальным полем

    \Notas Безвихревое поле имеет незамкнутые векторные линии, а вихревое -- замкнутые

    \hypertarget{irrotationalfieldproperty}{}

    \begin{MyTheorem}
        \ThNs{1} Свойство безвихревого поля: $\Rot \vec{F} = 0 \Longleftrightarrow \exists u(x, y, z) \ | \ \vec\nabla u = \vec{F}$
    \end{MyTheorem}


    $\Box$ \fbox{\Longrightarrow}

    $rot \overrightarrow{F} =
    \begin{vmatrix}
        \overrightarrow{i}          & \overrightarrow{j}          & \overrightarrow{j}          \\
        \frac{\partial}{\partial x} & \frac{\partial}{\partial y} & \frac{\partial}{\partial z} \\
        P & Q & R
    \end{vmatrix} = \left(\frac{\partial R}{\partial y} - \frac{\partial Q}{\partial z}\right)\overrightarrow{i} + \left(\frac{\partial P}{\partial z} - \frac{\partial R}{\partial x}\right)\overrightarrow{j} + \left(\frac{\partial Q}{\partial x} - \frac{\partial P}{\partial y}\right)\overrightarrow{k} = 0$

    $\Longleftrightarrow
    \begin{cases}
        \frac{\partial R}{\partial y} = \frac{\partial Q}{\partial z} \\
        \frac{\partial P}{\partial z} = \frac{\partial R}{\partial x} \\
        \frac{\partial Q}{\partial x} = \frac{\partial P}{\partial y}
    \end{cases}$

    Рассмотрим $u = u(x, y, z) \ | \ \frac{\partial u}{\partial x} = P, \frac{\partial u}{\partial y} = Q, \frac{\partial u}{\partial z} = R$ - удовлетворяет системе равенств

    $\overrightarrow{F} = (P, Q, R) = \left(\frac{\partial u}{\partial x}, \frac{\partial u}{\partial y}, \frac{\partial u}{\partial z}\right) = \overrightarrow{\triangledown} u$

    \fbox{\Longleftarrow} $\overrightarrow{F} = \overrightarrow{\triangledown}u$ - дана

    $rot \overrightarrow{F} = \overrightarrow{\triangledown} \times \overrightarrow{F} = \overrightarrow{\triangledown} \times (\overrightarrow{\triangledown} u) = (\overrightarrow{\triangledown} \times \overrightarrow{\triangledown}) u = 0$

    $\Box$

    \Nota Доказали, что если векторное поле является градиентом какого-то скалярного, то его вихрь равен нулю: $rot \overrightarrow{grad} u = 0$

    \Def $\overrightarrow{F} = \overrightarrow{\triangledown} u \quad$ Поле $u(x, y, z)$ называется потенциалом поля $\overrightarrow{F}$

    Таким образом, доказано, что безвихревое поле потенциально

    \hypertarget{solenoidalfieldproperty}{}

    \ThN{2} Свойство соленоидального поля

    $div (rot \overrightarrow{F}) = 0$

    $\Box$

    $div (rot \overrightarrow{F}) = div \overrightarrow{a} = \overrightarrow{\triangledown} \overrightarrow{a} = \overrightarrow{\triangledown} (\overrightarrow{\triangledown} \times \overrightarrow{F}) = (\overrightarrow{\triangledown} \times \overrightarrow{\triangledown}) \cdot \overrightarrow{F} = 0$

    $\Box$

    \hypertarget{integralcharacteristics}{}

    \subsection{6.4. Интегральные характеристики. Теоремы теории поля}

    \Mems 1) Поток поля $\overrightarrow{F}: \Pi = \iint_S \overrightarrow{F}d\overrightarrow{\sigma}$

    \Def 2) Циркуляция поля $\overrightarrow{F}: \Gamma = \oint_L Pdx + Qdy + Rdz$

    \Nota Запишем \Ths на векторном языке

    \hypertarget{theoremGaussOstrogradskyyinvectorform}{}

    1* \textbf{Гаусса-Остроградского}

    $\iint_S Pdydz + Qdxdz + Rdxdy = \iiint_T \left(\frac{\partial P}{\partial x} + \frac{\partial Q}{\partial y} + \frac{\partial R}{\partial z}\right) dxdydz$

    $\iint_S (P, Q, R) (dydz, dxdz, dxdy) = \iint_S (P, Q, R) (\cos\alpha d\sigma, \cos\beta d\sigma, \cos\gamma d\sigma) =
    \iint_S \overrightarrow{F} \overrightarrow{n} d\sigma = \iint_S \overrightarrow{F} d\overrightarrow{\sigma}$

    $\iiint_T \left(\frac{\partial P}{\partial x} + \frac{\partial Q}{\partial y} + \frac{\partial R}{\partial z}\right) dxdydz = \iiint_T (\overrightarrow{\triangledown} \overrightarrow{F}) = \iiint_T div \overrightarrow{F}$

    \fbox{$\iint_S \overrightarrow{F} d\overrightarrow{\sigma} = \iiint_T div \overrightarrow{F}$}

    \mediumvspace

    \hypertarget{theoremStokesinvectorform}{}

    2* \textbf{Стокса}

    $Pdx + Qdy + Rdz = \overrightarrow{F}d\overrightarrow{l}$

    $\oint_L \overrightarrow{F}d\overrightarrow{l} = \iint_S rot \overrightarrow{F} \overrightarrow{n} d\sigma = \iint_S rot \overrightarrow{F} d\overrightarrow{\sigma}$

    \mediumvspace

    \hypertarget{theoremaboutpotentialinvectorform}{}

    3* \textbf{\Ths о потенциале}

    $\forall L \ \oint_L \overrightarrow{F}d\overrightarrow{l} = 0 \Longleftrightarrow rot \overrightarrow{F} = 0 \Longleftrightarrow \exists u(x, y, z) \ | \ \overrightarrow{\triangledown} u = \overrightarrow{F}$

    (см. \Ths интеграла НЗП)

    \Ex $\overrightarrow{F} = x\overrightarrow{i} + xy \overrightarrow{j}, L: x = y, x = -y, x = 1$

    По формуле Грина (Стокса) $\oint_L \overrightarrow{F} d\overrightarrow{l} = \iint_{D} \left(\frac{\partial Q}{\partial x} - \frac{\partial P}{\partial y}\right) dxdy =
    \iint_D y dxdy \quad rot \overrightarrow{F} \neq 0$

    $\oint_L xdx + xydy = \int_{L_1} + \int_{L_2} + \int_{L_3} = \int_0^1 (x + x^2) dx + \int_{-1}^1 y dy - \int_0^1 (x + x^2) dx = \int_{-1}^1 y dy = 0$

    \subsection{6.5. Механический смысл}

    \hypertarget{divergencemechanicalmeaning}{}

    1* Дивергенция

    Гаусс-Остроградский: $\iiint_T div \overrightarrow{F} dv = \Pi$

    \Ths о среднем: $\exists M_1 \in T \ | \ \iiint_T div \overrightarrow{F} dv = div \overrightarrow{F} \Big|_{M_1} \cdot V_T = \Pi$

    $div \overrightarrow{F} \Big|_{M_1} = \frac{\Pi}{V_T}$, точка $M_0, S$ и $T$ выбраны произвольно

    $\letsymbol V_T \to 0$, тогда $div \overrightarrow{F} \Big|_{M_1 \to M_0} = \lim_{V_T \to 0} \frac{\Pi}{V_T}$ - поток через границу бесконечно малого объема с центром $M_0$, отнесенный к $V_T$ - мощность источника в $M_0$

        Таким образом, дивергенция поля -- мощность источников

        \Notas Смысл утверждения $\Div (\Rot \vec F) = 0$ -- поле вихря свободно от источников

        \Notas Утверждение $\Rot (\overrightarrow{\operatorname{grad}} u) = 0$ -- поле потенциалов свободно от вихрей

    \mediumvspace

    \hypertarget{rotormechanicalmeaning}{}

    2* Ротор

        По \Ths Стокса циркуляция $\Gamma = \iint_S \Rot \vec{F} d\vec{\sigma}$

        По \Ths о среднем существует точка $M_1 \ \Big| \ \iint_S \Rot \vec{F} d\vec{\sigma} = \Rot \vec{F} \Big|_{M_1} \cdot S = \Gamma$

        $\Rot \vec{F} \Big|_{M_1} = \frac{\Gamma}{S}$, будем стягивать поверхность $S$ к точке $M_0$, тогда $\Rot \vec{F} \Big|_{M_0} = \lim_{S \to 0} \frac{\Gamma}{S}$ -- циркуляция по бесконечно малому контуру с центром $M_0$

\end{document}




