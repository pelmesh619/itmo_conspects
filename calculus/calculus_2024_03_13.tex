\documentclass[12pt]{article}
\usepackage{preamble}

\pagestyle{fancy}
\fancyhead[LO,LE]{Математический анализ}
\fancyhead[CO,CE]{13.03.2024}
\fancyhead[RO,RE]{Лекции Далевской О. П.}


\begin{document}
    Для линейной параметризации форма дифференциала сохраняется

    $d^2 z = \left(\frac{\partial }{\partial x}dx + \frac{\partial}{\partial y}dy\right)^2 z \stackrel{\text{инвариант}}{=} z^{(n)}_t dt^n$

    Введем функцию: $z(x(t), y(t)) \stackrel{\text{обозн}}{=} \varphi (t)$ -- она $(n + 1)$ раз дифференцируема (композиция $(n + 1)$ дифференцируемых и линейных функций)

    Заметим, что $x = x_0 + \Delta x t \stackrel{t_0 = 0}{=} x_0$, $y = y_0 + \Delta y t \stackrel{t_0 = 0}{=} y_0$, тогда $M \stackrel{t \to t_0 = 0}{\rightarrow} M_0$

    То есть $z(M_0) = z(x_0, y_0) = z(x(t_0), y(t_0)) = \varphi (t_0) = \varphi(0)$

    Таким образом $\varphi(t)$ как функция одной переменной может быть разложена в окрестности $t_0 = 0$ по формуле Маклорена

    \[\varphi(t) = \varphi(0) + \frac{d\varphi(0)}{1!} \Delta t + \dots + \frac{d^{n}\varphi(0)}{n!} \Delta t^n + o((\Delta t)^n)\]

    Вернемся к $z(x, y)$ ($\Delta t = t - t_0 = 1$):

    \[z(x, y) = z(M) = z(M_0) + \frac{dz(M_0)}{1!} + \frac{d^2 z(M_0)}{2!} + \dots + \frac{d^n z(M_0)}{n!} + r_n(x, y)\]

    где $r_n(x, y) = r_n(t) \stackrel{\text{Лагр.}}{=} \frac{\varphi^{(n+1)}(\theta \Delta t)}{(n + 1)!} \Delta t = \frac{\varphi^{(n+1)}(\theta \Delta t)}{(n + 1)!}$

    $r_n(x, y)$ должен быть б. м. по отношению к $(\Delta \rho)^n$, то есть $r_n(x, y) = o((\Delta \rho)^n)$

    ($r_n(t) \stackrel{n \to \infty}{\rightarrow} 0$, если $\varphi(t)$ нужное число раз дифференцируема $\Rightarrow$ ограничена, $r_n(t)$ -- ограниченная бесконечно малая)

    \Nota В дальнейшем для исследования $z(x, y)$ на экстремум достаточно разложения по формуле Тейлора до 2-ого порядка включительно.
    Покажем сходимость $r_n(x, y) \stackrel{(\Delta \rho)^n \to 0}{\rightarrow} 0$ на примере $\displaystyle r_2 (x, y) = \frac{d^3 z(M_{\text{сред.}})}{3!}$


    $r_2(x, y) = \frac{1}{3!} \left(\frac{\partial}{\partial x} \Delta x + \frac{\partial}{\partial y} \Delta y\right)^3 z =
    \frac{1}{3!} \left(\frac{\partial^3 z}{\partial x^3} (\Delta x)^3 + 3 \frac{\partial^3 z}{\partial x^2 \partial y} (\Delta x)^2 \Delta y +
    3 \frac{\partial^3 z}{\partial x \partial y^2} (\Delta y)^2 \Delta x \frac{\partial^3 z}{\partial y^3} (\Delta y)^3\right)$

    Вообще говоря, значения частных производных берутся в различных средних точках

    $r_2(x, y) = \frac{1}{3!} (z_{xxx}(\mu_1)(\Delta x)^3 + 3 z_{xxy}(\mu_2)(\Delta x)^2 \Delta y + z_{xyy}(\mu_3)(\Delta y)^2 \Delta x + 3 z_{yyy}(\mu_4)(\Delta y)^3) = \Big|$ вынесем $(\Delta \rho)^3$

    $= \frac{(\Delta \rho)^3}{3!} \left(\text{огран.} \cdot \frac{(\Delta x)^3}{(\Delta \rho)^3} + \text{огран.} \cdot \frac{(\Delta x)^2 \Delta y}{(\Delta \rho)^3} + \text{огран.} \cdot \frac{(\Delta y)^2 \Delta x}{(\Delta \rho)^3} + \text{огран.} \cdot \frac{(\Delta y)^3}{(\Delta \rho)^3}\right)$

    $\frac{(\Delta x)^3}{(\Delta \rho)^3} = \frac{(\Delta x)^3}{\sqrt{(\Delta x)^2 + (\Delta y)^2}^3} \stackrel{\Delta x \to 0}{\rightarrow} 0$, то есть дробь и выражение выше ограничены

    $\frac{r_2(x, y)}{(\Delta \rho)^2} = \frac{1}{3!} \frac{(\Delta \rho)^3 \cdot \text{огр.}}{(\Delta \rho)^2} = \frac{1}{3!} \Delta \rho \cdot \text{огр.} \stackrel{\Delta \rho \to 0}{\rightarrow} 0$

    \subsection{4.7. Геометрия ФНП}


    \subsubsection{4.7.1. Линии и поверхности уровня}

    Положим $z = \const$.

    В сечении плоскостью $z = c$ образуется кривая $l$ с уравнением $\begin{cases}z = c \\ \varphi(x, y) = 0 \leftarrow \text{уравнение $l_\text{проек}$ на $Oxy$}\end{cases}$

    Кривая $l$ с уравнением $z(x, y) = c$ называется линией уровня Ф$_2$П $z = z(x, y)$

    \Def Поверхность уровня $\mathcal{P}$ -- это поверхность с уровнем $u(x, y, z) = c$

    Физ. смысл: Пусть $u : \Real^3 \rightarrow \Real$ (значения функции $u(x, y, z)$ - скаляры).
    Тогда говорят, что в $\Real^3$ задано скалярное поле.
    Например, поле температур, давления, плотности и т. д.

    Тогда $u = c$ -- поверхности постоянных температур, давления и т. п. (изотермические, изобарные, эквипотенциальные)

    \Ex Конус: $z = -\sqrt{x^2 + y^2}$

    \includegraphics[height=90mm]{calculus/images/calculus_2024_03_13_1}

    Линии уровня $z = c$:

    \begin{enumerate}
        \item $c > 0 \quad \emptyset$
        \item $c = 0 \quad x = y = 0 - $ точка $(0, 0)$
        \item $c < 0 \quad -|c| = -\sqrt{x^2 + y^2} \quad c^2 = x^2 + y^2$
    \end{enumerate}


    \subsubsection{4.7.2. Производная по направлению, градиент}

    \underline{Задача}. Дано скалярное поле $u = u(x, y, z)$ (напр. давления). Как меняется давление при перемещении в заданном направлении?

    Это задача о нахождении скорости изменения $u(x, y, z)$ в заданном направлении $\vec{s}$

    Из $M_0(x_0, y_0, z_0)$ движемся в $M(x, y, z)$ в направлении $\vec{s}$, $x = x_0 + \Delta x$, $y = y_0 + \Delta y$, $z = z_0 + \Delta z$

    $\Delta s = \sqrt{(\Delta x)^2 + (\Delta y)^2 + (\Delta z)^2} \quad \Big| \cdot \frac{1}{\Delta s}$

    $1 = \sqrt{\left(\frac{\Delta x}{\Delta s}\right)^2 + \left(\frac{\Delta y}{\Delta s}\right)^2 + \left(\frac{\Delta z}{\Delta s}\right)^2}$

    $(\frac{\Delta x}{\Delta s}, \frac{\Delta y}{\Delta s}, \frac{\Delta z}{\Delta s}) = (\cos\alpha, \cos\beta, \cos\gamma) = \overrightarrow{s^0}$

    Потребуем, чтобы $u(x, y, z)$ имела непрерывность $u_x, u_y, u_z$ в $D$

    То есть $u(x, y, z)$ дифференцируема и

    $\Delta u = du + o(\Delta s) = u_x \Delta x + u_y \Delta y + u_z \Delta x + o(\Delta s) \quad \Big| \cdot \frac{1}{\Delta s}$

    $\frac{\Delta u}{\Delta s} = u_x \cos\alpha + u_y \cos\beta + u_z \cos\gamma + \frac{o(\Delta s)}{\Delta s}$ - предельный переход

    $\frac{\partial u}{\partial s} = \frac{\partial u}{\partial x} \cos\alpha + \frac{\partial u}{\partial y} \cos\beta + \frac{\partial u}{\partial z} \cos\gamma$

    \Nota Изначально $\Delta u = du + \text{(б. м.)} \Delta x + \text{(б. м.)} \Delta y + \text{(б. м.)} \Delta z \quad \Big| \cdot \frac{1}{\Delta s}$

    $\frac{\Delta u}{\Delta s} = \frac{du}{\Delta s} + \text{(б. м.)} \cos\alpha$, $\text{(б. м.)} \cos\alpha \rightarrow 0$

    \hypertarget{derivativeoffunctionindirection}{}

    \Def Производной функции $u = u(x, y, z)$ в направлении $\vec{s}$ называют величину $\frac{\partial u}{\partial s} = \frac{\partial u}{\partial x} \cos\alpha + \frac{\partial u}{\partial y} \cos\beta + \frac{\partial u}{\partial z} \cos\gamma$, где $\alpha, \beta, \gamma$ - направления $\vec{s}$

    \Nota Производная в определении -- число, но $\frac{\partial u}{\partial s} \vec{s^0}$ -- вектор скорости

    \Nota Заметим, что если $\vec \imath, \vec \jmath, \vec{k}$ -- декартовы орты, то
    $\frac{\partial u}{\partial i} = \frac{\partial u}{\partial x} 1 + \frac{\partial u}{\partial y} 0 + \frac{\partial u}{\partial z} 0 = \frac{\partial u}{\partial x}$

    И аналогично в других направлениях: $\frac{\partial u}{\partial j} = \frac{\partial u}{\partial y}, \frac{\partial u}{\partial k} = \frac{\partial u}{\partial z}$

    Составим вектор $\frac{\partial u}{\partial x} \vec \imath + \frac{\partial u}{\partial y} \vec \jmath + \frac{\partial u}{\partial z} \vec k \stackrel{\text{обозн}}{=} \vec\nabla u$

    \hypertarget{gradientdefinition}{}

    $\vec\nabla$ -- набла-оператор (оператор Гамильтона); $\vec\nabla = \left(\frac{\partial}{\partial x}; \frac{\partial}{\partial y}; \frac{\partial}{\partial z}\right)$ -- условный вектор

    \Def $\overrightarrow{\operatorname{grad}} \ u \stackrel{def}{=} \vec\nabla u$ -- называют градиентом функции $u(x, y, z)$

    \hypertarget{gradientproperties}{}

    Свойства градиентов:

    \begin{MyTheorem}
        \ThNs{1} $\frac{\partial u}{\partial s} = \text{проек.}_{\vec{s}} \vec\nabla u$
    \end{MyTheorem}

    \begin{MyTheorem}
        \ThNs{2} $\vec\nabla u$ -- направление наибольшего значения $\frac{\partial u}{\partial s}$
    \end{MyTheorem}

    \begin{MyTheorem}
        \ThNs{3} $\vec{s} \perp \vec\nabla u \Longrightarrow \frac{\partial u}{\partial s} = 0$
    \end{MyTheorem}

    \begin{MyTheorem}
        \ThNs{4} $u = u(x, y), u = c$ -- линии уровня $l$. Тогда $\vec\nabla u \perp l$
    \end{MyTheorem}

    Доказательства:

    \begin{enumerate}
        \item \begin{MyProof}
            $\frac{\partial u}{\partial s} = \left(\left(\frac{\partial}{\partial x}; \frac{\partial}{\partial y}; \frac{\partial}{\partial z}\right) \cdot \vec{s^0}\right) u = \left(\frac{\partial}{\partial x}; \frac{\partial}{\partial y}; \frac{\partial}{\partial z}\right) u \cdot \vec{s^0} =
            \vec\nabla u \cdot \vec{s^0}$
            
            $|\vec\nabla u \cdot \vec{s^0}| = |\vec\nabla u| |\vec{s^0}| \cos(\widehat{\vec\nabla u, \vec{s^0}}) =
            |\vec\nabla u| \cos(\widehat{\vec\nabla u, \vec{s^0}}) = \text{пр.}_{\vec{s}} \vec\nabla u$
        \end{MyProof}

        \item \begin{MyProof}
            $\frac{\partial u}{\partial s} = |\vec\nabla u| \cos\varphi$, где $\varphi$ - угол между $\vec{s}$ и $\vec\nabla u$

            Косинус принимает наибольшее значение, если угол между $\vec{s}$ и $\vec\nabla u$ равен нулю, то есть направления векторов совпадает. Значит, при $\vec{s} = \vec\nabla u$ производная принимает наибольшее значение
        \end{MyProof}

        \item \begin{MyProof}
            Из доказательства \ThNs{2} следует, что если $\vec{s}$ сонаправлен с $\vec\nabla u$, то производная принимает наибольшее значение. Следовательно, если $\vec{s} \perp \vec\nabla s$, то $\cos\varphi = 0$, $\frac{\partial u}{\partial s} = 0$
        \end{MyProof}

        \item \begin{MyProof}
            $u = c$ -- уравнение $l_{\text{пр}}$ в плоскости $Oxy$, то есть $u(x, y) = c$ мы можем рассмотреть как неявную функцию $u(x, y(x)) - c = 0$

            Производная неявной функции: $\frac{dy}{dx} = -\frac{u_x}{u_y} = k_l$ -- угловой коэффициент касательной к $l$

            $\vec\nabla u = (u_x, u_y) \quad \frac{u_y}{u_x} = k_{\text{град.}}$ -- наклон вектора градиента.

            Очевидно $k_l \cdot k_{\text{град.}} = -1 \Longrightarrow \vec\nabla u \perp l$
        \end{MyProof}
    \end{enumerate}



\end{document}
