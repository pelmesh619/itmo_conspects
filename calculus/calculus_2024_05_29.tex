\documentclass[12pt]{article}
\usepackage{preamble}

\pagestyle{fancy}
\fancyhead[LO,LE]{Математический анализ}
\fancyhead[CO,CE]{29.05.2024}
\fancyhead[RO,RE]{Лекции Далевской О. П.}


\begin{document}
    Поток $\Pi$ и циркуляцию $\Gamma$ называют интегральными характеристиками поля, тогда как дивергенцию $\Div \vec F$ и ротор $\Rot \vec F$ -- дифференциальными

    \Nota Ранее выяснили, что смысл
    
    \begin{itemize}
        \item потока $\Pi = \oiint_S \vec F d\vec\sigma$ -- количество пройденной жидкости через поверхность за единицу времени;
        \item дивергенции $\Div \vec F \Big_{M_0} = \lim_{V \to 0} \frac{\Pi}{V}$ -- мощность точечного источника (сколько жидкости он \enquote{производит} или \enquote{потребляет})
        \item теоремы Гаусса-Остроградского: поток через замкнутую поверхность равен суммарной мощности источников внутри
    \end{itemize}

    Выясним смысл ротора и циркуляции на примере конкретного поля

    \Ex $\vec{F} = -\omega y \vec\imath + \omega x \vec\jmath$ -- поле линейных скоростей вращающегося твердого тела, где $\vec{\omega} = const$ -- угловая скорость

    Выберем контур $L$, ограничивающий область $S$

    Найдем $\Gamma_L = \oint_L \vec{F}d\vec{l} = \oint_L (-\omega y) dx + \omega x dy \stackrel{\Ths \text{Стокса}}{=}
    \iint_S \Rot \vec{F} \vec{n} d\sigma = \iint_S \left(\frac{\partial Q}{\partial x} - \frac{\partial P}{\partial y}\right)\cos \gamma d\sigma =
    \iint_S 2\omega \cos \gamma d\sigma$

    Так как ротор сонаправлен оси $Oz$, получаем $\cos \gamma = 1$

    $\iint_S 2\omega \cos \gamma d\sigma = 2\omega \iint_S d\sigma = 2\omega S$

    Раньше в интеграле видно, что $\Rot \vec{F} \vec{n} \Longrightarrow |\Rot \vec{F}| = 2\omega$

    \hypertarget{rotormechanicalmeaning2}{}

    То есть механический смысл ротора -- удвоенная угловая скорость вращающегося тела (или диска)

    \Nota Чтобы уточнить смысл $\Gamma$, рассмотрим такое же поле жидких скоростей (водоворот)
    $\vec{v} = -\omega y \vec\imath + \omega x \vec\jmath$ и погруженное в него колесо с лопатками (водяная мельница)

    В качестве контура $L$ берем обод колеса, а его располагаем под углом $\gamma$ к вектору $\vec{\omega}$

    Все равно $\Gamma_L = \iint_S 2\omega \cos \gamma d\sigma = 2\omega\cos \gamma S$

    Если $\gamma = 0$ (мельница расположена в плоскости водоворота), то $\Gamma_L = 2\omega S$ -- максимальная мощность вращения нашей мельницы

    Если, например, $\gamma = \frac{\pi}{2}$ (мельница расположена перпендикулярно водовороту), то $\Gamma_L = 0$ - колесо перпендикулярно полю, поэтому оно не вращается

    \section{6.6. Приложения к физике}

    1* Уравнение неразрывности (в гидромеханике)

    \Nota Здесь потребуются формулы:

    $\frac{du(x(t), y(t), z(t))}{dt} = \frac{\partial u}{\partial t} + \frac{\partial u}{\partial x}\frac{dx}{dt} + \dots$

    $\overrightarrow{\triangledown} \cdot (f\overrightarrow{F}) = \overrightarrow{\triangledown} f \cdot \overrightarrow{F} + f \cdot (\overrightarrow{\triangledown} \overrightarrow{F})$, где $f$ - скалярное поле, $\overrightarrow{F}$ - векторное поле

    \underline{Задача} Дано $\overrightarrow{F} = \rho \overrightarrow{v}$ - поле скоростей жидкости с весом $\rho = \rho(x, y, z, t)$

    Через площадку $dS$ за время $dt$ протекает $d\Pi = \rho v_n dt dS$ или за ед. времени $d\Pi = \rho v_n dS$

    Приращение жидкости за единицу времени $|dm| = |\frac{\partial \rho}{\partial t} dV|$

    Поток жидкости равен ее убыли в объеме $V$, то есть $\Pi = \oiint_S \rho v_n dS = -\iiint_V \frac{\partial \rho}{\partial t} dV$

    Применяя Г.-О.: $\Pi = \iiint_V \div (\rho \overrightarrow{v}) dV = -\iiint_V \frac{\partial \rho}{\partial t} dV \Longleftrightarrow \iiint_V (div (\rho \overrightarrow{v}) + \frac{\partial \rho}{\partial t}) dV = 0 \quad \forall V$ (поэтому подынт. функ. = 0)

    $\Longleftrightarrow \overrightarrow{\triangledown} (\rho \overrightarrow{v}) + \frac{\partial \rho}{\partial t} = 0$

    Учтем: $\frac{d \rho}{dt} = \frac{\partial \rho}{\partial t} +  \frac{\partial \rho}{\partial x} \frac{dx}{dt} +  \frac{\partial \rho}{\partial y} \frac{dy}{dt} + \frac{\partial \rho}{\partial z} \frac{dz}{dt} = \frac{\partial \rho}{\partial t} + \overrightarrow{\triangledown} \rho \overrightarrow{v}$

    % TODO

    $\overrightarrow{\triangledown} (\rho \overrightarrow{v}) = \overrightarrow{\triangledown} \rho \cdot \overrightarrow{v} + \rho \overrightarrow{\triangledown} \overrightarrow{v} \Longleftrightarrow \overrightarrow{\triangledown} \rho \overrightarrow{v} = \overrightarrow{\triangledown} (\rho \overrightarrow{v}) - \rho \overrightarrow{\triangledown} \overrightarrow{v}$

    $\frac{d\rho}{dt} + \rho div \overrightarrow{v} = 0$ - уравнение неразрывности (при несжимаемой жидкости $div \overrightarrow{v} = 0$)

    2* Уравнения Максвелла

    Экспериментально: 1) $\int_L \overrightarrow{H} d\overrightarrow{l} = \iint_S \overrightarrow{r}d\overrightarrow{\sigma}$ - закон Био-Савара

    2) $\int_L \overrightarrow{E} d\overrightarrow{l} = -\frac{\partial}{\partial t} \iint_S \overrightarrow{B}d\overrightarrow{\sigma}$ - закон Фарадея

    где $\overrightarrow{H}$ - магнитная сила, $\overrightarrow{r}$ - полный ток, $\overrightarrow{E}$ - электрическая сила, $\overrightarrow{B}$ - магнитная индукция

    Максвелл: $\overrightarrow{r}$ = ток проводимости + ток смещения = $\lambda \overrightarrow{E} + \varepsilon \frac{\partial \overrightarrow{E}}{\partial t}$

    $\lambda$ - коэффициент проводимости, $\varepsilon, \mu$ - проницаемость

    1) $\oint_L \overrightarrow{H} d\overrightarrow{l} = \iint_S (\lambda \overrightarrow{E} + \varepsilon \frac{\partial \overrightarrow{E}}{\partial t}) d\overrightarrow{\sigma}$

    Стокс: $\iint_S rot \overrightarrow{H} d\overrightarrow{\sigma} - \iint_S (\lambda \overrightarrow{E} + \varepsilon \frac{\partial \overrightarrow{E}}{\partial t}) d\overrightarrow{\sigma} = 0$

    В векторной форме: $rot \overrightarrow{H} = (\lambda \overrightarrow{E} + \varepsilon \frac{\partial \overrightarrow{E}}{\partial t})$ - источники магнитного поля - токи проводимости и смещения

    2) $\oint_L \overrightarrow{E}d\overrightarrow{l} = \iint_S rot \overrightarrow{E}d\overrightarrow{\sigma} = -\iint_S \frac{\partial \overrightarrow{B}}{\partial t}d\overrightarrow{\sigma} \Longleftrightarrow
    rot \overrightarrow{E} = -\frac{\partial \overrightarrow{B}}{\partial t}$ - изменение индукции дает эл. ток в соленоиде.

    3) $\overrightarrow{\triangledown} \varepsilon \overrightarrow{E} = \rho$

    4) $\overrightarrow{\triangledown} \mu \overrightarrow{H} = 0$

\end{document}




