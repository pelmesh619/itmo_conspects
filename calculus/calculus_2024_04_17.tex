\documentclass[12pt]{article}
\usepackage{preamble}

\pagestyle{fancy}
\fancyhead[LO,LE]{Математический анализ}
\fancyhead[CO,CE]{17.04.2024}
\fancyhead[RO,RE]{Лекции Далевской О. П.}


\begin{document}
    \Nota В строгом определении интегральная сумма строится так:

    $\overset{\frown}{M_{i-1}M_i}$ -- элементарная дуга

    $\Delta l_i$ -- длина элемента

    $\Delta s_i$ -- длина стягивающей дуги

    $\Delta l_i \approx \Delta s_i$

    $M_{\text{ср.}}(\xi_i, \eta_i)$ -- средняя точка элемента

    \[\sigma_n = \sum_{i = 1}^n f(\xi_i, \eta_i) \Delta s_i\]

    \hypertarget{curvilinearintegralofsecondkind}{}

    Определим криволинейный интеграл II рода. Задача (вычисление работы силы вдоль пути): вдоль пути $\overset{\frown}{AB}$ действует сила $\overrightarrow{F} = (P(x, y), Q(x, y))$. Найдем элементарную работу $dA = \vec{F}_{\text{ср.}} d\vec{s}$, где $d\vec{s}$ -- элементарное приращение

    $d\vec{s} = (dx, dy) = (\cos\alpha ds, \sin\alpha ds)$

    $\vec{F}_{\text{ср.}}$ -- значение силы на элементарном участке в какой-либо его точке

    Тогда $dA = (P(x, y), Q(x, y)) \cdot (dx, dy) = P(x, y)dx + Q(x, y)dy$, а по всей кривой $A = \int_{AB} dA = \int_{AB} Pdx + Qdy$ -- интеграл II рода (в проекциях)

    \Nota В проекциях, потому что $F_x = P, F_y = Q$, таким образом скалярное произведение записано в проекциях

    При этом часто рассматривают по отдельности: $\int_{AB} f(x, y) dx$ и $\int_{AB} g(x, y) dy$

    \hypertarget{connectionbetweencurvilinearintegrals}{}

    \Nota Связь интегралов I и II рода:

    $\int_L Pdx + Qdy = \int_L (P, Q)(dx, dy) = \int_L (P, Q) (\cos\alpha, \cos\beta) \underset{\approx dl}{\undergroup{ds}} =
    \int_L (P\cos\alpha + Q\cos\beta)dl$

    Обозначим $\vec{\tau} = (\cos\alpha, \cos\beta)$

    По теореме Лагранжа $\exists (\xi, \eta) \in$ элементарной дуге, касательная которой параллельна $ds$

    Тогда $d\vec{s} = \vec{\tau}ds \approx \vec{\tau}dl$, где $\vec{\tau}$ -- единичный вектор, касательной в $(\xi, \eta)$

    Тогда $\int_L Pdx + Qdy \stackrel{\text{пред. в вект. форме}}{=\joinrel=} \int_L \overrightarrow{F}\overrightarrow{\tau} dl =
    \int_L \overrightarrow{F}\underset{\text{ориент. эл. дуги}}{\undergroup{\overrightarrow{dl}}}$

    \hypertarget{curvilinearintegraloffirstkindproperties}{}

    \mediumvspace

    \underline{Свойства}:

    \Notas Свойства, не зависящие от прохода дуги, аналогичны свойствам определенного интеграла

    \begin{itemize}
        \item Направление обхода:

        \begin{multicols}{2}
            I рода:

            $\int_{AB} f(x, y)dl = \int_{BA} f(x, y)dl$

            II рода:

            $\int_{AB}Pdx + Qdy = -\int_{BA}Pdx + Qdy$
        \end{multicols}
    \end{itemize}

    \Def Часто рассматривают замкнутую дугу, называемую контур. Тогда интегралы обозначаются так: $\oint_K f dl$ и $\oint_K Pdx + Qdy$.

    Если $K$ (контур) обходят против часовой стрелки, то обозначают $\oint_{K^+}$, иначе $\oint_{K^-}$

    \hypertarget{curvilinearintegraloffirstkindcalculation}{}

    Вычисление сводится к $\int_a^b dx$ или $\int_\alpha^\beta dy$ или $\int_\tau^T dt$

    \begin{enumerate}
        \item Параметризация дуги $L$:

        $\begin{cases}
            x = \varphi(t) \\
            y = \psi(t)
        \end{cases} \varphi, \psi \in C^1_{[\tau, T]} \quad\quad \begin{matrix}
            A(x_A, y_A) = (\varphi(\tau), \psi(\tau)) \\
            B(x_B, y_B) = (\varphi(T), \psi(T))
        \end{matrix}$

        При этом задании $L \quad y = y(x), x \in [a, b]$ или $x = x(y), y \in [\alpha, \beta]$ -- частные случаи параметризации

        \item \begin{multicols}{2}
            I рода:

            $\int_{L} f(x, y) dl = \left[dl = \sqrt{\varphi_t^{\prime 2} + \psi_t^{\prime 2}}|dt|\right] = \\
            = \int_\tau^T f(t) \sqrt{\varphi_t^{\prime 2} + \psi_t^{\prime 2}}|dt|$

            II рода:

            $\int_{L = \overset{\frown}{AB}}Pdx + Qdy = [dx = \varphi_t^\prime dt, dy = \psi_t^\prime dt] = \\
            = \int_\tau^T (P\varphi^\prime + Q\psi^\prime)dt$

        \end{multicols}
    \end{enumerate}

    \Ex Дуга $L$ -- отрезок прямой от $A(1, 1)$ до $B(3, 5)$. Вычислим $\int_{AB} (x + y) dl$ двумя способами:

    \begin{enumerate}
        \item $\int_{AB} (x + y) dl = \begin{bmatrix}AB: \frac{x - 1}{2} = \frac{y - 1}{4} \\
        \text{или } y = 2x - 1, x \in [1, 3] \\
        f(x, y) = x + 2x - 1 = 3x - 1 \\
        dl = \sqrt{1 + y^{\prime 2}}dx = \sqrt{5}dx\end{bmatrix} =
        \int_1^3 (3x - 1) \sqrt{5}dx = \sqrt{5} \left(\frac{3x^2}{2} - x\right) \Big|_1^3 = \sqrt{5}(12 - 2) = 10\sqrt{5}$

        \item $\int_{AB} (x + y) dx + (x + y) dy = \begin{bmatrix}x \uparrow^3_1, y \uparrow^5_1 \\
        y = 2x - 1, x = \frac{y + 1}{2} \\
        dx = dx, dy = dy\end{bmatrix} = \int_1^3 (x + 2x - 1) dx + \int^5_1 \left(\frac{y + 1}{2} + y\right) dy = \\
        \left(\frac{3x^2}{2} - x\right) \Big|_1^3 + \frac{1}{2} \left(\frac{3y^2}{2} + y\right) \Big|_1^5 = 10 + 20 = 30$
    \end{enumerate}

    \hypertarget{formulaGreen}{}

    \begin{MyTheorem}
        \Ths Формула Грина

        Пусть дана область $D \subset \Real^2$, которая обходится в правильном направлении ($\uparrow Ox, \uparrow Oy$)

        $K$ -- гладкая замкнутая кривая (контур), которая ограничивает $D$

        В области $D$ действует $\vec{F} = (P(x, y), Q(x, y))$ -- непрерывные дифференциалы

        Тогда $\iint_D \left(\frac{\partial Q}{\partial x} - \frac{\partial P}{\partial y}\right) dxdy = \oint_{K^+} Pdx + Qdy$
    \end{MyTheorem}

    \begin{MyProof}
        $\iint_D \left(\frac{\partial Q}{\partial x} - \frac{\partial P}{\partial y}\right) dxdy =
        \iint_D \frac{\partial Q}{\partial x} dxdy - \iint_D \frac{\partial P}{\partial y} dxdy =
        \int^\beta_\alpha dy \int_{x=x_1(y)}^{x=x_2(y)} \frac{\partial Q}{\partial x} dx -
        \int^b_a dx \int_{y=y_1(x)}^{y=y_2(x)} \frac{\partial P}{\partial y} dy =
        \int^\beta_\alpha \left(Q(x, y)\Big|^{x=x_2(y)}_{x=x_1(y)}\right)dy - \int^b_a \left(P(x, y)\Big|^{y=y_2(x)}_{y=y_1(x)}\right)dx = \\
        \int_\alpha^\beta (Q(x_2(y), y) - Q(x_1(y), y)) dy - \int_a^b (P(x, y_2(x)) - P(x, y_1(x))) dx =
        \int_{NST}Q dy - \int_{NMT} Q dy - \int_{MTS} P dx + \int_{MNS}Pdx =
        \underset{\oint_{K^+}Qdy}{\undergroup{\int_{NST} Qdy + \int_{TMN} Qdy}} +
        \underset{\oint_{K^+}Pdx}{\undergroup{\int_{STM} Qdy + \int_{MNS} Qdy}} =
        \oint_{K^+} Pdx + Qdy$
    \end{MyProof}
\end{document}



