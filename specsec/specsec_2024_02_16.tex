\documentclass[12pt]{article}
\usepackage{preamble}

\pagestyle{fancy}
\fancyhead[LO,LE]{Специальные разделы \\ высшей математики}
\fancyhead[CO,CE]{16.02.2024}
\fancyhead[RO,RE]{Лекции Далевской О. П.}


\begin{document}
    \hypertarget{orthogonalbasisinspace}{}

    \begin{MyTheorem}
        \Ths Во всяком $E^n$ можно выделить ортонормированный базис
    \end{MyTheorem}

    \begin{MyProof}
        В $E^n_{\|\cdot\|} \ \exists B = \{\beta_1, \dots, \beta_n\}$ - базис

        Покажем, что можно выделить ортонормированный базис $\mathcal{E} = \{e_1, \dots, e_n\}$ при помощи метода
        математической индукции

        База: построим один ортогональный вектор для $\beta_1 = e_1^\prime$ (потом $e_1 = \frac{e_1^\prime}{\|e_1^\prime\|}$)

        Рассмотрим $e_2^\prime = \beta_1 - \lambda e^\prime_1$. Требуем $e_2^\prime \perp e_1^\prime$, то есть $(e_1^\prime, e_2^\prime) = 0$

        Отсюда найдем нужный $\lambda: (e_1^\prime, e_2^\prime) = (e_1^\prime, \beta_2 - \lambda e_1^\prime) = (e_1^\prime, \beta_2) - \lambda (e_1^\prime, e_1^\prime) = 0$

        Тогда $\lambda = \frac{(e_1^\prime, \beta_2)}{(e_1^\prime, e_1^\prime)}$

        Переход: Пусть построена система ортогональных векторов $\{e_1^\prime, e_2^\prime, \dots, e_k^\prime\}$

        Построим $k + 1$ систему:

        Рассмотрим $e_{k+1}^\prime = \beta_{k + 1} - \lambda_k e_k^\prime - \lambda_{k - 1}^\prime e_{k - 1}^\prime - \dots - \lambda_1 e_1^\prime \quad (*)$

        Требуем $e_{k+1}^\prime \perp e_i \quad \forall i \in [1;k]$

        $(e_{k+1}^\prime, e_k^\prime) = (\beta_{k + 1}, e_k^\prime) - \lambda_k (e_k^\prime, e_k^\prime) = 0$, так как $(e_i^\prime, e_j^\prime) = 0 \quad i \neq j$

        $\lambda_k = \frac{(\beta_{k + 1}, e_k^\prime)}{(e_k^\prime, e_k^\prime)}$

        Аналогично: $(e_{k+1}^\prime, e_{k - 1}^\prime) = (\beta_{k+1}, e_{k - 1}^\prime) - \lambda_{k - 1}(e_{k - 1}^\prime, e_{k-1}^\prime)$

        $\lambda_{k - 1} = \frac{(\beta_{k + 1}, e_{k - 1}^\prime)}{(e_{k - 1}^\prime, e_{k - 1}^\prime)}$

        Получаем $e_{k+1}^\prime = \beta_{k + 1} - \sum_{i = 1}^k \frac{(\beta_{k + 1}, e_{i}^\prime)}{(e_{i}^\prime, e_{i}^\prime)}$
    \end{MyProof}

    Изложенный метод называется методом ортогонализации базиса, при этом $(*)$ определяет ненулевой вектор, 
    иначе получим нулевую тривиальную линейную комбинацию векторов $\beta_i$ ($e_i$ выражается через них), но это невозможно, 
    так как вектора базисные.
    При этом полученную систему стоит нормировать

    \Ex Формула скалярного произведения в ортонормированном базисе

    $E_{\|\cdot\|}, B = \{\beta_1, \dots, \beta_n\}$ - какой-либо базис

    Рассмотрим $x = x_1 \beta_1 + x_2 \beta_2 + \dots + x_n \beta_n$ и $y = y_1 \beta_1 + \dots + y_n \beta_n$

    Найдем $(x, y)$, как произведение компонент: $(x_1 \beta_1 + \dots + x_n \beta_n, y_1 \beta_1 + \dots + y_n \beta_n) = \sum_{i = 1}^n \sum_{j = 1}^n x_i y_j (\beta_i, \beta_j)$

    Обозначим $(\beta_i, \beta_j) = a_{ij} \in \Real$

    Таким образом, $(x, y) = \sum_i \sum_j a_{ij} x_i y_j$ - дальше назовем квадратичной формой

    Ранее (в аналитической геометрии) $(a, b) = \sum_{i = 1}^n a_i b_i$ - произведение координат векторов $\vec{a}, \vec{b}$ в
    декартовой прямоугольной системе координат (с ортонормированным базисом)

    Действительно: если $\beta_i = e_i$, $\beta_j = e_j$, вектора $e_i, e_j$ принадлежат ортонормированному базису, 
    а $a_{ij} = \begin{cases}1, i = j \\ 0, i \neq j\end{cases}$, то $(x, y) = \sum_{i = 1}^n x_i y_i$

    Причем $x = x_1 e_1 + \dots + x_n e_n \Longrightarrow x_i = (x, e_i)$

    \Ex Система функций, непрерывных на $[0, 2\pi]$

    $\Phi = \{1, \sin t, \cos t, \sin 2t, \dots, \sin nt, \cos nt\}$

    Система ортогональна (\Lab), но не нормированная (\Lab)

    $\Phi_{\|\cdot\|} = \{\frac{1}{\sqrt{2\pi}}, \frac{1}{\sqrt{\pi}}\sin t, \frac{1}{\sqrt{\pi}} \cos t, \dots\}$ - нормированная система

    Тогда функция, определенная и непрерывная на $[0, 2\pi]$ может быть разложена по базису $\Phi_{\|\cdot\|}$
    и ее координат (как вектора): $f_i = \int_0^{2\pi} f \cdot e_i dx$, где $e_i \in \Phi_{\|\cdot\|}$

\end{document}
