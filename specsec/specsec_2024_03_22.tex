\documentclass[12pt]{article}
\usepackage{preamble}

\pagestyle{fancy}
\fancyhead[LO,LE]{Специальные разделы \\ высшей математики}
\fancyhead[CO,CE]{22.03.2024}
\fancyhead[RO,RE]{Лекции Далевской О. П.}


\begin{document}
    \Notas $\Ker \mathcal{A}$ и $\IM \mathcal{A}$ - подпространства $V$ ($\mathcal{A} : V \rightarrow V$)

    В общем случае $\Ker \mathcal{A} \subset V, \IM \mathcal{A} \subset W \ (\mathcal{A} : V \rightarrow W)$

    \mediumvspace

    Заметим, что если $\Ker \mathcal{A} = \texttt{0}$, то $\mathcal{A}$ - взаимно-однозначен

    \begin{MyProof}
        Докажем от противного:

        $\sqsupset \mathcal{A}$ - не взаимно-однозначен, то есть 
        $\exists x_1, x_2 \in V (x_1 \neq x_2) \ | \ \mathcal{A}x_1 = \mathcal{A}x_2 \Longleftrightarrow 
        \mathcal{A} (x_1 - x_2) = \texttt{0} \Longrightarrow x_1 - x_2 \in \Ker \mathcal{A}$ - 
        противоречие, так как $\Ker \mathcal{A} = \texttt{0}$
    \end{MyProof}

    \Notas Обратное также верно:

    \begin{MyProof}
        $\mathcal{A}$ - взаимно-однозначен $\Longleftrightarrow y_1 = y_2 \Longrightarrow x_1 = x_2$

        Докажем от противного: $\dim \Ker \mathcal{A} \neq 0$, значит найдется $x = x_1 - x_2 \in \Ker \mathcal{A}$ $(x_1 \neq x_2)$, 
        причем по определению ядра $\mathcal{A} x = \mathcal{A} (x_1 - x_2) = \texttt{0}$

        А так как $\mathcal{A} (x_1 - x_2) = \texttt{0}$, то $\mathcal{A} x_1 = \mathcal{A} x_2 \Longrightarrow x_1 = x_2$ - противоречие
    \end{MyProof}

    \Nota Также очевидно, что

    $\Ker \mathcal{A} = 0 \Longleftrightarrow \IM \mathcal{A} = V$

    $\Ker \mathcal{A} = V \Longrightarrow \IM \mathcal{A} = \texttt{0}$ и $\mathcal{A} = 0$

    \hypertarget{theoremaboutdimensions}{}

    \begin{MyTheorem}
        \Ths $\mathcal{A} : V \rightarrow V$, тогда $\dim \Ker \mathcal{A} + \dim \IM \mathcal{A} = \dim V$
    \end{MyTheorem}

    \begin{MyProof}
        Так как $\Ker \mathcal{A}$ - подпространство $V$, то можно построить дополнение до прямой суммы (взяв базисные векторы ядра, дополнить их набор до базиса $V$: $e^k_1, \dots e^k_m, e^k_{m+1}, \dots e^k_n$)

        Обозначим дополнение $W$, тогда $Ker \mathcal{A} \xor W = V \Longrightarrow \dim \Ker \mathcal{A} + \dim W = \dim V$

        Докажем, что $W$ и $\IM \mathcal{A}$ - изоморфны

        $\mathcal{A} : W \rightarrow \IM \mathcal{A}$

        $\mathcal{A} : Ker \mathcal{A} \rightarrow \texttt{0}$

        Докажем, что $\mathcal{A}$ действует из $W$ в $\IM \mathcal{A}$ взаимно-однозначно

        $\sqsupset \mathcal{A}$ не взаимно-однозначный, тогда $\exists x_1, x_2 \in W (x_1 \neq x_2) \ | \ \mathcal{A}x_1 = \mathcal{A}x_2 \in \IM \mathcal{A}$

        Из этого $\mathcal{A}(x_1 - x_2) = \texttt{0} \Longrightarrow x_1 - x_2 \stackrel{\text{обозн.}}{=} x \in \Ker \mathcal{A}$, причем $x \neq 0$, так как $x_1 \neq x_2$

        Но так как $W$ - дополнение до прямой суммы ($Ker \mathcal{A} \xor W = V$, то есть $W \union \Ker \mathcal{A} = \texttt{0}$), 
        а $x \in W \union \Ker \mathcal{A}$ - противоречие ($x \neq 0$)

        Из этого следует, что $\mathcal{A}$ - линейный и взаимно-однозначный $\Longrightarrow \dim W = \dim \IM \mathcal{A}$

        Получается, что $V$ можно представить как прямую сумму $W_1 \xor W_2$, причем 
        $W_1 = \Ker \mathcal{A}, W_2 = \IM \mathcal{A}$
    \end{MyProof}

    \Def Рангом оператора $\mathcal{A}$ называется $\dim \IM \mathcal{A}$: $\rang \mathcal{A} \stackrel{def}{=} \dim \IM \mathcal{A}$ 
    (также обозначается $\operatorname{r}(\mathcal{A})$ или $\operatorname{rank} \mathcal{A}$)

    \Nota Сравним ранг оператора с рангом его матрицы

    $\mathcal{A} x = y \quad \mathcal{A} : V^n \rightarrow W^m$

    $A$ - матрица $\mathcal{A}, x = x_1 e_1 + x_2 e_2 + \dots + x_n e_n, y = y_1 f_1 + \dots + y_m f_m$

    $\mathcal{A}x = y \Longleftrightarrow \begin{pmatrix}
         a_{11} & \dots & a_{1n} \\
         \vdots & \ddots & \vdots \\
         a_{m1} & \dots & a_{mn}
    \end{pmatrix} \begin{pmatrix}
         x_1 \\
         \vdots \\
         x_n
    \end{pmatrix} = \begin{pmatrix}
         y_1 \\
         \vdots \\
         y_m
    \end{pmatrix}$

    Или при преобразовании базиса $Ae_i = e^\prime_i$:

    $\begin{pmatrix}
         a_{11} & \dots & a_{1n} \\
         \vdots & \ddots & \vdots \\
         a_{m1} & \dots & a_{mn}
    \end{pmatrix} \begin{pmatrix}
         e_1 \\
         \vdots \\
         e_n
    \end{pmatrix}^T = \begin{pmatrix}
         e_1^\prime \\
         \vdots \\
         e_m^\prime
    \end{pmatrix}$

    Здесь $\begin{pmatrix}
         e_1 \\
         \vdots \\
         e_n
    \end{pmatrix}^T$ - это матрица $\begin{pmatrix}
         e_1 & \dots & e_n
    \end{pmatrix} = \begin{pmatrix}
         e_{11} & e_{12} & \dots \\
         \vdots & \vdots & \vdots \\
         e_{n1} & e_{n2} & \dots
    \end{pmatrix}$

    \Nota Поиск матрицы $\mathcal{A}$ можно осуществить, найдя ее в \enquote{домашнем} базисе $\{e_i\}$, то есть $A (e_1, \dots, e_n) = (e_1^\prime, \dots, e_m^\prime)$

    Затем, можно найти матрицу в другом (нужном) базисе, используя формулы преобразований (см. \Ths позже)

    Тогда $\Ker \mathcal{A} = K$ - множество векторов, которые решают систему

    $AX = \texttt{0} \quad (\dim K = m = \dim \text{ФСР} = n - \rang A)$ и при этом $\dim K = n - \dim \IM \mathcal{A}$

    $\rang \mathcal{A} = \rang A = \dim \IM \mathcal{A}$

    Следствия (без доказательств):

    \begin{enumerate}
        \item $\rang(\mathcal{AB}) \leq \rang(\mathcal{A})$ (или $\rang \mathcal{B}$)

        \item $\rang(\mathcal{AB}) \geq \rang(\mathcal{A}) + \rang(\mathcal{B}) - \dim V$
    \end{enumerate}

    \Nota Рассмотрим преобразование координат, как линейный оператор $T : V^n \rightarrow V^n$ (переход из системы $Ox_i \rightarrow Ox_i^\prime$, $i = 1..n$)

    $\dim \IM T = n, \dim \Ker T = 0 \Longrightarrow T$ - взаимно-однозначен

    Поставим задачу отыскания матрицы в другом базисе, используя $T_{e \to e^\prime}$

    \subsection[p2\_6]{2.6. Преобразование матрицы оператора при переходе к другому базису}

    \hypertarget{transformationtodifferentbasis}{}

    \begin{MyTheorem}
        \Ths $\mathcal{A} : V^n \rightarrow V^n$

        $\{e_i\} \stackrel{\text{об}}{=} e$ и $\{e^\prime_i\} \stackrel{\text{об}}{=} e^\prime$ - базисы пространства $V$

        $\mathcal{T} : V^n \rightarrow V^n$ - преобразование координат, то есть $Te_i = e^\prime_i$

        $\sqsupset A, A^\prime$ - матрицы $\mathcal{A}$ в базисах $e$ и $e^\prime$

        Тогда $A^\prime = TAT^{-1}$ ($A^\prime_{e^\prime} = T_{e\to e^\prime}AT^{-1}_{e\to e^\prime}$)
    \end{MyTheorem}

    \begin{MyProof}
        Пусть $y = \mathcal{A}x$, где $x, y$ - векторы в базисе $e$ ($x_e = x^\prime_{e^\prime}$ - один вектор)

        $y^\prime = \mathcal{A} x^\prime$, где $x^\prime, y^\prime$ - векторы в базисе $e^\prime$

        $\mathcal{T}x = x^\prime, \mathcal{T}y = y^\prime$

        $y = Ax$, $y^\prime = A^\prime x^\prime$, тогда $Ty = A^\prime (Tx) \quad \Big| \cdot T^{-1}$

        $T^{-1}Ty = (T^{-1}A^\prime T)x$
        
        $Ax = y = (T^{-1}A^\prime T)x$

        $A = T^{-1}A^\prime T \Longrightarrow A^\prime = TA T^{-1}$
    \end{MyProof}

\end{document}
