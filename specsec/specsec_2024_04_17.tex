\documentclass[12pt]{article}
\usepackage{preamble}

\pagestyle{fancy}
\fancyhead[LO,LE]{Специальные разделы \\ высшей математики}
\fancyhead[CO,CE]{17.04.2024}
\fancyhead[RO,RE]{Лекции Далевской О. П.}


\begin{document}
    \Nota Особое значение имеют симметричные билинейные формы. 
    Если рассмотреть матрицы симметричную билинейную форму как матрицу самосопряженного оператора, то можно найти базис
    (ортонормированный базис собственных векторов), в котором матрица билинейной формы диагонализируется

    Этот базис называется каноническим базисом билинейной формы

    \subsection[p3\_2]{3.2. Квадратичные формы}

    \hypertarget{quadraticform}{}

    \Def Квадратичной формой называется форма $\mathcal{B}(u, u)$, порожденная билинейной формой $\mathcal{B}(u, v)$

    \Ex Поверхность: $u = (x, y), v = (x, y, z)$

    $\mathcal{B}(u, u) = b_{11}u_1 u_1 + b_{12} u_1 u_2 + b_{21} u_2 u_1 + b_{22} u_2 u_2 = b_{11} x^2 + b_{12}xy + b_{21}xy + b_{22}y^2$

    \begin{tabular}{@{\hspace{-0.4em}}r@{\hspace{0.2em}}l}
        \setlength{\tabcolsep}{0pt} $\mathcal{B}(v, v)$ & $= \beta_{11} x^2 + \beta_{12}xy + \beta_{13}xz + \beta_{21} xy + \beta_{22}y^2 + \beta_{23}yz + \beta_{31} xz + \beta_{32}yz + \beta_{33}z^2$ \\
        & $= \beta_{11} x^2 + \beta_{22} y^2 + \beta_{33} z^2 + (\beta_{12} + \beta_{21}) xy + (\beta_{23} + \beta_{32}) yz + (\beta_{13} + \beta_{31}) xz$
    \end{tabular}

    \Mems Ранее уравнение поверхности второго порядка (без линейной группы, то есть сдвига): $a_{11}x^2 + 2a_{12}xy + a_{22}y^2 + 2a_{23}yz + 2a_{13}xz + a_{33}z^2 = c$

    \Nota Заметим, что здесь коэффициент $a_{ij}$ соответствуют матрице симметричной билинейной форме:

    $B(v, v) = \begin{pmatrix}a_{11} & a_{12} & a_{13} \\ a_{12} & a_{22} & a_{23} \\ a_{13} & a_{23} & a_{33}\end{pmatrix}$

    Если диагонализировать $B(v, v)$, то уравнение поверхности приводится к каноническому виду:

    $\mathcal{B}(v, v)_{\text{канон.}} = c_{11}x^2 + c_{22}y^2 + c_{33}z^2$

    Поэтому квадратичная форма, соответствующая поверхности второго порядка, рассматривается, как форма, порожденная симметричной билинейной формой

    \hypertarget{positivedefinedoperator}{}

    \Def Положительно определенная форма

    \begin{enumerate}[label*=\arabic*) ]
        \item Оператор $\mathcal{A}$ называется положительно определенным, если $\exists \gamma > 0 \ | \ \forall x \in V \quad (\mathcal{A}x, x) \geq \gamma \|x\|^2$

        \item $\mathcal{A}$ называется положительным, если
        $\forall x \in V, \ x \neq 0 \quad (\mathcal{A}x, x) > 0$
    \end{enumerate}

    \Nota Можно говорить о положительно определенном операторе $\mathcal{A}: V^n \rightarrow V^n$

    \begin{MyTheorem}
        \Ths Определения 1), 2) $\Longleftrightarrow \forall \lambda_i$ - собственное число $\mathcal{A}$, $\lambda_i > 0$
    \end{MyTheorem}

    \begin{MyProof}
        \fbox{$\Longrightarrow$} $\lambda_i$ - собственное число, $e_i$ - соответствующий ему собственный вектор

        $\forall x \in V \quad x = \overset{n}{\underset{i = 1}{\sum}} c_i e_i$

        $(\mathcal{A}x, x) = \left(\overset{n}{\underset{i = 1}{\sum}} c_i \overset{\lambda_i e_i}{\overgroup{\mathcal{A}e_i}}, \overset{n}{\underset{i = 1}{\sum}} c_i e_i\right) =
        \overset{n}{\underset{i = 1}{\sum}} \lambda_i c_i^2 \geq \overset{n}{\underset{i = 1}{\sum}}\lambda_{\min} c_i^2 =
        \lambda_{\min} \overset{n}{\underset{i = 1}{\sum}}c_i^2 = \lambda_{\min} \|x\|^2$

        Если $0 < \lambda_{\min} < \lambda_i, \lambda_i \neq \lambda_{\min}$, то $(\mathcal{A}x, x) > 0$

        \fbox{$\Longleftarrow$} 1) $\Longleftrightarrow \exists \gamma > 0 \ | \ (\mathcal{A}x, x) \geq \gamma \|x\|^2 \quad \forall x \in V$ в том числе $x = e_i \neq 0$

        $(\mathcal{A}e_i, e_i) = \lambda_i (e_i, e_i) = \lambda_i > 0 \ \forall i$

    \end{MyProof}

    \Notas $\det A$ инвариантен при замене базиса, $\det A = \lambda_1 \cdot \dots \cdot \lambda_n > 0$. Тогда $\exists \mathcal{A}^{-1}$

    \hypertarget{criterionSilvester}{}

    \begin{MyTheorem}
    \ThNs{Критерий Сильвестра}

    $\mathcal{A}: V^n \to V^n$ - положительно определен \Longleftrightarrow $\forall k = 1..n \ \Delta_k =
    \begin{vmatrix}a_{11} & \dots & a_{1k} \\ \vdots & \ddots & \vdots \\ a_{k1} & \dots & a_{kk}\end{vmatrix} > 0$ 

    Угловой минор матрицы положительно определенного оператора больше нуля

    \end{MyTheorem}

    \begin{MyProof}

        \fbox{$\Longrightarrow$} $\mathcal{A}$ - положительно определен, значит, $\mathcal{A}$ диагонализируется в базисе $\Set{e_1, \dots, e_n}$ собственных векторов.
        Тогда, $\mathcal{A}$ диагонализируется в базисе $\Set{e_1, \dots, e_k}$, $k \leq n$

        $A_k = \begin{pmatrix}a_{11} & \dots & a_{1k} \\ \vdots & \ddots & \vdots \\ a_{k1} & \dots & a_{kk}\end{pmatrix} \quad
        \Delta_k = \det A_k \stackrel{inv}{=} \begin{vmatrix}\lambda_{1} & \dots & 0 \\ \vdots & \ddots & \vdots \\ 0 & \dots & \lambda_{k}\end{vmatrix} > 0$

        \fbox{$\Longleftarrow$} Метод математической индукции

        $\forall k = 1..n, \Delta_k > 0$, тогда:

        \begin{enumerate}
            \item База: для $k = 1 \quad \mathcal{A}$ - положительно определен

            $\mathcal{A}x = a_{11}x \quad |a_{11}| > 0 \Longrightarrow \mathcal{A}$ - положительно определен

            \item Шаг индукции: $\mathcal{A}_{n-1}$ - положительно определен \Longrightarrow $\mathcal{A}_n$ - положительно определен

            $\mathcal{A}$ диагонализируется в базисе ${e_i}$, в этом базисе:
            
            $\mathcal{A}_e x =
            \begin{vmatrix}\lambda_{1} & \dots & 0 \\ \vdots & \ddots & \vdots \\ 0 & \dots & \lambda_{n}\end{vmatrix}x =
            \overset{n - 1}{\underset{i = 1}{\sum}}\lambda_i c_i e_i + \lambda_n c_n e_n \quad$ Для $i \leq n - 1$ все $\lambda_i > 0$

            $(\mathcal{A}x, x) = \left(\overset{n - 1}{\underset{i = 1}{\sum}} \lambda_i c_i e_i + \lambda_n c_n e_n,
            \overset{n - 1}{\underset{i = 1}{\sum}} c_i e_i\right) = \overset{> 0}{\overgroup{\overset{n - 1}{\underset{i = 1}{\sum}} \lambda_i c_i^2}} + \lambda_n c_n^2$ - знак зависит от $\lambda_n$
        \end{enumerate}

        $\Delta_n = \underset{> 0}{\undergroup{\lambda_1 \cdot \dots \cdot \lambda_{n-1}}} \cdot \lambda_n
        \Longrightarrow \lambda_n > 0 \Longrightarrow (\mathcal{A}x, x) > 0$

    \end{MyProof}

    \Ex Поверхность: $x^2 + y^2 + z^2 = 1$

    $\mathcal{B}(u, u) = \begin{pmatrix}1 & \dots & 0 \\ \vdots & \ddots & \vdots \\ 0 & \dots & 1\end{pmatrix},
    \quad \Delta_k = 1 > 0 \ \forall k$

    Положительная определенность - наличие экстремума\footnote{Точнее положительная определенность матрицы Гессе $\left(\frac{\partial^2 f}{\partial x_i \partial x_j}\right)_{i,j}$ в критической точке, в которой $\triangledown f = 0$, является достаточным условием для наличия в этой точке строгого локального минимума функции}

    \Def Оператор $\mathcal{A}$ называется отрицательно определенным, если $-\mathcal{A}$ - положительно определенный

    \Notas Для $-\mathcal{A}$ работает критерий Сильвестра: $\Delta_k(-\mathcal{A}) =
    \begin{vmatrix}-a_{11} & \dots & -a_{1n} \\ \vdots & \ddots & \vdots \\ -a_{n1} & \dots & -a_{nn}\end{vmatrix} = (-1)^k \Delta_k (\mathcal{A}) > 0$

    Таким образом, $\mathcal{A}$ - отрицательно определен $\Longleftrightarrow \Delta_k$ чередует знаки

    \Nota Аналогично операторам определяются положительно или отрицательно билинейные формы

    $\mathcal{B}(u, v) = \overset{n}{\underset{j = 1}{\sum}}\overset{n}{\underset{i = 1}{\sum}} b_{ij} u_i v_j = \sum_{j = 1}^n v_j \sum_{i = 1}^n b_{ij} u_i = (\mathcal{A} u, v)$

    Так как $\mathcal{B}(u, v)$ и  $\mathcal{B}(u, u)$ - числа, то $\mathcal{B}$ называется положительно определенным, если $\mathcal{B}(u, v) > 0$

    \Nota После приведения $\mathcal{B}(u, v)$ к каноническому виду, получаем $\mathcal{B}(u, u)_{\text{канон.}} = \lambda_1 x_1^2 + \dots + \lambda_n x_n^2$

    В общем случае $\lambda_i$ любого знака, но можно доказать, что количества $\lambda_i > 0, \lambda_j < 0, \lambda_k = 0$ постоянны по отношению к способу приведения
    к каноническому виду (так называемый закон инерции квадратичной формы)

\end{document}

