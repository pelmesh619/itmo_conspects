\documentclass[12pt]{article}
\usepackage{preamble}

\pagestyle{fancy}
\fancyhead[LO,LE]{Специальные разделы \\ высшей математики}
\fancyhead[CO,CE]{17.04.2024}
\fancyhead[RO,RE]{Лекции Далевской О. П.}


\begin{document}
    Nota. Особое значение имеют симметричные билинейные формы

    Если рассмотреть матрицы симм. Б. Ф. как матрицу самосопряженного оператора, то можно найти базис
    (ортонормированный базис собственных векторов), в котором матрица Б. Ф. диагонализируется

    Этот базис называется каноническим базисом билинейной формы

    Def. Квадратичной формой, порожденной Б. Ф. $\mathcal{B}(u, v)$, называется форма $\mathcal{B}(u, u)$

    Ex. Поверхность

    $u = (x, y), v = (x, y, z)$

    $\mathcal{B}(u, u) = b_{11}u_1 u_1 + b_{12} u_1 u_2 + b_{21} u_2 u_1 + b_{22} u_2 u_2 = b_{11} x^2 + b_{12}xy + b_{21}xy + b_{22}y^2$

    $\mathcal{B}(v, v) = \beta_{11} x^2 + \beta_{12}xy + \beta_{13}xz + \beta_{21} xy + \beta_{22}y^2 + \beta_{23}yz + \beta_{31} xz + \beta_{32}yz + \beta_{33}z^2$

    Mem. Ранее уравнение поверхности второго порядка (без линейной группы, то есть сдвига)

    \[a_{11}x^2 + 2a_{12}xy + a_{22}y^2 + 2a_{23}yz + 2a_{13}xz + a_{33}z^2 = c\]

    Nota. Заметим, что здесь коэфф. $a_{ij}$ соответствуют матрице симметричной Б. Ф.:

    $B(v, v) = \begin{pmatrix}a_{11} & a_{12} & a_{13} \\ a_{12} & a_{22} & a_{23} \\ a_{13} & a_{23} & a_{33}\end{pmatrix}$

    Если диагонализировать $B(v, v)$, то приведем уравнение поверхности к каноническому виду:

    $\mathcal{B}(v, v)_{\text{канон.}} = c_{11}x^2 + c_{22}y^2 + c_{33}z^2$

    Поэтому квадратичная форма, соответствующая поверхности второго порядка, рассматривается, как форма, порожденная симметричной билинейной формой

    Def. Положительно определенная форма

    Nota. Можно говорить о положительно определенном операторе $\mathcal{A}: V^n \rightarrow V^n$

    1) Оператор $\mathcal{A}$ называется положительно определенным, если

    $\exists \gamma > 0 \ | \ \forall x \in V \quad (\mathcal{A}x, x) \geq \gamma \|x\|^2$

    2) $\mathcal{A}$ называется положительным, если

    $\forall x \in V, \ x \neq 0 \quad (\mathcal{A}x, x) > 0$

    Th. 1), 2) \Longleftrightarrow $\ \forall \lambda_i$ - с. число $\mathcal{A}$, $\lambda_i > 0$

    $\Box \Longrightarrow \quad \lambda_i$ - с. число, $e_i$ - соответствующий им с. вектора

    $\forall x \in V \quad x = \overset{n}{\underset{i = 1}{\Sigma}} c_i e_i$

    $(\mathcal{A}x, x) = (\overset{n}{\underset{i = 1}{\Sigma}} c_i \overset{\lambda_i e_i}{\overgroup{\mathcal{A}e_i}}, \overset{n}{\underset{i = 1}{\Sigma}} c_i e_i) =
    \overset{n}{\underset{i = 1}{\Sigma}} \lambda_i c_i^2 \geq \overset{n}{\underset{i = 1}{\Sigma}}\lambda_{\min} c_i^2 =
    \lambda_{\min} \overset{n}{\underset{i = 1}{\Sigma}}c_i^2 = \lambda_{\min} \|x\|^2$

    Если $0 < \lambda_{\min} < \lambda_i, \lambda_i \neq \lambda_{\min}$, то $(\mathcal{A}x, x) > 0$

    \Longleftarrow \quad 1) \Longleftrightarrow $\exists \gamma > 0 \ | \ (\mathcal{A}x, x) \geq \gamma \|x\|^2 \quad \forall x \in V$ в том числе $x = e_i \neq 0$

    $(\mathcal{A}e_i, e_i) = \lambda_i (e_i, e_i) = \lambda_i > 0 \ \forall i$

    $\Box$

    Nota. $\det A$ инвариантен при замене базиса, $\det A = \lambda_1 \cdot \dots \cdot \lambda_n > 0$. Тогда $\exists \mathcal{A}^{-1}$

    Th. Критерий Сильвестра

    $\mathcal{A}: V^n \to V^n$ - положительно определен \Longleftrightarrow $\forall k = 1..n \ \Delta_k =
    \begin{vmatrix}a_{11} & \dots & a_{1n} \\ \vdots & \ddots & \vdots \\ a_{n1} & \dots & a_{nn}\end{vmatrix}$

    $\Box \Longrightarrow \quad \mathcal{A}$ - пол. опред.

    $\mathcal{A}$ диагонализируется в базисе $\Set{e_1, \dots, e_n}$ собственных векторов.
    Тогда, $\mathcal{A}$ диагонализируется в базисе $\Set{e_1, \dots, e_k}$, $k \leq n$

    $A_k = \begin{pmatrix}a_{11} & \dots & a_{1k} \\ \vdots & \ddots & \vdots \\ a_{k1} & \dots & a_{kk}\end{pmatrix} \quad
    \Delta_k = \det A_k \stackrel{inv}{=} \begin{vmatrix}\lambda_{1} & \dots & 0 \\ & \vdots & \ddots & \vdots \\ 0 & \dots & \lambda_{k}\end{vmatrix} > 0$

    $\Longleftarrow$ ММИ

    $\forall k = 1..n, \Delta_k > 0$

    1) Для $k = 1 \quad \mathcal{A}$ - пол. опр.

    2) $\mathcal{A}_{n-1}$ - пол. опр. \Longrightarrow $\mathcal{A}_n$ - пол. опр.

    1) $\mathcal{A}x = a_{11}x \quad |a_{11}| > 0 \Longrightarrow \mathcal{A}$ - пол. опр.

    2) $\mathcal{A}$ диагон. \quad $\mathcal{A}_e x =
    \begin{vmatrix}\lambda_{1} & \dots & 0 \\ & \vdots & \ddots & \vdots \\ 0 & \dots & \lambda_{n}\end{vmatrix}x =
    \overset{n - 1}{\underset{i = 1}{\Sigma}}\lambda_i c_i e_i + \lambda_n c_n e_n \quad$ Для $i \leq n - 1$ все $\lambda_i > 0$

    $(\mathcal{A}x, x) = (\overset{n - 1}{\underset{i = 1}{\Sigma}} \lambda_i c_i e_i + \lambda_n c_n e_n,
    \overset{n - 1}{\underset{i = 1}{\Sigma}} c_i e_i) = \overset{> 0}{\overgroup{\overset{n - 1}{\underset{i = 1}{\Sigma}} \lambda_i c_i^2}} + \lambda_n c_n^2$ - знак зависит от $\lambda_n$

    $\Delta_n = \underset{> 0}{\undergroup{\lambda_1 \cdot \dots \cdot \lambda_{n-1}}} \cdot \lambda_n
    \Longrightarrow \lambda_n > 0 \Longrightarrow (\mathcal{A}x, x) > 0$

    $\Box$

    Ex. Поверхность: $x^2 + y^2 + z^2 = 1$

    $\mathcal{B}(u, u) = \begin{pmatrix}1 & \dots & 0 \\ & \vdots & \ddots & \vdots \\ 0 & \dots & 1\end{pmatrix},
    \quad \Delta_k = 1 > 0 \ \forall k$

    Положительная определенность - наличие экстремума

    Def. Оператор $\mathcal{A}$ называется отрицательно определенным, если $-\mathcal{A}$ - положительно определенный

    Nota. Для $-\mathcal{A}$ работает критерий Сильвестра: $\Delta_k(-\mathcal{A}) =
    \begin{vmatrix}-a_{11} & \dots & -a_{1n} \\ \vdots & \ddots & \vdots \\ -a_{n1} & \dots & -a_{nn}\end{vmatrix} = (-1)^k \Delta_k (\mathcal{A}) > 0$

    Таким образом, $\mathcal{A}$ - отриц. опред. $\Longleftrightarrow \Delta_k$ чередует знаки

    Nota. Аналогично операторы определяются положительно или отрицательно билинейные формы

    $\mathcal{B}(u, v) = \overset{n}{\underset{j = 1}{\Sigma}}\overset{n}{\underset{i = 1}{\Sigma}} b_{ij} u_i v_j \stackrel{?}{=} \dots$ через оператор

    Так как $\mathcal{B}(u, v)$ и  $\mathcal{B}(u, u)$ - числа, то $\mathcal{B}$ - называется пол. опред., если $\mathcal{B}(u, v) > 0$

    Nota. После приведения $\mathcal{B}(u, v)$ к каноническому виду, получаем

    $\mathcal{B}(u, u)_{\text{канон.}} = \lambda_1 x_1^2 + \dots + \lambda_n x_n^2$

    В общем случае $\lambda_i$ любого знака

    Но можно доказать, что количества $\lambda_i > 0, \lambda_j < 0, \lambda_k = 0$ постоянны по отношению к способу приведения
    к каноническому виду (т. н. закон инерции квадратичной формы)

\end{document}

