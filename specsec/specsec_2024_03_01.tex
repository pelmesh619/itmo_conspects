\documentclass[12pt]{article}
\usepackage{preamble}

\pagestyle{fancy}
\fancyhead[LO,LE]{Специальные разделы \\ высшей математики}
\fancyhead[CO,CE]{01.03.2024}
\fancyhead[RO,RE]{Лекции Далевской О. П.}


\begin{document}
    \Nota Изоморфизм $E^n \rightarrow E^{\prime n}$ позволяет переносить свойства скалярного произведения
    из одного в другое пространство

    \Ex $\|x + y\| \leq \|x\| + \|y\|$ - арифметические векторы со скалярным произведением $(x, y) = \sum^n_{i = 1} x_i y_i$

    $E^{\prime n} \in C_{[a;b]}$ со скалярным произведением $(f, g) = \int^b_a f \cdot g dx$

    $\sqrt{\int^b_a (f \cdot g)^2 dx} \leq \sqrt{\int^b_a f^2 dx} + \sqrt{\int^b_a g^2 dx}$

    \hypertarget{perpendicularproblem}{}

    \subsection{1.4. Задача о перпендикуляре}

    Постановка: Нужно опустить перпендикуляр из точки пространства $E^n$ на подпространство $G$

    \includegraphics[height=90mm]{specsec/images/specsec_2024_03_01_1}

    Точка $M$ - конец вектора $x$ в пространстве $E^n$.
    Нужно найти $M_0$ (конец вектора $x_0$, проекции $x$ на $G$), причем $x_0 + h = x$,
    где $h \perp G$. Правда ли что, длина перпендикулярного вектора $h$ - минимальная длина от точки $M$ до $G$?

    \begin{MyTheorem}
        \Ths $h \perp G, x_0 \in G, x = x_0 + h$. Тогда $\forall x^\prime \in G (x^\prime \neq x_0) \ \ \|x - x^\prime\| > \|x - x_0\|$
    \end{MyTheorem}

    \begin{MyProof}
        $\|x - x^\prime\| = \|x - x_0 + x_0 - x^\prime\| \stackrel{\text{по теореме Пифагора}}{=\joinrel=\joinrel=\joinrel=} \|x - x_0\| + \|x_0 - x^\prime\| = \|h\| + \|x_0 - x^\prime\| > \|x - x_0\|$
    \end{MyProof}

    \Nota $x_0$ называется ортогональной проекцией, возникает вопрос о ее вычислении (так находятся основания перпендикуляров)

    \mediumvspace

    \textit{Алгоритм:} представим $x_0 = \lambda_1 e_1 + \lambda_2 e_2 + \dots + \lambda_k e_k$, $\{e_i\}^k_{i=1}$ - базис $G$ (необязательно ортонормированный)

    Дан вектор $x$, пространство $G$, нужно найти $\lambda_i$

    $h = x - x_0$, $h \perp G \quad (h, e_i) = 0$, так как $h \perp e_i \ \forall i$

    $(x - x_0, e_i) = (x, e_i) - (x_0, e_i) = 0 \Longrightarrow (x, e_i) = (x_0, e_i)$

    Тогда $\forall i \ (x_0, e_i) = (\lambda_1 e_1 + \dots + \lambda_k e_k, e_i) = \lambda_1 (e_1, e_i) + \dots + \lambda_k (e_k, e_i)$. 
    Здесь $(e_k, e_i)$ - числа, а $\lambda_i$ - неизвестные переменные. Из этого получаем СЛАУ:

    $\begin{pmatrix}
    (e_1, e_1) & (e_1, e_2) & \ldots & (e_1, e_k)\\
    \ldots & \ldots & \ldots & \ldots\\
    (e_k, e_1) & (e_k, e_2) & \ldots & (e_k, e_k)\\
    \end{pmatrix} \times 
    \begin{pmatrix}
    \lambda_1\\
    \ldots\\
    \lambda_k \\
    \end{pmatrix} = \Gamma \times \begin{pmatrix}
    \lambda_1\\
    \ldots\\
    \lambda_k \\
    \end{pmatrix} = \begin{pmatrix}
    (x,e_1)\\
    \ldots\\
    (x,e_k) \\
    \end{pmatrix}$

    \Nota В матрице $\Gamma$ нет нулевых строк, так как $e_i$ - вектор базиса и $e_i^2 \neq 0$

    Таким образом по теореме Крамера $\exists! (\lambda_1, \dots, \lambda_k)$

    \hypertarget{grammatrix}{}

    \Def Матрицу $\Gamma = \{(e_i, e_j)\}_{i, j = 1\dots k}$ называют матрицей Грама

    \smallvspace

    В простейшем случае, $\Gamma = I = \begin{pmatrix}
    1 & 0 & \ldots\\
    0 & 1 & \ldots\\
    \ldots & \ldots & 1\\
    \end{pmatrix}$, если базис ортонормированный

    \smallvspace

    Далее, $I$ - единичная матрица Грама

    \Nota Тогда $I \times \begin{pmatrix}
    \lambda_1\\
    \ldots\\
    \lambda_k \\
    \end{pmatrix} = \begin{pmatrix}
    \lambda_1\\
    \ldots\\
    \lambda_k \\
    \end{pmatrix} = \begin{pmatrix}
    (x,e_1)\\
    \ldots\\
    (x,e_k) \\
    \end{pmatrix}$

    \subsubsection{Приложения задачи о перпендикуляре}

    \begin{enumerate}
        \item Метод наименьших квадратов

        В качестве простейшей модели зависимости $y = y(x)$ берем линейную функцию $y = \lambda x$

        Ищем минимально отстоящую прямую от данных $(x_i, y_i)$, то есть ищем $\lambda$

        Определим расстояние (в этом методе) как $\sigma^2 = \sum^n_{i=1} (y_i - y_{0i})^2 = \sum^n_{i=1} (y_i - \lambda x_i)^2$ - наша 
        задача состоит в минимизации этой величины\footnote{Эта величина также известна как \textit{дисперсия}}

        Таким образом, ищем $y_0$ (ортогональная проекция) такой, что $(y - y_0)^2 = \sigma^2$ минимальна. 
        Найдем производную функции $\sigma^2(\lambda)$:

        $\left(\sigma^2(\lambda)\right)^\prime = \sum^n_{i = 1} (2\lambda x_i^2 - 2 x_i y_i) = 0 \Longrightarrow 
        \sum^n_{i = 1} \lambda x_i^2 = \sum^n_{i = 1} x_i y_i$

        Отсюда получаем $\lambda = \frac{\sum_{i = 1}^n x_i y_i}{\sum_{i = 1}^n x_i^2}$

        В общем случае для аппроксимирующей функции $f(x, \lambda_1, \dots, \lambda_k)$ с $k$ неизвестными
        параметрами составляем $\sigma^2(\lambda_1, \dots, \lambda_k) = \sum^n_{i = 1} (y_i - f(x_i, \lambda_1, \dots, \lambda_k))^2$,
        
        решаем систему
        \begin{cases}
            \frac{\partial \sigma^2}{\partial \lambda_1} = 0 \\
            \vdots \\
            \frac{\partial \sigma^2}{\partial \lambda_k} = 0
        \end{cases}
        и получаем $\lambda_1, \dots, \lambda_k$

        \item Многочлен Фурье

        $P(t) = \frac{a_0}{2} + a_1 \cos t + b_1 \sin t + \dots a_n \cos nt + b_n \sin nt$ - линейная комбинация

        Функции ${1, \cos t, \sin t, \dots, \cos nt, \sin nt}$ - ортогональны

        Задача в том, чтобы для функции $f(t)$, определенной на отрезке $[0;2\pi]$, 
        найти минимально отстоящий многочлен $P(t)$ при том,
        что расстояние определяется как $\sigma^2 = \int_0^{2\pi} (f(t) - P(t))^2 dt$

        Нужно найти $a_i$ и $b_i$ - обычные скалярные произведения $a_i = k \int_0^{2\pi} f(t) \cos(it) dt$, $\displaystyle b_i = m \int_0^{2\pi} f(t) \sin(it) dt$ ($k, m$ - нормирующие множители)
    \end{enumerate}

    \clearpage

    \section[p2]{2. Линейный оператор}

    \hypertarget{linearoperatordefinition}{}

    \subsection[p2\_1]{2.1. Определение}

    \Def \textit{Линейный оператор} - это отображение $V^n \stackrel{\mathcal{A}}{\Longrightarrow} W^m$
    ($V^n, W^m$ - линейные пространства размерностей $n \neq m$ в общем случае),
    которое $\forall x \in V^n$ сопоставляет один какой-либо $y \in W^m$ и
    \fbox{$\mathcal{A} (\lambda x_1 + \mu x_2) = \lambda \mathcal{A} x_1 + \mu \mathcal{A} x_2 = \lambda y_1 + \mu y_2$}

    \Nota Заметим, что если 0 представим как $0 \cdot x$, где $x \neq 0$, то
    $\mathcal{A}(0) = \mathcal{A}(0 \cdot x) = 0 \cdot \mathcal{A}x \stackrel{0 \cdot y}{=} 0$

    \Notas Если $V = W$, то $\mathcal{A}$ называют линейным преобразованием, но далее будем рассматривать в основном операторы $\mathcal{A}: \ \ V \rightarrow V$, $\mathcal{A}: \ \ V^n \rightarrow W^n$


    \ExN{1} $V = \Real^2$ - пространство направленных отрезков

    $\mathcal{A} : V \rightarrow V$

    $\mathcal{A}x = y = \lambda y_1 + \mu y_2$ для таких $\mathcal{A}$ как сдвиг, поворот, гомотетия, симметрия

    \ExN{2} $V^n = W^m$, где $m < n$

    $\mathcal{A}$ - оператор проектирования (убедиться, что он линейный)

    \ExN{3} $V^n$ - пространство числовых строк длины $n$

    $\mathcal{A}: V^n \rightarrow V^n$

    $x = (x_1, \dots, x_n), y = (y_1, \dots, y_n)$

    Выражение $\mathcal{A}x = y$ можно представить как 
    \begin{pmatrix}
    a_{11} & \ldots & a_{1n}\\
    \vdots & \ddots & \vdots\\
    a_{n1} & \ldots & a_{nn}\\
    \end{pmatrix} $x = y$


    \subsection[p2\_2]{2.2. Действия с операторами}

    \Def Пусть $\mathcal{A}, \mathcal{B}: V \rightarrow W$, тогда определены операции:

    \begin{enumerate}
        \item Сумма операторов: $(\mathcal{A} + \mathcal{B})x \stackrel{def}{=} \mathcal{A}x + \mathcal{B}x = \mathcal{C}x$
        \item Произведение оператора на число: $(\lambda\mathcal{A})x \stackrel{def}{=} \lambda(\mathcal{A}x)$ - $\lambda\mathcal{A} = \mathcal{D}x$
    \end{enumerate}

    \Nota Сформируем линейное пространство из операторов $\mathcal{A}: V \rightarrow W$

    \begin{enumerate}
        \item Ассоциативность сложения (очевидно)
        \item Коммутативность (очевидно)
        \item Нейтральный элемент $\mathcal{O}x = 0$
        \item Противоположный: $-\mathcal{A} = (-1) \cdot A$
        \item \dots \Lab
    \end{enumerate}

    \Def $\mathcal{I}$ - тождественный оператор, если $\forall x \in V \ \mathcal{I}x = x$

\end{document}
