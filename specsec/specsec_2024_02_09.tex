\documentclass[12pt]{article}
\usepackage{preamble}

\pagestyle{fancy}
\fancyhead[LO,LE]{Специальные разделы \\ высшей математики}
\fancyhead[CO,CE]{09.02.2024}
\fancyhead[RO,RE]{Лекции Далевской О. П.}


\begin{document}
    \section{1. Евклидовы пространства}

    \subsection{1.1. Скалярное произведение}

    Пусть $L$ - линейное пространство (ЛП). Тогда $\forall x, y \in L$ величину $\underset{x, y \to c \in \Real}{c = (x, y)}$
    будем называть скалярным произведением


    \hypertarget{scalarproductdefinition}{}

    \begin{enumerate}
        \item $(x, y) = (y, x)$
        \item $(\lambda x, y) = \lambda (x, y), \quad \lambda \in \Real$
        \item $(x + z, y) = (x, y) + (z, y)$
        \item $\forall x \in L\ (x, x) \geq 0$ и $(x, x) = 0 \Longrightarrow x = 0$
    \end{enumerate}

    \Notas Если векторы и коэффициенты комплексно-значные, то определения будут другими

    \Def Скалярная функция $c = (x, y)$ со свойствами 1-4 называется скалярным произведением элементов $x$ и $y$

    \hypertarget{euclidspacedefinition}{}

    \Def Линейное пространство со скалярным произведением называется \underline{Евклидовым}

    \ExN{1} ЛП - пространство геометрических векторов

    $(\vec{a}, \vec{b}) \stackrel{def}{=}
    \begin{cases}
        |\vec{a}||\vec{b}|\cos\varphi, \quad \vec{a}, \vec{b} \neq 0 \\
        0, \quad \vec{a} = 0 \lor \vec{b} = 0
    \end{cases}$

    \ExN{2} $L = C_{[a;b]}$

    $(f(x), g(x)) \stackrel{def}{=} \int^b_a f(x)g(x) dx$

    Очевидно, что свойства 1-3 выполняются, проверим 4:

    $\int^b_a f^2(x) dx = 0 \stackrel{?}{\Longrightarrow} f(x) = 0$

    \ExN{3} ЛП - пространство числовых строк вида $x = (x_1, x_2, \dots, x_n)$

    $(x, y) = x_1 y_1 + \dots x_n y_n = \sum_{i=1}^n x_i y_i$ - сумма произведений компонент

    \subsection{1.2. Свойства евклидова пространства - $E$}

    \hypertarget{inequalityofCauchyBunyakovsky}{}

    \begin{MyTheorem}
        \Ths Неравенство Коши-Буняковского

        $(x, y)^2 \leq (x, x)(y, y)$
    \end{MyTheorem}

    \begin{MyProof}
        Нетрудно заметить, что:

        $(\lambda x - y, \lambda x - y) = (\lambda x - y, \lambda x) - (\lambda x - y, y) =
        (\lambda x, \lambda x) - (y, \lambda x) - (\lambda x, y) + (y, y) = \lambda^2 (x, x) - 2\lambda (x, y) + (y, y)$

        Приравняем полученное выражение к 0, получаем квадратное уравнение. Решим относительно $\lambda$:

        $D = 4(x, y)^2 - 4(x, x)(y, y) \Longrightarrow \frac{D}{4} = (x, y)^2 - (x, x)(y, y)$

        Так как $(\lambda x - y, \lambda x - y) \geq 0$ (4-ое свойство скалярного произведения), то уравнение имеет $\leq 1$ корня, значит
        $\frac{D}{4} = (x, y)^2 - (x, x)(y, y) \leq 0$
    \end{MyProof}

    \subsection{1.3. Норма}

    \hypertarget{normdefinition}{}

    ЛП $= L, \forall x \in L$ определена функция так, что выполняется $x \to n \in \Real, n = \|x\|$

    \begin{enumerate}
        \item $\|x\| \geq 0$ и $\|x\| = 0 \Longrightarrow x = 0$
        \item $\|\lambda x\| = |\lambda| \cdot \|x\| \quad \lambda \in \Real$
        \item $\|x + y\| \leq \|x\| + \|y\| \quad \forall x, y \in L$ - неравенство треугольника
    \end{enumerate}

    \hypertarget{normalizedeuclidspace}{}

    Евклидово пространство с нормой называется нормированным

    \begin{MyTheorem}
        \Ths $E^n$ является нормированным, если $\|x\| = \sqrt{(x, x)}$
    \end{MyTheorem}

    \begin{MyProof}
        Свойства 1-2 очевидны, докажем 3 свойство:

        $\|x + y\| = \sqrt{(x + y, x + y)} \leq \sqrt{(x, x)} + \sqrt{(y, y)} = \|x\| + \|y\|$

        $\sqrt{(x, x) + 2(x, y) + (y, y)} \leq \sqrt{(x, x)} + \sqrt{(y, y)}$

        $(x, x) + 2(x, y) + (y, y) \leq (x, x) + (y, y) + 2\sqrt{(x, x)(y, y)}$

        $(x, y) \leq \sqrt{(x, x)(y, y)}$

        $(x, y)^2 \leq (x, x)(y, y)$ - верно по неравенству Коши-Буняковского
    \end{MyProof}

    Обобщим геометрические понятия ортогональности и косинуса угла на случай произвольных векторов

    \Def $x, y$ - ортогональны, если $(x, y) = 0$ и $x \neq 0$ и $y \neq 0 \quad x \perp y$

    \Def $\cos(\widehat{x, y}) = \frac{(x, y)}{\|x\|\cdot\|y\|}$ - косинус угла между векторами

    \Def $x, y \in E^n, x \perp y$, тогда $z = x + y$ - гипотенуза

    \begin{MyTheorem}
        \Ths $x \perp y$, тогда $\|x + y\|^2 = \|x\|^2 + \|y\|^2$
    \end{MyTheorem}

    \begin{MyProof}
        $\|x + y\|^2 = (x + y, x + y) = (x, x)^2 + \underset{= 0, x \perp y}{\undergroup{2(x, y)}} + (y, y)^2 = (x, x)^2 + (y, y)^2$
    \end{MyProof}

    \Def $B = \Set{e_i}_{i=1}^n$ - базис $L^n$

    На $L^n$ введены $(x, y)$ и $\|x\|$ (то есть $L^n \to E^n_{\|\cdot\|}$ - нормированное евклидово)

    \hypertarget{ortonormalizedbasis}{}

    $B$ называют ортонормированным базисом, если $(e_i, e_j) = \begin{cases}0, i \neq j \\ 1, i = j\end{cases}$

    \Nota Докажем, что всякая такая система из $n$ векторов линейно независима (то есть всякая нулевая комбинация тривиальная):

    $\sum_{i=1}^n \lambda_i e_i = 0 \stackrel{?}{\Longrightarrow} \forall \lambda_i = 0$

    $\left(e_k, \sum_{i=1}^n \lambda_i e_i\right) = \sum_{i=1}^n \lambda_i (e_k, e_i) \stackrel{k \neq i \Rightarrow (e_k, e_i) = 0}{=\joinrel=\joinrel=}
    \lambda_k \|e_k\|^2 = \lambda_k = 0 \quad \forall k$


\end{document}
