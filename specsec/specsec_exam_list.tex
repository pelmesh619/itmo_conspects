\documentclass[12pt]{article}
\usepackage{preamble}

\pagestyle{fancy}

\begin{document}
    \clearpage


    \section{X. Программа экзамена в 2023/2024}


    \begin{center}
        \textbf{Линейная алгебра.}
    \end{center}

    \begin{enumerate}
        \item Евклидово пространство: определение, неравенство Коши-Буняковского. Нормированное евклидово пространство.

        \hyperlink{scalarproductdefinition}{Скалярное произведение} - функция $(x, y)$, обладающая свойствами:

        \begin{enumerate}
            \item $(x, y) = (y, x)$
            \item $(\lambda x, y) = \lambda (x, y), \quad \lambda \in \Real$
            \item $(x + z, y) = (x, y) + (z, y)$
            \item $\forall x \in L\ (x, x) \geq 0$ и $(x, x) = 0 \Longrightarrow x = 0$
        \end{enumerate}

        \hyperlink{euclidspacedefinition}{Евклидовым} называет такое линейное пространство, на котором определено скалярное произведение

        \hyperlink{inequalityofCauchyBunyakovsky}{Неравенство Коши-Буняковского}: $(x, y)^2 \leq (x, x)(y, y)$

        \hyperlink{normdefinition}{Норма} - функция $\|x\|$, такая что

        \begin{enumerate}
            \item $\|x\| \geq 0$ и $\|x\| = 0 \Longrightarrow x = 0$
            \item $\|\lambda x\| = |\lambda| \cdot \|x\| \quad \lambda \in \Real$
            \item $\|x + y\| \leq \|x\| + \|y\| \quad \forall x, y \in L$ - неравенство треугольника
        \end{enumerate}

        \hyperlink{normalizedeuclidspace}{Нормированное Евклидово пространство}: $E^n$ является нормированным, если $\|x\| = \sqrt{(x, x)}$

        \item Ортонормированный базис, ортогонализация базиса. Матрица Грама. Инвариантность евклидовых пространств.

        \hyperlink{ortonormalizedbasis}{Ортонормированный базис} - такой базис, что $(e_i, e_j) = \begin{cases}
                                                                                                      0, i \neq j \\ 1, i = j
        \end{cases}$

        \hyperlink{orthogonalbasisinspace}{Теорема о существовании ортонормированного базиса} (доказывается по матиндукции)


        \hyperlink{grammatrix}{Матрица Грама}: Матрицу $\Gamma = {(e_i, e_j)}_{i, j = 1\dots k}$ называют матрицей Грама

        \item Ортогональность вектора подпространству, ортогональное дополнение. Задача о перпендикуляре.

        \hyperlink{perpendicularproblem}{Задача о перпендикуляре}: Постановка: Нужно опустить перпендикуляр из точки пространства $E^n$ на подпространство $G$

        Точка $M$ - конец вектора $x$ в пространстве $E^n$.
        Нужно найти $M_0$ (конец вектора $x_0$, проекции $x$ на $G$)

        \begin{MyTheorem}
            \Ths $h \perp G, x_0 \in G, x = x_0 + h$. Тогда $\forall x^\prime \in G (x^\prime \neq x_0) \ \ \|x - x^\prime\| > \|x - x_0\|$
        \end{MyTheorem}

        \item Линейный оператор: определение, основные свойства.

        \hyperlink{linearoperatordefinition}{Линейный оператор} - это отображение $V^n \stackrel{\mathcal{A}}{\Longrightarrow} W^m$

        \hyperlink{linearoperatorproperties}{Свойства}:

        1* $\lambda (\mathcal{A}\mathcal{B}) = (\lambda \mathcal{A})\mathcal{B}$

        2* $(\mathcal{A} + \mathcal{B}) \mathcal{C} = \mathcal{A}\mathcal{C} + \mathcal{B}\mathcal{C}$

        3* $\mathcal{A} (\mathcal{B} + \mathcal{C}) = \mathcal{A}\mathcal{B} + \mathcal{A}\mathcal{C}$

        4* $\mathcal{A} (\mathcal{B}\mathcal{C}) = (\mathcal{A}\mathcal{B}) \mathcal{C}$


        \item Обратный оператор. Взаимно-однозначный оператор.

        \hyperlink{reverselinearoperator}{Обратный оператор}: $\mathcal{B} : W \rightarrow V$ называется обратным оператором для $\mathcal{A} : V \rightarrow W$
        если $\mathcal{B}\mathcal{A} = \mathcal{A}\mathcal{B} = \mathcal{I}$ (обозначается $\mathcal{B} = \mathcal{A}^{-1}$)

        \hyperlink{onetoonelinearoperator}{Взаимно-однозначный оператор}: $\mathcal{A} : V \rightarrow W$ так, что $\mathcal{A}V = W$ и $\forall x_1 \neq x_2 (x_1, x_2 \in V) \quad
        \begin{cases}y_1 = \mathcal{A}x_1 \\ y_2 = \mathcal{A}x_2\end{cases} \Longrightarrow y_1 \neq y_2$

        Тогда $\mathcal{A}$ называется взаимно-однозначно действующим

        \item Матрица линейного оператора. Преобразование матрицы при переходе к новому базису.

        \hyperlink{operatorsmatrix}{Матрица оператора}: Матрица $A = {a_{ij}}_{i=1..m, j=1..n}$ называется матрицей оператора $\mathcal{A} : V^n \rightarrow W^m$ в базисе $\Set{e_j}^n_{j=1}$ пространства $V^n$

        \hyperlink{transformationtodifferentbasis}{Преобразование к другому базису}:     $\mathcal{T} : V^n \rightarrow V^n$ - преобразование координат, то есть $Te_i = e^\prime_i$

        Тогда $A^\prime = TAT^{-1}$ ($A^\prime_{e^\prime} = T_{e\to e^\prime}AT^{-1}_{e\to e^\prime}$)

        \item Ядро и образ оператора. Теорема о размерностях.

        \hyperlink{kernalandimageofoperator}{Ядро и образ}:

        Ядро оператора - $Ker \mathcal{A} \stackrel{def}{=} \Set{x \in V \ | \ \mathcal{A}x = \texttt{0}_W}$

        Образ оператора - $Im \mathcal{A} \stackrel{def}{=} \Set{y \in W \ | \ \mathcal{A}x = y}$

        \hyperlink{theoremaboutdimensions}{Теорема о размерностях}: $\mathcal{A} : V \rightarrow V$, тогда $\dim Ker \mathcal{A} + \dim Im \mathcal{A} = \dim V$


        \item Собственные числа и собственные векторы оператора. Теоремы о диагональной матрице оператора.

        \hyperlink{eigenvalue}{Собственное число} $\lambda$ - такое, что удовлетворяет вековому уравнению $|A - \lambda I| = 0$

        Кратность корня $\lambda_i$ называется алгебраической кратностью

        \hyperlink{eigenvector}{Собственный вектор} - такой вектор $x$, что $\mathcal{A}x = \lambda x$

        $U_{\lambda_i} = \Set{x \in V \ | \ \mathcal{A}x = \lambda_i x} \union \Set{0}$

        $\dim U_{\lambda_i}$ - геометрическая кратность числа $\lambda_i$

        \hyperlink{diagonalizedmatrixtheorem}{Теорема о диагонализации}: $\mathcal{A}$ - диаг.-ем $\Longleftrightarrow \exists$ базис из собственных векторов $\Longleftrightarrow$ сумма алгебраических кратностей равна сумме геометрических

        \item Сопряженный и самосопряженный операторы в вещественном евклидовом пространстве: определения, основные свойства. Свойства собственных чисел и собственных векторов самосопряженного оператора.

        \hyperlink{conjugateoperator}{Сопряженный оператор}: Оператор $\mathcal{A}^*$ называется сопряженным для $\mathcal{A} : V \to V$, если

        $(\mathcal{A}x, y) = (x, \mathcal{A}^* y)$

        $\mathcal{A}^*$ сопряженный для $\mathcal{A}$, если $A^* = A^T$ в любом ортонормированном базисе

        \hyperlink{conjugateoperatorproperties}{Свойства}:

        1) $\mathcal{I} = \mathcal{I}^*$

        2) $(\mathcal{A} + \mathcal{B})^* = \mathcal{A}^* + \mathcal{B}^*$

        3) $(\lambda \mathcal{A})^* = \lambda \mathcal{A}^*$

        4) $(\mathcal{A}^*)^* = \mathcal{A}$

        5) $(\mathcal{A}\mathcal{B})^* = \mathcal{B}^* \mathcal{A}^*$ (св-во транспонирования матриц)

        или $((\mathcal{AB})x, y) = (\mathcal{A}(\mathcal{B}x), y) = (\mathcal{B}x, \mathcal{A}^* y) = (x, \mathcal{B}^* \mathcal{A}^* y)$

        6) $\mathcal{A}^*$ - линейный оператор ($\mathcal{A}x = x^\prime, \mathcal{A}y = y^\prime \Longrightarrow \mathcal{A}(\lambda x + \mu y) = \lambda x^\prime + \mu y^\prime$)

        \hyperlink{selfconjugateoperator}{Самосопряженный оператор}: $\mathcal{A}$ называется самосопряженным, если $\mathcal{A} = \mathcal{A}^*$

        Следствие. $A^T = A \Longrightarrow$ матрица $A$ симметричная

        \hyperlink{selfconjugateoperatorproperties}{Свойства}:

        1) $\mathcal{A} = \mathcal{A}^*, \ \lambda : \ \mathcal{A}x = \lambda x (x \neq 0)$. Тогда, $\lambda \in \Real$

        2) $\mathcal{A} = \mathcal{A}^*, \ \mathcal{A}x_1 = \lambda_1 x_1, \mathcal{A}x_2 = \lambda_2 x_2$ и $\lambda_1 \neq \lambda_2$. Тогда $x_1 \perp x_2$

        \hyperlink{theoremabouteigenvectorsinselfconjugateoperator}{Теорема о базисе собственных векторов}: $\mathcal{A} = \mathcal{A}^*$ ($\mathcal{A} : V^n \to V^n$),
        тогда $\exists e_1, \dots, e_n$ - набор собственных векторов $\mathcal{A}$ и $\Set{e_i}$ - ортонормированный базис

        (другими словами: $\mathcal{A}$ - диагонализируем)

        \item Структура образа самосопряженного оператора. Проектор. Спектральное разложение оператора.

        \hyperlink{projector}{Проектор}: Оператор $P_i x = (x, e_i) e_i$ называется проектором на одномерное пространство, порожденное $e_i$ (линейная оболочка)

        \hyperlink{spectraldecomposition}{Спектральное разложение}: $\mathcal{A} = \sum_{i = 1}^{n} \lambda_i P_i$

        \item Ортогональная матрица и ортогональный оператор. Геометрический смысл ортогонального преобразования.

        \hyperlink{orthogonaloperator}{Ортогональный оператор}: $T$ - ортогональный оператор, если $(Tx, Ty) = (x, y)$

        Следствие: $\|Tx\| = \|x\|$, то есть $T$ сохраняет расстояние

        \hyperlink{orthogonalmatrix}{Ортогональная матрица}: Матрица $A$ называется ортогональной если $A^{-1} = A^T$



        \item Билинейные формы: определения, свойства. Матрица билинейной формы.

        \hyperlink{bilinearforms}{Билинейная форма}: $x, y \in V^n \quad$ Отображение $\mathcal{B}: V^n \to \Real$ (обозн. $\mathcal{B}(x, y)$)
        называется билинейной формой, если выполнены

        1) $\mathcal{B}(\lambda x + \mu y, z) = \lambda \mathcal{B}(x, z) + \mu \mathcal{B}(y, z)$

        2) $\mathcal{B}(x, \lambda y + \mu z) = \lambda \mathcal{B}(x, y) + \mu \mathcal{B}(x, z)$

        \hyperlink{bilinearformmatrix}{Матрица}: $\Set{e_i}_{i=1}^n$ - базис $V_n$, $u, v \in V^n$. Тогда $\mathcal{B}(u, v) =
        \sum_{j = 1}^{n}\sum_{i = 1}^{n} b_{ij} u_i v_j$, где $b_{ij} \in \Real$ - матрица


        \item Квадратичная форма: определения, приведение к каноническому виду.

        \hyperlink{quadraticform}{Квадратичная форма}: Квадратичной формой, порожденной Б. Ф. $\mathcal{B}(u, v)$, называется форма $\mathcal{B}(u, u)$


        \item Знакоопределенность квадратичной формы: необходимые и достаточные условия. Критерий Сильвестра.

        \hyperlink{positivedefinedoperator}{Положительно определенный оператор}: 1) Оператор $\mathcal{A}$ называется положительно определенным, если
        $\exists \gamma > 0 \ | \ \forall x \in V \quad (\mathcal{A}x, x) \geq \gamma \|x\|^2$

        2) $\mathcal{A}$ называется положительным, если
        $\forall x \in V, \ x \neq 0 \quad (\mathcal{A}x, x) > 0$


        \hyperlink{criterionSilvester}{Критерий Сильвестра}: $\mathcal{A}: V^n \to V^n$ - положительно определен $\Longleftrightarrow \\ \forall k = 1..n $ угловые миноры $ \Delta_k =
        \begin{vmatrix}a_{11} & \dots & a_{1k} \\ \vdots & \ddots & \vdots \\ a_{k1} & \dots & a_{kk}\end{vmatrix} > 0$


    \end{enumerate}

    \begin{center}
        \textbf{Дифференциальные уравнения.}
    \end{center}

    \begin{enumerate}
        \item Обыкновенное дифференциальное уравнение (ДУ): задача о радиоактивном распаде и задача о падении тела. Определение ДУ, решения ДУ и их геометрический смысл. Задача Коши.

        \hyperlink{radiumproblem}{Задача о распаде}: Скорость распада радия в текущий момент времени $t$ пропорциональна его наличному количеству $Q$. Требуется найти закон распада радия: $Q = Q(t)$

        если в начальный момент времени $t_0 = 0$ количество равнялось $Q_0$

        \fbox{$Q(t) = Ce^{-nt}$}

        \hyperlink{freefallingbodyproblem}{Задача о падении тела}: Тело массой $m$ брошено вверх с начальной скоростью $v_0$. Нужно найти закон движения $y = y(t)$.
        Сопротивлением воздуха пренебречь.

        $y(t) = \int (-gt + C_1) dt = $\fbox{$-\frac{gt^2}{2} + C_1 t + C_2 = y(t)$} - общий закон

        \fbox{$y^*(t) = v_0 t - \frac{gt^2}{2}$} - частный закон при $y(t_0) = 0, y^\prime(t_0) = v_0$

        \hyperlink{differentialequationdefinition}{Определение}: Уравнение $F(x, y(x), y^\prime(x), \dots, y^{(n)}(x)) = 0$ - называется обыкновенным ДУ $n$-ого порядка $(*)$

        Решением ДУ $(*)$ называется функция $y(x)$, которая при подстановке обращает $(*)$ в тождество

        \hyperlink{problemCauchy}{Задача Коши}: $\begin{cases}y(x_0) = y_0 \\ \vdots \\ y^{(n - 1)}(x_0) = y_0^{(n - 1)}\end{cases}$ - система начальных условий $(**)$

        Тогда $\begin{cases}(*) \\ (**)\end{cases}$ - задача Коши (ЗК)

        \item Уравнение с разделяющимися переменными.

        \hyperlink{equationwithseparablevariables}{УРП}: $m(x)N(y)dx + M(x)n(y)dy = 0$

        Решение: $\int \frac{m(x)}{M(x)}dx = \int\frac{-n(y)}{N(y)}dy$

        \item Однородное уравнение.

        \hyperlink{homogeneousequation}{ОУ}: \fbox{$P(x, y)dx + Q(x, y)dy = 0$, где $P(x, y), Q(x, y)$ - однородные функции одного порядка} - однородное уравнение

        Решение: $Cx = e^{\int\frac{dt}{f(t) - t}}$, где $t = \frac{y}{x}$

        \item Уравнение в полных дифференциалах.

        \hyperlink{equationincompletedifferentials}{Уравнение в полных дифференциалах}: \fbox{$P(x, y)dx + Q(x, y)dy = 0 \quad \frac{\partial P}{\partial y} = \frac{\partial Q}{\partial x}$} - УПД

        Решение: $\Phi(x, y) = \int^{(x,y)}_{(x_0,y_0)} Pdx + Qdy = 0$

        \item Линейное уравнение первого порядка. Метод Лагранжа.

        \hyperlink{lineardifferentialequation}{ЛДУ}: \fbox{$y^\prime + p(x)y = q(x)$} - ЛДУ$_1$

        \hyperlink{methodLagrange}{Метод Лагранжа}: Принцип: если удалось найти частное решение ДУ$_\text{однор}$ (обозначим $y_0$), то общее решение ДУ$_\text{неод}$
        можно искать в виде $y = C(x)y_0$

        Решение: $y_0 = e^{-\int p(x) dx}, C(x) = \int q(x) e^{\int p(x)dx} dx$

        $y = e^{-\int p(x) dx} \int q(x) e^{\int p(x)dx} $

        \item Теорема существования и единственности решения задачи Коши. Особые решения.

        \hyperlink{existenceanduniquenessofsolution}{Теорема существования и единственности}: 
        \begin{MyTheorem}
            \Ths Если $\exists U(M_0) \ | \
            \begin{cases}
                f(x,y) \in C_{U(M_0)} \\
                \frac{\partial f}{\partial y}\text{ - огр. в } U(M_0),
            \end{cases}$ то в $M_0\ \exists! y(x)$ - решение ДУ
        \end{MyTheorem}


        \item Уравнения n-ого порядка, допускающие понижение порядка.

        \hyperlink{differentilaequationshigherdegree}{ДУ высших порядков}:     1* Непосредственно интегрирование

        $y^{(n)} = f(x)$

        Решение: $y^{(n - 1)} = \int f(x) dx + C_1$

        $y^{(n - 2)} = \int (\int f(x) dx + C_1) dx + C_2$

        2* ДУ$_2$, не содержащие $y(x)$

        $F(x, y^\prime(x), y^{\prime\prime}(x)) = 0$

        Замена $y^\prime(x) = z(x)$, получаем:

        $F(x, z(x), z^\prime(x)) = 0$ - ДУ$_1$

        3* ДУ$_2$, не содержащие $x$

        $F(y(x), y^\prime(x), y^{\prime\prime}(x)) = 0$

        Замена $y^\prime(x) = z(y) \quad y^{\prime\prime}(x) = \frac{dz(y(x))}{dx} = \frac{dz}{dx} \frac{dy}{dx} = z^\prime_y y^\prime = z^\prime z$


        \item Линейные однородные дифференциальные уравнения (ЛОДУ): определения, решение ЛОДУ2 с постоянными коэффициентами для случая различных вещественных корней характеристического уравнения.

        \hyperlink{lineardifferentialequationhigherdegree}{Определение}: $a_0(x) y^{(n)}(x) + a_1(x)y^{(x)} + \dots + a_{n - 1}y^\prime(x) + a^n(x)y = f(x)$, где $y = y(x)$ - неизв. функция, - это ЛДУ$_n$

        \hyperlink{lineardifferentialequationseconddegreewithconstants}{Решение ЛОДУ$_2$}: $y^{\prime\prime} + p y^\prime + q y = f(x), \quad p, q \in \Real$

        $\forall p, q \in \Real \exists $ уравнение: $\quad \lambda^2 + p\lambda + q = 0$ и $\lambda_{1,2} \in \mathbb{C} \ | \ \lambda_1 + \lambda_2 = -p, \lambda_1 \lambda_2 = q$ - корни

        \hyperlink{ldesgdifferentrealsolutions}{1 случай}: $\lambda_{1.2} \in \Real, \lambda_1 \neq \lambda_2 \Longrightarrow y(x) = C_1 e^{\lambda_1 x} + C^2 e^{\lambda_2 x}$

        \item Решение ЛОДУ2 с постоянными коэффициентами для случая вещественных кратных корней характеристического уравнения.

        \hyperlink{ldesgequalrealsolutions}{2 случай}: $\lambda_1 = \lambda_2 = \lambda \in \Real \Longrightarrow y(x) = (C_1 x + C_2)e^{\lambda x}$


        \item Решение ЛОДУ2 с постоянными коэффициентами для случая комплексных корней характеристического уравнения.

        \hyperlink{ldesgcomplexsolutions}{3 случай}: $\lambda = \alpha \pm i \beta \in \mathbb{C} \Longrightarrow y(x) = C_1 e^{\alpha x} \sin \beta x + C_2 e^{\alpha x} \cos \beta x$


        \item Свойства решений ЛОДУ2: линейная независимость решений, определитель Вронского. Теоремы 1,2.

        \hyperlink{linearindependance}{Линейная независимость}: \Defs $y_1, y_2$ -- линейно независимы $\Longleftrightarrow C_1 y_1 + C_2 y_2 = 0 \Longrightarrow \forall C_1 = 0 \Longleftrightarrow \nexists k : y_2 = k y_1, k \in \Real$
        
        \hyperlink{determinantWronski}{Определитель Вронского}: \Defs $W \stackrel{\text{обозн}}{=} \begin{vmatrix}y_1(x) & y_2(x) \\ y_1^\prime(x) & y_2^\prime(x)\end{vmatrix} = \begin{vmatrix}y_1(x) & y_2(x) \\ \frac{d y_1(x)}{dx} & \frac{d y_2(x)}{dx}\end{vmatrix}$ -- определитель Вронского или вронскиан

        \begin{MyTheorem}
            \ThNs{1} $\letsymbol y_1, y_2$ - частные решения ЛОДУ, то есть $Ly_1 = 0, Ly_2 = 0$

            Тогда $Ly = 0$, если $y = C_1 y_1 + C_2 y_2$
        \end{MyTheorem}

        \begin{MyTheorem}
            \ThNs{2} $y_1, y_2$ -- линейно зависимы $\Longrightarrow W = 0$ на $[a;b]$
        \end{MyTheorem}    

        \item Свойства решений ЛОДУ2: линейная комбинация решений, линейная зависимость решений. Определитель Вронского. Теоремы о вронскиане.

        \begin{MyTheorem}
            \ThNs{3} $x_0 \in [a;b]$, пусть $W(x_0) = W_0$. Тогда: 
            
            $\begin{matrix}W_0 = 0 \Longrightarrow W(x) = 0 \forall x \in [a;b] \\
            W_0 \neq 0 \Longrightarrow W(x) \neq 0 \forall x \in [a;b]\end{matrix}$
        \end{MyTheorem}

        \begin{MyTheorem}
            \ThNs{4} $y_1, y_2$ -- линейно независимы $\Longrightarrow W(x) \neq 0$ на $[a;b]$
        \end{MyTheorem}
    

        \item Свойства решений ЛОДУ2: линейная комбинация решений, линейная зависимость решений. Теорема о структуре общего решения ЛОДУ2. Фундаментальная система решений (определение).

        \begin{MyTheorem}
            \ThNs{5} $y_1, y_2$ -- линейно независимые решения ЛОДУ, тогда $\overline{y}(x) = C_1 y_1 + C_2 y_2$ -- общее решение ЛОДУ$_2$
        \end{MyTheorem}

        \item Свойства решений ЛНДУ2: теоремы о структуре общего решения и решении ДУ с суммой правых частей.

        \begin{MyTheorem}
            \ThNs{6} Решение ЛНДУ $Ly = f(x)$
    
            $\overline{y}(x): L\overline{y} = 0$ -- общее решение ЛОДУ
    
            $y^*(x): Ly^*(x) = f(x)$ -- частное решение ЛНДУ
    
            Тогда $y(x) = \overline{y} + y^*$ -- общее решение ЛНДУ
        \end{MyTheorem}


        \item Структура решения ЛОДУn: линейная независимость решений, нахождение фундаментальной системы решений по корням характеристического уравнения.

        \hyperlink{fssforlde2}{ФСР для всех случаев}:

        \begin{enumerate}
            \item Всякому $\lambda_i$ -- одиночному $\Real$-корню ХрУ сопоставляем $y_i = e^{\lambda_i x}$
    
            \item $\Real$-корню $\lambda$ кратности $s$ сопоставляем набор $\Set{y_1, y_2, \dots, y_s} = \Set{e^{\lambda x}, xe^{\lambda x}, \dots, x^{s - 1} e^{\lambda x}}$
    
            \item Всякой одиночной паре $\lambda_{j_1,j_2} = \alpha_j \pm i\beta_j$ соответствует пара $\{e^{\alpha x} \cos\beta x, e^{\alpha x} \sin\beta x\}$
    
            \item Комплексной паре $\lambda = \alpha \pm i\beta$ кратности $t$ соответствует набор 
            
            $\{e^{\alpha x} \cos \beta x, e^{\alpha x} \sin \beta x, x e^{\alpha x} \cos \beta x, \dots, x^{t - 1}e^{\alpha x} \cos\beta x, x^{t - 1}e^{\alpha x} \sin\beta x\}$
        \end{enumerate}

        $\overline{y} = l_{\{\text{ФСР}\}}$

        \item Решение ЛНУ2 с постоянными коэффициентами: специальная правая часть, поиск частного решения методом неопределенных коэффициентов.

        \hyperlink{specialrightpart}{Специальная правая часть}: для линейного неоднородного уравнения второго порядка $y^\prime^\prime + py^\prime + qy = f(x)$ с постоянными коэффициентами, где правая часть $f(x)$ является специальной, частное решение $y^*$ ищут методом неопределённых коэффициентов: 
        
        \begin{enumerate}
            \item Анализ корней характеристического уравнения $\lambda^2 + p\lambda + q = 0$;
            \item Определение структуры $y^*$ в виде $x^r e^{ax} (\overline{P}_l(x) \cos(bx) + \overline{Q}_l(x) \sin(bx))$, где
            $a, b$ берутся из $f(x) = e^{ax}(P_n(x)\cos(bx) + Q_m(x)\sin(bx))$,
            $l = \max(n, m)$,
            $r$ -- кратность корня $k = a \pm ib$ в характеристическом уравнении ($r = 0$ при отсутствии совпадения)
            \item Подстановку $y^*$ в уравнение и определение коэффициентов $\overline{P}_l$, $\overline{Q}_l$ приравниванием аналогичных членов. Метод эффективен для правых частей вида $e^{\alpha x}$, $x^k$, $\cos \beta x$, $\sin \beta x$ и их произведений.
        \end{enumerate}


        \item Решение ЛНУ2: метод вариации произвольных постоянных (Лагранжа).

        \hyperlink{methodLagrangesecondorder}{Метод Лагранжа}: подход для решения линейных неоднородных дифференциальных уравнений второго порядка $y^\prime^\prime + p(x)y^\prime + q(x)y = f(x)$ с произвольной непрерывной правой частью:
        \begin{enumerate} 
            \item Нахождение фундаментальной системы решений (ФСР) $y_1(x), y_2(x)$ соответствующего однородного уравнения

            \item Поиск частного решения в виде $y^*(x) = C_1(x)y_1(x) + C_2(x)y_2(x)$, где функции $C_1(x), C_2(x)$ заменяют константы из общего решения однородного уравнения

            \item Решение системы для производных $C_1^\prime(x), C_2^\prime(x)$:
            
            $\begin{cases} 
            C_1^\primey_1 + C_2^\primey_2 = 0 \\
            C_1^\primey_1^\prime + C_2^\primey_2^\prime = f(x)
            \end{cases}$
            
            с использованием вронскиана $W = y_1y_2^\prime - y_2y_1^\prime$
            \item Интегрирование выражений: 
            
            $C_1(x) = \int -\frac{y_2 f}{W}dx, \quad C_2(x) = \int \frac{y_1 f}{W}dx$
        \end{enumerate}


        \item Системы дифференциальных уравнений: определения, решение методом исключения.

        \hyperlink{desystem}{Система ДУ}: \Defs Пусть дан набор функций $y_1, \dots, y_n$. Система, связывающие эти функции, то есть
        $\begin{cases}
             F_1(x_1, y_1, \dots y_n, \dots, y_1^{(n)}, \dots y_n^{(n)}) = 0 \\
             \vdots
        \end{cases}$, называется системой дифференциальных уравнений (СДУ)

        \hyperlink{exceptionmethod}{Метод исключения}: чтобы свести к ДУ систему ДУ $\begin{cases}
            \dot x_1 = \varphi_1(t, x_1, \dots, x_n) \\
            \vdots \\
            \dot x_n = \varphi_n(t, x_1, \dots, x_n) \\
       \end{cases}$ 
       
       нужно исключить $n - 1$ выражение $\dot x_i$, для этого взять производные $\frac{d^{n - 1} x_i}{dt^{n - 1}}$

        \item Системы дифференциальных уравнений: определения, решение матричным методом в случае различных вещественных собственных чисел.

        \hyperlink{matrixmethod}{Матричный метод}: 

        
        $\begin{cases}
            y^{\prime}_1 = a_{11}y_1 + a_{12}y_2 + \dots + a_{1n} y_n \\
            \vdots \\
            y^{\prime}_n = a_{n1}y_n + a_{n2}y_2 + \dots + a_{nn} y_n
        \end{cases} \quad a_{ij} \in \Real$

        Обозначим $(y_1, \dots, y_n) = Y$ -- вектор функций, $\{a_{ij}\} = A$ -- матрица СДУ

        Тогда СДУ запишется в виде $Y^\prime = AY$ (однородная СДУ, так как нет $f(x)$)

        Пусть $\lambda_1, \dots, \lambda_n$ -- собственные числа $A$ и $h_i$ -- собственный вектор для $\lambda_i$

        Будем искать решение $Y$ в виде $Y = \ln e^{\lambda_i x}$

        Подставим в СДУ: $Y^\prime = \lambda_i h_i = e^{\lambda_i x} = A \underset{Y}{\undergroup{h_i e^{\lambda_i x}}} = AY$


        \item Теория устойчивости: определение устойчивости по Ляпунову, фазовая плоскость, траектории ДУ. Примеры устойчивого и неустойчивого решения.

        \hyperlink{stability}{Устойчивость}: Решение СДУ $x = x(t), y = y(t)$ называется устойчивым по Ляпунову при $t \to +\infty$, если
        $\forall \varepsilon > 0 \ \exists \delta > 0 \ | \ \underset{ \scriptsize \begin{cases}|\tilde{x}_0 - x_0| < \delta \\ |\tilde{y}_0 - y_0| < \delta\end{cases}}{\forall x, y} \forall t > 0 \begin{cases}|\tilde{x}_0 - x_0| < \varepsilon \\ |\tilde{y}_0 - y_0| < \varepsilon\end{cases}$

        Или $\begin{matrix}\Delta x (t) \to 0 \\ \Delta y (t) \to 0\end{matrix}$ при $t \to +\infty$ и $\begin{cases}\Delta x_0 \to 0 \\ \Delta y_0 \to 0\end{cases}$


    \end{enumerate}


\end{document}