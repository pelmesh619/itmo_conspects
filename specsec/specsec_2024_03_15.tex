\documentclass[12pt]{article}
\usepackage{preamble}

\pagestyle{fancy}
\fancyhead[LO,LE]{Специальные разделы \\ высшей математики}
\fancyhead[CO,CE]{15.03.2024}
\fancyhead[RO,RE]{Лекции Далевской О. П.}


\begin{document}
    \Def Пусть $\mathcal{A} : V \rightarrow W; \ \mathcal{B} : U \rightarrow V$, тогда 
    $\mathcal{A}\mathcal{B}$ - произведение операторов (композиция), причем
    $(\mathcal{A}\mathcal{B}) x = \mathcal{A}(\mathcal{B}x); \quad x \in U$

    \hypertarget{linearoperatorproperties}{}

    \underline{Свойства}: 

    \begin{enumerate}[label*=\arabic*$^\circ$]
        \item $\lambda (\mathcal{A}\mathcal{B}) = (\lambda \mathcal{A})\mathcal{B}$

        \item $(\mathcal{A} + \mathcal{B}) \mathcal{C} = \mathcal{A}\mathcal{C} + \mathcal{B}\mathcal{C}$

        \item $\mathcal{A} (\mathcal{B} + \mathcal{C}) = \mathcal{A}\mathcal{B} + \mathcal{A}\mathcal{C}$

        \item $\mathcal{A} (\mathcal{B}\mathcal{C}) = (\mathcal{A}\mathcal{B}) \mathcal{C}$
    \end{enumerate}

    \Lab доказать

    \Nota Можно обобщить $4^\circ$ на $n$ равных $\mathcal{A}$

    \Defs $\mathcal{A}^n = \underset{n \text{ раз}}{\underbrace{\mathcal{A} \cdot \mathcal{A} \cdot \dots \cdot \mathcal{A}}}$ - степень оператора

    Свойства: $\mathcal{A}^{m + n} = \mathcal{A}^n \cdot \mathcal{A}^m$

    \subsection[p2\_3]{2.3. Обратимость оператора}

    \hypertarget{onetoonelinearoperator}{}

    \Def $\mathcal{A} : V \rightarrow W$ так, что $\mathcal{A}V = W$ и $\forall x_1 \neq x_2 (x_1, x_2 \in V) \quad
    \begin{cases}y_1 = \mathcal{A}x_1 \\ y_2 = \mathcal{A}x_2\end{cases} \Longrightarrow y_1 \neq y_2$

    Тогда $\mathcal{A}$ называется взаимно-однозначно действующим

    Nota: Проще сказать \enquote{линейный изоморфизм}

    \begin{MyTheorem}
        \Ths $\Set{x_i}$ - линейно независима $\stackrel{\mathcal{A}x = y}{\Longrightarrow} \{y_i\}$ - линейно независима

        В обратную сторону верно, если $\mathcal{A}$ - взаимно-однозначен
    \end{MyTheorem}

    \begin{MyProof}
        Пусть $\mathcal{A} : V \rightarrow W$ и $\texttt{0}_V, \texttt{0}_W$ - нули $V$ и $W$ соответственно

        \begin{enumerate}
            \item $\mathcal{A}(\texttt{0}_V) = \mathcal{A}\left(\sum^k_{i=1} 0 \cdot e_i\right) = \sum^k_{i=1} 0 \cdot \mathcal{A}e_i = \texttt{0}_W$

            \item Докажем, что если ${x_i} \subset V$ - линейно независима, то ${y_i} \subset W$ - линейно независима

            Составим $\sum^m_{j=1} \lambda_j y_j = \texttt{0}_W$ 
            
            От противного пусть $\{y_i\}$ - линейно зависима, тогда $\exists \lambda_k \neq 0$

            При этом $\forall j \ \ y_j = \mathcal{A}x_j$ (т. к. $\mathcal{A}$ - взаимно-однозначен, 
            то $n^\prime = m^\prime$: кол-во $x_i$ и $y_i$ равно)

            $\sum^{m^\prime}_{j=1} \lambda_j \mathcal{A}x_j \stackrel{\text{линейность}}{=} \mathcal{A} (\sum^{m^\prime}_{j=1} \lambda_j x_j) = \texttt{0}_W$

            Так как $\mathcal{A}\texttt{0}_V = \texttt{0}_W$, то $\texttt{0}_W$ - образ $x = \texttt{0}_V$, 
            но так как $\mathcal{A}$ - взаимно-однозначен, то
            $\nexists x^\prime \neq x \ | \ \mathcal{A}(x^\prime) = \texttt{0}_W$

            Значит $\sum^{m^\prime}_{j=1} \lambda_j x_j = \texttt{0}_V$, но 
            $\exists \lambda_k \neq 0 \Longrightarrow \{x_j\}$ - линейно зависима - \underline{противоречие}

            \item Пусть теперь $\{y_i\}$ - линейно независима, а $\{x_i\}$ (по предположению от противного) - линейно зависима

            $\sum^{n^\prime}_{i = 1} \lambda_i x_i \stackrel{\exists \lambda_k \neq 0}{=} \texttt{0}_V \quad \Big| \mathcal{A}$

            $\sum^{n^\prime}_{i = 1} \lambda_i \mathcal{A}x_i = \texttt{0}_W$

            При этом $\exists \lambda_k \neq 0 \Longrightarrow \{y_i\}$ - линейно зависима - \underline{противоречие}

        \end{enumerate}
    \end{MyProof}

    Следствие: $\dim V = \dim W \Longrightarrow \mathcal{A}$ - линейный изоморфизм

    \hypertarget{reverselinearoperator}{}

    \Def $\mathcal{B} : W \rightarrow V$ называется обратным оператором для $\mathcal{A} : V \rightarrow W$,
    если $\mathcal{B}\mathcal{A} = \mathcal{A}\mathcal{B} = \mathcal{I}$ (обозначается $\mathcal{B} = \mathcal{A}^{-1}$)

    Следствие: $\mathcal{A}\mathcal{A}^{-1} x = x$

    \begin{MyTheorem}
        \Ths $\mathcal{A}x = \texttt{0}$ и $\exists \mathcal{A}^{-1}$, тогда $x = \texttt{0}$
    \end{MyTheorem}

    \begin{MyProof}
        $\mathcal{A}^{-1}\mathcal{A} x = \mathcal{A}^{-1}(\mathcal{A} x) = \mathcal{A}^{-1} \texttt{0}_W = \texttt{0}_V \Longrightarrow x = \texttt{0}$
    \end{MyProof}

    \begin{MyTheorem}
        \Ths Необходимые и Достаточные условия существования $\mathcal{A}^{-1}$

        $\exists \mathcal{A}^{-1} \Longleftrightarrow \mathcal{A}$ - взаимно-однозначный
    \end{MyTheorem}

    \begin{MyProof}
        \fbox{$\Longrightarrow$} $\exists \mathcal{A}^{-1}$, но $\sqsupset \mathcal{A}$ - не взаимно-однозначен, то есть
        $\exists x_1, x_2 \in V (x_1 \neq x_2) \ | \ \mathcal{A}x_1 = \mathcal{A}x_2 \Longleftrightarrow \mathcal{A}x_1 - \mathcal{A}x_2 = \texttt{0} \Longleftrightarrow
        \mathcal{A}(x_1 - x_2) = \texttt{0}_W \stackrel{\exists \mathcal{A}^{-1}}{\Longrightarrow} x = \texttt{0}_V \Longleftrightarrow x_1 = x_2$ - противоречие


        \fbox{$\Longleftarrow$} Так как $\mathcal{A}$ - изоморфизм (не учитывая линейность), 
        то $\exists \mathcal{A}^\prime$ - обратное отображение (не обязательно линейное)

        Докажем, что $\mathcal{A}^\prime : W \rightarrow V$ - линейный оператор

        $\mathcal{A}$ - взаимно-однозначен $\Longleftrightarrow \forall x_i \longleftrightarrow y_i \quad \Big| \cdot \lambda_i, \sum$

        $\mathcal{A}\left(\sum \lambda_i x_i\right) = \mathcal{A} x = y = \sum \lambda_i y_i \quad$ и $y$ имеет только один прообраз $x$

        Применим $\mathcal{A}^\prime$ к $y = \sum \lambda_i y_i$, получим $\mathcal{A}^\prime y = x = \sum \lambda_i x_i$ - единственный прообраз $y$

        Таким образом, $\mathcal{A}^\prime$ переводит линейную комбинацию в такую же линейную комбинацию прообразов, то есть $\mathcal{A}^\prime$ - линейный: $\mathcal{A}^\prime = \mathcal{A}^{-1}$
    \end{MyProof}

    \subsection[p2\_4]{2.4. Матрица линейного оператора}

    Пусть $\mathcal{A} : V^n \rightarrow W^m$

    Возьмем вектор $x \in V^n$ и разложим по какому-либо базису $\{e_j\}^n_{j=1}$

    $\mathcal{A}x = \mathcal{A} \left(\sum^n_{j=1} c_j e_j\right) = \sum^n_{j=1} c_j \mathcal{A}e_j$

    $\mathcal{A} e_j \stackrel{\text{образ базисного вектора}}{=} y_j \stackrel{\{f_i\} - \text{ базис } W^m}{=} \sum^m_{i=1} a_{ij}f_i$

    $\mathcal{A}x = \sum^n_{j=1} c_j \mathcal{A}e_j = \sum^n_{j=1} c_j \sum^m_{i=1} a_{ij}f_i = \sum^n_{j=1} \sum^m_{i=1} c_j a_{ij} f_i = \sum^m_{i=1} \sum^n_{j=1} c_j a_{ij} f_i$

    Иллюстрация:

    $\begin{pmatrix}
         a_{11} & \dots & a_{1n} \\
         \vdots & \ddots & \vdots \\
         a_{m1} & \dots & a_{mn} \\
    \end{pmatrix} \begin{pmatrix}
         c_{1} \\
         \vdots \\
         c_{n} \\
    \end{pmatrix} = \begin{pmatrix}
         b_{1} \\
         \vdots \\
         b_{m} \\
    \end{pmatrix}$

    \hypertarget{operatorsmatrix}{}

    \Def Матрица $A = \{a_{ij}\}_{i=1..m, j=1..n}$ называется матрицей оператора $\mathcal{A} : V^n \rightarrow W^m$ в базисе $\{e_j\}^n_{j=1}$ пространства $V^n$

    \begin{enumerate}
        \item Для каждого ли оператора $\mathcal{A}$ существует матрица $A$?

        При выбранном базисе $\{e_j\} \ \forall \mathcal{A} \ \exists A$ (алгоритм выше)

        \item Для каждой ли матрицы $A$ существует оператор $\mathcal{A}$?

        $\forall A_{m\times n}$ можно взять пару ЛП $V^n, W^m$ и определить $\mathcal{A} : V^n \rightarrow W^m$ по правилу $\mathcal{A}e_V = e^\prime_W$

        \item Если существует матрица $A$ для оператора $\mathcal{A}$, то она единственная?

        Такая $A$ единственная $\Longrightarrow$ в разных базисах матрицы ЛО $\mathcal{A} \quad A_e \neq A_{e^\prime}$

        \item Если существует оператор $\mathcal{A}$ для матрицы $A$, то он единственный?

        \Lab
    \end{enumerate}

    \Nota Далее будем решать две задачи:

    \begin{enumerate}
        \item преобразование координат как действие оператора

        \item поиск наиболее простой матрицы в некотором базисе
    \end{enumerate}

    \subsection[p2\_5]{2.5. Ядро и образ оператора}

    \hypertarget{kernalandimageofoperator}{}

    \Defs Ядро оператора $\Ker \mathcal{A} \stackrel{def}{=} \{x \in V \ | \ \mathcal{A}x = \texttt{0}_W\}$

    \Defs Образ оператора $\IM \mathcal{A} \stackrel{def}{=} \{y \in W \ | \ \mathcal{A}x = y\}$

\end{document}


