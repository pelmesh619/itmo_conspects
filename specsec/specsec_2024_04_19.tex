\documentclass[12pt]{article}
\usepackage{preamble}

\pagestyle{fancy}
\fancyhead[LO,LE]{Специальные разделы \\ высшей математики}
\fancyhead[CO,CE]{19.04.2024}
\fancyhead[RO,RE]{Лекции Далевской О. П.}


\begin{document}
    \section[p4]{4. Дифференциальные уравнения}

    \subsection[p4\_1]{4.1. Общие понятия}

    \subsubsection{4.1.1. Постановка задачи}

    \hypertarget{radiumproblem}{}

    \begin{tcolorbox}
        \textit{Pr. 1.} Скорость распада радия в текущий момент времени $t$ пропорциональна его наличному количеству $Q$. Требуется найти закон распада радия:

        \[Q = Q(t),\]

        если в начальный момент времени $t_0 = 0$ количество равнялось $Q(t_0) = Q_0$

        Коэффициент пропорциональности $k$ найден эмпирически.
    \end{tcolorbox}

    \underline{Решение.} Имеем уравнение $\frac{dQ(t)}{dt} = kQ \quad$, ищем $Q(t)$

    $dQ(t) = kQdt$

    $\underset{\text{содержит только }Q}{\undergroup{\frac{dQ(t)}{Q}}} = \underset{\text{содержит только }t}{\undergroup{kdt}}$ \hfill \enquote{разделение переменных}

    $d \ln Q = kdt = dkt$ \hfill вносим $k$ в дифференциал

    Получаем $d(\ln Q - kt) = 0$. Находим семейство первообразных:

    $\ln Q - kt = \tilde{C} \Longrightarrow \ln Q = \tilde{C} + kt$

    $Q = e^{\tilde{C} + kt} \stackrel{e^{\tilde{C}} = C}{=\joinrel=\joinrel=\joinrel=} Ce^{kt}$

    По смыслу $k < 0$, так как $Q$ уменьшается. Обозначим $n = -k, n > 0$

    Тогда \fbox{$Q(t) = Ce^{-nt}$}

    \mediumvspace

    Получили вид закона распада. Выбор константы $C$ определен начальными условиями (НУ):

    $t_0 = 0 \quad Q(t_0) = Q_0 = C$

    Тогда, закон -- \fbox{$Q^*(t) = Q_0 e^{-nt}$}

    \Nota Оба закона -- общий $Q(t) = Ce^{-nt}$ и частный $Q^*(t) = Q_0 e^{-nt}$ --
    являются решением дифференциального уравнения:

    \begin{multicols}{2}
        \begin{center}
        Явный вид

        $Q^\prime(t) = kQ$

        В дифференциалах

        $d \ln Q(t) - kdt = 0$
        \end{center}
    \end{multicols}

    \vspace{5mm}

    \hypertarget{freefallingbodyproblem}{}

    \begin{tcolorbox}
        \textit{Pr. 2} \quad Тело массой $m$ брошено вверх с начальной скоростью $v_0$. Нужно найти закон движения $y = y(t)$.
        Сопротивлением воздуха пренебречь.
    \end{tcolorbox}

    По II закону Ньютона:

    $m\vec{a} = m\vec{g} \Longleftrightarrow \vec{a} = \vec{g}$

    $a = $\fbox{$\frac{d^2 y}{dt^2} = -g$} - дифференциальное уравнение

    \underline{Решение.} \quad $y^{\prime\prime}(t) = -g$

    $(y^{\prime}(t))^\prime = -g$

    $y^{\prime}(t) = -\int g dt = -gt + C_1$

    $y(t) = \int (-gt + C_1) dt = $\fbox{$-\frac{gt^2}{2} + C_1 t + C_2 = y(t)$} - общий закон

    Коэффициенты $C_{1,2}$ ищем из начальных условий

    В задаче нет условия для $y(t_0)$. Возьмем $y_0 = y(t_0) = 0$

    Кроме того $y^\prime(t_0) = v(t_0) = v_0$

    Таким образом, $\begin{cases}y(t_0) = 0 \\ y^\prime(t_0) = v_0\end{cases}$

    Найдем $C_1$: $y^\prime(t_0) = y(0) = -gt_0 + C_1 = v_0 \quad C_1 = v_0$

    Найдем $C_2$: $y(t_0) = y(0) = -\frac{gt^2}{2} + C_1 t + C_2 = C_2 = 0$

    Частный закон: \fbox{$y^*(t) = v_0 t - \frac{gt^2}{2}$}

    \subsubsection{4.1.2. Основные определения}

    \hypertarget{differentialequationdefinition}{}

    \DefN{1} Уравнение $F(x, y(x), y^\prime(x), \dots, y^{(n)}(x)) = 0$ - называется обыкновенным ДУ $n$-ого порядка $(*)$

    \Exs $Q^\prime + nQ = 0$ и $y^{\prime\prime} + g = 0$

    \DefN{2} Решением ДУ $(*)$ называется функция $y(x)$, которая при подстановке обращает $(*)$ в тождество

    \DefNs{2$^\prime$} Если $y(x)$ имеет неявное задание $\Phi(x, y(x)) = 0$, то $\Phi(x, y)$ называется интегралом уравнения $(*)$

    \Notas Разделяют общее решение ДУ - семейство функций, при этом каждое из них - решение; и
    частное решение - отдельная функция

    \DefN{3} Кривая с уравнением $y = y(x)$ или $\Phi(x, y(x)) = 0$ называют интегральной кривой

    \hypertarget{problemCauchy}{}

    \DefN{4} $\begin{cases}y(x_0) = y_0 \\ \vdots \\ y^{(n - 1)}(x_0) = y_0^{(n - 1)}\end{cases}$ - система начальных условий $(**)$

    Тогда $\begin{cases}(*) \\ (**)\end{cases}$ - задача Коши (ЗК)

    \Notas Задача Коши может не иметь решений или иметь множество решений

    \begin{MyTheorem}
        \Ths $y^\prime = f(x, y)$ - ДУ

        $M_0(x_0, y_0) \in D$ - точка, принадлежащая ОДЗ

        Если $f(x, y)$ и $\frac{\partial f}{\partial y}$ непрерывны в $M_0$, то задача Коши

        \[\begin{cases}y^\prime = f(x, y) \\ y(x_0) = y_0\end{cases}\]

        имеет единственное решение $\varphi(x, y) = 0$, удовлетворяющее начальным условиям (без док-ва)
    \end{MyTheorem}

    \Nota Преобразуем ДУ: $\underset{F(x, y(x), y^\prime(x))}{\undergroup{y^\prime - f(x, y)}} = 0$

    См. определения обыкновенных и особых точек

    \DefN{5} Точки, в которых нарушаются условия теоремы, называются особыми, а решения, у которых каждая точка особая,
    называются особыми

    \DefN{6} Общим решением ДУ $(*)$ называется $y = f(x, C_1, C_2, \dots, C_n)$

    \Notas $\Phi(x, y(x), C_1, \dots, C_n) = 0$ - общий интеграл

    \DefN{7} Решением $(*)$ с определенными значениями $C_1^*, \dots, C_n^*$ называется частным

    \Nota Форма записи:

    Разрешенное относительно производной $y^\prime = f(x, y)$

    Сведем к виду: $\frac{dy}{dx} = \frac{P(x, y)}{-Q(x, y)} \Longrightarrow -Q(x, y)dy = P(x, y)dx \Longrightarrow $
    \fbox{$P(x, y)dx + Q(x, y)dy = 0$} - форма в дифференциалах



\end{document}

