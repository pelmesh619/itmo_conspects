$subject$=Физические основы компьютерных \\ и сетевых технологий
$teacher$=Лекции Музыченко Я. Б.
$date$=23.12.2024

\section{15. Электрический ток}

\Def Электрический ток - упорядоченное движение зарядов. Электрический ток может быть обусловлен движением как положительных, так и отрицательных зарядов.
За положительное направление тока принимают направление движения положительных зарядов.

Чтобы ток мог возникнуть в веществе, в нем должны быть свободные заряды. Также одним из условием возникновения тока является разность потенциалов

Движение тока идет от большего потенциала к меньшему.

Ток характеризуется величиной силы тока $I = \frac{dq}{dt}$ - количество зарядов за определенный промежуток времени через поперечное сечение проводника

Тогда $q = \int_{t_1}^{t_2} Idt$

$I = \frac{dq}{dt} = \frac{N \cdot |\overline{e}|}{t} = \frac{n \cdot V \cdot |\overline{e}|}{t} = \frac{n \cdot S \cdot dl \cdot |\overline{e}|}{dt} = n |\overline{e}| v S$

$n = \frac{N}{V}$ - концентрация частиц, $v$ - скорость упорядоченного движения

$\vec{j} = n |\overline{e}| \vec{v}$ - плотность тока, тогда $I = \int \vec{j} d\vec{S}$

Для постоянного тока $\oint \vec{j} d\vec{S} = -\frac{dq}{dt} = 0$ - условие непрерывности

Однородным проводником называется участок проводник, на котором не действуют сторонние силы неэлектрической природы

\textbf{Закон Ома}: \fbox{$I = \frac{U}{R} = \frac{\varphi_1 - \varphi_2}{R}$}

Для однородного проводника цилиндрической природы $R = \frac{\rho l}{S}$, где $\rho$ - удельное сопротивление

$\sigma = \frac{1}{\rho}$ - удельная проводимость \hfill $[\sigma] = \text{Ом}^{-1} \text{м}^{-1} = \text{См}$

Закон Ома также является одним из уравнением Максвелла: $dI = \frac{dU}{dR} = \frac{E dl ds}{\rho dl} = \frac{E ds}{\rho}$.
Получаем $j = \frac{dI}{dS} = \frac{1}{\rho} E = \sigma E$ - закон Ома в дифференциальной форме

Электродвижущая сила (ЭДС) - работа сторонних сил по переносу единичного положительного заряда

$\varepsilon = \frac{A_\text{стор}}{q} \qquad U = \varphi_1 - \varphi_2 = \frac{A_{\text{эл. сил}}}{q}$

$I = \frac{\varepsilon}{R + r}$ - закон Ома для полной замкнутой цепи ($\varphi_1 = \varphi_2$)

\textbf{Закон Джоуля-Ленца}: \fbox{$Q = I^2 R \Delta t = UI \Delta t = \frac{U^2}{R} \Delta t$}

В интегральной форме: $Q = \int I^2(t) Rdt$

$dQ = \frac{U^2}{R} dt = \frac{E^2 dl^2 ds}{\rho dl} dt = \frac{1}{\rho} E^2 dV dt$

Получаем количество теплоты за единицу времени и на единицу объема $\omega = \frac{dQ}{dt dV} = \frac{1}{\rho} E^2$ - заком Джоуля-Ленца в дифференциальной форме

\textbf{Правила Кирхгофа}:

1. Алгебраическая сумма токов, сходящихся в узле, равна нулю: $\sum I_i = 0$

2. Алгебраическая сумма произведений сил токов в отдельных участках произвольного замкнутого контура на их сопротивления равна алгебраической сумме ЭДС, действующих на этом контуре: $\sum I_i R_i = \sum \varepsilon_i$

% картинка мейби

