\documentclass[12pt]{article}
\usepackage{preamble}

\pagestyle{fancy}

\begin{document}
    \clearpage

    \section{X. Программа экзамена в 2024/2025}

    \begin{enumerate}
        \item Предмет изучения физики. Основные понятия механики.  
        \item Способы описания движения: векторный, координатный.  
        \item Траекторный способ описания движения. Тангенциальное и нормальное ускорения. 
        \item Кинематика движения материальной точки по окружности. Плоское движение твердого 
        тела.  
        \item Динамика материальной точки. Системы отсчета. Принцип относительности Галилея. 
        \item Фундаментальные взаимодействия. Сила. Законы Ньютона. 
        \item Импульс материальной точки и системы м.т. II закон Ньютона в импульсной форме.  
        \item Закон сохранения импульса. 
        \item Работа. Мощность. Энергия.  
        \item Потенциальная энергия. Взаимосвязь силы и потенциальной энергии. 
        \item Кинетическая энергия. Взаимосвязь силы и кинетической энергии. 
        \item Консервативные и неконсервативные силы. Закон сохранения энергии.  
        \item Центральное соударение двух тел. Абсолютно упругий и абсолютно неупругий удары.  
        \item Момент инерции твердого тела. Теорема Штейнера. 
        \item Момент импульса. Уравнение моментов. Закон сохранения момента импульса.  
        \item Динамика вращения твердого тела. Аналогии между поступательными и вращательными 
        величинами. 
        \item Закон Кулона. Напряженность электрического поля. Принцип суперпозиции. 
        Непрерывное распределение заряда.  
        \item Теорема Гаусса для вектора напряженности электрического поля в вакууме в 
        интегральной и дифференциальной форме.  
        \item Потенциальность электрического поля. Теорема о циркуляции в интегральной и 
        дифференциальной форме. Выводы из теоремы о циркуляции. 
        \item Взаимосвязь потенциала и напряженности электростатического поля. Уравнение 
        Пуассона.  
        \item Аналогии между гравитационным и электростатическим полем.  
        \item Проводники в электростатическом поле. Принцип электростатической защиты.  
        \item Электрический диполь. Напряженность и потенциал диполя. 
        \item Электрический диполь в электрическом поле. Сила и момент сил, действующих на 
        диполь. Энергия диполя. 
        \item Диэлектрик в электрическом поле. Механизм поляризации. Поляризованность и 
        электрическое смещение. 
        \item Теорема Гаусса для векторов напряженности, электрического смещения и 
        поляризованности.  
        \item Граничные условия для векторов напряженности, электрического смещения, 
        поляризованности на границе раздела двух сред.   
        \item Электрическая емкость уединенного проводника и конденсатора. Расчет емкости 
        плоского, цилиндрического и сферического конденсаторов. 
        \item Электрическая энергия и ее локализация в пространстве. Энергия конденсатора.   
        \item Постоянный ток. Сила и плотность тока. Сторонние силы. Уравнение непрерывности. 
        \item Закон Ома в интегральной и дифференциальной форме. Электродвижущая сила. Правила 
        Кирхгофа. 
        \item Работа и мощность постоянного тока. Закон Джоуля-Ленца в интегральной и 
        дифференциальной форме.
    \end{enumerate}

\end{document}