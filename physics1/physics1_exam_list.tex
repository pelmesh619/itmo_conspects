\documentclass[12pt]{article}
\usepackage{preamble}

\pagestyle{fancy}

\begin{document}
    \clearpage

    \section{X. Программа экзамена в 2024/2025}

    Примерный список тем, которые будут на экзамене

    \begin{enumerate}
        \item Способы описания движения: векторный, координатный, траекторный. Траектория движения, пройденный путь, перемещение. Линейная скорость (средняя и мгновенная) как векторная величина. Среднее и мгновенное значение модуля вектора скорости. Вычисление пройденного пути, его графическое представление.
        \item Линейное ускорение как векторная величина. Нормальное и тангенциальное ускорения, их направления. Полное ускорение, его направление и формула. Формулы кинематики движения с постоянным ускорением: зависимость радиус-вектора от времени, зависимость скорости от времени и формула для разности квадратов скоростей.
        \item Роль начальных условий в кинематике на примерах равномерного и равнопеременного движений.
        \item Вектор угловой скорости: его направление и модуль. Правило буравчика (правого винта). Период обращения и частота при равномерном вращении, их связь с угловой скоростью. Вектор углового ускорения: его направление и модуль. Связь между линейной и угловой скоростью, линейным и угловым ускорением. Формулы кинематики равнопеременного движения материальной точки по окружности (по аналогии с соответствующими формулами поступательного движения).
        \item Инерциальные системы отсчета. Преобразования Галилея. Принцип относительности Галилея. Закон сложения скоростей. I закон Ньютона.
        \item Масса, сила, II и III законы Ньютона. Принцип суперпозиции сил. Единица силы в системе СИ. Силы в механике: гравитационная, кулоновская, упругости, трения, натяжения нити, реакции опоры.
        \item Импульс материальной точки (МТ). Элементарный импульс силы как элементарное приращение импульса. Второй закон Ньютона в импульсной форме. Полный импульс силы как разность конечного и начального импульсов МТ. Центр масс (центр инерции) Основное уравнение динамики системы МТ. Скорость и ускорение центра масс системы МТ. Определение замкнутой механической системы МТ. Закон сохранения импульса.
        \item Элементарная механическая работа и ее представления: с помощью косинуса угла между соответствующими векторами и с помощью проекции одного вектора на другой. Полная механическая работа, ее графическое представление. Примеры вычисления работы сил тяжести, упругости, трения. Средняя и мгновенная мощности. Единицы работы и мощности в системе СИ.
        \item Консервативные и неконсервативные силы. Потенциальное поле сил, его определение. Примеры: поле силы тяжести и поле центральных сил. Примеры неконсервативных сил: трение скольжения, сила сопротивления среды.
        \item Кинетическая и потенциальная энергия МТ и системы МТ. Полная механическая энергия МТ и системы МТ. Связь изменения полной механической энергии с работой неконсервативных сил. Закон сохранения полной механической энергии системы МТ. Центральный удар шаров: абсолютно упругий и абсолютно неупругий удары.
        \item Неинерциальные системы отсчета. Второй закон Ньютона в неинерциальных системах отсчета. Силы инерции: поступательная, центробежная, сила Кориолиса, и примеры их проявления.
        \item Динамика вращения твердого тела. Вектор момента силы относительно неподвижной точки. Момент силы относительно оси (скаляр). Плечо силы. Примеры определения направления вектора момента силы. Пара сил, момент пары сил. Вектор момента импульса МТ. Плечо импульса. Примеры определения направления момента импульса. Момент импульса твердого тела, его аналогия с импульсом МТ.
        \item Момент инерции тела: определение и примеры для сравнительного анализа момента инерции различных модельных систем (кольцо, диск, шар, стержень). Теорема Штейнера, примеры ее применения.
        \item Основной закон динамики вращательного движения твердого тела в векторной форме, его аналогия со вторым законом Ньютона для поступательного движения тела. Закон сохранения момента импульса.
        \item Кинетическая энергия катящегося и вращающегося тела. Работа и мощность в динамике вращательного движения.
        \item Постулаты специальной теории относительности: принцип относительности Эйнштейна и принцип постоянства скорости света. Формулы Лоренца для преобразований координат и времени при переходе от одной инерциальной системы к другой. Преобразование скоростей в релятивистской механике. Преобразование длин отрезков и промежутков времени. Пространственно-временной интервал (интервал между событиями).
        \item Элементы релятивистской динамики. Релятивистское выражение для импульса. Релятивистское уравнение динамики МТ. Релятивистское выражение для энергии. Связь между энергией покоя и массой покоя.Основные положения теории электромагнетизма
        \item Электрические заряды в природе. Закон сохранения электрического заряда. Закон Кулона. Электрическое поле. Напряженность электрического поля. Принцип суперпозиции. Силовые линии вектора напряженности электрического поля. Потенциал электростатического поля. Связь между потенциалом и напряженностью электростатического поля.
        \item Поток вектора напряженности электростатического поля. Теорема Гаусса для вектора напряженности электростатического поля (с доказательством для поля точечного заряда). Применение теоремы Гаусса для вектора напряженности электростатического поля на примере вычисления электростатического поля бесконечной заряженной плоскости.
        \item Проводник в электрическом поле. Электростатическое поле внутри и снаружи проводника. Общая задача электростатики. Уравнение Пуассона. Метод электрических изображений.
        \item Электрический диполь. Поле электрического диполя. Электрический диполь во внешнем поле (сила, момент силы, потенциальная энергия).
        \item Диэлектрик в электрическом поле. Физические основы поляризации диэлектрика: неполярные и полярные диэлектрики, упругая и ориентационная поляризации, образование связанного заряда. Вектор поляризованности. Теорема Гаусса для вектора поляризованности. Граничные условия для вектора поляризованности.
        \item Вектор электрической индукции. Теорема Гаусса для вектора электрической индукции. Теорема о циркуляции для вектора напряженности поля. Диэлектрическая проницаемость среды. Граничные условия для векторов напряженности поля и электрической индукции.
        \item Электроемкость. Электроемкость плоского конденсатора. Формулы для эквивалентной емкости при параллельном и последовательном соединении конденсаторов. Энергия электрического поля. Пример: энергия заряженного плоского конденсатора.
        \item Электрический ток, сила тока, плотность тока, единицы измерения этих величин. Соотношение между плотностью тока и скоростью упорядоченного движения зарядов. Электрическое сопротивление проводника. Удельное сопротивление и удельная проводимость. Единицы измерения. Соединения проводников. Формулы для эквивалентного сопротивления при параллельном и последовательном соединении. Закон Ома для однородного участка цепи в дифференциальной и интегральной формах.
        \item Сторонние силы. Электродвижущая сила. Падение напряжения на участке цепи. Единицы измерения. Закон Ома для неоднородного участка в обеих формах. Правила Кирхгофа Выделение теплоты Джоуля (суть явления). Закон Джоуля-Ленца в дифференциальной и интегральной формах.
        \item Основные законы магнитостатики: закон Био-Савара-Лапласа, теорема о циркуляции магнитной индукции, теорема Гаусса для вектора магнитной индукции. Сила Лоренца и сила Ампера. Магнитный диполь во внешнем поле: сила, момент сил, энергия.
        \item Магнитный дипольный момент в магнитном поле (сила, момент сил, потенциальная энергия).
        \item Намагничение вещества. Магнитные моменты электронов и атомов. Гипотеза Ампера. Классификация магнетиков: диа- и парамагнетики, ферромагнетики.
        \item Намагниченность. Токи намагничивания. Теорема о циркуляции вектора намагниченности Вектор напряженности магнитного поля. Магнитная проницаемость. Теорема о циркуляции вектора напряженности магнитного поля. Граничные условия на границе раздела двух магнетиков.
        \item Закон электромагнитной индукции, правило Ленца. Описание различных способов возникновения электродвижущей силы индукции. Явление самоиндукции. Индуктивность. Пример: индуктивность соленоида.
        \item Взаимная индукция. Взаимная индуктивность. Примеры: затухание тока при размыкании цепи, устройство трансформатора.
        \item Ток смещения (пример - разряд конденсатора). Уравнения Максвелла в интегральной и дифференциальной форме. Физический смысл уравнений Максвелла
    \end{enumerate}

\end{document}