$subject$=Физические основы компьютерных \\ и сетевых технологий
$teacher$=Решение задач из сборника
$date$=

\clearpage

\begin{tcolorbox}
    \textbf{Задача 2.3.5} Сила тока $I(t)$ в проводнике меняется со временем $t$ по
    уравнению $I(t) = 4 + 2t$, где $I$ выражено в амперах и $t$
    в секундах.

    \begin{enumerate}
        \item Какое количество электричества проходит через поперечное сечение
        проводника за время от $t_1 = 2$ с до $t_2 = 6$ с? 
        \item При какой силе постоянного тока через поперечное сечение 
        проводника за это же время проходит
        такое же количество электричества?
    \end{enumerate}

    \begin{UpsideDown}
        \footnotesize
        \underline{Ответ}: 48 Кл, 12 А.
    \end{UpsideDown}
\end{tcolorbox}

Количество заряда за промежуток времени $dt$ определяется как $dq = I dt$, тогда $q = \int_2^6 Idt = 4t + t^2 \Big|_2^6 = 48$

При $I = \mathrm{const}$ получаем, что $I = \frac{q}{t} = \frac{48}{4} = 12$

\bigvspace

\underline{Ответ}: 48 Кл, 12 А.

\begin{tcolorbox}
    \textbf{Задача 2.3.15} Из кусочка алюминия массой $m = 21.2$ г изготавливают
    цилиндрический провод длиной $l = 10$ м. Найти его сопротивление. Каков диаметр провода? 
    Плотность алюминия $\rho_m = 2.70 \cdot 10^3$ кг/мз.

    \begin{UpsideDown}
        \footnotesize
        \underline{Ответ}: 0.32 Ом, 1 мм.
    \end{UpsideDown}
\end{tcolorbox}

Удельное сопротивление алюминия $\rho_R = 0.028 \frac{\text{Ом} \cdot \text{мм}^2}{\text{м}} = 2.8 \cdot 10^{-8} \text{Ом} \cdot \text{м}$

Сопротивление проводника вычисляется по формуле $R = \frac{\rho_R l}{S} = \frac{\rho_R l}{\frac{V}{l}} = \frac{\rho_R l^2 \rho}{m} = 
\frac{2.8 \cdot 10^{-8} \cdot 10^2 \cdot 2.7 \cdot 10^3}{0.0212} = 0.356$ Ом

Диаметр проводника $d = 2r = 2\sqrt{\frac{S}{\pi}} = 2\sqrt{\frac{m}{\pi \rho l}} = 1 \cdot 10^{-3} \ \text{м} = 1 \ \text{мм}$

\bigvspace

\underline{Ответ}: 0.356 Ом, 1 мм.

\begin{tcolorbox}
    \textbf{Задача 2.3.26} Две батареи с ЭДС $\varepsilon_1 = 20$ В, $\varepsilon_2 = 30$ В и внутренним
    сопротивлением $r_1 = 4$ Ом, $r_2 = 6$ Ом соединены параллельно. Каковы
    ЭДС и внутреннее сопротивление источника тока, которым можно заменить 
    эти батареи без изменения тока в нагрузке?

    \begin{UpsideDown}
        \footnotesize
        \underline{Ответ}: 24 В, 2.4 Ом.
    \end{UpsideDown}
\end{tcolorbox}

Источники тока соединены параллельно, тогда их общее внутреннее сопротивление $r = \frac{r_1 r_2}{r_1 + r_2} = \frac{4 \cdot 6}{10}= 2.4$ Ом

Общая сила тока источников равна $I = I_1 + I_2 = \frac{\varepsilon_1}{r_1} + \frac{\varepsilon_2}{r_2} = 10$ А

Тогда у заменяющей батареи должна быть ЭДС $\varepsilon = Ir = 24$ В

\bigvspace

\underline{Ответ}: $24$ В, $2.4$ Ом.


\begin{tcolorbox}
    \textbf{Задача 2.3.56} Аккумулятор замыкается один раз на сопротивление $R_1 = 20$ Ом, 
    другой раз - на сопротивление $R_2 = 5$ Ом. При этом количество
    тепла, выделяющееся во внешней цепи в единицу времени, одинаково.
    Найти внутреннее сопротивление аккумулятора.
    
    \begin{UpsideDown}
        \footnotesize
        \underline{Ответ}: 10 Ом.
    \end{UpsideDown}
\end{tcolorbox}

По закону Джоуля-Ленца $Q = I_1^2 R_1 \Delta t = I_2^2 R_2 \Delta t$, где $I_1 = \frac{\varepsilon}{R_1 + r}$, $I_2 = \frac{\varepsilon}{R_2 + r}$

Получаем $\fraC{\varepsilon^2 R_1}{(R_1 + r)^2} = \frac{\varepsilon^2 R_2}{(R_2 + r)^2}$

$(R_2 + r)^2 \cdot R_1 = (R_1 + r)^2 \cdot R_2 \Longrightarrow 100 + 40r + 4r^2 = 400 + 40r + r^2 \Longrightarrow r = 10$

\bigvspace

\underline{Ответ}: 10 Ом.
