\documentclass[12pt]{article}
\usepackage{preamble}

\pagestyle{fancy}
\fancyhead[LO,LE]{Физические основы компьютерных \\ и сетевых технологий}
\fancyhead[RO,RE]{Лекции Музыченко Я. Б.}

\renewcommand{\thesection}{}

\begin{document}

    \tableofcontents
    \clearpage

        % begin physics1_2024_09_02.tex

    \section{0. Вводная лекция}

    Задается вопрос: зачем обучающимся программистам нужна физика в учебном плане?

    Приводятся цитаты Л. Богуславского, одного из крупнейших IT инвесторов, и Б. Страуструпа, которые считают,
    что такие фундаментальные дисциплины, как математика, физика, иностранный язык, способствуют развитию
    мышления человека

    Такие компании, как Bell Labs и IBM создали прорывные изобретения в области физики, на основе которых
    построены компьютерные технологии

    В 3-ем семестре курс физики будет состоять из классической механики и основ электричества

    В 4-ом семестре будут темы магнетизма, колебаний, волн и волновых процессов

    В 5-ом семестре будут рассматриваться оптика, основы квантовой физики и квантовые вычисления

    Занятия состоят из лекций, практических и лабораторных занятий.
    Всего в 3-ем семестре будут 5 лабораторных работ


    \section{1. Современная физическая картина мира. Кинематика материальной точки}

    \textbf{Физика} - раздел естествознания, изучающий свойства и формы движения материи.
    Под материей понимают вещество и поля.

    Научный метод: сначала проводятся наблюдения и эксперименты, из которых выдвигается гипотеза и ищется
    адекватная математическая модель, эта гипотеза проверяется, и если она подтверждается,
    то формируется \textit{теория}

    Пример - открытие Нептуна: в 1781-1845 годах наблюдались аномалии в движении Урана, в 1845 проведение расчетов
    координат новой планеты, а в 1846 обнаружилась новая планета

    Принцип соответствия (Н. Бор, 1923 г.) - каждая новая теория должна включать предыдущую как частный случай

    Изучаемые объекты: вселенная, галактики, звездные системы и планеты, экосистемы, макротела, молекулы, атомы, ядра,
    элементарные частицы

    Всего в физике существуют 4 фундаментальных взаимодействия:

    \begin{tabular}{c|c|c}
        Взаимодействие   & Квант поля & Область взаимодействия                        \\ \hline
        Гравитационное   & гравитон   & масса                                         \\
        Электромагнитное & фотон      & все заряженные частицы, атомы, электротехника \\
        Слабое           & бозон      & радиоактивный распад                          \\
        Сильное          & глюон      & атомные ядра, фундаментальные частицы         \\
    \end{tabular}

    Механика - раздел физики, изучающий механическое движение, то есть движение тел в пространстве и времени.
    Механическое движение тел ОТНОСИТЕЛЬНО.

    \begin{tabular}{c|c|c|}
        & $\ll 3 \cdot 10^8$ м/с & $\approx 3 \cdot 10^8$ м/с \\ \hline
        $\gg 1$ нм & Классическая           & Релятивистская             \\ \hline
        $\ll 1$ нм & Квантовая              & Квантовая теория поля      \\ \hline
    \end{tabular}

    Материальная точка - тело, размерами которого можно пренебречь в условиях данной задачи

    Абсолютно твердое тело (АТТ) - система материальных точек, расстояние между которыми не меняется
    в процессе движения (деформации в процессе движения пренебрежимо малы)

    Тело отсчета - тело, относительно которого определяется положение других тел в пространстве

    Система отсчета - совокупность тела отсчета, связанной с ним системы координат и синхронизированных между собой часов

    % векторный, координатный и естественный способы

    Степени свободы - число независимых скалярных величин, однозначно определяющих положение тела в пространстве

    Материальная точка: $3$ степени свободы

    Система N материальных точек: $3N$ степени свободы

    АТТ: $6$ степеней свободы

    Система единиц (\deutscht{le System International d'unites}), 1960

    $[t] = $ с \quad $[S, l] = $ м

    7 основных единиц:

    $[S] = $ м \quad $[T] = $ К

    $[m] = $ кг \quad $[\nu] = $ моль

    $[t] = $ с \quad $[l] = $ Кд

    $[q] = $ Кл

    Изначально все физические единицы основывались на материальных предметов, из-за которых точности единиц была низкой,
    но недавно все единицы были переопределены на основе физических констант.

    В природе нет абсолютно точных вычислений. Измерение любой физической величины без погрешности не имеет смысла!
    % end physics1_2024_09_02.tex

    % begin physics1_2024_09_09.tex

    \section{2. Кинематика материальной точки}

    \begin{tcolorbox}[colframe=blue!25, colback=blue!10, title=\textbf{План лекции}]

        \footnotesize
        \begin{itemize}
            \item Основные способы описания движения

            \item Основные понятия кинематики

            \item Кинематика поступательного и вращательного движения

            \item Прямая и обратная задачи кинематики

            \item Численные методы при решении задач
        \end{itemize}
    \end{tcolorbox}

    \Def Кинематика - раздел механики, изучающий движение тел, независимо от причин, вызывающих это движение.

    \Def Траектория - линия, по которой движется материальная точка в пространстве

    \Def Путь - длина траектории

    \Def Перемещение - вектор, проведенный из начальной точки в конечную

    \smallvspace

    \textbf{Способы описания движения}

    \begin{multicols}{3}

        \begin{tcolorbox}
            Векторный способ
        \end{tcolorbox}

        \begin{tcolorbox}
            Координатный способ
        \end{tcolorbox}

        \begin{tcolorbox}
            Естественный (траекторный) способ
        \end{tcolorbox}

    \end{multicols}

    \begin{multicols}{3}

        Положение точки может быть однозначно определено с помощью радиус-вектора

        \smallvspace

        Положение точки может быть однозначно определено с помощью трех скалярных координат

        Положение точки определяется дуговой координатой

        \phantom{.}

    \end{multicols}

    \textbf{Векторный способ}

    $\vec{r_1}, \vec{r_2}$ - радиус-векторы, определяющие положения материальной точки в 1 и 2

    $\Delta \vec{r} = \vec{r_2} - \vec{r_1}$ - перемещение материальной точки

    \Def Скорость - векторная физическая величина, характеризующая быстроту перемещения материальной точки

    Средняя скорость - $\Pair{\vec{v}} = \frac{\Delta \vec{r}}{\Delta t}$

    Мгновенная скорость - $\vec{v} = \lim_{\Delta t \to 0} \frac{\Delta \vec{r}}{\Delta t} = \frac{d \vec{r}}{dt}$

    Средняя путевая скорость - $v_{\text{ср}} = \frac{\Delta S}{\Delta t}$

    \Def Ускорение - векторная физическая величина, характеризующая быстроту изменения скорости материальной точки

    Среднее ускорение - $\Pair{\vec{a}} = \frac{\Delta \vec{v}}{\Delta t}$

    Мгновенное ускорение - $\vec{a} = \lim_{\Delta t \to 0} \frac{\Delta \vec{v}}{\Delta t} = \frac{d \vec{v}}{dt}$

    % годограф скорости

    \textbf{Координатный способ}

    В координатном способе положение точки описано 3 координатами $x, y, z$ (в данном случае в ДПСК)

    $|r| = \sqrt{x^2 + y^2 + z^2}$

    $\vec{r} (t) = r_x (t) \vec{\imath} + r_y (t) \vec{\jmath} + r_z (t) \vec{k} = x(t) \vec{\imath} + y(t) \vec{\jmath} + z(t) \vec{k}$

    $\vec{v} (t) = \frac{d \vec{r}}{dt} = \frac{dx}{dt} \vec{\imath} + \frac{dy}{dt} \vec{\jmath} + \frac{dz}{dt} \vec{k}$

    $\vec{v} (t) = v_x(t)\vec{\imath} + v_y(t) \vec{\jmath} + v_z(t) \vec{k}$

    $|\vec{v}| = \sqrt{v_x^2 + v_y^2 + v_z^2}$


    % ускорение

    Прямая задача:

    $\vec{r}(t), x(t), y(t), z(t) \longrightarrow \vec{v}(t), \vec{a}(t), v_x, v_y, v_z, a_x, a_y, a_z$

    Решением является дифференцирование

    Обратная задача:

    $\vec{a}(t), a_x, a_y, a_z \longrightarrow \vec{v}(t), \vec{r}(t), x(t), y(t), z(t)$

    Для обратной задачи решением является интегрирование

    $\vec{v} = \frac{d\vec{r}}{dt} \quad d\vec{r} = \vec{v}dt \quad \Delta\vec{r} = \int_{t_1}^{t_2} \vec{v}dt$

    \begin{tcolorbox}
        $\vec{r} = \vec{r_0} + \Delta \vec{r} = \vec{r_0} + \int_{t_1}^{t_2} \vec{v}dt$
    \end{tcolorbox}

    Аналогично для ускорения

    Численное решение ОДУ (обыкновенного дифференциального уравнения) $\frac{dy}{dx} = f(x, y)$ на отрезке $[x_0, x_n]$ при условии $y(x_0) = y_0$

    Разбиваем отрезок $[x_0, x_n]$ на конечное число частей введением узловых точек

    Шаг разбиения: $h = \frac{x_N - x_0}{N}$

    По определению производной $\frac{dy}{dx} = \frac{y_{i + 1} - y_i}{h}$, из этого:

    \begin{tcolorbox}
        Формула Эйлера: $y_{i + 1} = y_i + hf(x_i, y_i)$
    \end{tcolorbox}

    $dy = f(x, y) dx$

    $\Delta y = y_{i + 1} - y_i = \int_{x_i}^{x_{i + 1}} f(x, y) dx$

    \textbf{Естественный (траекторный) способ}

    Если траектория точки заранее известна, то положение точки задается дуговой координатой $l(t)$

    $\vec{v} = v_\tau \vec{\tau} \quad v_\tau \frac{dl}{dt} |\vec{\tau}| = 1$

    $\vec{a} = \frac{d\vec{v}}{dt} = \frac{dv_\tau}{dt} \vec{\tau} + \frac{d\vec{\tau}}{dt} v_\tau \quad\quad\quad \frac{d\tau}{dt} = \frac{d\tau}{dl} \cdot \frac{dl}{dt} = \frac{d\tau}{dl} v_\tau$

    $\vec{a} = \frac{d\vec{v}}{dt} = \frac{dv_\tau}{dt} \vec{\tau} + \frac{d\vec{\tau}}{dt} v_\tau^2 $

    $d\tau = \tau d\alpha$

    $dl = R d\alpha \quad\quad\quad d\vec{\tau} \uparrow\uparrow \vec{n}$

    $R$ - радиус кривизны траектории

    $\vec{\alpha} = \frac{dv_\tau}{dt} \vec{\tau} + \frac{1}{R} v_\tau^2 \vec{n} \quad\quad\quad \vec{a} = \vec{a_\tau} + \vec{a_n}$

    Тангенциальное ускорение отвечает за изменение модуля скорости, направлено по касательной к траектории движения

    Нормальное ускорение отвечает за изменение направления вектора скорости, направлено к центру кривизны траектории
    % end physics1_2024_09_09.tex



\end{document}

