\documentclass[12pt]{article}
\usepackage{preamble}

\pagestyle{fancy}
\fancyhead[LO,LE]{Физические основы компьютерных \\ и сетевых технологий}
\fancyhead[CO,CE]{28.10.2024}
\fancyhead[RO,RE]{Лекции Музыченко Я. Б.}

\fancyfoot[L]{\scriptsize исходники найдутся тут: \\ \url{https://github.com/pelmesh619/itmo_conspects} \Cat}

\begin{document}
    \section{8. Тепловые явления.}

    Тепловые явления в физике изучают 2 раздела: молекулярная кинетическая теория (МКТ) и термодинамика. 
    МКТ обычно изучает макроскопические системы, используя статистику, а термодинамика описывает
    макросистемы, исходя из глобальных параметров

    Здесь же исследователи выделили основные положения МКТ: все тела состоят из очень большого числа частиц, 
    и эти частицы постоянно находятся в хаотичном, беспорядочном движении - броуновском движении

    Возьмем поршень и посчитаем давление на него - силу на единицу площади:

    $p = \frac{F}{S} \Longrightarrow F = p \cdot S$

    Работа силы давления:

    $dA = \vec{F} \cdot d\vec{s} = p \cdot S \cdot dx$

    $A = \int pS dx = \int p dV$

    Или знакомая со школы формула $A = p\Delta V$ при $p = const$ (изобарный процесс)

    Внутренняя энергия молекул идеального газа $U = \frac{i}{2} \nu R T$

    $i$ - степень свободы

    $\nu$ - количество вещества (в молях)

    $R = 8.31\ \frac{\text{Дж}}{\text{моль} \cdot \text{К}}$ - универсальная газовая постоянная

    $T$ - температура ($T = t^\circ C + 273.15$ К)

    Или для одной молекулы $U = \frac{i}{2} kT$

    $k = 1.38 \cdot 10^{-23}\ \frac{\text{Дж}}{\text{К}}$ - постоянная Больцмана

    На каждую степень свободы молекулы приходится $\frac{1}{2}kT$

    У инертных газов степень свободы - 3

    У двухатомных газов степень свободы - 5 (еще 2 вращательных)

    У молекул газов, состоящих из более 2 атомов, степень свободы - 6

    $Q = A + \Delta U$ - количество теплоты, которое получает газ, преобразовывается в работу и изменение внутренней энергии

    Закон сохранения тепловой энергии - \textit{первое начало термодинамики}

    Равновесное состояние - состояние системы, при котором нет направленного движения вещества или энергии 
    между ее составляющими или между системой и окружающей средой. 
    Обратимым может быть только равновесный процесс

    Второе начало термодинамики гласит: энтропия либо остаётся неизменной, либо возрастает в неравновесных процессах, 
    достигая максимума при установлении термодинамического равновесия

    Элементарное приращение энтропии: $dS = \frac{dQ}{T}$

    $\Delta S = \int_1^2 \frac{dQ}{T}$

    Для обратимых процессов $\Delta S = 0 \Longrightarrow S = \mathrm{const}$

    Для необратимых $\Delta S > 0 \Longrightarrow S \uparrow$

    

\end{document}

