$subject$=Физические основы компьютерных \\ и сетевых технологий
$teacher$=Лекции Музыченко Я. Б.
$date$=16.12.2024

\section{14. Конденсаторы. Энергия электрического поля. Электреты}

\subsection{Конденсаторы}

Рассмотрим еще раз емкости различных конденсаторов:

$C = \frac{q}{U}$ - емкость конденсатора означает, сколько 

\begin{itemize}
    \item Плоский конденсатор

    Внутри плоского конденсатора находится равномерное поле $E = \frac{\sigma}{\varepsilon\varepsilon_0}$

    $C = \frac{q}{U} = \frac{q}{Ed} = \frac{q \varepsilon \varepsilon_0}{\sigma d} = \frac{S \varepsilon \varepsilon_0}{d}$

    \item Сферический конденсатор

    Сферический конденсатор представляет собой две концентрические сферы разного радиуса и противополных знаков заряда.

    $C = \frac{q}{U} = \frac{q}{\int \vec{E}d\vec{r}} = \frac{q}{\int Edr} = \frac{q}{\int_{R_1}^{R_2} \frac{kq}{r^2} dr} = \frac{R_1 R_2}{k(R_2 - R_1)} = \frac{4\pi \varepsilon_0 R_1 R_2}{R_2 - R_1}$

    Внешняя сфера напряженности внутри не создает

    \item Цилиндрический конденсатор

    Цилиндрический конденсатор представляет собой два коаксиальных цилиндра разных радиусов

    $C = \frac{q}{U} = \frac{q}{\int_{R_1}^{R_2} \frac{2k\tau}{r} dr} = \frac{q}{2k\tau \ln\frac{R_2}{R_1}} = \frac{l}{2k\ln\frac{R_2}{R_1}} = \frac{2\pi \varepsilon_0 l}{\ln \frac{R_2}{R_1}}$

\end{itemize}

Параллельное соединение конденсаторов дает сумму их емкостей: $C = C_1 + C_2 + \dots + C_n$

Последовательное соединение конденсаторов дает сумму обратных к емкостям: $\frac{1}{C} = \frac{1}{C_1} + \frac{1}{C_2} + \dots + \frac{1}{C_n}$

\subsection{Энергия электрического поля}

Работа всех сил взаимодействия произвольной системы зарядов равна убыли энергии взаимодействия зарядов этой системы: $dA = -dW$

Энергия всей системы:

$W = W_{12} + W_{13} + \dots + W_{21} + W_{23} + \dots$

Энергия взаимодействия пары зарядов $W_{12} = \frac{1}{2} (W_{12} + W_{21})$

Получаем $W = \frac{1}{2} \sum_{i = 1}^n W_i = \frac{1}{2} \sum_{i = 1}^n q_i \varphi_i = \frac{1}{2}\sum_{i = 1}^n \left(q_i \sum_{j \neq i} \varphi_{ji}\right)$

Энергия взаимодействия всех зарядов определяется как половина суммы произведения зарядов и потенциала остального поля в этом заряде

Для конденсатора $W = \frac{1}{2} \sum q_i \varphi_i = \frac{1}{2} |q \varphi_2 - q \varphi_1| = \frac{qU}{2} = \frac{CU^2}{2} = \frac{q^2}{2C}$

$W = \frac{\varepsilon \varepsilon_0 S}{2d} (Ed)^2 = \frac{\varepsilon \varepsilon_0 E^2}{2} \cdot V = \frac{E D}{2} \cdot V = \frac{D^2}{2\varepsilon \varepsilon_0} \cdot V$

$\varepsilon \varepsilon_0 E^2 = \frac{W}{V} = \omega$ - объемная плотность распределения энергии $(\vec{E} = \mathrm{const})$

В неоднородном поле $\omega = \frac{dW}{dV}$, а $W = \int \omega dV$

Энергия сферического конденсатора:

$\frac{\varepsilon \varepsilon_0 E^2}{2} = \frac{\varepsilon \varepsilon_0 k^2 q^2}{2r^4 \varepsilon^2}$

$W = \int \frac{\varepsilon_0 k^2 q^2}{2\varepsilon} \frac{1}{r^4} dV = \frac{\varepsilon_0 k^2 q^2}{2\varepsilon} \int \frac{4\pi}{r^2} dr$

Также конденсаторы способны быстро разряжаться, высвобождая огромное количество энергии.

\subsection{Электреты}

Пироэлектрики - кристаллические диэлектрики, поляризуемые в отсутствии внешнего поля. При нагревании пироэлектрики поляризуется - пироэлектрический эффект.
Также возможен обратный эффект - нагревание диэлектрика под действием внешнего поля

\mediumvspace

В пьезоэлектриках поляризация возникает под действием механической деформации - прямой пьезоэлектрический эффект. Также 
возможен обратный эффект - деформация под действием внешнего поля

\mediumvspace

Сегнетоэлектрики (или же ферроэлектрики) способны поляризоваться спонтанно в некотором температурном интервале. 
При периодическом изменении напряженности внешнего поля вектор поляризации сегнетоэлектрика изменяется немоментально, 
образуя на графике петлю гистерезиса

С помощью сегнетоэлектриков можно сделать энергонезависимую память (FRAM)

Другие свойства сегнетоэлектриков: в некотором температурном интервале проницаемость может достигать 10000, наличие областей спонтанной поляризации (доменов), 
сложная зависимость $D$ и $P$ от $E$, зависимость $\varepsilon$ от внешнего поля, $D$ и $P$ зависят от предшествующего состояния
