$subject$=Математическая статистика
$teacher$=Лекции Блаженова А. В.
$date$=18.02.2025

\section{Лекция 2.}

\subsection{Точечная оценка}

Пусть имеется выборка $\vec{X} = (X_1, X_2, \dots, X_n)$ объемом $n$

Пусть требуется найти приближенную оценку $\theta^*$ неизвестного параметра $\theta$

Находим ее при помощи некоторой функции обработки данных $\theta^* = \theta^*(X_1, \dots, X_n)$

\Def Такая функция называется статистикой

\Defs А оценка $\theta^*$ называется точечной оценкой

\subsubsection{Свойство точечных оценок}

\begin{enumerate}
    \item Состоятельность

    \Defs Статистика $\theta^* = \theta^*(X_1, \dots, X_n)$ неизвестного параметра называется
    состоятельной, если $\theta^* \overset{p}{\longrightarrow} \theta$ при $n \to \infty$

    \mediumvspace

    \item Несмещенность

    \Defs Оценка $\theta^*$ параметра $\theta$ называется несмещенной, если 
    математическое ожидание $E \theta^* = \theta$
    
    \Notas Оценка $\theta^*$ называется асимптотически несмещенной, если 
    $E \theta^* \overset{p}{\longrightarrow} \theta$ при $n \to \infty$

    \mediumvspace

    \item Эффективность 

    \Defs Оценка $\theta^*_1$ не хуже $\theta^*_2$, если $E (\theta^*_1 - \theta)^2 \leq E (\theta^*_2 - \theta)^2$.
    Или, если $\theta^*_1$ и $\theta^*_2$ несмещенные, то $D \theta^*_1 \leq D \theta^*_2$

    \Defs Оценка $\theta^*$ называется эффективной, если она не хуже всех остальных оценок

    \Notas Не существует эффективной оценки в классе всех возможных оценок

    \begin{MyTheorem}
        \Ths В классе несмещенных оценок существует эффективная оценка
    \end{MyTheorem}

    \mediumvspace

    \item Асимптотическая нормальность

    \Defs Оценка $\theta^*$ параметра $\theta$ называется асимптотически нормальной, если 
    $\sqrt{n} (\theta^* - \theta) \rightrightarrows N(0, \sigma^2 (\theta))$ при $n \to \infty$
    
\end{enumerate}

\subsection{Точечные оценки моментов}

\Def Выборочным средним $\overline{x}$ называется величина $\overline{x} = \frac{1}{n} \sum_{i = 1}^n X_i$

\Defs Выборочной дисперсией $D^*$ называется величина $D^* = \frac{1}{n} \sum_{i = 1}^n (X_i - \overline{x})^2$

\Defs Исправленной дисперсией $S^2$ называется величина $S^2 = \frac{n}{n - 1} D^* = \frac{1}{n - 1} \sum_{i = 1}^n (X_i - \overline{x})^2$

\Defs Выборочным средним квадратическим отклонением называется величина $\sigma^* = \sqrt{D^*}$

\Defs Исправленным средним квадратическим отклонением называется величина $S = \sqrt{S^2}$

\Defs Выборочным $k$-ым моментом называется величина $\overline{x^k} = \frac{1}{n} \sum_{i = 1}^n X_i^k$

\Defs Модой $\mathrm{Mo}^*$ называется варианта $x_k$ с наибольшей частотой $n_k = \max_i (n_1, n_2, \dots, n_m)$

\Defs Выборочной медианой $\mathrm{Me}^*$ называется варианта $x_i$ в середине вариационного ряда $\begin{cases}\mathrm{Me}^* = 
X_{(k)}, & \text{если } n = 2k - 1 \\ \frac{X_{(k)} + X_{(k + 1)}}{2}, & \text{если } n = 2k\end{cases}$

\begin{MyTheorem}
    \Ths $\overline{x}$ - состоятельная несмещенная оценка теоретического матожидания $ЕX = a$

    1) $E \overline{x} = a$

    2) $\overline{x} \overset{p}{\longrightarrow} a$ при $n \to \infty$
\end{MyTheorem}

\begin{MyProof}
    1) $E \overline{x} = E\left(\frac{X_1 + \dots + X_n}{n}\right) = \frac{1}{n} \sum_{i = 1}^n E X_i = 
    \frac{1}{n} n E X_1 = E X_1 = a$

    2) $\overline{x} = \frac{\overline{x}_1 + \dots + \overline{x}_n}{n} \overset{p}{\longrightarrow} a$ 
    согласно Закону Больших Чисел
\end{MyProof}

\Nota Если второй момент конечен, то $\overline{x}$ - асимптотически нормальная оценка. По ЦПТ $\frac{S_n - n E X_1}{\sqrt{n} \sqrt{D X_1}} = \sqrt{n} \frac{\overline{x} - E X_1}{\sqrt{D X_1}} \rightrightarrows N(0, 1)$
или $\sqrt{n} (\overline{x} - E X_1) \rightrightarrows N(0; D X_1)$

\begin{MyTheorem}
    \Ths Выборочный $k$-ый момент является состоятельной несмещенной оценкой теоретического $k$-ого момента

    1) $\overline{E X^k} = E X^k$

    2) $\overline{X^k} \overset{p}{\longrightarrow} X^k$
\end{MyTheorem}

Это следует из предыдущей теоремы, если взять $X^k$ вместо $X$

\begin{MyTheorem}
    \Ths Выборочной дисперсией $D^*$ и $S^2$ являются состоятельными оценками теоретической дисперсией, при этом $D^*$ - смещенная оценка, а
    $S^2$ - несмещенная оценка
\end{MyTheorem}

\begin{MyProof}
    Заметим, что $D^* = \overline{X^2} - \overline{X}^2$

    $E D^* = E(\overline{X^2} - \overline{X}^2) = E\overline{X^2} - E (\overline{X}^2) = 
    E X^2 - E (\overline{X}^2)$

    Так как $D \overline{X} = E(\overline{X^2}) - (E \overline{X})^2$, то $E X^2 - E (\overline{X}^2) = 
    E X^2 - ((E\overline{X})^2 + D\overline{X}) = (E X^2 - EX) - D\overline{X} = D X - D \overline{X} = D X - D \left(\frac{X_1 + \dots + X_n}{n}\right) = 
    DX - \frac{1}{n^2} \sum_{i = 1}^n D X_i = DX - \frac{1}{n^2} n D X_1 = DX - \frac{1}{n} DX = \frac{n - 1}{n} DX$, то есть $D^*$ - смещенная вниз оценка

    $E S^2 = E(\frac{n}{n - 1} D^*) = \frac{n}{n - 1} \frac{n - 1}{n} DX = DX \Longrightarrow S^2$ - несмещенная вниз оценка 

    2. $D^* = \overline{X^2} - \overline{X}^2 \overset{p}{\longrightarrow} E X^2 - (E X)^2 = DX$ - состоятельная оценка

    $S^2 = \frac{n}{n - 1} D^* \overset{p}{\longrightarrow} DX$
\end{MyProof}

\Nota Отсюда видим, что выборочная дисперсия - асимптотически несмещенная оценка. Поэтому при большом (обычно не меньше 100) объеме выборке можно
считать обычную выборочную дисперсию

\subsection{Метод моментов (Пирсона)}

Постановка задачи: пусть имеется выборка объема $n$ неизвестного распределения, но известного типа,
которое задается $k$ параметрами: $\theta = (\theta_1, \theta_2, \dots, \theta_k)$. Требуется дать оценки данным
неизвестным параметрам

Идея метода состоит в том, что сначала находим оценки $k$ моментов, а затем с помощью теоретических формул
из теории вероятности даем оценки этих параметров

Пусть $\vec{X}$ - выборка из абсолютно непрерывного распределения $F_\theta$ с плотностью известного типа, 
которая задается $k$ параметрами $f_\theta (x, \theta_1, \dots, \theta_k)$

Тогда теоретические моменты находим по формуле $m_i = \int_{-\infty}^{\infty} x^i f_\theta (x, \theta_1, \dots, \theta_k) dx = h_i(\theta_1, \dots, \theta_n)$

Получаем систему из $k$ уравнений с $k$ неизвестными. В эти уравнения подставляем найденные оценки
моментов и, решая получившуюся систему уравнений, находим нужные оценки параметров

$\begin{cases}
    \overline{x} = h_1(\theta_1^*, \dots, \theta_n^*) \\ 
    \overline{x^2} = h_2(\theta_1^*, \dots, \theta_n^*) \\ 
    \dots \\
    \overline{x^k} = h_k(\theta_1^*, \dots, \theta_n^*) \\ 
\end{cases}$

\Nota Оценки по методу моментов как правило состоятельные, но часто смещенные

\Ex Пусть $X \in U(a, b)$. Обработав статданные, нашли оценки первого и второго моментов:

$\overline{x} = 2.25; \overline{x^2} = 6.75$

Найти оценки параметров $a^*, b^*$

Плотность равномерного распределения $f_{(a, b)} (x) = \begin{cases}0, & x < a \\ \frac{1}{b - a} & a \leq x \leq b, \\ 0, x > b\end{cases}$

$EX = \int_a^b x \frac{1}{b - a} dx = \frac{a + b}{2}$

$EX = \int_a^b x^2 \frac{1}{b - a} dx = \frac{a^2 + ab + b^2}{3}$

\mediumvspace

Получаем:

$\begin{cases}
    \overline{x} = \frac{a^* + b^*}{2} \\ 
    \overline{x^2} = \frac{a^*^2 + a^* b^* + b^*^2}{3} \\ 
\end{cases} \Longleftrightarrow \begin{cases}
    \frac{a^* + b^*}{2} = 4.5 \\ 
    a^*^2 + a^* b^* + b^*^2 = 20.25 \\ 
\end{cases} \Longleftrightarrow \begin{cases}
    \frac{a^* + b^*}{2} = 4.5 \\ 
    a^* b^* = 0 \\ 
\end{cases} \Longleftrightarrow \begin{cases}
    a^* = 0 \\ 
    b^* = 4.5 \\ 
\end{cases}$



