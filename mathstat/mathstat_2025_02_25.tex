$subject$=Математическая статистика
$teacher$=Лекции Блаженова А. В.
$date$=25.02.2025

\section{Лекция 3.}

\subsection{Метод максимального правдоподобия}

\hypertarget{maximum_likelihood_estimation}{}

Пусть имеется выборка $\vec{X} = (X_1, \dots, X_n)$ из распределения известного типа, определяемого неизвестными параметрами 
$\theta = (\theta_1, \dots, \theta_n)$

Идея метода состоит в следующем: подбираем параметры таким образом, чтобы вероятность получения
данной выборки при случайном эксперименте была наибольшей.

Если распределение дискретное, то $P_{\theta} (X_1 = x_1, X_2 = x_2, \dots, X_n = x_n) = P(X_1 = x_1) \dots P(X_n = x_n)$

\Def Функцией правдоподобия $L(\vec{X}, \theta)$ называется функция $L(\vec{X}, \theta) = P(X_1 = x_1) \dots P(X_n = x_n) = \prod_{i = 1}^n P(X_i = x_i)$ при дискретном распределении

и $L(\vec{X}, \theta) = f_\theta(x_1) \dots f_\theta(x_n) = \prod_{i = 1}^n f_\theta(x_i)$ в абсолютно непрерывном распределении

\Def Логарифмической функцией правдоподобия называется функция $\ln L(\vec{X}, \theta)$

\Nota Так как $y = \ln x$ возврастающая функция, точки максимума совпадают, а такую функцию правдоподобия становится легче дифференцировать

\Def Оценкой максимального правдоподобия $\hat{\theta}$ называется значение $\theta$, при котором функция правдоподобия 
$L(\vec{X}, \theta)$ достигает наибольшего значения (при фиксированных значениях выборки)

\ExN{1} Пусть $\vec{X} = (X_1, \dots, X_n)$ - выборка из распределения Пуассона $\Pi_\lambda$ с неизвестным $\lambda > 0$

\Mem Для распределения Пуассона $P(X = x_i) = \frac{\lambda^{x_i}}{x_i!} e^{-\lambda}$

Получаем функцию максимального правдоподобия $L(\vec{X}, \lambda) = \prod_{i = 1}^n \frac{\lambda^{x_i}}{x_i!} e^{-\lambda} = 
\frac{\lambda^{\sum_{i = 1}^n x_i}}{\prod_{i = 1}^n x_i!} e^{-n\lambda} = \frac{\lambda^{n \overline{x}}}{\prod_{i = 1}^n x_i!} e^{-n\lambda}$

$\ln L(\vec{X}, \lambda) = n \overline{x} \ln \lambda - \ln \prod_{i = 1}^n x_i! - n\lambda$

$\frac{\partial \ln L}{\partial \lambda} = \frac{n \overline{x}}{\lambda} - n = 0 \Longrightarrow \hat{\lambda} = \overline{x}$ - оценка максимального правдоподобия

Убедимся, что этот экстремум - максимум: $\frac{\partial^2 \ln L}{\partial \lambda^2} = -\frac{n \overline{x}}{\lambda} < 0 \Longrightarrow \hat{\lambda} = \overline{x}$ - точка максимума

\ExN{2} Пусть $(X_1, \dots, X_n)$ из $N(a, \sigma^2)$

$f_{a, \sigma^2} (x) = \frac{1}{\sigma \sqrt{2\pi}} e^{-\frac{(x - a)^2}{2\sigma^2}}$

$L(\vec{X}, a, \sigma^2) = \prod_{i = 1}^n \frac{1}{\sigma \sqrt{2\pi}} e^{-\frac{(x_i - a)^2}{2\sigma^2}} = 
\frac{1}{\sigma^n (2\pi)^{\frac{n}{2}}} e^{-\frac{\sum_{i = 1}^n (x_i - a)^2}{2\sigma^2}}$

$\ln L(\vec{X}, a, \sigma^2) = -n\ln \sigma - \frac{n}{2} \ln 2\pi - \frac{1}{2\sigma^2}\sum_{i = 1}^n (x_i - a)^2$

$\frac{\partial \ln L}{\partial a} = -\frac{1}{2\sigma^2} \sum_{i = 1}^n -2(x_i - a) = \frac{1}{\sigma^2} \sum_{i = 1}^n (x_i - a) = \frac{n\overline{x} - na}{\sigma^2}$

$\frac{\partial \ln L}{\partial \sigma} = -\frac{n}{\sigma} - \sum_{i = 1}^n (x_i - a)^2 \frac{1}{2} \cdot (-2) \cdot \sigma^{-3} = \frac{1}{\sigma^3} \sum_{i = 1}^n (x_i - a)^2 - \frac{n}{\sigma}$

$\begin{cases}
    \frac{n\overline{x} - na}{\sigma^2} = 0 \\
    \frac{1}{\sigma^3} \sum_{i = 1}^n (x_i - a)^2 - \frac{n}{\sigma} = 0
\end{cases} \Longrightarrow \begin{cases}
    \hat{a} = \overline{x} \\
    \widehat{\sigma^2} = \frac{1}{n} \sum_{i = 1}^n (x_i - a)^2 = D^*
\end{cases} $

\ExN{3} Пусть $(X_1, \dots, X_n)$ из $U(0, \theta)$. Найти оценку $\theta$ этого распределения.

Воспользуемся методом моментов:

$EX = \frac{a + b}{2} = \frac{\theta}{2} \Longrightarrow \overline{x} = \frac{\theta^*}{2} \Longrightarrow \theta^* = 2\overline{x}$

Воспользуемся методом максимального правдоподобия:

$f_\theta = \begin{cases}0, & x < 0 \\ \frac{1}{\theta}, & 0 \leq x \leq \theta \\ 0, & x > \theta \end{cases}$

$X_{(n)} = \max_i (X_1, \dots, X_n)$

$L(\vec{X}, \theta) = \prod_{i = 1}^n f_\theta (x_i) = 
\begin{cases}
    0, & \text{если } \theta < X_{(n)} \\ 
    \frac{1}{\theta^n}, & \text{если } \theta \geq X_{(n)}
\end{cases}$

$L(\vec{X}, \theta)$ достигает наибольшего значения при наименьшем значении $\theta^n$, то есть при $\hat{\theta} = X_{(n)}$

Сравним оценки:

$\theta^* = 2 \overline{x}$ - несмещенная оценка, так как $E\theta^* = 2E\overline{x} = 2E X = \theta$

$E(\theta^* - \theta)^2 = D\theta^* = D2\overline{x} = 4D\overline{x} = 4\frac{Dx}{n} = \frac{4}{n}\frac{\theta^2}{12} = \frac{\theta^2}{3n}$

Изучим распределение $X_{(n)}$: $F_{X_{(n)}}(x) = P(X_{(n)} < x) = P(X_1 < x, X_2 < x, \dots, X_n < x) = 
P(X_1 < x) \dots P(X_n < x) = F_{X_1}(x) \dots F_{X_n}(x) = F^n_{(x_1)}(x)$

$F_{X_1} (x) = \begin{cases}
    0, & x < 0 \\
    \frac{x}{\theta}, & 0 \leq x \leq \theta \\
    1, & x > \theta
\end{cases} \Longrightarrow F_{X_{(n)}}(x) = \begin{cases}
    0, & x < 0 \\
    \frac{x^n}{\theta^n}, & 0 \leq x \leq \theta \\
    1, & x > \theta
\end{cases} \Longrightarrow f_{X_{(n)}}(x) = \begin{cases}
    0, & x < 0 \\
    n\frac{x^{n - 1}}{\theta^n}, & 0 \leq x \leq \theta \\
    1, & x > \theta
\end{cases}$

$EX_{(n)} = \int_0^\theta x \cdot \frac{nx^{n - 1}}{\theta^n} dx = \frac{n}{\theta^n} \int_0^\theta x^n dx = \frac{n x^{n + 1}}{\theta^n (n + 1)} \Big|_0^\theta = 
\frac{n\theta}{n + 1}$ - смещенная вниз оценка

$\tilde{\theta} = \frac{n + 1}{n} X_{(n)}$ - несмещенная оценка (будем считать, что эффективность не изменилась)

$E\tilde{\theta}^2 = E(\frac{n + 1}{n} X_{(n)})^2 = \frac{(x + 1)^2}{n^2} E X_{(n)} = \frac{(n + 1)^2}{n^2} \int_0^\theta x^2 \frac{n x^{n - 1}}{\theta^n} dx = 
\frac{(n + 1)^2 \theta^2}{n (n + 2)}$

$D\tilde{\theta} = E\tilde{\theta}^2 - (E\tilde{\theta})^2 = \frac{\theta^2}{n(n + 2)}$

$D\tilde{\theta} = \frac{\theta^2}{n(n + 2)} < \frac{\theta^2}{3n} = D\theta^*$

Таким образом, оценка по методу правдоподобия сходится быстрее, чем оценка по методу моментов, поэтому она лучше

Отсюда следует, что при равномерном распределении выборочное среднее не является эффективной оценкой
для математического ожидания; вместо нее половина максимального элемента выборки будет лучше

\Nota Эффективной здесь будет несмещенная оценка $\frac{n + 1}{2n} X_{(n)}$

В общем случае для $U(a, b)$ будет такая эффективная оценка матожидания - $\frac{X_{(1)} + X_{(n)}}{2}$, длины интервала - $\frac{n + 1}{n - 1} (X_{(n)} - X_{(1)})$

\Nota При методе максимального правдоподобия обычно получаем состоятельные и эффективные оценки, но часто смещенные

\subsection{Неравенство Рао-Крамера}

Пусть $X \in F_\theta$ - семейство распределений с параметром $\theta \in \Real$

\Def Носителем семейства распределений $F_\theta$ называется множество $C \subset \Real$
такое, что $P(X \in C) = 1 \ \forall X \in F_\theta$

$f_\theta(x) = \begin{cases}
    \text{плотность } f_\theta(x) \text{ при непрерывном распределении} \\
    P_\theta(X = x) \text{ при дискретном распределении}
\end{cases}$

\hypertarget{fishers_information}{}

\Def Информацией Фишера $I(\theta)$ семейства распределений $F_\theta$ называется величина 
$I(\theta) = E\left(\frac{\partial}{\partial \theta} \ln f_\theta(X)\right)^2$ при условии, что
она существует

\Def Семейство распределений $F_\theta$ называется регулярным, если:

\begin{itemize}
    \item существует носитель $C$ семейства $F_\theta$ такой, что $\forall x \in C \ $ функция $\ln f_\theta(x)$ непрерывно дифференцируема по $\theta$
    \item информация Фишера $I(\theta)$ существует и непрерывна по $\theta$
\end{itemize}

\hypertarget{rao_kramer_inequality}{}

\begin{MyTheorem}
    \Ths Пусть $(X_1, \dots, X_n)$ - выборка объема $n$ из регулярного семейства $F_\theta$,

    $\theta^* = \theta^*(X_1, \dots, X_n)$ - несмещенная оценка параметра $\theta$, дисперсия которой
    $D\theta^*$ ограничена в любой замкнутой ограниченной области параметра $\theta$

    Тогда \fbox{$D\theta^* \geq \frac{1}{n I(\theta)}$}
\end{MyTheorem}

\underline{Следствие}: если при данных услових получили $D\theta^* = \frac{1}{n I(\theta)}$, то оценка $\theta^*$ является эффективной 
(то есть дальше улучшать уже некуда)

\Ex Пусть $(X_1, \dots, X_n)$ из $N(a, \sigma^2)$ (то есть $F_a = N(a, \sigma^2)$, $\sigma^2$ зафиксируем)

Проверим эффективность $a^* = \overline{x}$

Плотность $f_a(x) = \frac{1}{\sigma \sqrt{2\pi}} e^{-\frac{(x - a)^2}{2\sigma^2}}$, носитель - вся прямая $\Real$

$\ln f_a(x) = -\ln \sigma - \frac{1}{2} \ln 2\pi - \frac{(x - a)^2}{2\sigma^2}, \quad\quad a \in (-\infty, \infty)$

$\frac{\partial}{\partial a} \ln f_a(x) = \frac{1}{2\sigma^2} \cdot 2(x - a) = \frac{x - a}{\sigma^2}$ - непрерывна для всех $a \in \Real$

$I(a) = E\left(\frac{\partial}{\partial a} \ln f_a(X)\right)^2 = E\left(\frac{X - a}{\sigma^2}\right)^2 = \frac{1}{\sigma^4} E(X - a)^2 = \frac{E(X - EX)^2}{\sigma^4} = 
\frac{DX}{\sigma^4} = \frac{1}{\sigma^2}$ - непрерывна по $a$

Из этого следует, что $N(a, \sigma^2)$ - регулярное семейство относительно параметра $a$

$Da^* = D\overline{x} = \frac{DX}{n} = \frac{\sigma^2}{n}$ - ограничена по параметру $a$

По неравенству Рао-Крамера $Da^* = \frac{\sigma^2}{n} =\joinrel= \frac{1}{nI(a)} = \frac{1}{n} \sigma^2$; из 
этого следует, что $a^*$ - эффективная оценка параметра $a$

\Nota Аналогично можно показать, что $S^2$ - несмещенная эффективная оценка для параметра $\sigma^2$


