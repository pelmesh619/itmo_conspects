$subject$=Математическая статистика
$teacher$=Лекции Блаженова А. В.
$date$=25.03.2025

\section{Лекция 7.}

\subsection{Критерии для проверки гипотез о распределении}

\subsubsection{Простая параметрическая гипотеза}

Пусть имеется выборка $(X_1, \dots, X_n)$ объема $n$ из неизвестного распределения $\mathcal{F}$. 
Проверяется простая гипотеза $H_0 : \mathcal{F} = \mathcal{F}_1$ против $H_1 : \mathcal{F} \neq \mathcal{F}_1$, где $\mathcal{F}_1$ - распределение известного типа с 
известными нами параметрами $\theta = (\theta_1, \dots, \theta_m)$

\begin{enumerate}[label*=\Roman*. ]
    \item \textbf{Критерий Колмогорова}

    Если $\mathcal{F}_1$ - \underline{абсолютно непрерывное} распределение с функцией распределения $F(x)$, то применим критерий

    $\letsymbol K = \sqrt{n} \sup_x |F^*(x) - F(x)|$, где $F^*(x)$ - выборочная функция распределения

    То есть используем теорему Колмогорова: если $H_0 : \mathcal{F} = \mathcal{F}_1$, то $K =\sqrt{n} \sup_x |F^*(x) - F(x)| 
    \rightrightarrows \mathcal{K}$ - распределение Колмогорова 
    с функцией распределения $F_\mathcal{K}(x) = \sum_{j = -\infty}^\infty (-1)^j e^{-2j^2 x^2}$

    Для уровня значимости $\alpha$ находим квантиль $t_\alpha$ такой, что $P(\xi \geq t_\alpha) = \alpha$, 
    где $\xi \in \mathcal{K}$

    \begin{cases}
        H_0 : \mathcal{F} = \mathcal{F}_1, & \text{если } K < t_\alpha \\
        H_1 : \mathcal{F} \neq \mathcal{F}_1, & \text{если } K \geq t_\alpha \\
    \end{cases}

    \item \textbf{Критерий \enquote{хи-квадрат} Пирсона}

    Пусть выборка разбита на $k$ интервалов $A_1, A_2, \dots, A_k$, $A_i = [a_{i - 1}, a_1)$

    $n_i$ - соответствующая частота интервала

    При распределении $\mathcal{F}_1$ теоретические вероятности попадания в эти интервалы $p_i = F_{\mathcal{F}_1}(a_i) - F_{\mathcal{F}_1}(a_{i - 1})$.
    Тогда $n_i^\prime = p_i \cdot n$ - теоретические частоты


    В качестве статистики критерия выберем $\chi^2_\text{набл.} = \sum_{i = 1}^k \frac{(n_i - n_i^\prime)^2}{n_i^\prime}$

    \begin{MyTheorem}
        \ThNs{Пирсона} Если $H_0 : \mathcal{F} = \mathcal{F}_1$ верна, то $\chi^2_\text{набл.} \rightrightarrows \chi^2_{k - 1}$ - 
        распределение \enquote{хи-квадрат} с $k - 1$ степенями свободы
    \end{MyTheorem}

    Критерий: $\letsymbol t_\alpha$ - квантиль $\chi^2_{k - 1}$ уровня $\alpha$

    \begin{cases}
        H_0 : \mathcal{F} = \mathcal{F}_1, & \text{если } \chi^2_\text{набл.} < t_\alpha \\
        H_1 : \mathcal{F} \neq \mathcal{F}_1, & \text{если } \chi^2_\text{набл.} \geq t_\alpha \\
    \end{cases}

    \Nota Часто обозначают $t_\alpha = \chi^2_\text{теор.}$

    \Notas При этом частота каждого интервала должна быть не меньше 5, а объем выборки - не меньше 50. 
    Число интервалов лучше брать по формуле Стерджесса
\end{enumerate}

\subsubsection{Сложная параметрическая гипотеза}

Здесь мы будем проверять гипотезу $H_0 : \mathcal{F} \in \mathcal{F}_\theta$ против $H_1 : \mathcal{F} \not\in \mathcal{F}_\theta$, где $\mathcal{F}_\theta$ - распределение 
известного типа с неизвестными параметрами

\begin{enumerate}[label*=\Roman*. ]
    \setcounter{enumi}{2}

    \item \textbf{Критерий \enquote{хи-квадрат} Фишера}

    Пусть выборка разбита на $k$ интервалов $A_1, A_2, \dots, A_k$, $A_i = [a_{i - 1}, a_1)$

    $n_i$ - соответствующая частота интервала $A_i$

    \Def Оценка максимального правдоподобия по частотам называется значения неизвестных параметров,
    при которых вероятность появления таких частот явялется максимальной

    Пусть $\hat \theta = (\hat \theta_1, \dots, \hat \theta_n)$ - оценка максимального правдоподобия 
    по частотам неизвестных параметров. Тогда теоретические вероятности попадания в интервал считаем по формуле 
    $p_i = F_{\mathcal{F}_\theta} (a_i) - F_{\mathcal{F}_\theta} (a_{i - 1})$, теоретическая частота - $n_i^\prime = n p_i$

    В качестве статистики критерия берется функция $\chi^2_\text{набл.} = \sum_{i = 1}^k \frac{(n_i - n_i^\prime)^2}{n_i^\prime}$

    \begin{MyTheorem}
        \ThNs{Фишера} 
        
        Если $H_0 : \mathcal{F} \in \mathcal{F}_\theta$ верна, 
        то $\chi^2_\text{набл.} = \sum_{i = 1}^k \frac{(n_i - n_i^\prime)^2}{n_i^\prime} \rightrightarrows \chi^2_{k - m - 1}$ - 
        распределение \enquote{хи-квадрат}, где $m$ - число параметров неизвестного распределения
    \end{MyTheorem}
    
    $\letsymbol t_\alpha$ - квантиль $\chi^2_{k - m - 1}$ уровня $\alpha$

    \begin{cases}
        H_0 : \mathcal{F} = \mathcal{F}_1, & \text{если } \chi^2_\text{набл.} < t_\alpha \\
        H_1 : \mathcal{F} \neq \mathcal{F}_1, & \text{если } \chi^2_\text{набл.} \geq t_\alpha \\
    \end{cases}

    \Nota Часто в качестве оценки неизвестных параметров берется просто оценка максимального правдоподобия
    
\end{enumerate}

\Ex Имеется выборка в виде вариационного ряда $(5.2; \dots; 22.8)$, $n = 120$, при разбиении на $k = 8$ интервалов получили

\begin{tabular}{c|c|c|c|c|c|c|c|c|c}
    $A_i$ & $[5.2, 7.4)$ & $[7.4, 9.6)$ & $[9.6, 11.8)$ & $[11.8, 14)$ & $[14, 16.2)$ & $[16.2, 18.4)$ & $[18.4, 20.6)$ & $[20.6, 22.8)$ & $\sum$ \\
    \hline
    $n_i$ & 12 & 17 & 14 & 13 & 18 & 14 & 13 & 19 & 120
\end{tabular}

Проверить гипотезу о равномерном распределении $H_0 : \mathcal{F} \in U(a, b)$ против $H_1 : \mathcal{F} \not\in U(a, b)$ при уровне значимости $\alpha = 0.05$

Дадим оценку параметров методом максимального правдоподобия: $\hat a = 5.2 \quad \hat b = 22.8$

Теоретическая вероятность будет $p_i^\prime = \frac{1}{8}$, теоретическая частота - $n_i = 15$

$\chi^2_\text{набл.} = \sum_{i = 1}^k \frac{(n_i - n_i^\prime)^2}{n_i^\prime} = \frac{(12 - 15)^2}{15} + 
\frac{(17 - 15)^2}{15} + \frac{(14 - 15)^2}{15} + \frac{(13 - 15)^2}{15} + \frac{(18 - 15)^2}{15} + 
\frac{(14 - 15)^2}{15} + \frac{(19 - 15)^2}{15} = 3.2$

При $\alpha = 0.05$ и $S = k - m - 1 = 5$ квантиль $\chi^2_S$ уровня $\alpha$ равен $t_\alpha = 11.07$

Так как $\chi^2_\text{набл.} < t_\alpha$, нулевая гипотеза о равномерном распределении принимается

\subsection{Критерии для проверки однородности}

\begin{enumerate}[label*=\Roman*. ]
    \setcounter{enumi}{3}

    \item \textbf{Критерий Колмогорова-Смирнова}

    Пусть имеются 2 независимых выборки $(X_1, \dots, X_n)$ и $(Y_1, \dots, Y_m)$ объемов $n$ и $m$ из неизвестных непрерывных распределений $\mathcal{F}$ и $\mathcal{J}$

    Проверяется $H_0 : \mathcal{F} = \mathcal{J}$ (данные однородны) против $H_1 : \mathcal{F} \neq \mathcal{J}$

    В качестве статистики критерия берется функция $K = \sqrt{\frac{nm}{n + m}} \sup_x |F^*(x) - G^*(x)|$, где $F^*$ и $G^*$ - 
    соответствующие выборочные функции распределения

    \begin{MyTheorem}
        \ThNs{Колмогорова-Смирнова}

        Если $H_0 : \mathcal{F} = \mathcal{J}$ верна, то $K \rightrightarrows \mathcal{K}$ - распределение Колмогорова
    \end{MyTheorem}

    Критерий: $t_\alpha$ - квантиль $\mathcal{K}$ уровня значимости $\alpha$

    \begin{cases}
        H_0 : \mathcal{F} = \mathcal{J}, & \text{если } K < t_\alpha \\
        H_1 : \mathcal{F} \neq \mathcal{J}, & \text{если } K \geq t_\alpha \\
    \end{cases}

\end{enumerate}

\subsubsection{Проверки однородности выборок из нормальных совокупностей}


\begin{enumerate}[label*=\Roman*. ]
    \setcounter{enumi}{4}

    \item \textbf{Критерий Фишера}

    Пусть имеют две независимые выборки $(X_1, \dots, X_n)$ и $(Y_1, \dots, Y_m)$ объемов $n$ и $m$ из нормальных распределений $X \in N(a_1, \sigma^2_1)$ и $Y \in N(a_2, \sigma^2_2)$

    Проверяется $H_0 : \sigma_1 = \sigma_2$ против $H_1 : \sigma_1 \neq \sigma_2$

    В качестве статистики критерия берется функция $K = \frac{S_X^2}{S_Y^2}$, где $S_X^2$, $S_Y^2$ - соответствующие исправленные дисперсии, причем
    $S_X^2 \geq S_Y^2$

    \begin{MyTheorem}
        \Ths Если $H_0 : \sigma_1 = \sigma_2$ верна, то $K = \frac{S_X^2}{S_Y^2} \in F(n - 1, m - 1)$ - распределение Фишера-Снедекера
    \end{MyTheorem}

    \begin{MyProof}
        По пункту 3 основной теоремы $\frac{(n - 1)S^2}{\sigma^2} \in \chi^2_{n - 1}$. Если $H_0$ верна, то $K = \frac{S_X^2}{S_Y^2} = \frac{(n - 1) S_X^2 \sigma_2^2}{\sigma_1^2 (m - 1) S_Y^2} \frac{(m - 1)}{(n - 1)} = \frac{\frac{\chi^2_{n - 1}}{n - 1}}{\frac{\chi^2_{m - 1}}{m - 1}} \in F(n - 1, m - 1)$
    \end{MyProof}

    $t_\alpha$ - квантиль $F(n - 1, m - 1)$ уровня $\alpha$

    \begin{cases}
        H_0 : \sigma_1 = \sigma_2, & \text{если } K < t_\alpha \\
        H_1 : \sigma_1 \neq \sigma_2, & \text{если } K \geq t_\alpha \\
    \end{cases}

    \Nota Здесь критерий согласия работает чуть иным образом: при верной альтернативной гипотезе $K = \frac{S_X^2}{S_Y^2} \ConvergesInProbability \frac{\sigma_1^2}{\sigma_2^2} > 1$

    Если нулевая гипотеза отклоняется, то отклоняется общая гипотеза об однородности. А если основная гипотеза принимается, то 
    применяем критерий Стьюдента

    \item \textbf{Критерий Стьюдента}

    Пусть $(X_1, \dots, X_n)$ и $(Y_1, \dots, Y_m)$ из нормальных распределений $X \in N(a_1, \sigma^2)$ и $Y \in N(a_2, \sigma^2)$

    Проверяется $H_0 : a_1 = a_2$ против $H_1 : a_1 \neq a_2$

    \begin{MyTheorem}
        \Ths $\sqrt{\frac{nm}{n + m}} \frac{(\overline x - a_1) - (\overline y - a_2)}{\sqrt{\frac{(n - 1) S_X^2 + (m - 1) S^2_Y}{n + m - 2}}} \in T_{n + m - 2}$
    \end{MyTheorem}

    \begin{MyProof}
        По пункту 5 основной теоремы считаем, что числитель и знаменатель независимы 

        $\sqrt{\frac{nm}{n + m}} \frac{(\overline x - a_1) - (\overline y - a_2)}{\sqrt{\frac{(n - 1) S_X^2 + (m - 1) S^2_Y}{n + m - 2}}} = 
        \sqrt{\frac{nm}{n + m}} \frac{\frac{\overline x - a_1}{\sigma} - \frac{\overline y - a_2}{\sigma}}{\sqrt{\frac{(n - 1) S_X^2 + (m - 1) S^2_Y}{\sigma^2 (n + m - 2)}}}$

        По пункту 1 основной теоремы $\sqrt{n} \frac{\overline x - a_1}{\sigma}, \sqrt{m} \frac{\overline y - a_2}{\sigma} \in N(0, 1) \Longrightarrow \\
        \frac{\overline x - a_1}{\sigma} \in N\left(0, \frac{1}{n}\right), \frac{\overline y - a_2}{\sigma} \in N\left(0, \frac{1}{m}\right) \Longrightarrow \\
        \frac{\overline x - a_1}{\sigma} - \frac{\overline y - a_2}{\sigma} \in N\left(0, \sqrt{\frac{n + m}{nm}}\right) \Longrightarrow \\
        \sqrt{\frac{nm}{n + m}} \left(\frac{\overline x - a_1}{\sigma} - \frac{\overline y - a_2}{\sigma}\right) \in N(0, 1)$

        По пункту 3 основной теоремы $\frac{(n - 1)S^2_X}{\sigma^2} \in \chi^2_{n - 1}, \frac{(m - 1)S^2_Y}{\sigma^2} \in \chi^2_{m - 1} \Longrightarrow
        \frac{(n - 1) S_X^2 + (m - 1) S^2_Y}{\sigma^2 (n + m - 2)} \in \frac{\chi^2_{n + m - 2}}{m + n - 2}$

        Из этого $\sqrt{\frac{nm}{n + m}} \frac{(\overline x - a_1) - (\overline y - a_2)}{\sqrt{\frac{(n - 1) S_X^2 + (m - 1) S^2_Y}{n + m - 2}}} \in \frac{N(0, 1)}{\frac{\chi^2_{n + m - 2}}{n + m - 2}} = T_{n + m - 2}$
    \end{MyProof}

    В качестве статистики возьмем $\sqrt{\frac{nm}{n + m}} \frac{(\overline x - a_1) - (\overline y - a_2)}{\sqrt{\frac{(n - 1) S_X^2 + (m - 1) S^2_Y}{n + m - 2}}}$, 
    по теореме при $a_1 = a_2$ получаем, что $K \in T_{n + m - 2}$

    Если верна альтернативная гипотеза, то $K \longrightarrow \infty$

    Критерий: $t_\alpha$ - квантиль $|T_{n + m - 2}|$ уровня $\alpha$

    \begin{cases}
        H_0 : a_1 = a_2, & \text{если } K < t_\alpha \\
        H_1 : a_1 \neq a_2, & \text{если } K \geq t_\alpha \\
    \end{cases}

    \Nota Если при обоих критериях согласились с нулевой гипотезой, то соглашаемся с гипотезой об однородности выборок

    \Notas Критерий хорошо работает, если выборки из нормальных распределении (или очень близких к ним)

\end{enumerate}

