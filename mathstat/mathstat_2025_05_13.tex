$subject$=Математическая статистика
$teacher$=Лекции Блаженова А. В.
$date$=13.05.2025

\section{Лекция 14.}

\subsection{Моделирование случайных величин}

\subsubsection{Датчики случайных чисел}

Пусть $\eta \in U(0, 1)$

\Def Члены последовательности $y_1, y_2, \dots, y_n$, которые можно рассматривать как экспериментальные данные случайной величины $\eta$, называются псевдослучайными числами. А устройство или алгоритмы для их получения - датчики (или генераторы) случайных чисел

\begin{enumerate}[label*=\Roman*. ]
    \item Физические датчики

    Примерами физических датчиков может быть секундомер, случайной величиной может быть миллисекунды случайно остановленного секундомер, или европейская рулетка с 36 ячейками для шарика

    Простейшим способом случайное число можно сгенерировать, подбрасывая монету, где за одну сторону принимается 1, а за другую - 0

    \begin{MyTheorem}
        \Ths Случайная величина $\eta \in U(0, 1) \Longleftrightarrow$ разряды $\xi_i$ в ее двоичной записи $\sum_{i = 1}^n 2^{-1} \xi_i$ имеют распределение Бернулли $B_{\frac{1}{2}}$
    \end{MyTheorem}

    Если требуется точность $2^{-n}$, то бросать монету нужно $n$ раз 

    Физические датчики довольно примитивными, однако их значения сложно передавать компьютеру. Также две последовательности случайных чисел могут отличаться при одинаковых условиях

    Знаменитый пример использования физических датчиков - \href{https://www.cloudflare.com/learning/ssl/lava-lamp-encryption/}{использование лавовых ламп компанией Cloudflare} для генерации чисел для использования в шифровании

    \item Таблицы случайных чисел

    Пусть имеется таблица псевдослучайных чисел - результат работы некоторого датчика. Случайным образом выбиралась строка и столбец и, начиная с этого места, выбиралась последовательность случайных чисел

    Такой способ использовался до широкого появления компьютеров и сейчас устарел

    \hypertarget{mathematical_rng}{}

    \item Математические датчики

    Обычно математический датчик - это рекуррентная последовательность вида $y_n = f(y_{n - 1})$, однако способ генерации может быть значительно сложнее

    В качестве математического датчика разберем мультипликативный датчик: задается большое число $m$, начальное число $k_0$ и множитель $a$, при этом $m$, $k_0$ и $a$ - взаимно простые

    Последовательность $y_n$ задается по формуле:

    $\begin{cases}
        k_n \equiv a k_{n - 1} (\operatorname{mod} m) & \text{- остаток от деления }ak_{n - 1}\text{ на }m \\
        y_n = \frac{k_n}{m} \in (0, 1)
    \end{cases}$

    Есть рекомендация использовать $m = 2^{31} - 1 = 2147483647$, $a = 630360016$ или $764261123$ (алгоритм Дж. Фишмана и Л. Мура)

    \mediumvspace

    Позднее предложили такой датчик (алгоритм Вичмена-Хилла):

    $a_1 = 171$, $m_1 = 30269$, $a_2 = 172$, $m_2 = 30307$, $a_3 = 170$, $m_3 = 30323$ 

    Члены $y_n^\prime, y_n^{\prime\prime}, y_n^{\prime\prime\prime}$ задаются как мультипликативные датчики, а итоговое значение вычисляется как дробная часть от их суммы: $y_n = \{y_n^\prime + y_n^{\prime\prime} + y_n^{\prime\prime\prime}\}$

    Преимущества: работает быстрее (числа меньше), период датчика - $3 \cdot 10^{13}$, а алгоритма Фишмана и Мура - $2 \cdot 10^9$

    Наблюдателю кажется, что, чем сложнее алгоритм датчика, тем более случайным он кажется

\end{enumerate}

\subsection{Моделирование непрерывного распределения}

\subsubsection{Квантильное преобразование (или метод обратной функции)}

На курсе теории вероятности выражали такую теорему:

\begin{MyTheorem}
    \Ths Пусть $F(x)$ - непрерывная, строго возрастающая функция распределения

    Если случайная величина $\eta \in U(0, 1)$, то $\xi = F^{-1}(\eta)$ имеет функцию распределения $F(x)$
\end{MyTheorem}

\Ex Показательное распределение $E_\alpha$: $F(x) = \begin{cases}0, & x < 0 \\ 1 - e^{-\alpha x}, & x \geq 0\end{cases}$

Обратной к ней функцией будет $x = -\frac{1}{\alpha} \ln (1 - y) \Longrightarrow \xi = -\frac{1}{\alpha} \ln \eta \in E_\alpha$

\Ex Нормальное распределение

Если $\xi \in N(0, 1)$, то $\sigma \xi + a \in N(a, \sigma^2)$, поэтому достаточно уметь моделировать стандартное нормальное распределение

$F(x) = \frac{1}{\sqrt{2\pi}} \int_{-\infty}^x e^{-\frac{z^2}{2}} dz$

Если $\eta \in U(0, 1)$, то $\xi = F^{-1}(\eta) \in N(0, 1)$ (\texttt{НОРМ.СТ.ОБР} в Excel)

\Nota Этот алгоритм простой и универсальный, но не самый эффективный

\subsubsection{Нормальные случайные числа}

\begin{enumerate}[label*=\Roman*. ]
    \item На основе ЦПТ

    Пусть $\eta_i \in U(0, 1)$. Тогда $E\eta = \frac{1}{2}$, $D\eta = \frac{1}{12}$, $S_n = \eta_1 + \dots + \eta_n$ и по ЦПТ:

    $\frac{S_n - nE\eta}{\sqrt{n D\eta}} = \frac{S_n - \frac{n}{2}}{\sqrt{\frac{n}{12}}} \rightrightarrows N(0, 1)$

    Уже при $n = 12$ получается неплохое приближение $S_{12} - 6 \approx N(0, 1)$ 

    \item Точное моделирование пары независимых значений $N(0, 1)$

    \begin{MyTheorem}
        Пусть $\eta_1, \eta_2 \in U(0, 1)$ и независимы. Тогда случайные величины $X, Y \in N(0, 1)$ и независимы:

        $X = \sqrt{-2\ln \eta_1} \cos (2 \pi \eta_2)$

        $Y = \sqrt{-2\ln \eta_1} \sin (2 \pi \eta_2)$
    \end{MyTheorem}
    
    \begin{MyProof}
        Пусть $X, Y \in N(0, 1)$ независимы, тогда плотность $f_{X, Y}(x, y) = f_X(x)f_Y(y) = \frac{1}{2\pi} e^{-\frac{1}{2} (x^2 + y^2)}$

        Перейдем к полярным координатам:

        $x = r \cos \varphi, y = r \sin \varphi, |J| = r$

        Плотность: $f_{\Phi, R}(\varphi, r) = \frac{1}{2\pi} e^{-\frac{1}{2}r^2} r$

        Так как $\varphi \in U(0, 2\pi)$, то $f_{\Phi}(\varphi) = \frac{1}{2\pi}$ и $f_{\Phi, R}(\varphi, r) = \underset{f_\Phi}{\underbrace{\frac{1}{2\pi}}} \cdot \underset{f_R}{\underbrace{r e^{-\frac{1}{2}r^2}}} \Longrightarrow \Phi$ и $R$ - независимы 

        Смоделируем $f_R$ методом обратной функции

        $F_R(r) = \int_0^r r e^{-\frac{r^2}{2}} dr = -\int_0^r r e^{-\frac{r^2}{2}} d\left(-\frac{r^2}{2}\right) = e^{-\frac{r^2}{2}} \Big|_0^r = 1 - e^{-\frac{r^2}{2}} = \eta_1$. Тогда $r = \sqrt{-2\ln \eta_1}$

        Аналогично $F_\Phi(\varphi) = \int_0^\varphi \frac{1}{2\pi} d\varphi = \frac{\varphi}{2\pi} = \eta_2 \Longrightarrow \varphi = 2\pi \eta_2$
    \end{MyProof}

    Такой датчик использует всего три сложных операции (логарифм, корень и косинус) и моделирует нормальное распределение очень точно
\end{enumerate}

\subsubsection{Быстрый показательный датчик}

\begin{MyTheorem}
    \Ths Пусть независимая случайная величина $\eta_1, \eta_2, \dots, \eta_n, \eta_{n+1}, \dots, \eta_{2n - 1} \in U(0, 1)$, $\xi_1, \xi_2, \dots, \xi_{n - 1}$ - упорядоченные значения случайных величин $\eta_{n + 1}, \dots, \eta_{2n - 1}$, $\xi_0 = 0$, $\xi_n = 1$

    Тогда случайная величина $\mu_i = -\frac{1}{\alpha}(\xi_i - \xi_{i - 1}) \ln (\eta_1 \cdot \eta_2 \cdot \dots \cdot \eta_n) \in E_\alpha$ и независимы, $1 \leq i \leq n$
\end{MyTheorem}

Экономим на вычислении значения (считаем логарифм один раз), но теряем при сортировке. Алгоритм оптимален при $n = 3$: $\eta_1, \eta_2, \eta_3, \eta_4, \eta_5 \in U(0, 1)$ и $\eta_4 < \eta_5$, тогда $\mu_1 = -\frac{1}{\alpha} \eta_4 \ln (\eta_1 \eta_2 \eta_3); \mu_2 = -\frac{1}{\alpha} (\eta_5 - \eta_4) \ln (\eta_1 \eta_2 \eta_3); \mu_3 = -\frac{1}{\alpha} (1 - \eta_5) \ln (\eta_1 \eta_2 \eta_3)$

\subsection{Моделирование дискретных случайных величин}

\begin{enumerate}[label*=\Roman*. ]
    \item Общий метод (или квантильное преобразование)

    Пусть $\xi$ - дискретная случайная величина с законом распределения $P(\xi = x_i) = p_i$

    Разбиваем единичный отрезок на отрезки длин $p_1, p_2$ и так далее

    Пусть $r_m = \sum_{i = 1}^m P_i$ - границы отрезков

    Если $\eta_i \in U(0, 1)$ и $\eta_i \in [r_{i-1}, r_{i})$, то $\xi_i = x_i$

    В частности, так можно смоделировать распределение Бернулли: делим отрезок на две части; если $\eta_i < 1 - p$, то $\xi_i = 0$, если $\eta_i \geq 1 - p$, то $\xi_i = 1$

    \item Биномиальное распределение

    $B_{n,p}: P(\xi = k) = C_n^k p^k q^{n - k}$, $k = 0, 1, \dots, n$ - число успехов при $n$ экспериментах
    
    Берем $n$ значений датчика $y_1, \dots, y_n \in U(0, 1)$. Если $y_i < 1 - p$, то $z_i = 0$, если $y_i \geq 1 - p$, то $z_i = 1$

    Тогда $\xi_i = \sum_{i = 1}^n z_i \in B(n, p)$

    \item Геометрическое распределение

    $G_p: P(\xi = k) = p q^{k - 1}$ - номер первого успеха в испытании

    Берем серию значений датчика $y_i \in U(0, 1)$ до тех пор, пока не будет $y_i \geq 1 - p$. Тогда $\xi = i$ - номер эксперимента

    \item Распределение Пуассона

    $\Pi_\lambda: P(\xi = k) = \frac{\lambda^k}{k!} e^{-\lambda}, k = 0, 1, \dots$

    Используем тот факт, что распределения показательное и Пуассона довольно тесно связаны

    \begin{MyTheorem}
        \Ths Пусть $\mu_1, \mu_2, \dots, \mu_n$ - независимые случайные величины с распределением $E_\lambda$

        Пусть $S_n = \mu_1 + \dots + \mu_n, N = \max n$ такое, что $S_N \in [0, 1]$

        Тогда $N \in \Pi_\lambda$
    \end{MyTheorem}

    Алгоритм: проводим серию $k$-ую серию испытаний до тех пор, пока $\prod^k_{i = 1} y_i < e^{-\lambda}$, тогда $\xi_i = k$

\end{enumerate}


