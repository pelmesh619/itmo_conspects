$subject$=Дополнительные главы \\ высшей математики
$date$=07.03.2025
$teacher$=Лекции Далевской О. П.

Заметим, что $w = e^z = e^x (\cos y + i \sin y)$ - многолистная функция, а $w = \Ln z = \ln \rho + i (\arg z + 2\pi k)$ - многозначная

\ExN{6} Тригонометрические и гиперболические

\begin{multicols}{2}
    \begin{center}
        $\sin z = \frac{e^{iz} - e^{-iz}}{2i}$

        $\cos z = \frac{e^{iz} + e^{-iz}}{2}$

        $\sh z = \frac{e^{z} - e^{-z}}{2}$

        $\ch z = \frac{e^{z} + e^{-z}}{2}$
    \end{center}
\end{multicols}

\Nota Рассмотрим уравнение $\sin z = A \in \Complex$

$\frac{e^{iz} - e^{-iz}}{2i} = A \Longrightarrow e^{2iz} - 2iAe^{iz} - 1 = 0$

При $t = e^{iz}$ получаем квадратное уравнение, у которого в $\Complex$ всегда будет два корня. 
Это значит, что в $\Complex$ $\sin$ и $\cos$ принимают любые значения (то есть $|\sin z| > 1$)

\subsection{2.4. Дифференцирование ФКП}

\Def $w = f(z), w : D \subset \Complex \longrightarrow \Complex, z_0 \in D$. \textbf{Производная} функции
$w(z_0)$ - это предел $\lim_{\Delta z \to 0} \frac{f(z_0 + \Delta z) - f(z_0)}{\Delta z} = \lim_{\Delta z \to 0} \frac{\Delta f}{\Delta z}$, 
если он существует и не зависит от пути $z \to z_0$

\Mem Дифференцирование $y = f(x)$:

\begin{tabular}{ll}
    В Ф$_1$П: & $\Delta y = f(x_0 + \Delta x) - f(x_0) \underset{A \in \Real}{=} A\Delta x + o(\Delta x)$ \\
    В Ф$_2$П: & $\Delta f = f(x_0 + \Delta x, y_0 + \Delta y) - f(x_0, y_0) = A\Delta x + B \Delta y + \alpha_1 + \alpha_2 = 
    \frac{\partial f}{\partial x} \Delta x + \frac{\partial f}{\partial y} \Delta y + o(\Delta x) + o(\Delta y)$
\end{tabular}

\Def $f(z)$ называется дифференцируемой в точке $z_0$, если $\exists f^\prime(z_0) \in \Complex$

\Defs Дифференцируемая в точке $z_0$ функция $w = f(z)$, производная $f^\prime(z_0)$ которой непрерывна в $z_0$,
называется аналитической (или аналитичной) функцией в $z_0$

\begin{MyTheorem}
    \Ths Критерий аналитичности (или Условие Коши-Римана)

    \begin{center}
        $f(x) = u(x, y) + i v(x, y)$ аналитична в точке $z_0 = x + iy$ 
        
        \rotatebox{90}{$\Longleftrightarrow$}
    
        $\exists \frac{\partial u}{\partial x}, \frac{\partial u}{\partial y}, \frac{\partial v}{\partial x}, \frac{\partial v}{\partial y}$ непрерывны в $z$ и
        $\begin{cases}\frac{\partial u}{\partial x} = \frac{\partial v}{\partial y} \\ \frac{\partial u}{\partial y} = -\frac{\partial v}{\partial x}\end{cases}$
    \end{center}

    Причем, $f^\prime(z) = u_x + i v_x = v_y - i u_y = u_x - i u_y = v_y + i v_x$
\end{MyTheorem}

\begin{MyProof}
    \fbox{\Longrightarrow} \ $f$ аналитическая в $z$ $\Longleftrightarrow \exists$ непрерывная 
    $f^\prime(z) = \\ = \lim_{\Delta z \to 0} \frac{\Delta f}{\Delta z} = [\text{предел не зависит от пути}] = 
    \lim_{\Delta x \to 0} \frac{f(x + \Delta x, y) - f(x, y)}{\Delta x} = \\
    = \lim_{\Delta x \to 0} \frac{u(x + \Delta x, y) + i v(x + \Delta x, y) - u(x, y) - v(x, y)}{\Delta x} = \\
    = \lim_{\Delta x \to 0} \left(\frac{u(x + \Delta x, y) - u(x, y)}{\Delta x} + i \frac{v(x + \Delta x, y) - v(x, y)}{\Delta x}\right) = \\
    = \lim_{\Delta x \to 0} \left(\frac{\Delta_x u}{\Delta x} + i \frac{\Delta_x v}{\Delta x}\right) = 
    \lim_{\Delta x \to 0} \frac{\Delta_x u}{\Delta x} + i \lim_{\Delta x \to 0} \frac{\Delta_x v}{\Delta x} = 
    \frac{\partial u}{\partial x} + i \frac{\partial v}{\partial x} = u_x + i v_x$

    Аналогично при $i \Delta y \to 0$ получаем $\lim_{\Delta y \to 0} \frac{f(x, y + \Delta y) - f(x, y)}{i \Delta y} = \\
    = \lim_{\Delta y \to 0} \left(\frac{u(x, y + \Delta y) - u(x, y)}{i \Delta y} + \frac{v(x, y + \Delta y) - v(x, y)}{\Delta y}\right) = 
    \lim_{\Delta y \to 0} \frac{\Delta_x v}{\Delta y} - i \lim_{\Delta y \to 0} \frac{\Delta_x u}{\Delta x} = 
    v_y - i u_y$

    Итак, $f^\prime(z) = u_x + i v_x = v_y + i u_y$

    Отсюда $u_x = v_y$ и $u_y = -v_x$

    \mediumvspace 

    \fbox{\Longleftarrow} \ $\exists$ непрерывные $u_x, u_y, v_x, v_y \Longleftrightarrow u(x, y), v(x, y)$ 
    дифференцируемы в $(x, y)$, тогда $\Delta u = u_x \Delta x + u_y \Delta y + \alpha_1 (x, y, \Delta x, \Delta y) + 
    \alpha_2 (x, y, \Delta x, \Delta y)$

    $\alpha_1 = o(\Delta \rho), \quad \Delta \rho = (\Delta x)^2 + (\Delta y)^2$

    $\Delta v = v_x \Delta x + v_y \Delta_y + \beta_1 + \beta_2$

    $\Delta f = u(x + \Delta x, y + \Delta y) + iv(x + \Delta x, y + \Delta y) - (u(x, y) + i v(x, y)) = 
    u_x \Delta x + u_y \Delta y + \alpha + i(v_x \Delta x + v_y \Delta y + \beta)$

    $\frac{\Delta f}{\Delta z} = \frac{u_x \Delta x + u_y \Delta y + i v_x \Delta x + i v_y \Delta y}{\Delta x + i \Delta y} + 
    \frac{\alpha + i \beta}{\Delta x + i \Delta y} = \frac{u_x(\Delta x + i \Delta y)}{\Delta x + i \Delta y} + 
    \frac{v_x(i \Delta x - \Delta y)}{\Delta x + i \Delta y} + \frac{\alpha + i\beta}{\Delta x + i \Delta y} = 
    u_x + v_x i + \frac{\alpha + i \beta}{\Delta x + i \Delta y}$

    Тогда $\lim_{\Delta z \to 0} \frac{\Delta f}{\Delta z} = 
    u_x + i v_x + \lim_{\substack{\Delta x \to 0 \\ \Delta y \to 0}} \frac{\alpha + i \beta}{\Delta x + i \Delta y} = 
    u_x + i v_x \Longleftrightarrow f^\prime = u_x + i v_y$, существует и непрерывна в $(x, y)$
\end{MyProof}

\Nota Используя Условие Коши-Римана, получим равенство $u_x + i v_x = v_y - i u_y = u_x - i u_y = v_y + i v_x$

\Notas Коши-Риман в ПСК:

\begin{cases}
    \frac{\partial u}{\partial \rho} = \frac{1}{\rho} \frac{\partial v}{\partial \varphi} \\
    \frac{\partial u}{\partial \varphi} = -\frac{1}{\rho} \frac{\partial v}{\partial \rho}
\end{cases}

Тогда $f^\prime(z) = \frac{1}{z} \left(\frac{\partial v}{\partial \varphi} - i \frac{\partial u}{\partial \varphi}\right) = \frac{\rho}{z} \left(\frac{\partial u}{\partial \rho} + i \frac{\partial v}{\partial \rho}\right)$

\begin{MyProof}
    $u_\rho = u_x \frac{\partial x}{\partial \rho} + u_y \frac{\partial y}{\partial \rho} = u_x \cos \varphi + u_y \sin \varphi$

    $v_\varphi = v_x \frac{\partial x}{\partial \varphi} + v_y \frac{\partial y}{\partial \varphi} = 
    -\rho v_x \sin \varphi + \rho v_y \cos \varphi = \rho u_y \sin \varphi + \rho u_x \cos \varphi = \rho u_\rho$

    \Lab $\frac{\partial u}{\partial \varphi} = -\frac{1}{\rho} \frac{\partial v}{\partial \rho}$
\end{MyProof}

\mediumvspace

\underline{Свойства аналитических функций}

Пусть $f, g$ - аналитические функции, тогда:

\begin{enumerate}[label=\arabic*$^\circ$]
    \item Линейность: $af + bg$ - аналитическая
    \item Композиция: $f(g(z))$ - аналитическая
    \item Произведение: $f \cdot g$ - аналитическая
\end{enumerate}

\Nota Доказательства свойств элементарные, все сводится к сведению к $u$ и $v$

\Ex $w = \frac{1}{z} = \frac{1}{x + iy} = \frac{x}{x^2 + y^2} - i \frac{y}{x^2 + y^2} = u(x, y) + i v(x, y)$

$u_x = \frac{x^2 + y^2 - 2x^2}{(x^2 + y^2)^2} = \frac{y^2 - x^2}{(x^2 + y^2)^2}$

$v_y = \frac{-x^2 - y^2 + 2y^2}{(x^2 + y^2)^2} = \frac{y^2 - x^2}{(x^2 + y^2)^2} = u_x$

$u_y = \frac{-2xy}{(x^2 + y^2)^2}$

$v_x = \frac{2xy}{(x^2 + y^2)^2} = -u_y$

Таким образом, $\frac{1}{z}$ - аналитическая функция

\Ex $w = \overline{z} = x - iy$

$u_x = 1$, $v_y = -1 \neq u_x$ - не аналитическая функция


