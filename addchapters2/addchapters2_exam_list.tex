\clearpage

\section{X. Программа экзамена в 2024/2025}

\subsection{Часть 1.}

\begin{enumerate}
    \item Функции комплексного переменного (ФКП). Геометрия ФКП.
    \item Предел, непрерывность функции комплексного переменного.
    \item Элементарные функции. Степень и корень. Показательная функция и логарифм. Тригонометрические и гиперболические функции.
    \item Дифференцирование и аналитичность функции комплексной переменной. Условия Коши-Римана. Свойства аналитических функций.
    \item Понятие конформного отображения. Геометрический смысл производной.
    \item Интеграл по комплексной переменной. Теорема Коши. Первообразная.
\end{enumerate}

    

\subsection{Часть 2.}

\begin{enumerate}
    \setcounter{enumi}{6}

    \item Числовые ряды. Регулярная функция. Функциональные ряды. Признаки сходимости.
    \item Степенные ряды. Теорема Абеля. Ряд Тейлора.
    \item Бесконечная дифференцируемость регулярной функции. Формула n-ой производной.
    \item Ряды Лорана в конечной и бесконечно удаленной точке. Главная и правильная части.
    \item Классификация изолированных особых точек однозначной аналитической функции.
    \item Вычет аналитической функции в изолированной особой точке. Определение и формулы вычисления вычета. Основные теоремы теории вычетов.
\end{enumerate}