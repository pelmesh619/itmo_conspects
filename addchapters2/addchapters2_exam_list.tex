\clearpage

\section{X. Программа экзамена в 2024/2025}

\subsection{Часть 1.}

\begin{enumerate}
    \item Функции комплексного переменного (ФКП). Геометрия ФКП.

    \Defs \hyperlink{complex_function}{Функция} $f : D \subset \Complex \longrightarrow G \subset \Complex \ \overset{def}{\Longleftrightarrow} \ $ отображение такое, 
    что $\forall z \in D \ \exists w \in G \ | \ f(z) = w$

    \Defs Если $\forall z \in D \ \exists! w \in G$, то $f$ называется однозначной функцией

    \Defs Если $\forall z_1, z_2 \in D (z_1 \neq z_2) \Longrightarrow f(z_1) \neq f(z_2)$, 
    то $f$ называется однолистной функцией
    
    \Exs $w = \sqrt{z}$ - неоднозначная функция, $w = z^2$ - неоднолистная функция

    \Notas Если $f(z)$ однозначна и однолистна, то $f(z)$ - взаимно однозначное соответствие (биекция). Тогда $\exists g(x) \ | \ g(f(x)) = x$
    
    Комплексную функцию $f(z)$ можно представить как $u(x, y) + i v(x, y)$, где $x + iy = z$

    \item Предел, непрерывность функции комплексного переменного.

    \DefNs{\hyperlink{complex_limit}{Предел}} $L \in \Complex, f : D \longrightarrow G, \quad L \overset{def}{=} \lim_{z \to z_0} f(z) \Longrightarrow
    \forall \varepsilon > 0 \ \exists \underset{\delta = \delta(\varepsilon)}{\delta > 0} \ \Big| \ z \in D, z \in \overset{\circ}{U}_\delta(z_0) \ f(x) \in U_\varepsilon(L)$

    В определении существование и значение $L$ не должно зависеть от пути, по которому $z$ приближается к точке сгущения $z_0$.
    Может быть так, что для любого направления стремления предел есть, но в общем смысле не существует

    \DefNs{\hyperlink{continuity}{Непрерывность} функций в точке $z_0$} $f : D \longrightarrow G, z_0 \in D$, $f(z)$ называется непрерывной в $z_0$, если $\lim_{z \to z_0} f(z) = f(z_0)$

    На языке приращений: $\Delta f = f(z_0 + \Delta z) - f(z_0) \underset{\Delta z \to 0}{\longrightarrow} 0$

    \item Элементарные функции. Степень и корень. Показательная функция и логарифм. Тригонометрические и гиперболические функции.

    \hyperlink{elementary_functions}{Элементарные функции}:

    \ExN{1} Линейная $f(z) = az + b, \qquad\qquad a, b \in \Complex \quad a \neq 0$

    Эта функция однозначная, однолистная $\Longrightarrow \exists f^{-1}(z) = g(z) = \frac{z - b}{a}$. 
    Линейная функция - композиция из поворота, растяжения и сдвига
    
    \ExN{2} Степенная $w = z^n, \quad n \in \Natural$ - однозначная, может быть неоднолистной
    
    Для $n \in \Rational$ функция становится неоднозначной
    
    \Exs $w = z^2 \qquad\qquad z = \rho e^{i\varphi}, w = \rho^2 e^{2i\varphi}$. Область однолистности $z^2$ - множество точек, для которых $\arg z \in [0; \pi)$.
    Точку $w = 0$ называют точкой разветвления
    
    \Exs $w = z^{-1} = \frac{1}{z}$, при этом $w(0) = \infty, w(\infty) = 0$.    
    Для $z \in \Complex \setminus \{0\}$ функция обратима
    
    Преобразование $|w| = \frac{1}{|z|}$ называется инверсией, а $\arg w = -\arg z$ дает симметрию относительно $\RE z$
    
    \ExN{3} Рациональная $f(z) = \frac{P_n(z)}{Q_m(z)}, \qquad\qquad n, m \in \Natural$
    
    \ExN{4} Показательная $w = e^z = e^x \cdot e^{iy} = e^x (\cos y + i \sin y)$ - многолистная функция. \underline{Свойства}: 
    
    \begin{enumerate}
        \item $e^{z_1 + z_2} = e^{z_1} \cdot e^{z_2}$
        \item $\left(e^{z_1}\right)^{z_2} = e^{z_1 z_2}$
        \item $e^{z + 2\pi i} = e^{z} \cdot e^{2\pi i} = e^z$ - показательная функция периодична с периодом $2\pi i$
    \end{enumerate}
    
    \ExN{5} Логарифмическая $w = \Ln z$ - многозначная функция
    
    Если $e^w = e^{u + vi} = e^u (\cos v + i \sin v) = z = |z| e^{i\arg z}$, то $u = \ln |z|$, $v = \arg z + 2\pi k$. Тогда \fbox{$\Ln z = \ln |z| + i (\arg z + 2\pi k)$}
    
    $\ln z = \Ln z$ при $k = 0$ - т. н. главное значение
    
    \ExN{6} Тригонометрические и гиперболические

    \begin{multicols}{2}
        \begin{center}
            $\sin z = \frac{e^{iz} - e^{-iz}}{2i}$

            $\cos z = \frac{e^{iz} + e^{-iz}}{2}$

            $\sh z = \frac{e^{z} - e^{-z}}{2}$

            $\ch z = \frac{e^{z} + e^{-z}}{2}$
        \end{center}
    \end{multicols}

    В $\Complex$ область значений этих функций является $\Complex$ - эти функции не ограничены


    \item Дифференцирование и аналитичность функции комплексной переменной. Условия Коши-Римана. Свойства аналитических функций.

    \Def $f(z)$ называется \hyperlink{derivative}{дифференцируемой} в точке $z_0$, если $\exists f^\prime(z_0) \in \Complex$

    \Defs Дифференцируемая в точке $z_0$ функция $w = f(z)$, производная $f^\prime(z_0)$ которой непрерывна в $z_0$, называется аналитической (или аналитичной) функцией в $z_0$

    \begin{MyTheorem}
        \Ths \hyperlink{cauchy_riemann}{Критерий аналитичности (или Условие Коши-Римана)}
    
        \begin{center}
            $f(x) = u(x, y) + i v(x, y)$ аналитична в точке $z_0 = x + iy$ 
            
            \rotatebox{90}{$\Longleftrightarrow$}
        
            $\exists \frac{\partial u}{\partial x}, \frac{\partial u}{\partial y}, \frac{\partial v}{\partial x}, \frac{\partial v}{\partial y}$ непрерывны в $z$ и
            $\begin{cases}\frac{\partial u}{\partial x} = \frac{\partial v}{\partial y} \\ \frac{\partial u}{\partial y} = -\frac{\partial v}{\partial x}\end{cases}$
        \end{center}
    
        Причем, $f^\prime(z) = u_x + i v_x = v_y - i u_y = u_x - i u_y = v_y + i v_x$
    \end{MyTheorem}
    
    \underline{\hyperlink{analytic_function_properties}{Свойства аналитических функций}}: пусть $f, g$ - аналитические функции, тогда:

    \begin{enumerate}[label=\arabic*$^\circ$]
        \item Линейность: $af + bg$ - аналитическая
        \item Композиция: $f(g(z))$ - аналитическая
        \item Произведение: $f \cdot g$ - аналитическая
        \item $f(z)$ аналитична в $D \ (f : D \longrightarrow D^\prime)$, $f^\prime(z) \neq 0 \ \forall z \in D$. 
        Тогда $\exists g(w) = f^{-1}(z) \ (g : D^\prime \longrightarrow D)$ и $\forall z_0 \in D \ f^\prime_z (z_0) = \frac{1}{g^\prime_w (w_0)}$, где $w_0 = w(z)$
        
        \item $f(z) = u(x, y) + i v(x, y)$ аналитична в $D$. Тогда $u(x, y), v(x, y)$ -- гармонические функции в $D$

        \item Если $f(z) = u(x, y) + i v(x, y)$ аналитична в $D$ и известна $u(x, y)$ или $v(x, y)$, то $f(z)$ определяется однозначно с точностью до $\operatorname{const}$
    \end{enumerate}

    \item Понятие конформного отображения. Геометрический смысл производной.

    \Def \hyperlink{conformal_map}{Конформное отображение} -- отображение $w(z)$, сохраняющее углы (между образами и прообразами) и постоянство растяжений

    \begin{MyTheorem}
        \Ths Условия конформности: \begin{cases}\text{дифференцируемость} \\ \text{однолистность} \\ f^\prime(z) \neq 0 \text{ в } D\end{cases} $\Longleftrightarrow$ конформно
    \end{MyTheorem}

    \hyperlink{geometrical_meaning_of_derivative}{Геометрический смысл}: Функция $w = f(z)$ в точке $z_0$ поворачивает точку у окрестности на угол $\alpha = \arg f^\prime(z_0)$ и растягивает отрезки $[z_0, z]$ в $k = |f^\prime(z_0)|$ раз


    \item Интеграл по комплексной переменной. Теорема Коши. Первообразная.
    
    В $\Complex$ задана кусочно-гладкая кривая $K$ (с концами в точках $M$ и $N$) параметрическими уравнениями: 
    \begin{cases}
        x = \varphi(t) & \qquad t \in [\alpha, \beta] \subset \Real \\
        y = \psi(t) & \qquad \varphi, \psi \text{ -- } \Real \text{-функции} \\
    \end{cases}

    Тогда $z(t) = \varphi(t) + i \psi(t)$ - задание $K$ в $\Complex$. Введем отображение $w = f(z)$, действующее на $K$

    Определим интегральные суммы:

    \begin{enumerate}
        \item Дробление отрезка $MN$ на частичные дуги: $M = z_0, z_1, \dots, z_{n - 1}, z_n = N$

        Тогда $\alpha = t_0, t_1, \dots, t_{n - 1}, t_n = \beta$

        \item Выбор средних точек в отрезках кривой $\zeta_i = (\xi_i, \eta_i)$

        \item Сопоставим интегральную сумму $\sigma_n = \sum_{i = 1}^n f(\zeta_i) \Delta z_i$

        \item \hyperlink{complex_integral}{Интегралом} от $w = f(z)$ по кривой $K$ называется $\lim_{\substack{n \to \infty \\ \tau = \max \Delta z_i \to 0}} \sigma_n = 
        \int_K f(z) dz$, если он существует, конечен и не зависит от способа разбиения, выбора средних точек и т. д.
    \end{enumerate}

    При этом интеграл можно представить как $\lim_{n \to \infty} \sigma_n = \int_K f(z) dz = \int_K udx - vdy + i \int_K udy + vdx$

    \begin{MyTheorem}
        \ThNs{\hyperlink{cauchy_for_simply_connected_space}{Теорема Коши для односвязной области}} $f(z)$ аналитическая и однозначная в односвязной области $D$
    
        Если $f(z)$ непрерывна на $\Gamma_D$, то $\oint_{\Gamma_D} f(z) dz = 0$
    \end{MyTheorem}

    \begin{MyTheorem}
        \ThNs{\hyperlink{cauchy_for_connected_space}{Теорема Коши для многосвязной области}} Дана многосвязная область $D$, $f(z)$ - аналитична в $D$ и непрерывна на $\Gamma_D$
    
    
        Граница $\Gamma_D = \Gamma_0 \cup \Gamma_1 \cup \Gamma_2 \cup \dots \cup \Gamma_n$, где положительным обходом области
        считается тот, при котором область обхода слева
    
        Тогда $\int_{\Gamma_D^+} f(z) dz = 0$ или $\int_{\Gamma_0 \CounterClockwiseArrow} f(z) dz = \sum_{i = 1}^n \int_{\Gamma_i \CounterClockwiseArrow} f(z) dz$ 
    \end{MyTheorem}

    По теореме Барроу $\Phi(x) = \int_{x_0}^x f(t) dt$ - интеграл с переменным верхним пределом

    Тогда $\Phi(x)$ - дифференцируема, и $\Phi^\prime(x) = f(x)$, то есть $\Phi(x)$ - \hyperlink{antiderivative}{первообразная} $f(x)$

    \begin{MyTheorem}
        \Ths $f(z)$ непрерывна в односвязной области $D$ и $\forall \Gamma \subset D \ \int_\Gamma f(z) dz = 0$

        Тогда при фиксированном $z_0 \in D \ \Phi(z) = \int_{z_0}^z f(\zeta) d\zeta$ аналитична в $D$ и $\Phi^\prime(z) = f(z)$
    \end{MyTheorem}

    $\Phi(z) = \int_{z_0}^z f(\zeta) d\zeta$ называют первообразной для $f(z)$

    Следствие - формула Ньютона-Лейбница: \fbox{$\int_{z_1}^{z_2} f(\zeta) d\zeta = \Phi(z_2) - \Phi(z_1)$}

\end{enumerate}

    

\subsection{Часть 2.}

\begin{enumerate}
    \setcounter{enumi}{6}

    \item Числовые ряды. Регулярная функция. Функциональные ряды. Признаки сходимости.

    \DefNs{1} $z_1 + z_2 + \dots + z_n + \dots = \sum_{n = 1}^\infty z_n$, где $z_n \in \Complex$ - \hyperlink{complex_series}{числовой ряд}

    \DefNs{2} Сумма ряда - $S = \lim_{n \to \infty} S_n = \lim_{n \to \infty} \sum_{k = 1}^n z_k$. Если сумма существует и конечна, то ряд называют сходящимся. 

    \Defs $f(z)$ называется регулярной в точке $z_0$, если $f(z_0) = \sum_{n = 1}^\infty c_n$, где $c_n \in \Complex$

    \Def $\sum_{n = 1}^\infty u_n(z)$, где $u_n(z) : \ D \subset \Complex \longrightarrow \Complex$ - функциональный ряд

    \begin{MyTheorem}
        \ThNs{Признак Вейерштрасса}
    
        $\exists \sum_{n = 1}^\infty \alpha_n$, $\alpha_n \in \Real_0^+$, $\sum_{n = 1}^\infty \alpha_n \in \Real$,
        $|u_n(z)| \leq \alpha_n \ \forall z \in D \Longrightarrow \sum_{n = 1}^\infty u_n(z)$ сходится равномерно в $D$
    \end{MyTheorem}

    \item Степенные ряды. Теорема Абеля. Ряд Тейлора.

    \Def \hyperlink{complex_power_series}{Степенной ряд} - $\sum_{n = 0}^\infty c_n (z - a)^n \qquad \left(a = 0: \sum_{n = 0}^\infty c_n z^n\right)$, $c_n \in \Complex$. Область сходимости - круг с центром $a$, $|z - a| \leq R$ - радиус сходимости 

    $\lim_{n \to \infty} \left|\frac{c_{n + 1} (z - a)^{n + 1}}{c_n (z - a)^n}\right| = \lim_{n \to \infty} \left|\frac{c_{n + 1}}{c_n}\right| |z - a| < 1 \Longrightarrow |z - a| < \left|\frac{c_n}{c_{n + 1}}\right|$

    \begin{MyTheorem}
        \ThNs{\hyperlink{abels_theorem}{Абеля}} 
        
        Если степенной ряд сходится в точке $z_1$, то он сходится абсолютно и равномерно 
        в любой точке $z_2$ такой, что $|z - z_1| > |z - z_2|$

        Если степенной ряд расходится в точке $z_1$, то он расходится
        в любой точке $z_2$ такой, что $|z - z_1| < |z - z_2|$
    \end{MyTheorem}

    \begin{MyTheorem}
        \ThNs{Почленное дифференцирование суммы ряда}

        $\sum_{n = 0}^\infty c_n z^n = f(z)$ - сходящийся в круге радиуса $R \neq 0$. Тогда $f(z)$ дифференцируема и 
        $f^\prime(z) = \sum_{n = 0}^\infty n c_n z^{n - 1}$
    \end{MyTheorem}

    Ряд Тейлора $f(z) \underset{|z - a| < \rho}{=} \sum_{n = 0}^\infty \frac{f^{(n)} (a)}{n!} (z - a)^n$

    \item Бесконечная дифференцируемость регулярной функции. Формула n-ой производной.

    \begin{MyTheorem}
        \Ths $f(z)$ аналитическая в области $D$ $\Longrightarrow f(z)$ регулярна в области $D$
    \end{MyTheorem}

    Для функции $f(z)$ в точке $a \ $ $\frac{1}{2\pi i} \int_{\gamma_\rho} \frac{f(\zeta) d\zeta}{(\zeta - a)^{n + 1}} = \frac{f^{(n)}(a)}{n!}$

    \item Ряды Лорана в конечной и бесконечно удаленной точке. Главная и правильная части.
    \item Классификация изолированных особых точек однозначной аналитической функции.
    \item Вычет аналитической функции в изолированной особой точке. Определение и формулы вычисления вычета. Основные теоремы теории вычетов.
\end{enumerate}