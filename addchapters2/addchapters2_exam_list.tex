\clearpage

\section{X. Программа экзамена в 2024/2025}

\begin{enumerate}
    \item Комплексное число и действия над комплексными числами.
    \item Функции комплексного переменного (ФКП). Геометрия ФКП.
    \item Предел, непрерывность функции комплексного переменного.
    \item Элементарные функции. Степень и корень. Показательная функция и логарифм. Тригонометрические и гиперболические функции.
    \item Дифференцирование и аналитичность функции комплексной переменной. Условие Коши-Римана. Свойства аналитических функций.
    \item Интеграл по комплексной переменной. Основные свойства. Теорема Коши.
    \item Интеграл Коши. Интегралы, зависящие от параметра.
    \item Числовые ряды. Функциональные ряды. Равномерно сходящиеся ряды функций комплексной переменной. Теоремы Вейерштрасса.
    \item Степенные ряды. Теорема Абеля. Ряд Тейлора.
    \item Единственность определения аналитической функции. Нули аналитической функции.
    \item Ряды Лорана.
    \item Классификация изолированных особых точек однозначной аналитической функции.
    \item Вычет аналитической функции в изолированной особой точке. Определение и формулы вычисления вычета. Основная теорема теории вычетов.
    \item Вычисление определенных интегралов с помощью вычетов.
\end{enumerate}