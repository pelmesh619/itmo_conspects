\clearpage

\section{X. Программа экзамена в 2024/2025}

\subsection{Часть 1.}

\begin{enumerate}
    \item Функции комплексного переменного (ФКП). Геометрия ФКП.

    \Defs \hyperlink{complex_function}{Функция} $f : D \subset \Complex \longrightarrow G \subset \Complex \ \overset{def}{\Longleftrightarrow} \ $ отображение такое, 
    что $\forall z \in D \ \exists w \in G \ | \ f(z) = w$

    \Defs Если $\forall z \in D \ \exists! w \in G$, то $f$ называется однозначной функцией

    \Defs Если $\forall z_1, z_2 \in D (z_1 \neq z_2) \Longrightarrow f(z_1) \neq f(z_2)$, 
    то $f$ называется однолистной функцией
    
    \Exs $w = \sqrt{z}$ - неоднозначная функция, $w = z^2$ - неоднолистная функция

    \Notas Если $f(z)$ однозначна и однолистна, то $f(z)$ - взаимно однозначное соответствие (биекция). Тогда $\exists g(x) \ | \ g(f(x)) = x$
    
    Комплексную функцию $f(z)$ можно представить как $u(x, y) + i v(x, y)$, где $x + iy = z$

    \item Предел, непрерывность функции комплексного переменного.

    \DefNs{\hyperlink{complex_limit}{Предел}} $L \in \Complex, f : D \longrightarrow G, \quad L \overset{def}{=} \lim_{z \to z_0} f(z) \Longrightarrow
    \forall \varepsilon > 0 \ \exists \underset{\delta = \delta(\varepsilon)}{\delta > 0} \ \Big| \ z \in D, z \in \overset{\circ}{U}_\delta(z_0) \ f(x) \in U_\varepsilon(L)$

    В определении существование и значение $L$ не должно зависеть от пути, по которому $z$ приближается к точке сгущения $z_0$.
    Может быть так, что для любого направления стремления предел есть, но в общем смысле не существует

    \DefNs{\hyperlink{continuity}{Непрерывность} функций в точке $z_0$} $f : D \longrightarrow G, z_0 \in D$, $f(z)$ называется непрерывной в $z_0$, если $\lim_{z \to z_0} f(z) = f(z_0)$

    На языке приращений: $\Delta f = f(z_0 + \Delta z) - f(z_0) \underset{\Delta z \to 0}{\longrightarrow} 0$

    \item Элементарные функции. Степень и корень. Показательная функция и логарифм. Тригонометрические и гиперболические функции.

    \hyperlink{elementary_functions}{Элементарные функции}:

    \ExN{1} Линейная $f(z) = az + b, \qquad\qquad a, b \in \Complex \quad a \neq 0$

    Эта функция однозначная, однолистная $\Longrightarrow \exists f^{-1}(z) = g(z) = \frac{z - b}{a}$. 
    Линейная функция - композиция из поворота, растяжения и сдвига
    
    \ExN{2} Степенная $w = z^n, \quad n \in \Natural$ - однозначная, может быть неоднолистной
    
    Для $n \in \Rational$ функция становится неоднозначной
    
    \Exs $w = z^2 \qquad\qquad z = \rho e^{i\varphi}, w = \rho^2 e^{2i\varphi}$. Область однолистности $z^2$ - множество точек, для которых $\arg z \in [0; \pi)$.
    Точку $w = 0$ называют точкой разветвления
    
    \Exs $w = z^{-1} = \frac{1}{z}$, при этом $w(0) = \infty, w(\infty) = 0$.    
    Для $z \in \Complex \setminus \{0\}$ функция обратима
    
    Преобразование $|w| = \frac{1}{|z|}$ называется инверсией, а $\arg w = -\arg z$ дает симметрию относительно $\RE z$
    
    \ExN{3} Рациональная $f(z) = \frac{P_n(z)}{Q_m(z)}, \qquad\qquad n, m \in \Natural$
    
    \ExN{4} Показательная $w = e^z = e^x \cdot e^{iy} = e^x (\cos y + i \sin y)$ - многолистная функция. \underline{Свойства}: 
    
    \begin{enumerate}
        \item $e^{z_1 + z_2} = e^{z_1} \cdot e^{z_2}$
        \item $\left(e^{z_1}\right)^{z_2} = e^{z_1 z_2}$
        \item $e^{z + 2\pi i} = e^{z} \cdot e^{2\pi i} = e^z$ - показательная функция периодична с периодом $2\pi i$
    \end{enumerate}
    
    \ExN{5} Логарифмическая $w = \Ln z$ - многозначная функция
    
    Если $e^w = e^{u + vi} = e^u (\cos v + i \sin v) = z = |z| e^{i\arg z}$, то $u = \ln |z|$, $v = \arg z + 2\pi k$. Тогда \fbox{$\Ln z = \ln |z| + i (\arg z + 2\pi k)$}
    
    $\ln z = \Ln z$ при $k = 0$ - т. н. главное значение
    
    \ExN{6} Тригонометрические и гиперболические

    \begin{multicols}{2}
        \begin{center}
            $\sin z = \frac{e^{iz} - e^{-iz}}{2i}$

            $\cos z = \frac{e^{iz} + e^{-iz}}{2}$

            $\sh z = \frac{e^{z} - e^{-z}}{2}$

            $\ch z = \frac{e^{z} + e^{-z}}{2}$
        \end{center}
    \end{multicols}

    В $\Complex$ область значений этих функций является $\Complex$ - эти функции не ограничены


    \item Дифференцирование и аналитичность функции комплексной переменной. Условия Коши-Римана. Свойства аналитических функций.

    \Def $f(z)$ называется \hyperlink{derivative}{дифференцируемой} в точке $z_0$, если $\exists f^\prime(z_0) \in \Complex$

    \Defs Дифференцируемая в точке $z_0$ функция $w = f(z)$, производная $f^\prime(z_0)$ которой непрерывна в $z_0$, называется аналитической (или аналитичной) функцией в $z_0$

    \begin{MyTheorem}
        \Ths \hyperlink{cauchy_riemann}{Критерий аналитичности (или Условие Коши-Римана)}
    
        \begin{center}
            $f(x) = u(x, y) + i v(x, y)$ аналитична в точке $z_0 = x + iy$ 
            
            \rotatebox{90}{$\Longleftrightarrow$}
        
            $\exists \frac{\partial u}{\partial x}, \frac{\partial u}{\partial y}, \frac{\partial v}{\partial x}, \frac{\partial v}{\partial y}$ непрерывны в $z$ и
            $\begin{cases}\frac{\partial u}{\partial x} = \frac{\partial v}{\partial y} \\ \frac{\partial u}{\partial y} = -\frac{\partial v}{\partial x}\end{cases}$
        \end{center}
    
        Причем, $f^\prime(z) = u_x + i v_x = v_y - i u_y = u_x - i u_y = v_y + i v_x$
    \end{MyTheorem}
    
    \underline{\hyperlink{analytic_function_properties}{Свойства аналитических функций}}: пусть $f, g$ - аналитические функции, тогда:

    \begin{enumerate}[label=\arabic*$^\circ$]
        \item Линейность: $af + bg$ - аналитическая
        \item Композиция: $f(g(z))$ - аналитическая
        \item Произведение: $f \cdot g$ - аналитическая
    \end{enumerate}

    \item Понятие конформного отображения. Геометрический смысл производной.
    \item Интеграл по комплексной переменной. Теорема Коши. Первообразная.
\end{enumerate}

    

\subsection{Часть 2.}

\begin{enumerate}
    \setcounter{enumi}{6}

    \item Числовые ряды. Регулярная функция. Функциональные ряды. Признаки сходимости.
    \item Степенные ряды. Теорема Абеля. Ряд Тейлора.
    \item Бесконечная дифференцируемость регулярной функции. Формула n-ой производной.
    \item Ряды Лорана в конечной и бесконечно удаленной точке. Главная и правильная части.
    \item Классификация изолированных особых точек однозначной аналитической функции.
    \item Вычет аналитической функции в изолированной особой точке. Определение и формулы вычисления вычета. Основные теоремы теории вычетов.
\end{enumerate}